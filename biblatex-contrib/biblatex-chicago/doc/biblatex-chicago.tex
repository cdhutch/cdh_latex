%
% This file documents the biblatex-chicago package, which allows users
% of the biblatex package to format references according to the
% Chicago Manual of Style, 16th edition.
%
\documentclass[a4paper]{article}
\usepackage[T1]{fontenc}
\usepackage{textcomp}
\usepackage[latin1]{inputenc}
\usepackage[american]{babel}
\usepackage[autostyle]{csquotes}
\usepackage{vmargin}
\setpapersize{A4}
\setmarginsrb{1.65in}{.9in}{1.75in}{.6in}{0pt}{0pt}{12pt}{24pt}
\setlength{\marginparwidth}{1in}
\usepackage[colorlinks,urlcolor=blue,linkcolor=blue]{hyperref}
\usepackage[osf]{mathpazo}
\usepackage[scaled]{helvet}
\usepackage[pdftex]{xcolor}
%\usepackage[dvips]{xcolor}
\newcommand{\mycolor}[1]{\textcolor[HTML]{228B22}{#1}}
\usepackage{multicol}
% Some generic settings.
\newcommand{\cmd}[1]{\texttt{\textbackslash #1}}
\setlength{\parindent}{0pt}
\newcommand{\mymarginpar}[1]{\marginpar{\flushright#1}}
\newcommand{\colmarginpar}[1]{\mymarginpar{\mycolor{#1}}}
\newcommand{\mybigspace}{\vspace{\baselineskip}}
\newcommand{\mylittlespace}{\vspace{.5\baselineskip}}
\makeatletter
\renewcommand{\section}{\@startsection
  {section}%
  {1}%
  {0mm}%
  {\baselineskip}%
  {\baselineskip}%
  {\sffamily\normalsize\bfseries}}%
\renewcommand{\subsection}{\@startsection
  {subsection}%
  {1}%
  {0mm}%
  {\baselineskip}%
  {.5\baselineskip}%
  {\sffamily\normalsize\bfseries}}%
\renewcommand{\subsubsection}{\@startsection
  {subsubsection}%
  {1}%
  {0mm}%
  {\baselineskip}%
  {.5\baselineskip}%
  {\sffamily\normalsize\bfseries}}%
\renewcommand{\paragraph}{\@startsection
  {paragraph}%
  {1}%
  {\z@}%
  {\baselineskip}%3.25ex \@plus1ex \@minus.2ex}%
  {0mm}%
  {}}%
\makeatother
\begin{document}
\begin{center}
  \sffamily\large\bfseries The biblatex-chicago package: \\
  Style files for biblatex

\vspace{.3\baselineskip}
\sffamily\normalsize\bfseries David Fussner\qquad Version 0.9.9g (beta) \\
\href{mailto:djf027@googlemail.com}{djf027@googlemail.com}\\ \today

\end{center} 
\setcounter{tocdepth}{3}
\begin{multicols}{2}
\footnotesize
\tableofcontents
\end{multicols}
\normalsize
\vspace{-.5\baselineskip}
\section{Notice}
\label{sec:Notice}

\textbf{Please be advised that this package is beta software.  Philipp
  Lehman's \textsf{biblatex} package has now reached a stable state,
  and is unlikely to require wholesale changes to styles written for
  it.  The \textsf{biblatex-chicago} package has, for the last few
  releases, implemented the 16th edition of the \emph{Chicago Manual
    of Style}.  I am, at the moment, barely maintaining the
  15th-edition files for those who need or want them.  I have marked
  them as \enquote{strongly deprecated,} and I encourage all users to
  switch to the newer specification, which is receiving just about all
  of my development time.  If the title-formatting changes in the
  author-date style have been an obstacle, please note that the
  \textsf{authordate-trad} style keeps the traditional title
  formatting but switches everything else to the 16th-edition spec.  I
  have summarized the changes between the two editions in
  section~\ref{sec:history} below, especially the ones that may
  require alterations to your .bib files.  (The 15th-edition
  documentation is still available, also, in
  \textsf{biblatex-chicago15.pdf}.)  I also strongly encourage all
  users who haven't already done so to switch to \textsf{Biber} as
  their backend; it has long been a requirement for the author-date
  styles, but it is now becoming indispensable for accessing all the
  features of the notes \&\ bibliography style, as well.\mylittlespace\\
  I have tried to implement as much of the \emph{Manual's}
  specification as possible, though undoubtedly some gaps remain.  One
  user has recently argued that I should attempt to include legal
  citations, so in the long term it may be that I return to this
  issue.  In the meantime, if it seems like this package could be of
  use to you, yet it doesn't do something you need/want it to do,
  please feel free to let me know, and of course any suggestions for
  solving problems more elegantly or accurately would be most
  welcome.}

\mylittlespace\textbf{Important Note:} If you have used
\textsf{biblatex-chicago} before, please make sure you have read the
RELEASE file that came with the package.  It details the changes
you'll need to make to your .bib database in order for it to work
properly with this release.  If you are new to these styles, please
read on.

\section{Quickstart}
\reversemarginpar

The \textsf{biblatex-chicago} package is designed for writers who wish
to use \LaTeX\ and \textsf{biblatex}, and who either want or need to
format their references according to one of the specifications defined
by the \emph{Chicago Manual of Style}.  This package includes two
versions of the \emph{Manual's} \enquote{author-date} system, favored
by many disciplines in the sciences and social sciences, and also its
\enquote{notes \&\ bibliography} style, generally favored in the
humanities.  The latter code produces a full reference in a first
footnote, shorter references in subsequent notes, and a full reference
in the bibliography. Some authors prefer to use the shorter note form
even for the first occurrence, relying on the bibliography to provide
the full information.  This, too, is supported by the code.  The
author-date styles produce a short, in-text citation inside
parentheses --- (Author Year) --- keyed to a list of references where
entries start with the same name and year.

\enlargethispage{-\baselineskip}

\mylittlespace The documentation you are reading covers all three of
these Chicago styles and their variants.  Much of what follows is
relevant to all users, but I have decided, after some experimentation,
to keep the instructions for the two author-date styles separate from
those pertaining to the notes \&\ bibliography style, at least in
sections~\ref{sec:Spec} and \ref{sec:authdate}.  Information provided
under one style will often duplicate that found under the other, but
efficiency's loss should, I hope, be clarity's gain, and much of what
you learn using one style will be applicable without alteration to the
other.  Within the author-date section, the \textsf{authordate-trad}
information really only appears separately in
section~\ref{sec:fields:authdate}, s.v.\ \enquote{title.} Throughout
the documentation, any \mycolor{green} text
\colmarginpar{\textsf{New!}} indicates something \mycolor{new} in this
release.

\mylittlespace Here's a list of things you will need in order to use
\textsf{biblatex-chicago}:

\begin{itemize}{}{}
\item Philipp Lehman's \textsf{biblatex} package, of course!  You
  should use the current version(s) --- 1.7 or 2.9a at the time of
  writing --- as my code has been most extensively tested with, and
  relies on features and bug fixes only available in, this current
  release.  Lehman's tools require several packages, and he strongly
  recommends several more:
  \begin{itemize}{}{}
  \item e-\TeX\ (required)
  \item \textsf{etoolbox} --- available from CTAN (required)
  \item \textsf{keyval} --- a standard package (required)
  \item \textsf{ifthen} --- a standard package (required)
  \item \textsf{url} --- a standard package (required)
  \item \textsf{babel} --- a standard package (\emph{strongly}
    recommended)
  \item \textsf{csquotes} --- available from CTAN (recommended).
    Please upgrade to the latest version of \textsf{csquotes} (5.1b).
  \item \textsf{bibtex8} --- a replacement for \textsc{Bib}\TeX, which
    can, with the right com\-mand-line switches, process very large
    .bib files.  It also does the right thing when alphabetizing
    non-ASCII entries.  It is available from CTAN, but please be aware
    that this database parser no longer suffices if you are using the
    Chicago author-date style with any version of \textsf{biblatex}
    from version 1.5 onwards.  For that style, and to take full
    advantage of all the features of the notes \&\ bibliography style,
    in particular its enhanced handling of cross references, you must
    use the following:
  \item \textsf{Biber} --- the next-generation \textsc{Bib}\TeX\
    replacement, which is available from SourceForge.  You should use
    the latest version, 1.9, to work with \textsf{biblatex} 2.9a and
    \textsf{biblatex-chicago}, and it is required for users who are
    either using the author-date styles or processing a .bib file in
    Unicode.  See \textsf{cms-dates-sample.pdf} and, for example, the
    \textsf{crossref} documentation in section~\ref{sec:entryfields},
    below, for more details.
  \end{itemize}
\item The line:
  \begin{quote}
    \cmd{usepackage[notes]\{biblatex-chicago\}}
  \end{quote}
  in your document preamble to load the notes \&\ bibliography style,
  the line:
  \begin{quote}
    \cmd{usepackage[authordate,backend=biber]\{biblatex-chicago\}}
  \end{quote}
  to load the author-date style, or the line:
  \begin{quote}
    \cmd{usepackage[authordate-trad,%
      backend=biber]\{biblatex-chicago\}}
  \end{quote}

  to load the traditional variant of the author-date style.  (You can
  use \texttt{notes15} or \texttt{authordate15} to load the
  15th-edition styles.  Please see \textsf{bibla\-tex-chicago15.pdf}
  for the details.)  Any other options you usually pass to
  \textsf{biblatex} can be given to \textsf{biblatex-chicago} instead,
  but loading it this way sets up a number of other parameters
  automatically.  You can also load the package via the usual
  \cmd{usepackage\{biblatex\}}, adding either
  \texttt{style=chicago\-notes} or \texttt{style=chicago-authordate},
  but this is mainly for those who wish to set much of the low-level
  formatting of their document themselves.  Please see
  sections~\ref{sec:loading} and \ref{sec:loading:auth} below for a
  fuller discussion of the issues involved here.
\item You can use \cmd{usepackage[notes,short]\{biblatex-chicago\}} to
  get the short note format even in the first reference of a notes \&\
  bibliography document, letting the bibliography provide the full
  reference.
\item If you are accustomed to using the \textsf{natbib} compatibility
  option with \textsf{biblatex}, then you can continue to do so with
  \textsf{biblatex-chicago}.  If you are using
  \cmd{usepackage\{biblatex-chicago\}} to load the package, then the
  option must be the plain \texttt{natbib} rather than
  \texttt{natbib=true}.  If you use the latter, you'll get a
  \textsf{keyval} error.  Please see sections~\ref{sec:useropts} and
  \ref{sec:authuseropts}, below.
\item By far the simplest setup is to use \textsf{babel}, and to have
  \texttt{american} as the main text language.  (\textsf{Polyglossia}
  should work, too, but I haven't tested it.)  As before,
  \textsf{babel}-less setups, and also those choosing \texttt{english}
  as the main text language, should work out of the box.
  \textsf{Biblatex-chicago} also now provides (at least partial)
  support for British, Finnish, French, German, Icelandic, and
  Norwegian.  Please see below (section~\ref{sec:international}) for a
  fuller explanation of all the options.
\item \textsf{chicago-auth\-ordate.cbx},
  \textsf{chi\-cago-authordate.bbx},
  \textsf{chi\-cago-authordate-trad.cbx},
  \textsf{chicago-auth\-ordate-trad.bbx}, \textsf{chicago-notes.bbx},
  \textsf{chi\-cago-notes.cbx}, \textsf{cms-am\-erican.lbx},
  \textsf{cms-british.lbx}, \textsf{cms-finnish.lbx},
  \textsf{cms-french.lbx}, \textsf{cms-ger\-man.lbx},
  \textsf{cms-icelandic.lbx}, \textsf{cms-ngerman.lbx},
  \textsf{cms-norsk.lbx}, \textsf{cms-norwe\-gian.lbx},
  \textsf{cms-nynorsk.lbx}, and \textsf{biblatex-chicago.sty}, all
  from \textsf{biblatex-chicago}, installed either in a system-wide
  \TeX\ directory, or in the working directory where you keep your
  *.tex files.  (To use the 15th-edition styles, you'll also require
  \textsf{chicago-notes15.bbx}, \textsf{chicago-notes15.cbx},
  \textsf{chicago-authordate15.bbx}, and
  \textsf{chicago-authordate15.cbx}.)  The .zip file from CTAN
  contains several subdirectories to help keep the growing number of
  files organized, so the files listed above can be found in the
  \texttt{latex/} subdirectory, itself further divided into the
  \texttt{bbx/}, \texttt{cbx/}, and \texttt{lbx/} subdirectories.  If
  you install in a system-wide directory, I suggest using the standard
  layout and creating
  \texttt{<TEXMFLOCAL>/tex/latex/biblatex-contrib/biblatex-chicago},
  where\ \texttt{<TEXMFLOCAL>} is the root of your local \TeX\
  installation --- for example, and depending on your system and
  preferences, \texttt{/usr/share/texmf\-local},
  \texttt{/usr/local/share/texmf}, or \texttt{C:\textbackslash{}Local
    TeX Files\textbackslash}.  Then you can copy the contents of the
  \texttt{latex/} directory there, subdirectories and all.  (If you
  install into your working directory, then you'll need to copy the
  files directly there, without subdirectories.)  Of course, if you
  choose to place them anywhere in the \texttt{texmf} tree, you'll
  need to update the file name database to make sure \TeX\ can find
  them.
\item Philipp Lehman's very clear and detailed documentation of the
  \textsf{biblatex} system, available in his package as
  \textsf{biblatex.pdf}.  Here he explains why you might want to use
  the system, the rules for constructing .bib files for it, and the
  (numerous) methods at your disposal for modifying the formatted
  output.
\item The annotated bibliography files \textsf{notes-test.bib} and
  \textsf{dates-test.bib}, which will acquaint you with most of the
  details on how to get started constructing your own .bib files for
  use with the two \textsf{biblatex-chicago} styles.
\item The files \textsf{cms-notes-sample.pdf},
  \textsf{cms-dates-sample.pdf}, and \textsf{cms-trad-sam\-ple.pdf}.
  The first shows how my system processes \textsf{notes-test.bib} and
  \textsf{cms-notes-sample.tex}, in both footnotes and bibliography,
  the second and third are the result of processing
  \textsf{dates-test.bib} with \textsf{cms-dates-sample.tex} or
  \textsf{cms-trad-sample.tex}.  All of these files are in
  \texttt{doc/examples/}.
\item The file you are reading, \textsf{biblatex-chicago.pdf}, which
  aims to be as complete a description as possible of the rules for
  creating a .bib file that will, when processed by \LaTeX\ and
  \textsc{Bib}\TeX, at least somewhat ease the burden when you try to
  implement the \emph{Chicago Manual of Style}'s specifications.
  These docs may seem frustratingly over-long, but remember that you
  only need to read the part(s) that apply to the style in which you
  are interested.  Much of the information in section~\ref{sec:Spec}
  is duplicated in section~\ref{sec:authdate}, so even if you have a
  need for multiple styles then using one will be excellent
  preparation for the others.  If you have used a previous version of
  this package, please pay particular attention to the sections on
  Obsolete and Deprecated Features, starting on
  page~\pageref{deprec:obsol}.  You will find the seven previous files
  in the \texttt{doc/} subdirectory once you've extracted
  \textsf{biblatex-chicago.zip}.  If you wish to place them in a
  system-wide directory, I would recommend
  \texttt{<TEXMFLOCAL>/doc/latex/biblatex-contrib/biblatex-chicago},
  all the while remembering, of course, to update the file name
  database afterward.  (Let me reiterate, also, that if you currently
  have quoted material in your .bib file, and are using \cmd{enquote}
  or the standard \LaTeX\ mechanisms there, then the simplest
  procedure is always to use \cmd{mkbibquote} instead in order to
  ensure that punctuation works out right.)
\item Access to a copy of \emph{The Chicago Manual of Style} itself,
  which naturally contains incomparably more information than I can
  hope to present here.  It should always be your first port of call
  when any doubts arise as to exactly what the specification requires.
\end{itemize}

\subsection{License}
\label{sec:lppl}

Copyright � 2008--2014 David Fussner.  This package is
author-maintained.  This work may be copied, distributed and/or
modified under the conditions of the \LaTeX\ Project Public License,
either version 1.3 of this license or (at your option) any later
version.  The latest version of this license is in
http://www.latex-project.org/lppl.txt and version 1.3 or later is part
of all distributions of \LaTeX\ version 2005/12/01 or later.  This
software is provided \enquote{as is,} without warranty of any kind,
either expressed or implied, including, but not limited to, the
implied warranties of merchantability and fitness for a particular
purpose.

\subsection{Acknowledgements}
\label{sec:acknowl}

Even a cursory glance at the cbx and bbx files in the package will
demonstrate how much of Lehman's code from \textsf{biblatex} I've
adapted and re-used, and I've also followed some of the advice he gave
to others in the \texttt{comp.text.tex} newsgroup.  He has been
instrumental in improving the contextual capitalization procedures of
which the style makes such frequent use, and his advice on
constructing \textsf{biblatex-chicago.sty} was invaluable.  The code
for formatting the footnote marks, and that for printing the
separating rule only after a run-on note, I've adapted from the
\textsf{footmisc} package by Robin Fairbairns, and I've borrowed ideas
for the \texttt{shorthandibid} option from Dominik Wa�enhoven's
\textsf{biblatex-dw} package.  I've adapted Audrey Boruvka's
\cmd{textcite} code from
\href{http://tex.stackexchange.com/questions/67837/citations-as-nouns-in-biblatex-chicago}{Stackexchange}
for the notes \&\ bibliography style, and her page-number-compression
code for both styles from the
\href{http://tex.stackexchange.com/questions/44492/biblatex-chicago-style-page-ranges}{same
  site}.  I am very grateful to Antti-Juhani Kaijahano for the Finnish
localization, to Baldur Kristinsson for providing the Icelandic
localization, and to H�kon Malmedal for the Norwegian localizations.
Kazuo Teramoto and Gildas Hamel both sent patches to improve the
package, and there may be other \LaTeX\ code I've appropriated and
forgotten, in which case please remind me.  Finally, Charles Schaum
and Joseph Reagle Jr.\ were both extremely generous with their help
and advice during the development of this package, and have both
continued indefatigably to test it and suggest needed improvements.
They were particularly instrumental in encouraging the greatest
possible degree of compatibility with other \textsf{biblatex} styles.
Indeed, if the task of adapting .bib files for use with the Chicago
style seems onerous now, you should have tried it before they got
their hands on it.

\section{Detailed Introduction}
\label{sec:Intro}

The \emph{Chicago Manual of Style}, implemented here in its 16th
edition, has long, in America at least, been one of the most
influential style guides for writers and publishers.  While one's
choices are now perhaps more extensive than ever, the \emph{Manual} at
least still provides a widely-recognized, and widely-utilized,
standard.  Indeed, when you add to this the sheer completeness of the
specification, its detailed instructions for referencing an enormous
number of different kinds of source material, then your choice (or
your publisher's choice) of the \emph{Manual} as a style guide seems
set to be a happy one.

%% %\enlargethispage{-\baselineskip}

\mylittlespace These very strengths, however, also make the style
difficult to use.  Admittedly, the \emph{Manual} emphasizes
consistency within a work, as opposed to rigid adherence to the
specification, at least when writer and publisher agree (14.70).
Sometimes a publisher demands such adherence, however, and anyone who
has attempted to produce it may well come away with the impression
that the specification itself is somewhat idiosyncratic in its
complexity, and I can't help but agree.  In the notes \&\ bibliography
style, the numerous differences in punctuation (and strings
identifying translators, editors, and the like) between footnotes and
bibliographies and the sometimes unusual location of page numbers; in
both styles the distinction between \enquote{journal} and
\enquote{magazine,} and the formatting differences between (e.g.)\ a
work from antiquity and one from the Renaissance, all of these tend to
overburden the writer who wants to comply with the standard.  Many of
these complexities, in truth, make the specification very nearly
impossible to implement straightforwardly in a system like
\textsc{Bib}\TeX\ --- options multiply, each requiring a particular
sort of formatting, until one almost reaches the point of believing
that every individual book or article should have its own entry type.
Completeness and usability tend each to exclude the other, so the code
you have before you is a first attempt to achieve the former without
utterly sacrificing the latter.

\subsection*{What \textsf{biblatex-chicago} can and can't do}
\label{sec:bltries}

In short, the \textsf{biblatex} style files in this package try to
simplify the task of following the two Chicago specifications along
with their major variants.  In the notes \&\ bibliography style, the
two sorts of reference are treated separately (as are the two
different note forms, long and short), and you can choose always to
use the short note form, even at the first citation.  In the two
author-date styles, a series of options allows you to choose which
date (original printing, reprint, or both) appears in citations and at
the head of entries in the list of references.  In all styles,
punctuation is placed within quotation marks when needed, and as a
general rule as many parts of the style as possible are implemented as
transparently as possible.  Thanks to advice I received from Joseph
Reagle Jr.\ and Charles Schaum while these files were a work in
progress, I have attended as carefully as I can to backward
compatibility with the standard \textsf{biblatex} styles, and have
attempted to minimize both any changes you need to make to achieve
compliance with the Chicago specification, and indeed also any changes
necessary to switch between the two Chicago styles.  There is no doubt
room for improvement on this score, but even now, for a substantial
number of entries, any well-constructed .bib file that works for other
\textsf{biblatex} styles will \enquote{just work} under
\textsf{biblatex-chicago}.  By no means, however, will all entries in
such a .bib file produce equally satisfactory results.  Using this
documentation and the examples in \textsf{dates-test.bib} and/or
\textsf{notes-test.bib}, it should be possible to achieve compliance,
though the amount of revision necessary to do so will vary
significantly from .bib file to .bib file.  Conversely, once you have
created a database for \textsf{biblatex-chicago}, it won't necessarily
work well with other \textsf{biblatex} styles.  Indeed, most, quite
possibly all, users will find that they need to use special formatting
macros within the .bib file that would make such a file unusable in
any other context.  I strongly recommend, if you want to experiment
with this style, that you work on a copy of any .bib files that are
important to you, until you have determined that this package does
what you need/want it to do.

% %\enlargethispage{\baselineskip}

\mylittlespace When I first began working on this package, I made the
decision to alter as little as possible the main files from Lehman's
\textsf{biblatex}, so that my .bbx and .cbx files would use his
original \LaTeX\ .sty file and \textsc{Bib}\TeX\ .bst file.  As you
proceed, you will no doubt encounter some of the consequences of this
decision, with certain fields and entry types in the .bib file having
less-than-memorable names because I chose to use the supplementary
ones provided by \textsf{biblatex.bst} rather than alter that file.  I
intended then, if it turned out that anyone besides myself actually
used \textsf{biblatex-chicago}, to ask Mr.\ Lehman to include more
descriptive names for these few entry types and fields in
biblatex.bst, if he were willing.  As luck would have it, several new
types appeared in \textsf{biblatex} 0.8, many of which I have
incorporated as replacements for the custom entry types I defined
before.  If a consensus emerges about how best to assign the data to
various fields in such entries, then I shall adopt it.  In the
meantime, as you will see below, I have made two of the old custom
types obsolete, and recycled the third for an entirely new purpose.
Needless to say, I'm open to advice and suggestions on this score.

\section{The Specification:\ Notes\,\&\,Bibliography}
\label{sec:Spec}

In what follows, I attempt to explain all the parts of
\textsf{biblatex-chicago-notes} that might be considered somehow
\enquote{non standard,} at least with respect to the styles included
with \textsf{biblatex} itself, though in the section on entry fields I
have also duplicated a lot of the information in
\textsf{biblatex.pdf}, which I hope won't badly annoy expert users of
the system.  Headings in \mycolor{green} \colmarginpar{\textsf{New in
    this release}} indicate material new to this release, or
occasionally old material that has undergone significant revision.
Numbers in parentheses refer to sections of the \emph{Chicago Manual
  of Style}, 16th edition.  The file \textsf{notes-test.bib} contains
many examples from the \emph{Manual} which, when processed using
\textsf{biblatex-chicago-notes}, should produce the same output as you
see in the \emph{Manual} itself, or at least compliant output, where
the specifications are vague or open to interpretation, a state of
affairs which does sometimes occur.  I have provided
\textsf{cms-notes-sample.pdf}, which shows how my system processes
\textsf{notes-test.bib}, and I have also included the reference keys
from the latter file below in parentheses.

\subsection{Entry Types}
\label{sec:entrytypes}

The complete list of entry types currently available in
\textsf{biblatex-chicago-notes}, minus the odd \textsf{biblatex}
alias, is as follows: \textbf{article}, \textbf{artwork},
\textbf{audio}, \mycolor{\textbf{book}},
\mycolor{\textbf{bookinbook}}, \textbf{booklet},
\mycolor{\textbf{collection}}, \textbf{customc}, \textbf{image},
\mycolor{\textbf{inbook}}, \textbf{incollection},
\textbf{inproceedings}, \textbf{inreference}, \textbf{letter},
\textbf{manual},\textbf{misc}, \textbf{music},
\mycolor{\textbf{mvbook}}, \mycolor{\textbf{mvcollection}},
\mycolor{\textbf{mvproceedings}}, \mycolor{\textbf{mvreference}},
\textbf{online} (with its alias \textbf{www}), \textbf{patent},
\textbf{periodical}, \mycolor{\textbf{proceedings}},
\textbf{reference}, \textbf{report} (with its alias
\textbf{techreport}), \textbf{review}, \textbf{suppbook},
\textbf{supp\-collection}, \textbf{suppperiodical}, \textbf{thesis}
(with its aliases \textbf{mastersthesis} and \textbf{phdthesis}),
\textbf{unpublished}, and \textbf{video}.

\mylittlespace What follows is an attempt to specify all the
differences between these types and the standard provided by
\textsf{biblatex}.  If an entry type isn't discussed here, then it is
safe to assume that it works as it does in the standard styles.  In
general, I have attempted not to discuss specific entry fields here,
unless such a field is crucial to the overall operation of a given
entry type.  As a general and important rule, most entry types require
very few fields when you use \textsf{biblatex-chicago-notes}, so it
seemed to me better to gather information pertaining to fields in the
next section.

\mybigspace The \mymarginpar{\textbf{article}} \emph{Chicago Manual of
  Style} (14.170) recognizes three different sorts of periodical
publication, \enquote{journals,} \enquote{magazines,} and
\enquote{newspapers.} The first (14.172) includes \enquote{scholarly
  or professional periodicals available mainly by subscription,} while
the second refers to \enquote{weekly or monthly} publications that are
\enquote{available either by subscription or in individual issues at
  bookstores or newsstands or online.}  \enquote{Magazines} will tend
to be \enquote{more accessible to general readers,} and typically
won't have a volume number.  Indeed, by fiat I declare that should you
need to refer to a journal that identifies its issues mainly by year,
month, or week, then for the purposes of
\textsf{biblatex-chicago-notes} such a publication is a
\enquote{magazine,} and not a \enquote{journal.}

\mylittlespace Now, for articles in \enquote{journals} you can simply
use the traditional \textsc{Bib}\TeX\ --- and indeed \textsf{biblatex}
--- \textsf{article} entry type, which will work as expected and set
off the page numbers with a colon, as required by the \emph{Manual}.
If, however, you need to refer to a \enquote{magazine} or a
\enquote{newspaper,} then you need to add an \textsf{entrysubtype}
field containing the exact string \texttt{magazine}.  The main
formatting differences between a \texttt{magazine} (which includes
both \enquote{magazines} and \enquote{newspapers}) and a plain
\textsf{article} are that the year isn't placed within parentheses,
and that page numbers are set off by a comma rather than a colon.
Otherwise, the two sorts of reference have much in common.  (For
\textsf{article}, see \emph{Manual} 14.175--198; batson,
beattie:crime, friedman:learning, garaud:gatine, garrett, hlatky:hrt,
kern, lewis, loften:hamlet, mcmillen:antebellum, rozner:liberation,
saberhagen:beluga, warr:ellison, white:callimachus. For
\textsf{entrysubtype} \texttt{magazine}, see 14.181, 14.199--202;
assocpress:gun, morgenson:market, reaves:rosen, sten\-ger:privacy.)

\mylittlespace It gets worse.  The \emph{Manual} treats reviews (of
books, plays, performances, etc.) as a sort of recognizable subset of
\enquote{journals,} \enquote{magazines,} and \enquote{newspapers,}
distinguished mainly by the way one formats the title of the review
itself.  In \textsf{biblatex 0.7}, happily, Lehman provided a
\textsf{review} entry type which will handle a large subset of such
citations, though not all.  The key rule is this: if a review has a
separate, non-generic title (gibbard; osborne:poison) in addition to
something that reads like \enquote{review of \ldots,} then you need an
\textsf{article} entry, with or without the \texttt{magazine}
\textsf{entrysubtype}, depending on the sort of publication containing
the review.  If the only title is the generic \enquote{review of
  \ldots,} for example, then you'll need the \textsf{review} entry
type, with or without this same \textsf{entrysubtype} toggle using
\texttt{magazine}.  On \textsf{review} entries, see below.  (The
curious reader will no doubt notice that the code for formatting any
sort of review still exists in \textsf{article}, as it was initially
designed for \textsf{biblatex 0.6}, but this new arrangement is
somewhat simpler and therefore, I hope, better.)

%%\enlargethispage{-\baselineskip}

\mylittlespace In the case of a review with a specific as well as a
generic title, the former goes in the \textsf{title} field, and the
latter in the \textsf{titleaddon} field.  Standard \textsf{biblatex}
intends this field for use with additions to titles that may need to
be formatted differently from the titles themselves, and
\textsf{biblatex-chicago-notes} uses it in just this way, with the
additional wrinkle that it can, if needed, replace the \textsf{title}
entirely, and this in, effectively, any entry type, providing a fairly
powerful, if somewhat complicated, tool for getting \textsc{Bib}\TeX\
to do what you want.  Here, however, if all you need is a
\textsf{titleaddon}, then you want to switch to the \textsf{review}
type, where you can simply use the \textsf{title} field instead.

\mylittlespace No less than eight more things need explication here.
First, since the \emph{Manual} specifies that much of what goes into a
\textsf{titleaddon} field stays unformatted --- no italics, no
quotation marks --- this plain style is the default for such text,
which means that you'll have to format any titles within
\textsf{titleaddon} yourself, e.g., with \cmd{mkbibemph\{\}}.  Second,
the \emph{Manual} specifies a similar plain style for the titles of
other sorts of material found in \enquote{magazines} and
\enquote{newspapers,} e.g., obituaries, letters to the editor,
interviews, the names of regular columns, and the like.  References
may contain both the title of an individual article and the name of
the regular column, in which case the former should go, as usual, in a
\textsf{title} field, and the latter in \textsf{titleaddon}.  As with
reviews proper, if there is only the generic title, then you want the
\textsf{review} entry type.  (See 14.203, 14.205, 14.208;
morgenson:market, reaves:rosen.)

\mylittlespace Third, the 16th edition of the \emph{Manual} has, I
believe, subtly changed its recommendations in the case of
\enquote{unsigned newspaper articles or features} (14.207).
Unfortunately, these changes aren't entirely clear to me.  First, it
suggests that such pieces are \enquote{best dealt with in text or
  notes.}  If, however, \enquote{a bibliography entry should be
  needed, the name of the newspaper stands in place of the author.}
The examples it provides, therefore, suggest quite different
treatments of the same material in notes and bibliography, and they
don't at any point that I can see recommend a format for short notes.
I've implemented these recommendations fairly literally, which means
that in an \textsf{article} entry, \textsf{entrysubtype}
\texttt{magazine}, or in a \textsf{review} entry,
\textsf{entrysubtype} \texttt{magazine}, and \emph{only} in such
entries, a missing \textsf{author} field results in the name of the
periodical (in the \textsf{journaltitle} field) being used as the
missing author, but \emph{only} in the bibliography and in short
notes.  In long notes, the \textsf{title} will appear first, before
the \textsf{journaltitle}.  Note that the use of the name of the
newspaper as an author creates sorting issues in the bibliography,
issues that will mostly be solved for you if you use \textsf{Biber} as
the backend.  If you don't, or if the \textsf{journaltitle} begins
with a definite or indefinite article with which you can't dispense,
then you'll need a \textsf{sortkey} field to ensure that the
bibliography entry is alphabetized correctly.  (See
lakeforester:pushcarts and, for the sorting issue,
\cmd{DeclareSortingScheme} in section~\ref{sec:formatopts} below.)

\enlargethispage{-\baselineskip}

\mylittlespace Fourth, Bertold Schweitzer has pointed out, following
the \emph{Manual} (14.192), that while an \textsf{issuetitle} often
has an \textsf{editor}, it is not too unusual for a \textsf{title} to
have, e.g., an \textsf{editor} and/or a \textsf{translator}.  In order
to allow as many permutations as possible on this theme, I have
brought the \textsf{article} entry type into line with most of the
other types in allowing the use of the \textsf{namea} and
\textsf{nameb} fields in order to associate an editor or a translator
specifically with the \textsf{title}.  The \textsf{editor} and
\textsf{translator} fields, in strict homology with other entry types,
are associated with the \textsf{issuetitle} if one is present, and
with the \textsf{title} otherwise.  The usual string concatenation
rules still apply --- cf.\ \textsf{editor} and \textsf{editortype} in
section~\ref{sec:entryfields}, below.

\mylittlespace Fifth, if you've been using
\textsf{biblatex-chicago-notes} for a while, you may remember using
the single-letter \cmd{bibstring} mechanism in order to help
\textsf{biblatex} decide where to capitalize a wide variety of strings
in numerous entry fields.  This mechanism was particularly common in
all the periodical types, but if you've had a look in
\textsf{notes-test.bib} while following this documentation, you'll
have noticed that it no longer appears there.  The regular whole-word
bibstrings still work as normal, but the single-letter ones are now
obsolete, replaced by Lehman's macro \cmd{autocap}, which itself only
occurs twice in \textsf{notes-test.bib}.  Basically, in certain
fields, just beginning your data with a lowercase letter activates the
mechanism for capitalizing that letter depending on its context within
a note or bibliography entry.  Please see \textbf{\textbackslash
  autocap} below for the details, but both the \textsf{titleaddon} and
\textsf{note} fields are among those treating their data this way, and
since both appear regularly in \textsf{article} entries, I thought the
problem merited a preliminary mention here.

\mylittlespace Sixth, if you need to cite an entire issue of any sort
of periodical, rather than one article in an issue, then the
\textsf{periodical} entry type, once again with or without the
\texttt{magazine} toggle in \textsf{entrysubtype}, is what you'll
need.  (You can also use the \textsf{article} type, placing what would
normally be the \textsf{issuetitle} in the \textsf{title} field and
retaining the usual \textsf{journaltitle} field, but this arrangement
isn't compatible with standard \textsf{biblatex}.)  The \textsf{note}
field is where you place something like \enquote{special issue} (with
the small \enquote{s} enabling the automatic capitalization routines),
whether you are citing one article or the whole issue
(conley:fifthgrade, good:wholeissue).  Indeed, this is a somewhat
specialized use of \textsf{note}, and if you have other sorts of
information you need to include in an \textsf{article},
\textsf{periodical}, or \textsf{review} entry, then you shouldn't put
it in the \textsf{note} field, but rather in \textsf{titleaddon} or
perhaps \textsf{addendum} (brown:bremer).

\mylittlespace Seventh, if you wish to cite a television or radio
broadcast, the \textsf{article} type, \textsf{entrysubtype}
\texttt{magazine} is the place for it.  The name of the program would
go in \textsf{journaltitle}, with the name of the episode in
\textsf{title}, and the network's name in the \textsf{usera} field.
Of course, if the piece you are citing has only a generic name (an
interview, for example), then the \textsf{review} type would be the
best place for it.  (8.185, 14.221; see bundy:macneil for an example
of how this all might look in a .bib file.  Commercial recordings of
such material would need one of the audiovisual entry types, probably
\textsf{audio} or \textsf{video} [friends:leia], while recordings from
archives fit best into \textsf{misc} entries with an
\textsf{entrysubtype} [coolidge:speech, roosevelt:speech].)

\mylittlespace Finally, the 16th edition of the \emph{Manual}
(14.243--6) specifies that blogs and other, similar online material
should be presented like \textsf{articles}, with \texttt{magazine}
\textsf{entrysubtype} (ellis:blog).  The title of the specific entry
goes in \textsf{title}, the general title of the blog goes in
\textsf{journaltitle}, and the word \enquote{\texttt{blog}} in the
\textsf{location} field (though you could just use special formatting
in the \textsf{journaltitle} field itself, which may sometimes be
necessary).  Comments on blogs, with generic titles like
\enquote{comment on} or \enquote{reply to,} need a \textsf{review}
entry with the same \textsf{entrysubtype}.  Such comments make
particular use of the \textsf{eventdate} and of the \textsf{nameaddon}
fields; please see the documentation of \textbf{review}, below.

% %\enlargethispage{-\baselineskip}

\mylittlespace If you're still with me, allow me to recommend that you
browse through \textsf{notes-test.bib} to get a feel for just how many
of the \emph{Manual}'s complexities the \textsf{article} and
\textsf{review} (and, indeed, \textsf{periodical}) types attempt to
address.  It may be that in future releases of
\textsf{biblatex-chicago-notes} I'll be able to simplify these
procedures somewhat, but in the meantime it might be of some comfort
that I have found in my own research that the unusual and/or limit
cases are really rather rare, and that the vast majority of sources
won't require any knowledge of these onerous details.

\mybigspace Arne \mymarginpar{\textbf{artwork}} Kjell Vikhagen has
pointed out to me that none of the standard entry types were
straightforwardly adaptable when referring to visual artworks.  The
\emph{Manual} doesn't give any thorough specifications for such
references, and indeed it's unclear that it believes it necessary to
include them in the bibliographical apparatus at all.  Still, it's
easy to conceive of contexts in which a list of artworks studied might
be desirable, and \textsf{biblatex} includes entry types for just this
purpose, though the standard styles leave them undefined.  The two I
chose to include in previous releases were \textsf{artwork} and
\textsf{image}, the former intended for paintings, sculptures,
etchings, and the like, the latter for photographs.  The 16th edition
of the \emph{Manual} has modified its specifications for presenting
photographs so that they are the same as for works in all other media.
The \textsf{image} type, therefore, is now merely a clone of the
\textsf{artwork} type, maintained mainly to provide backward
compatibility for users migrating from the old specification to the
current one.

\mylittlespace Constructing an entry is fairly straightforward.  As
one might expect, the artist goes in \textsf{author} and the name of
the work in \textsf{title}.  The \textsf{type} field is intended for
the medium --- e.g., oil on canvas, charcoal on paper, albumen print
--- and the \textsf{version} field might contain the state of an
etching.  You can place the dimensions of the work in \textsf{note},
and the current location in \textsf{organization},
\textsf{institution}, and/or \textsf{location}, in ascending order of
generality.  The \textsf{type} field, as in several other entry types,
uses \textsf{biblatex's} automatic capitalization routines, so if the
first word only needs a capital letter at the beginning of a sentence,
use lowercase in the .bib file and let \textsf{biblatex} handle it for
you.  (See \emph{Manual} 3.22, 8.193; leo:madonna, bedford:photo.)

\mylittlespace As a final complication, the \emph{Manual} (8.193) says
that \enquote{the names of works of antiquity \ldots\,are usually set
  in roman.}  If you should need to include such a work in the
reference apparatus, you can either define an \textsf{entrysubtype}
for an \textsf{artwork} entry --- anything will do --- or you could
use the \textsf{misc} entry type with an \textsf{entrysubtype}.
Fortunately, in this instance the other fields in a \textsf{misc}
entry function pretty much as in \textsf{artwork}.

\mybigspace Following \mymarginpar{\textbf{audio}} the request of
Johan Nordstrom, I have included three entry types, all undefined by
the standard styles, designed to allow users to present audiovisual
sources in accordance with the Chicago specifications.  The
\emph{Manual's} presentation of such sources (14.274--280), though
admirably brief, seems to me somewhat inconsistent; the proliferation
of online sources has made the task yet more complex.  For the 15th
edition I attempted to condense all the requirements into two new
entry types, but ended up relying on three.  For the 16th edition, in
particular, I also need to include the \textbf{online} and even the
\textbf{misc} entry types, which see, under the audiovisual rubric.  I
shall attempt to delineate the main differences here, and though there
are likely to be occasions when your choice of entry type is not
obvious, at the very least \textsf{biblatex-chicago} should help you
maintain consistency.

\mylittlespace The \textbf{music} type is intended for all musical
recordings that do not have a video component.  This means, for
example, digital media (whether on CD or hard drive), vinyl records,
and tapes.  The \textbf{video} type includes most visual media,
whether it be films, TV shows, tapes and DVDs of the preceding or of
any sort of performance (including music), or online multimedia.  The
\emph{Manual's} treatment (14.280) of the latter suggests that online
video excerpts, short pieces, and interviews should generally use the
\textbf{online} type (harwood:biden, horowitz:youtube, pollan:plant).
The \textbf{audio} type, our current concern, fills gaps in the
others, and presents its sources in a more \enquote{book-like} manner.
Published musical scores need this type --- unpublished ones would use
\textsf{misc} with an \textsf{entrysubtype} (shapey:partita) --- as do
such favorite educational formats as the slideshow and the filmstrip
(greek:filmstrip, schubert:muellerin, verdi:corsaro).  The
\emph{Manual} (14.277--280) sometimes uses a similar format for audio
books (twain:audio), though, depending on the sorts of publication
facts you wish to present, this sort of material may fall under
\textsf{music} (auden:reading).  Dated audio recordings that are part
of an archive, online or no, may best be presented in a \textbf{misc}
entry with an \textsf{entrysubtype} (coolidge:speech,
roosevelt:speech).

%\enlargethispage{\baselineskip}

\mylittlespace Once you've accepted the analogy of composer to
\textsf{author}, constructing an \textsf{audio} entry should be fairly
straightforward, since many of the fields function just as they do in
\textsf{book} or \textsf{inbook} entries.  Indeed, please note that I
compare it to both these other types as, in common with the other
audiovisual types, \textsf{audio} has to do double duty as an analogue
for both books and collections, so while there will normally be an
\textsf{author}, a \textsf{title}, a \textsf{publisher}, a
\textsf{date}, and a \textsf{location}, there may also be a
\textsf{booktitle} and/or a \textsf{maintitle} --- see
schubert:muellerin for an entry that uses all three in citing one song
from a cycle.  If the medium in question needs specifying, the
\textsf{type} field is the place for it.  Finally, the
\textsf{titleaddon} field can specify functions for which
\textsf{biblatex-chicago} provides no automated handling, e.g., a
librettist (verdi:corsaro).

\mybigspace This \colmarginpar{\textbf{book}} is the standard
\textsf{biblatex} and \textsc{Bib}\TeX\ entry type, but with this
release the package can now provide automatically abbreviated
references in notes and bibliography when you use a \textsf{crossref}
or an \textsf{xref} field.  The functionality is not enabled by
default, but you can enable it in the preamble or in the
\textsf{options} field using the new \mycolor{\texttt{booklongxref}}
option.  Please see \textbf{crossref} in section~\ref{sec:entryfields}
and \texttt{booklongxref} in section~\ref{sec:chicpreset}, below.
Cf.\ harley:ancient:cart, harley:cartography, and harley:hoc for how
this might look.

\mybigspace This \colmarginpar{\textbf{bookinbook}} type provides the
means of referring to parts of books that are considered, in other
contexts, themselves to be books, rather than chapters, essays, or
articles.  Such an entry can have a \textsf{title} and a
\textsf{maintitle}, but it can also contain a \textsf{booktitle}, all
three of which will be italicized when printed.  In general usage it
is, therefore, rather like the traditional \textsf{inbook} type, only
with its \textsf{title} in italics rather than in quotation marks.  As
with the \textsf{book} type, you can now enable automatically
abbreviated references in notes and bibliography, though this isn't
the default.  Please see \textbf{crossref} in
section~\ref{sec:entryfields} and \mycolor{\texttt{booklongxref}} in
section~\ref{sec:chicpreset}, below.  (Cf.\ \emph{Manual} 14.114,
14.127, 14.130; bernhard:boris, bernhard:ritter, and
bernhard:themacher for the new abbreviating functionality; also
euripides:orestes, plato:republic:gr.)

\mylittlespace \textbf{NB}: The Euripides play receives slightly
different presentations in 14.127 and 14.130.  Although the
specification is very detailed, it doesn't eliminate all choice or
variation.  Using a system like \textsc{Bib}\TeX\ should help to
maintain consistency.  

\mybigspace This \mymarginpar{\textbf{booklet}} is the first of two
entry types --- the other being \textsf{manual}, on which see below
--- which are traditional in \textsc{Bib}\TeX\ styles, but which the
\emph{Manual} (14.249) suggests may well be treated basically as
books.  In the interests of backward compatibility,
\textsf{biblatex-chicago-notes} will so format such an entry, which
uses the \textsf{howpublished} field instead of a standard
\textsf{publisher}, though of course if you do decide just to use a
\textsf{book} entry then any information you might have given in a
\textsf{howpublished} field should instead go in \textsf{publisher}.
(See clark:mesopot.)

\mybigspace This \colmarginpar{\textbf{collection}} is the standard
\textsf{biblatex} entry type, but with this release the package can
now provide automatically abbreviated references in notes and
bibliography when you use a \textsf{crossref} or an \textsf{xref}
field.  The functionality is not enabled by default, but you can
enable it in the preamble or in the \textsf{options} field using the
new \mycolor{\texttt{booklongxref}} option.  Please see
\textbf{crossref} in section~\ref{sec:entryfields} and
\texttt{booklongxref} in section~\ref{sec:chicpreset}, below.  See
harley:ancient:cart, harley:cartography, and harley:hoc for how this
might look.

\mybigspace This \mymarginpar{\textbf{customa}} entry type is now
obsolete, and any such entries in your .bib file will trigger an
error.  Please use the standard \textsf{biblatex} \textbf{letter} type
instead.

%\enlargethispage{\baselineskip}

\mybigspace This \mymarginpar{\textbf{customb}} entry type is now
obsolete, and any such entries in your .bib file will trigger an
error.  Please use the standard \textsf{biblatex} \textbf{bookinbook}
type instead.

\mybigspace This \mymarginpar{\textbf{customc}} entry type allows you
to include alphabetized cross-references to other, separate entries in
the bibliography, particularly to other names or pseudonyms, as
recommended by the \emph{Manual}.  (This is different from the usual
\textsf{crossref}, \textsf{xref}, and \textsf{userf} mechanisms, all
primarily designed to include cross-references to other works.  Cf.\
14.84,86).  The lecarre:cornwell entry, for example, would allow your
readers to find the more-commonly-used pseudonym \enquote{John Le
  Carr�} even if they were, for some reason, looking under his real
name \enquote{David John Moore Cornwell.}\ As I read the
specification, these cross-references are particularly encouraged,
bordering on required, when \enquote{a bibliography includes two or
  more works published by the same author but under different
  pseudonyms.}\ The following entries in \textsf{notes-test.bib} show
one way of addressing this: crea\-sey:ashe:blast,
crea\-sey:york:death, crea\-sey:mor\-ton:hide, ashe:crea\-sey,
york:crea\-sey and mor\-ton:crea\-sey.

\mylittlespace In these latter cases, you would need merely to place
the pseudonym in the \textsf{author} field, and the author's real
name, under which his or her works are presented in the bibliography,
in the \textsf{title} field.  To make sure the cross-reference also
appears in the bibliography, you can either manually include the entry
key in a \cmd{nocite} command, or you can put that entry key in the
\textbf{userc} field in the main .bib entry, in which case
\textsf{biblatex-chicago} will print the expanded abbreviation if and
only if you cite the main entry.  (Cf.\ \textsf{userc}, below.)

%%\enlargethispage{\baselineskip}

\mylittlespace Under ordinary circumstances, \textsf{biblatex-chicago}
will connect the two parts of the cross-reference with the word
\enquote{\emph{See}} --- or its equivalent in the document's language
--- in italics.  If you wish to present the cross-reference
differently, you can put the connecting word(s) into the
\textsf{nameaddon} field.

\mybigspace This \mymarginpar{\textbf{image}} entry type, left
undefined in the standard styles, was in previous releases of
\textsf{biblatex-chicago} intended for referring to photographs, but
the 16th edition of the \emph{Manual} has changed its specifications
for such works, which are now treated the same as works in all other
media.  This means that this entry type is now a clone of the
\textsf{artwork} type, which see.  I retain it here as a convenience
for users migrating from the old to the new specification.  (See 3.22,
8.193; bedford:photo.)

\mybigspace These
\mymarginpar{\mycolor{\textbf{inbook}}\\\textbf{incollection}} two
standard \textsf{biblatex} types have very nearly identical formatting
requirements as far as the Chicago specification is concerned, but I
have retained both of them for compatibility.  \textsf{Biblatex.pdf}
(�~2.1.1) intends the first for \enquote{a part of a book which forms
  a self-contained unit with its own title,} while the second would
hold \enquote{a contribution to a collection which forms a
  self-contained unit with a distinct author and its own title.}  The
\textsf{title} of both sorts will be placed within quotation marks,
and in general you can use either type for most material falling into
these categories.  There was an important difference between them, as
in previous releases of \textsf{biblatex-chicago} it was only in
\textsf{incollection} entries that I implemented the \emph{Manual's}
recommendations for space-saving abbreviations in notes and
bibliography when you cite multiple pieces from the same
\textsf{collection}.  These abbreviations are now activated by default
when you use the \textsf{crossref} or \textsf{xref} field in
\textsf{incollection} entries \emph{and} in \textsf{inbook} entries,
because although the \emph{Manual} (14.113) here specifies a
\enquote{multiauthor book,} I believe the distinction between the two
is fine enough to encourage similar treatments.  (For more on this
mechanism see \textbf{crossref} in section~\ref{sec:entryfields},
below, and the new option \mycolor{\texttt{longcrossref}} in
section~\ref{sec:chicpreset}.  Please note that it is also active by
default in \textsf{letter} and \textsf{inproceedings} entries.)  If
the part of a book to which you are referring has had a separate
publishing history as a book in its own right, then you may wish to
use the \textsf{bookinbook} type, instead, on which see above.  (See
\emph{Manual} 14.111--114; \textsf{inbook}: ashbrook:brain,
phibbs:diary, will:cohere; \textsf{incollection}: centinel:letters,
contrib:contrib, sirosh:visualcortex; ellet:galena, keating:dearborn,
and lippincott:chicago [and the \textsf{collection} entry
prairie:state] demonstrate the use of the \textsf{crossref} field with
its attendant abbreviations in notes and bibliography.)

\mylittlespace \textbf{NB}: The \emph{Manual} suggests that, when
referring to a chapter, one use either a chapter number or the
inclusive page numbers, not both.  If, however, you wish to refer in a
footnote to a specific page within the chapter,
\textsf{biblatex-chicago-notes} will always print the optional,
postnote argument of a \cmd{cite} command --- the page number, say ---
instead of any inclusive page numbers given in the .bib file
\textsf{incollection} entry.  This mechanism is quite general, that
is, any specific page reference given in any sort of \cmd{cite}
command overrides the contents of a \textsf{pages} field in a .bib
file entry.

\mybigspace This \mymarginpar{\textbf{inproceedings}} entry type works
pretty much as in standard \textsf{biblatex}.  Indeed, the main
differences between it and \textsf{incollection} are the lack of an
\textsf{edition} field and the possibility that an
\textsf{organization} may be cited alongside the \textsf{publisher},
even though the \emph{Manual} doesn't specify its use (14.226).
Please note, also, that the \textsf{crossref} and \textsf{xref}
mechanism for shortening citations of multiple pieces from the same
\textsf{proceedings} is operative here, just as it is in
\textsf{incollection} and \textsf{inbook} entries.  See
\textbf{crossref} in section~\ref{sec:entryfields} and the new option
\mycolor{\texttt{longcrossref}} in section~\ref{sec:chicpreset} for
more details.

\mybigspace This \mymarginpar{\textbf{inreference}} entry type is
aliased to \textsf{incollection} in the standard styles, but the
\emph{Manual} has particular requirements, so if you are citing
\enquote{[w]ell-known reference books, such as major dictionaries and
  encyclopedias,} then this type should simplify the task of
conforming to the specifications (14.247--248).  The main thing to
keep in mind is that I have designed this entry type for
\enquote{alphabetically arranged} works, which you shouldn't cite by
page, but rather by the name(s) of the article(s).  Because of the
formatting required by the \emph{Manual}, we need one of
\textsf{biblatex's} list fields for this purpose, and in order to keep
all this out of the way of the standard styles, I have chosen the
\textsf{lista} field.  You should present these article names just as
they appear in the work, separated by the keyword
\enquote{\texttt{and}} if there is more than one, and
\textsf{biblatex-chicago-notes} will provide the appropriate prefatory
string (\texttt{s.v.}, plural \texttt{s.vv.}), and enclose each in its
own set of quotation marks (ency:britannica).  In a typical
\textsf{inreference} entry, very few other fields are needed, as
\enquote{the facts of publication are often omitted, but the edition
  (if not the first) must be specified.}  In practice, this means a
\textsf{title} and possibly an \textsf{edition} field.

\mylittlespace There are quite a few other peculiarities to explain
here.  First of all, you should present any well-known works
\emph{only} in notes, not in a bibliography, as your readers are
assumed to know where to go for such a reference.  You can use the
\texttt{skipbib} option to achieve this.  For such works, and given
how little information will be present even in a full note, you may
wish to use \cmd{fullcite} or \cmd{footfullcite} in place of the short
form, especially if, for example, you are citing different versions of
an article appearing in different editions.

\mylittlespace If the work is slightly less well known, it may be that
full publication details are appropriate (times:guide), but this makes
things more complicated.  In previous releases of
\textsf{biblatex-chicago-notes}, you would have had to format the
\textsf{postnote} field of short notes appropriately, including the
prefatory string and quotation marks I mentioned above.  Now you can
put an article name in the \textsf{postnote} field of
\textsf{inreference} entries and have it formatted for you, and this
holds for both long and short notes, which could allow you to refer
separately to many different articles from the same reference work
using only one .bib entry.  (In a long note, any \textsf{postnote}
field stops the printing of the contents of \textsf{lista}.)  The only
limitation on this system is that the \textsf{postnote} field, unlike
\textsf{lista}, is not a list, and therefore for the formatting to
work correctly you can only put one article name in it.  Despite this
limitation, I hope that the current system might simplify things for
users who cite numerous works of reference.

\enlargethispage{-\baselineskip}

\mylittlespace If it seems appropriate to include such a work in the
bibliography, be aware that the contents of the \textsf{lista} field
will also be presented there, which may not be what you want.  A
separate \textsf{reference} entry might solve this problem, but you
may also need a \textsf{sortkey} field to ensure proper
alphabetization, as \textsf{biblatex} will attempt to use an
\textsf{editor} or \textsf{author} name, if either is present.  (Cf.\
mla:style, a \textsf{reference} entry that uses section numbers
instead of alphabetized headings, and \texttt{useeditor=false} in the
\textsf{options} field instead of a \textsf{sortkey} to ensure the
correct alphabetization.)

\mylittlespace Speaking of the \textsf{author}, this field holds the
author of the specific entry (in \textsf{lista}), not the author of
the \textsf{title} as a whole.  This name will be printed after the
entry's name (grove:sibelius).  If you wish to refer to a reference
work by author or indeed by editor, having either appear at the head
of the note (long or short) or bibliography entry, then you'll need to
use a \textsf{book} entry instead (cf.\ schellinger:novel), where the
\textsf{lista} mechanism will also work in the bibliography, but which
in every other way will be treated as a normal book, often a good
choice for unfamiliar or non-standard reference works.

\mylittlespace Finally, all of these rules apply to online reference
works, as well, for which you need to provide not only a \textsf{url}
but also, always, a \textsf{urldate}, as these sources are in constant
flux (wikiped:bibtex, grove:sibelius).

\mybigspace This \mymarginpar{\textbf{letter}} is the entry type to
use for citing letters, memoranda, or similar texts, but \emph{only}
when they appear in a published collection.  (Unpublished material of
this nature needs a \textsf{misc} entry, for which see below.)
Depending on what sort of information you need to present in a
citation, you may simply be able to get away with a standard
\textsf{book} entry, which may then be cited by page number (see
\emph{Manual} 14.78, 14.88; meredith:letters, adorno:benj).  If,
however, for whatever reason, you need to give full details of a
specific letter, then you'll need to use the \textsf{letter} entry
type, which attempts to simplify for you the \emph{Manual}'s rather
complicated rules for formatting such references.  (See 14.117;
jackson:paulina:letter, white:ross:memo, white:russ [a completely
fictitious entry to show the \textsf{crossref} mechanism], white:total
[a \textsf{book} entry, for the bibliography]).

\mylittlespace To start, the name of the letter writer goes in the
\textsf{author} field, while the \textsf{title} field contains both
the name of the writer and that of the recipient, in the form
\texttt{Author to Recipient}.  The \textsf{titleaddon} field contains,
optionally, the type of correspondence involved.  If it's a letter,
the type needn't be given, but if it's a memorandum or report or the
like, then this is the place to specify that fact.  Also, because the
\textsf{origdate} field only accepts numbers, if you want to use the
abbreviation \enquote{n.d.}  (or \cmd{bibstring\{nodate\}}) for
undated letters, then this is where you should put it.  If you need to
specify where a letter was written, then you can also use this field,
and, if both are present, remember to separate the location from the
type with a comma, like so: \texttt{memorandum, London}.
Alternatively, you can put the place of writing into the
\textsf{origlocation} field.  Most importantly, the date of the letter
itself goes in the \textsf{origdate} field (\texttt{year-month-day}),
which now allows a full date specification, while the publishing date
of the whole collection goes in the \textsf{date} field, instead of in
the obsolete \textsf{origyear}.  As in other entry types, then, the
\textsf{date} field now has its ordinary meaning of \enquote{date of
  publication.}  (You may have noticed here that the presentation of
the \textsf{origdate} in this sort of reference is different from the
date format required elsewhere by the \emph{Manual}.  This appears to
result from some recent changes to the specification, and it may be
that we could get away with choosing one or the other format for all
occurrences [6.45], but for the moment I hope this mixed solution will
suffice.)  Another difficulty arises when producing the short footnote
form, which requires you to provide a \textsf{shorttitle} field of the
form \enquote{\texttt{to Recipient},} the latter name as short as
possible while avoiding ambiguity.  The remaining fields are fairly
self explanatory, but do remember that the title of the published
collection belongs in \textsf{booktitle} rather than in
\textsf{title}.

\mylittlespace Finally, the \emph{Manual} specifies that if you cite
more than one letter from a given published collection, then the
bibliography should contain only a reference to said collection,
rather than to each individual letter, while the form of footnotes
would remain the same.  This should be possible using
\textsc{Bib}\TeX's standard \textsf{crossref} field, with each
\textsf{letter} entry pointing to a \textsf{collection} or
\textsf{book} entry, for example.  (If you are using \textsf{Biber},
then \textsf{letter} entries now correctly inherit fields from
\textsf{book} and \textsf{collection} entries, and also from the new
\textsf{mvbook} and \textsf{mvcollection} types --- \textsf{titles}
from the former provide a \textsf{booktitle} and from the latter a
\textsf{maintitle}.)  I shall discuss cross references at length later
(see esp.\ \textbf{crossref} in section~\ref{sec:entryfields}, below),
but I should mention here that \textsf{letter} is one of the entry
types in which a \textsf{crossref} or an \textsf{xref} field
automatically results in special shortened forms in notes and
bibliography if more than one piece from a single collection is cited.
(The other entry types are \textsf{inbook}, \textsf{incollection}, and
\textsf{inproceedings}; see 14.113 for the \emph{Manual}'s
specification.)  This ordinarily won't be an issue for \textsf{letter}
entries in the bibliography, as individual letters aren't included
there, but it is operative in notes, where you can disable it by
setting the new \mycolor{\texttt{longcrossref=true}} option, on which
see section~\ref{sec:chicpreset}, below.  To stop individual letters
turning up in the bibliography, you can use the \texttt{skipbib}
option in the \textsf{options} field.

% %\enlargethispage{\baselineskip}

\mybigspace This \mymarginpar{\textbf{manual}} is the second of two
traditional \textsc{Bib}\TeX\ entry types that the \emph{Manual}
suggests formatting as books, the other being \textsf{booklet}. As
with this latter, I have retained it in
\textsf{biblatex-chicago-notes} for backward compatibility, its main
peculiarity being that, in the absence of a named author, the
\textsf{organization} producing the manual will be printed both as
author and as publisher.  If you are using \textsf{Biber} you no
longer need a \textsf{sortkey} field to aid \textsf{biblatex's}
alphabetization routines, as the style takes care of this for you
(cf.\ section~\ref{sec:formatopts}, below).  You also don't need to
provide a \textsf{shortauthor} field, as the style will automatically
use \textsf{organization} in the absence of anything else.  Of course,
if you were to use a \textsf{book} entry for such a reference, then
you would need to define both \textsf{author} and \textsf{publisher}
using the name you here might have put in \textsf{organization}.  (See
14.92; chicago:manual, dyna:browser, natrecoff:camera.)

\mybigspace As \mymarginpar{\textbf{misc}} its name suggests, the
\textsf{misc} entry type was designed as a hold-all for citations that
didn't quite fit into other categories.  In
\textsf{biblatex-chicago-notes}, I have somewhat extended its
applicability, while retaining its traditional use.  Put simply, with
no \textsf{entrysubtype} field, a \textsf{misc} entry will retain
backward compatibility with the standard styles, so the usual
\textsf{howpublished}, \textsf{version}, and \textsf{type} fields are
all available for specifying an otherwise unclassifiable text, and the
\textsf{title} will be italicized.  (The \emph{Manual}, you may wish
to note, doesn't give specific instructions on how such citations
should be formatted, so when using the Chicago style I would recommend
you have recourse to this traditional entry type as sparingly as
possible.)

\mylittlespace If you do provide an \textsf{entrysubtype} field, the
\textsf{misc} type provides a means for citing unpublished letters,
memoranda, private contracts, wills, interviews, and the like, making
it something of an unpublished analogue to the \textsf{letter},
\textsf{article}, and \textsf{review} entry types (which see).  It
also works well for presenting online audio pieces, particularly dated
ones, like speeches.  Typically, such an entry will cite part of an
archive, and equally typically the text cited won't have a specific
title, but only a generic one, whereas an \textsf{unpublished} entry
will ordinarily have a specific author and title, and won't come from
a named archive.  The \textsf{misc} type with an \textsf{entrysubtype}
defined is the least formatted of all those specified by the
\emph{Manual}, so titles are in plain text, and any location details
take no parentheses in full footnotes.  (It is quite possible, though
somewhat unusual, for archival material to have a specific title,
rather than a generic one.  In these cases, you will need to enclose
the title inside a \cmd{mkbibquote} command manually.  Cf.\
coolidge:speech, roosevelt:speech, shapey:partita.)

\mylittlespace If you are wondering what to put in
\textsf{entrysubtype}, the answer is, currently, anything at all.  You
no longer need to put the exact string \texttt{letter} there in order
to move the date into closer proximity with the \textsf{title}.
Indeed, recent reconsideration of the \emph{Manual} has suggested that
the distinction to be drawn in this class of material hasn't to do
with \emph{where} the date is presented but, rather, with \emph{how}
it is presented.  As I now understand the specification, it draws a
distinction between archival material that is \enquote{letter-like}
(letters, memoranda, reports, telegrams) and that which isn't
(interviews, wills, contracts, speeches, or even personal
communications you've received and which you wish to cite).  This may
not always be the easiest distinction to draw, and in previous
releases of \textsf{biblatex-chicago} I have been ignoring it, but
once you've decided to classify it one way or the other you put the
date in the \textsf{origdate} field for letters, etc., and into the
\textsf{date} field for the others.

\mylittlespace In effect, whether it's a \textsf{letter} entry or a
\enquote{letter-like} \textsf{misc} entry (with
\textsf{entrysubtype}), it is by using the \textsf{origdate} field
that you identify when it was written, and the \textsf{origlocation},
if needed, identifies where it was written.  Other sorts of
\textsf{misc} entry (with \textsf{entrysubtype}) use the \textsf{date}
field (but still the \textsf{origlocation}).  This maintains
consistency of usage across entry types and also, I hope, improves
compliance when using the \textsf{misc} type for citing archival
material.  Remember, however, that without an \textsf{entrysubtype}
the entry will be treated as traditional \textsf{misc}, and the title
italicized.  In addition, defining \textsf{entrysubtype} activates the
automatic capitalization mechanism in the \textsf{title} field of
\textsf{misc} entries, on which see \textbf{\textbackslash autocap}
below.  (See 14.219-220, 14.231, 14.232-242; creel:house,
dinkel:agassiz, spock:interview.)

\mylittlespace As in \textsf{letter} entries, the titles of
unpublished letters are of the form \texttt{Author to Recipient}, and
further information can be given in the \textsf{titleaddon} field,
including the abbreviation \enquote{\texttt{n.d.}}\ (or
\cmd{bibstring\{nodate\}}) for undated examples.  The \textsf{note},
\textsf{organization}, \textsf{institution}, and \textsf{location}
fields (in ascending order of generality) allow the specification of
which manuscript collection now holds the letter, though the
\emph{Manual} specifies (14.238) that well-known depositories don't
usually need a city, state or country specified.  (The traditional
\textsf{misc} fields are all still available, also.)  Both the long
and short note forms can use the same \textsf{title}, but in both
cases you may need to use the \cmd{headlesscite} command to avoid the
awkward repetition of the author's name, though that name will always
appear in the bibliography (creel:house).  If you want to include the
date of a letter in a short note, I have provided the
\cmd{letterdatelong} command for inclusion in the postnote field of
the citation command.  (The standard \textsf{biblatex} command
\cmd{printdate} will work if you need to do the same for interviews.)

\enlargethispage{\baselineskip}

\mylittlespace As with \textsf{letter} entries, the \emph{Manual}
(14.233) suggests that bibliography entries contain only the name of
the manuscript collection, unless only one item from that collection
is cited.  The \textsf{crossref} field can be used, as well as the
\texttt{skipbib} option, for preventing the individual items from
turning up in the bibliography.  Obviously, this is a matter for your
discretion, and if you're using only short notes (see the
\texttt{short} option, section~\ref{sec:useropts} below), you may feel
the need to include more information in the note if the bibliography
doesn't contain a full reference to an individual item.

\mylittlespace Finally, if the \textsf{misc} entry isn't a letter,
remember that, as in \textsf{article} and \textsf{review} entries,
words like \texttt{interview} or \texttt{memorandum} needn't be
capitalized unless they follow a period --- the automatic
capitalization routines (with the \textsf{title} field starting with a
lowercase letter [see dinkel:agassiz, spock:interview, and
\textbf{\textbackslash autocap}]) will ensure correctness.  In all
this class of archived material, the \emph{Manual} (14.232) quite
specifically requires more consistency within your own work than
conformity to some external standard, so it is the former which you
should pursue.  I hope that \textsf{biblatex-chicago-notes} proves
helpful in this regard.

\mybigspace The \mymarginpar{\textbf{music}} 16th edition of the
\emph{Manual} has revised its recommendations more for this type than
for any other, so the notes which follow present several large changes
that you'll need to make to your .bib files.  The good news is that
some, though by no means all, of those changes involve considerable
simplifications.  \textsf{Music} is one of three audiovisual entry
types, and is intended primarily to aid in the presentation of musical
recordings that do not have a video component, though it can also
include audio books (auden:reading).  A DVD or VHS of an opera or
other performance, by contrast, should use the \textbf{video} type
instead, while an online music video will probably need an
\textbf{online} entry.  (Cf.\ \textsf{online} and \textsf{video};
handel:messiah, horowitz:youtube.)  Because \textsf{biblatex} --- and
\textsc{Bib}\TeX\ before it --- were designed primarily for citing
book-like objects, some choices needed to be made in assigning the
various roles found on the back of a CD to the fields in a typical
.bib entry.  I have also implemented several bibstrings to help in
identifying these roles within entries.  If you can think of a simpler
way to distribute the roles, please let me know, so that I can
consider making changes before anyone gets used to the current
equivalences.

\mylittlespace These equivalences, in summary form, are:

{\renewcommand{\descriptionlabel}[1]{\qquad\textsf{#1}}
\begin{description}
\item[author =] composer, songwriter, or performer(s),
  depending on whom you wish to emphasize by placing them at the head
  of the entry.
\item[editor, editora, editorb =] conductor, director or
  performer(s).  These will ordinarily follow the \textsf{title} of
  the work, though the usual \texttt{useauthor} and \texttt{useeditor}
  options can alter the presentation within an entry.  Because these
  are non-standard roles, you will need to identify them using the
  following:
\item[editortype, editoratype, editorbtype:] The most common roles,
  all associated with specific bibstrings (or their absence), will be
  \texttt{conductor}, \texttt{director}, \texttt{producer}, and,
  oddly, \texttt{none}.  The last is particularly useful when
  identifying the group performing a piece, as it usually doesn't need
  further specifying and this role prevents \textsf{biblatex} from
  falling back on the default \texttt{editor} bibstring.
\item[title, booktitle, maintitle:] As with the other audiovisual
  types, \textsf{music} serves as an analogue both to books and to
  collections, so the title will either be, e.g., the album title or a
  song title, in which latter case the album title would go into
  \textsf{booktitle}.  The \textsf{maintitle} might be necessary for
  something like a box set of \emph{Complete Symphonies}.
\item[publisher, series, number:] These three closely-associated
  fields are intended for presenting the catalog information provided
  by the music publisher.  The 16th edition generally only requires
  the \textsf{series} and \textsf{number} fields (nytrumpet:art),
  which hold the record label and catalog number, respectively.
  Alternatively, \textsf{publisher} would function as a synonym for
  \textsf{series} (holiday:fool), but there may be cases when you need
  or want to specify a publisher in addition to a label, as was the
  general requirement in the 15th edition.  (This might happen, for
  example, when a single publisher oversees more than one label.)  You
  can certainly put all of this information into one of the above
  fields, but separating it may help make the .bib entry more
  readable.
\item[howpublished/pubstate:] The 16th edition of the \emph{Manual}
  (14.276) has rather helpfully eliminated any reference to the
  specialized symbols (\texttt{\textcircledP} \&\
  \texttt{\textcopyright}) found in the 15th edition for presenting
  publishing information for musical recordings.  This means that the
  \textsf{howpublished} field is now obsolete, and you can remove it
  from \textsf{music} entries in your .bib files.  The
  \textsf{pubstate} field, therefore, can revert to its standard use
  for identifying reprints.  In \textsf{music} entries, putting
  \texttt{reprint} here will transform the \textsf{origdate} from a
  recording date for an entire album into an original release date for
  that album, notice of which will be printed towards the end of a
  note or bibliography entry.
\item[date, eventdate, origdate:] As though to compensate for the
  simplification I've just mentioned, the \textsf{Manual} now states
  that \enquote{citations without a date are generally unacceptable}
  (14.276).  Finding a date may take some research, but they will
  basically fall into two types, i.e., the date(s) of the recording or
  the copyright / publishing date(s).  Recording dates go either in
  \textsf{origdate} (for complete albums) or \textsf{eventdate} (for
  individual tracks).  The copyright or publishing dates go either in
  the \textsf{date} field (which applies to the current medium you are
  citing) or in the \textsf{origdate} field (which refers to the
  original release date).  You may have noticed that the
  \textsf{origdate} has two slightly different uses --- you can tell
  \textsf{biblatex-chicago} which sort you intend by using the string
  \texttt{reprint} in the \textsf{pubstate} field, which transforms
  the \textsf{origdate} from a recording date into an original release
  date.  The style will automatically prepend the bibstring
  \texttt{recorded} to the \textsf{eventdate} or, in the absence of
  this \textsf{pubstate} mechanism, to the \textsf{origdate}, or even
  to both, but you can modify what is printed there using the
  \mycolor{\textsf{userd}} field, which acts as a sort of date type
  modifier.  In \textsf{music} entries, \textsf{userd} will be
  prepended to an \textsf{eventdate} if there is one, barring that to
  the \textsf{origdate}, barring that to a \textsf{urldate}, and
  absent those three to the \textsf{date}.  (See floyd:atom,
  nytrumpet:art.)
 \item[type:] As in all the audiovisual entry types, the \textsf{type}
  field holds the medium of the recording, e.g., vinyl, 33 rpm,
  8-track tape, cassette, compact disc, mp3, ogg vorbis.
\end{description}}

The entries in \textsf{notes-test.bib} should at least give you a good
idea of how this all works, and that file also contains an example of
an audio book presented in a \textsf{music} entry.  If you browse the
examples in the \emph{Manual} you will see some variations in the
formatting choices there, from which I have made selections for
\textsf{biblatex-chicago}.  It wasn't always clear to me that these
variations were rules as opposed to possibilities, so I've ignored
some of them in the code.  Arguments as to why I'm wrong will, of
course, be entertained.  (Cf. 14.276--77; \textsf{eventdate},
\textsf{origdate}, \textsf{userd}; auden:reading, beethoven:sonata29,
bernstein:shostakovich, floyd:atom, holiday:fool, nytrumpet:art,
rubinstein:chopin.)

\mybigspace All \colmarginpar{\textbf{mvbook}\\\textbf{mvcollection}%
  \\\textbf{mvproceedings}\\\textbf{mvreference}} four of these entry
types are new to \textsf{biblatex-chicago}, and all function more or
less as in standard \textsf{biblatex}.  I would like, however, to
emphasize a couple of things.  First, each is aliased to the entry
type that results from removing the \enquote{mv} from their names.
Second, assuming you are using \textsf{Biber} and not
\textsc{Bib}\TeX, each has an important role as the target of
cross-references from other entries, the \textsf{title} of the
\textbf{mv*} entry \emph{always} providing a \textsf{maintitle} for
the entry referencing it.  If you want to provide a \textsf{booktitle}
for the referencing entry, please use another entry type, e.g.,
\textbf{collection} for \textbf{incollection} or \textbf{book} for
\textbf{inbook}.  (These distinctions are particularly important to the
correct functioning of the abbreviated references that
\textsf{biblatex-chicago}, in various circumstances, provides.  Please
see the documentation of the \textbf{crossref} field in
section~\ref{sec:entryfields}, below.)

\mylittlespace On the same subject, when multi-volume works are
presented in the reference apparatus, the \emph{Manual} (14.121--27)
requires that any dates presented should be appropriate to the
specific nature of the citation.  In short, this means that a date
range that is right for the presentation of a multi-volume work in its
entirety isn't right for citing, e.g., a single volume of that work
which appeared in one of the years contained in the date range.
Because child entries will by default inherit all the date fields from
their parent (including the \textsf{endyear} of a date range), I have
turned off the inheritance of \textsf{date} and \textsf{origdate}
fields from all of the \textbf{mv*} entry types to any other entry
type.  When the dates of the parent and of the child in such a
situation are exactly the same, then this unfortunately requires an
extra field in the child's .bib entry.  When they're not the same, as
will, I believe, often be the case, this arrangement saves a lot of
annoying work in the child entry to suppress wrongly-inherited fields.
Other sorts of parent entries aren't affected by this, and of course
you must be using \textsf{Biber} for the settings to apply.  See
harley:ancient:cart, harley:cartography, and harley:hoc for how this
might look.

\paragraph*{\protect\mymarginpar{\textbf{online}}}
\label{sec:online}

The \emph{Manual}'s scattered instructions (14.4--13, 14.166--169,
14.184--185, 14.200, 14.223, 14.243--246) for citing online materials
are slightly different from those suggested by standard
\textsf{biblatex}.  Indeed, this is a case where complete backward
compatibility with other \textsf{biblatex} styles may be impossible,
because as a general rule the \emph{Manual} considers relevant not
only where a source is found, but also the nature of that source,
e.g., if it's an online edition of a book (james:ambassadors), then it
calls for a \textsf{book} entry.  Even if you cite an intrinsically
online source, if that source is structured more or less like a
conventional printed periodical, then you'll probably want to use
\textsf{article} or \textsf{review} instead of \textsf{online}
(stenger:privacy, which cites \emph{CNN.com}).  The 16th edition's
suggestions for blogs lend themselves well to the \textsf{article}
type, too, while comments become, logically, \textsf{reviews}
(14.243--6; ellis:blog, ac:comment).  Otherwise, for online documents
not \enquote{formally published,} the \textsf{online} type is usually
the best choice (evanston:library, powell:email).  Online videos, in
particular short pieces or those that present excerpts of some longer
event or work, and also online interviews, usually require this type,
too.  (See harwood:biden, horowitz:youtube, pollan:plant, but cp.\
weed:flatiron, a complete film, which requires a \textsf{video} entry.
Online audio pieces, particularly dated ones from an archive, work
best as \textsf{misc} entries with an \textsf{entrysubtype}:
coolidge:speech, roosevelt:speech.)  Some online materials will, no
doubt, make it difficult to choose an entry type, but so long as all
locating information is present, then perhaps that is enough to
fulfill the specification, or at least so I'd like to hope.

\mylittlespace Constructing an \textsf{online} .bib file entry is much
the same as in \textsf{biblatex}.  The \textsf{title} field would
contain the title of the page, the \textsf{organization} field could
hold the title or owner of the whole site.  If there is no specific
title for a page, but only a generic one (powell:email), then such a
title should go in \textsf{titleaddon}, not forgetting to begin that
field with a lowercase letter so that capitalization will work out
correctly.  It is worth remarking here, too, that the 16th edition of
the \emph{Manual} (14.7--8) prefers, if they're available, revision
dates to access dates when documenting online material.  See
\textsf{urldate} and \textsf{userd}, below.

%%\enlargethispage{\baselineskip}

\mybigspace The \mymarginpar{\textbf{patent}} \emph{Manual} is very
brief on this subject (14.230), but very clear about which information
it wants you to present, so such entries may not work well with other
\textsf{biblatex} styles.  The important date, as far as Chicago is
concerned, is the filing date.  If a patent has been filed but not yet
granted, then you can place the filing date in either the
\textsf{date} field or the \textsf{origdate} field, and
\textsf{biblatex-chicago-notes} will automatically prepend the
bibstring \texttt{patentfiled} to it.  If the patent has been granted,
then you put the filing date in the \textsf{origdate} field, and you
put the date it was issued in the \textsf{date} field, to which the
bibstring \texttt{patentissued} will automatically be prepended.  (In
other words, you no longer need to use a hand-formatted
\textsf{addendum} field, though you can place additional information
in that field if desired, and it will be printed in close association
with the dates.)  The patent number goes in the \textsf{number} field,
and you should use the standard \textsf{biblatex} bibstrings in the
\textsf{type} field.  Though it isn't mentioned by the \emph{Manual},
\textsf{biblatex-chicago-notes} will print the \textsf{holder} after
the \textsf{author}, if you provide one.  Finally, the 16th edition of
the \emph{Manual} has removed the quotation marks from around
\textsf{patent} titles, and also capitalized them sentence-style, both
of which seem to be the generally-accepted conventions.  The former
requires no intervention from you, but the latter may mean revision of
the \textsf{title} field to provide the lowercase letters manually.
See petroff:impurity.


\mybigspace This \mymarginpar{\textbf{periodical}} is the standard
\textsf{biblatex} entry type for presenting an entire issue of a
periodical, rather than one article within it.  It has the same
function in \textsf{biblatex-chicago-notes}, and in the main uses the
same fields, though in keeping with the system established in the
\textsf{article} entry type (which see) you'll need to provide
\textsf{entrysubtype} \texttt{magazine} if the periodical you are
citing is a \enquote{newspaper} or \enquote{magazine} instead of a
\enquote{journal.}  Also, remember that the \textsf{note} field is the
place for identifying strings like \enquote{special issue,} with its
initial lowercase letter to activate the automatic capitalization
routines.  (See \emph{Manual} 14.187; good:wholeissue.)

\mybigspace This \colmarginpar{\textbf{proceedings}} is the standard
\textsf{biblatex} and \textsc{Bib}\TeX\ entry type, but with this
release the package can now provide automatically abbreviated
references in notes and bibliography when you use a \textsf{crossref}
or an \textsf{xref} field.  The functionality is not enabled by
default, but you can enable it in the preamble or in the
\textsf{options} field using the new \mycolor{\texttt{booklongxref}}
option.  Please see \textbf{crossref} in section~\ref{sec:entryfields}
and \texttt{booklongxref} in section~\ref{sec:chicpreset}, below.

\mybigspace This \mymarginpar{\textbf{reference}} entry type is
aliased to \textsf{collection} by the standard \textsf{biblatex}
styles, but I intend it to be used in cases where you need to cite a
reference work but not an alphabetized entry or entries in that work.
This could be because it doesn't contain such entries, or perhaps
because you intend the citation to appear in a bibliography rather
than in notes.  Indeed, the only differences between it and
\textsf{inreference} are the lack of a \textsf{lista} field to present
an alphabetized entry, and the fact that any \textsf{postnote} field
will be printed verbatim, rather than formatted as an alphabetized
entry.  (See mla:style for an example of a reference work that uses
numbered sections rather than alphabetized entries, and that appears
in the bibliography as well.)

\mybigspace This \mymarginpar{\textbf{report}} entry type is a
\textsf{biblatex} generalization of the traditional \textsc{Bib}\TeX\
type \textsf{techreport}.  Instructions for such entries are rather
thin on the ground in the \emph{Manual} (8.183, 14.249), so I have
followed the generic advice about formatting it like a book, and hope
that the results conform to the specification.  Its main peculiarities
are the \textsf{institution} field in place of a \textsf{publisher},
the \textsf{type} field for identifying the kind of report in
question, the \textsf{number} field closely associated with the
\textsf{type}, and the \textsf{isrn} field containing the
International Standard Technical Report Number of a technical report.
As in standard \textsf{biblatex}, if you use a \textsf{techreport}
entry, then the \textsf{type} field automatically defaults to
\cmd{bibstring\{techreport\}}.  As with \textsf{booklet} and
\textsf{manual}, you can also use a \textsf{book} entry, putting the
report type in \textsf{note} and the \textsf{institution} in
\textsf{publisher}.  (See herwign:office.)

%%\enlargethispage{\baselineskip}

\mybigspace The \mymarginpar{\textbf{review}} \textsf{review} entry
type was added to \textsf{biblatex 0.7}, and it certainly eases the
task of coping with the \emph{Manual}'s complicated requirements for
citing periodicals of all sorts, though it doesn't, I admit, eliminate
all difficulties.  As its name suggests, this entry type was designed
for reviews published in periodicals, and if you've already read the
\textsf{article} instructions above --- if you haven't, I recommend
doing so now --- you'll know that \textsf{review} serves as well for
citing other sorts of material with generic titles, like letters to
the editor, obituaries, interviews, online comments and the like.  The
primary rule is that any piece that has only a generic title, like
\enquote{review of \ldots,} \enquote{interview with \ldots,} or
\enquote{obituary of \ldots,} calls for the \textsf{review} type.  Any
piece that also has a specific title, e.g., \enquote{\enquote{Lost in
    \textsc{Bib}\TeX,} an interview with \ldots,} requires an
\textsf{article} entry.  (This assumes the text is found in a
periodical of some sort.  Were it found in a book, then the
\textsf{incollection} type would serve your needs, and you could use
\textsf{title} and \textsf{titleaddon} there.  While we're on the
topic of exceptions, the \emph{Manual} includes an example --- 14.221
--- where the \enquote{Interview} part of the title is considered a
subtitle rather than a titleaddon, said part therefore being included
inside the quotation marks and capitalized accordingly.  Not having
the journal in front of me I'm not sure what prompted that decision,
but \textsf{biblatex-chicago} would obviously have no difficulty
coping with such a situation.)

\mylittlespace Once you've decided to use \textsf{review}, then you
need to determine which sort of periodical you are citing, the rules
for which are the same as for an \textsf{article} entry.  If it is a
\enquote{magazine} or a \enquote{newspaper}, then you need an
\textsf{entrysubtype} \texttt{magazine}.  The generic title goes in
\textsf{title} and the other fields work just as as they do in an
\textsf{article} entry with the same \textsf{entrysubtype}, including
the substitution of the \textsf{journaltitle} for the \textsf{author}
if the latter is missing. (See 14.202--203, 14.205, 14.208,
14.214--217, 14.221; barcott:review, bundy:macneil, Clemens:letter,
gourmet:052006, kozinn:review, nyt:obittrevor, nyt:trevorobit,
unsigned:ranke, wallraff:word.)  If, on the other hand, the piece
comes from a \enquote{journal,} then you don't need an
\textsf{entrysubtype}.  The generic title goes in \textsf{title}, and
the remaining fields work just as they do in a plain \textsf{article}
entry.  (See 14.215; ratliff:review.)

\mylittlespace Most of the onerous details are the same as I described
them in the \textbf{article} section above, but I'll repeat some of
them briefly here.  If anything in the \textsf{title} needs
formatting, you need to provide those instructions yourself, as the
default is completely plain.  \textsf{Author}-less reviews are treated
just like similar newspaper articles --- in short notes and in the
bibliography the \textsf{journaltitle} replaces the author and heads
the entry, while in long notes the \textsf{title} comes first.  The
sorting of such entries is an issue, solved if you use \textsf{Biber}
as your backend, and otherwise requiring manual intervention with a
\textsf{sortkey} or the like (14.217; gourmet:052006, nyt:trevorobit,
unsigned:ranke, and see \cmd{DeclareSortingScheme} in
section~\ref{sec:formatopts}, below.).  As in \textsf{misc} entries
with an \textsf{entrysubtype}, words like \enquote{interview,}
\enquote{review,} and \enquote{letter} only need capitalization after
a full stop, i.e., ordinarily in a bibliography and not a note, so
\textsf{biblatex-chicago-notes} automatically deals with this problem
itself if you start the \textsf{title} field with a lowercase letter.
The file \textsf{notes-test.bib} and the documentation of
\cmd{autocap} will provide guidance here.

\mylittlespace One detail of the \textsf{review} type is new, and
responds to the needs of the 16th edition of the \emph{Manual}.  As I
mentioned above, blogs are best treated as \textsf{articles} with
\texttt{magazine} \textsf{entrysubtype}, whereas comments on those
blogs --- or on any similar sort of online content --- need the
\textsf{review} type with the same \textsf{entrysubtype}.  What they
will frequently also need is a date of some sort closely associated
with the comment (14.246; ac:comment), so I have included the
\textsf{eventdate} in \textsf{review} entries for just this purpose.
It will be printed just after the \textsf{author} and before the
\textsf{title}.  If you need a timestamp in addition, then the
\textsf{nameaddon} field is the place for it, but you'll have to
provide your own parentheses, in order to preserve the possibility of
providing pseudonyms in square brackets that is the standard function
of this field in all other entry types, and possibly in the the
\textsf{review} type as well.

\mylittlespace For the reasons I explained in the \textsf{article}
docs above, I have brought the \textsf{article} and \textsf{review}
entry types into line with most of the other types in allowing the use
of the \textsf{namea} and \textsf{nameb} fields in order to associate
an editor or a translator specifically with the \textsf{title}.  The
\textsf{editor} and \textsf{translator} fields, in strict homology
with other entry types, are associated with the \textsf{issuetitle} if
one is present, and with the \textsf{title} otherwise.  The usual
string concatenation rules still apply --- cf.\ \textsf{editor} and
\textsf{editortype} in section~\ref{sec:entryfields}, below.

\mybigspace This \mymarginpar{\textbf{suppbook}} is the entry type to
use if the main focus of a reference is supplemental material in a
book or in a collection, e.g., an introduction, afterword, or forward,
either by the same or a different author.  In previous releases of
\textsf{biblatex-chicago} these three just-mentioned types of
material, and only these three types, could be referenced using the
\textsf{introduction}, \textsf{afterword}, or \textsf{foreword}
fields, a system that required you simply to define one of them in any
way and leave the others undefined.  The macros don't use the text
provided by such an entry, they merely check to see if one of them is
defined, in order to decide which sort of pre- or post-matter is at
stake, and to print the appropriate string before the \textsf{title}
in long notes, short notes, list of shorthands, and bibliography.  I
have retained this mechanism both for backward compatibility and
because it works without modification across multiple languages, but
have also added functionality which allows you to cite any sort of
supplemental material whatever, using the \textsf{type} field.  Under
this system, simply put the nature of the material, including the
relevant preposition, in that field, beginning with a lowercase letter
so \textsf{biblatex} can decide whether it needs capitalization
depending on the context.  Examples might be \enquote{\texttt{preface
    to}} or \enquote{\texttt{colophon of}.}  (Please note, however,
that unless you use a \cmd{bibstring} command in the \textsf{type}
field, the resultant entry will not be portable across languages.)

\enlargethispage{-\baselineskip}

\mylittlespace There are a few other rules for constructing your .bib
entry.  The \textsf{author} field refers to the author of the
introduction or afterword, while \textsf{bookauthor} refers to the
author of the main text of the work, if the two differ.  For the 16th
edition, the \emph{Manual} requires the inclusion of the page range of
the part in question, though \emph{only} in the bibliography.  I have
followed this advice literally, so the \textsf{pages} field of a
\textsf{suppbook} entry won't automatically appear in a long note.  If
you wish to include those pages in a note, then you'll need to repeat
them in the \textsf{postnote} field of the citation command.

\mylittlespace Finally, if the focus of the reference is the main text
of the book, but you want to mention the name of the writer of an
introduction or afterword for bibliographical completeness, then the
normal \textsf{biblatex} rules apply, and you can just put their name
in the appropriate field of a \textsf{book} entry, that is, in the
\textsf{foreword}, \textsf{afterword}, or \textsf{introduction} field.
(See \emph{Manual} 14.116; polakow:afterw, prose:intro).

\mybigspace This \mymarginpar{\textbf{suppcollection}} fulfills a
function analogous to \textsf{suppbook}.  Indeed, I believe the
\textbf{suppbook} type can serve to present supplemental material in
both types of work, so this entry type is an alias to
\textsf{suppbook}, which see.

\mybigspace This \mymarginpar{\textbf{suppperiodical}} type, new in
\textsf{biblatex} 0.8, is intended to allow reference to
generically-titled works in periodicals, such as regular columns or
letters to the editor.  Previous releases of
\textsf{biblatex-chicago-notes} provided the \textsf{review} type for
this purpose, and now you can use either of these, as I've added
\textsf{suppperiodical} as an alias of \textsf{review}.  Please see
above under \textbf{review} for the full instructions on how to
construct a .bib entry for such a reference.

%%\enlargethispage{-\baselineskip}

\mybigspace The \mymarginpar{\textbf{unpublished}}
\textsf{unpublished} entry type works largely as it does in standard
\textsf{biblatex}, though it's worth remembering that you should use a
lowercase letter at the start of your \textsf{note} field (or perhaps
an\ \cmd{autocap} command in the somewhat contradictory
\textsf{howpublished}, if you have one) for material that wouldn't
ordinarily be capitalized except at the beginning of a sentence.
Thanks to a bug report by Henry D. Hollithron, such entries will now
print information about any \textsf{editor}, \textsf{translator},
\textsf{compiler}, etc., that you include in the .bib file (14.228;
nass:address).

\mybigspace This \mymarginpar{\textbf{video}} is the last of the three
audiovisual entry types, and as its name suggests it is intended for
citing visual media, be it films of any sort or TV shows, broadcast,
on the Net, on VHS, DVD, or Blu-ray.  As with the \textsf{music} type
discussed above, certain choices had to be made when associating the
production roles found, e.g., on a DVD, to those bookish ones provided
by \textsf{biblatex}.  Here are the main correspondences:

{\renewcommand{\descriptionlabel}[1]{\qquad\textsf{#1}}
\begin{description}
\item[author:] This will not infrequently be left undefined, as the
  director of a film should be identified as such and therefore placed
  in the \textsf{editor} field with the appropriate
  \textsf{editortype} (see below).  You will need it, however, to
  identify the composer of, e.g., an oratorio on VHS (handel:messiah),
  or perhaps the provider of commentaries or other extras on a film
  DVD (cleese:holygrail).
\item[editor, editora, editorb =] director or producer, or possibly
  the performer or conductor in recorded musical performances.  These
  will ordinarily follow the \textsf{title} of the work, though the
  usual \texttt{useauthor} and \texttt{useeditor} options can alter
  the presentation within an entry.  Because these are non-standard
  roles, you will need to identify them using the following:
\item[editortype, editoratype, editorbtype:] The most common roles,
  all associated with specific bibstrings (or their absence), will
  likely be \texttt{director}, \texttt{produ\-cer}, and, oddly,
  \texttt{none}.  The last is particularly useful if you want to
  identify performers, as they usually don't need further specifying
  and this role prevents \textsf{biblatex} from falling back on the
  default \texttt{editor} bibstring.
\item[title, titleaddon, booktitle, booktitleaddon, maintitle:] As
  with the other audiovisual types, \textsf{video} serves as an
  analogue both to books and to collections, so the \textsf{title} may
  be of a whole film DVD or of a TV series, or it may identify one
  episode in a series or one scene in a film.  In the latter cases,
  the title of the whole would go in \textsf{booktitle}.  The
  \textsf{booktitleaddon} field, in a change from the 15th edition,
  may be useful for specifying the season and/or episode number of a
  TV series, while the \textsf{titleaddon} is for for any information
  that needs to come between the \textsf{title} and the
  \textsf{booktitle} (cleese:holygrail, friends:leia, handel:messiah).
  As in the \textsf{music} type, \textsf{maintitle} may be necessary
  for a boxed set or something similar.
\item[date, eventdate, origdate:] As with \textsf{music} entries, in
  order to follow the specifications of the 16th edition of the
  \emph{Manual}, I have had to provide three separate date fields for
  citing \textsf{video} sources, but their uses differ somewhat
  between the two types.  In both, the \textsf{date} will generally
  provide the publishing or copyright date of the medium you are
  referencing.  The \textsf{eventdate} will most commonly present
  either the broadcast date of a particular TV program, or the
  recording/performance date of, for example, an opera on DVD.  The
  style will automatically prepend the bibstring \texttt{broadcast} to
  such a date, though you can use the \mycolor{\textsf{userd}} field
  to change the string printed there.  (Absent an \textsf{eventdate},
  the \textsf{userd} field in \textsf{video} entries will modify the
  \textsf{urldate}, and absent those two it will modify the
  \textsf{date}.)  The \textsf{origdate} has more or less the same
  function, and appears in the same places, as it does in standard
  book-like entries, providing the date of first release of a film,
  though there isn't any \texttt{reprint} string associated with it in
  this entry type.  Cf.\ friends:leia, handel:messiah,
  hitchcock:nbynw.
\item[type:] As in all the audiovisual entry types, the \textsf{type}
  field holds the medium of the \textsf{title}, e.g., 8 mm, VHS, DVD,
  Blu-ray, MPEG.
\end{description}}

As with the \textsf{music} type, entries in \textsf{notes-test.bib}
should at least give you a good idea of how all this works.  (Cf.\
14.279--80; loc:city, weed:flatiron.)

\subsection{Entry Fields}
\label{sec:entryfields}

The following discussion presents, in alphabetical order, a complete
list of the entry fields you will need to use
\textsf{biblatex-chicago-notes}.  As in section \ref{sec:entrytypes},
I shall include references to the numbered paragraphs of the
\emph{Chicago Manual of Style}, and also to the entries in
\textsf{notes-test.bib}.  Many fields are most easily understood with
reference to other, related fields.  In such cases, cross references
should allow you to find the information you need.

\mybigspace As \mymarginpar{\textbf{addendum}} in standard
\textsf{biblatex}, this field allows you to add miscellaneous
information to the end of an entry, after publication data but, with
the single exception of the \textsf{online} entry type, before any
\textsf{url} or \textsf{doi} field.  In the \textsf{patent} entry type
(which see), it will be printed in close association with the filing
and issue dates.  In a few entry types --- \textsf{article},
\textsf{audio}, \textsf{music}, \textsf{periodical}, \textsf{review},
and \textsf{video} --- this information will come \emph{after} any
\textsf{pages} or \textsf{postnote} references present in long notes,
while in the remainder it comes \emph{before} such information,
allowing you in particular to use the field to identify a particular
type of book-like publication when such data won't fit well in another
part of an entry.  In any entry type, if your data begins with a word
that would ordinarily only be capitalized at the beginning of a
sentence, then simply ensure that that word is in lowercase, and the
style will take care of the rest.  Cf.\ \textsf{note}. (See
\emph{Manual} 14.119, 14.166--168; davenport:attention,
natrecoff:camera.)

\mybigspace In most \mymarginpar{\textbf{afterword}} circumstances,
this field will function as it does in standard \textsf{biblatex},
i.e., you should include here the author(s) of an afterword to a given
work.  The \emph{Manual} suggests that, as a general rule, the
afterword would need to be of significant importance in its own right
to require mentioning in the reference apparatus, but this is clearly
a matter for the user's judgment.  As in \textsf{biblatex}, if the
name given here exactly matches that of an editor and/or a translator,
then \textsf{biblatex-chicago-notes} will concatenate these fields in
the formatted references.

%\vspace{\baselineskip}

\mylittlespace As noted above, however, this field has a special
meaning in the \textsf{suppbook} entry type, used to make an
afterword, foreword, or introduction the main focus of a citation.  If
it's an afterword at issue, simply define \textsf{afterword} any way
you please, leave \textsf{foreword} and \textsf{introduction}
undefined, and \textsf{biblatex-chicago-notes} will do the rest. Cf.\
\textsf{foreword} and \textsf{introduction}. (See \emph{Manual} 14.91,
14.116; polakow:afterw.)

\paragraph*{\protect\mymarginpar{\textbf{annotation}}}
\label{sec:annote}

At the request of Emil Salim, \textsf{biblatex-chicago-notes} has, as
of version 0.9, added a package option (see \texttt{annotation} below,
section \ref{sec:useropts}) to allow you to produce annotated
bibliographies.  The formatting of such a bibliography is currently
fairly basic, though it conforms with the \emph{Manual's} minimal
guidelines (14.59).  The default in \textsf{chicago-notes.cbx} is to
define \cmd{DeclareFieldFormat\{an\-notation\}} using
\cmd{par}\cmd{nobreak} \cmd{vskip} \cmd{bibitemsep}, though you can
alter it by re-declaring the format in your preamble.  The
page-breaking algorithms don't always give perfect results here, but
the default formatting looks, to my eyes, fairly decent.  In addition
to tweaking the field formatting you can also insert \cmd{par} (or
even \cmd{vadjust\{\cmd{eject}\}}) commands into the text of your
annotations to improve the appearance.  Please consider the
\texttt{annotation} option a work in progress, but it is usable now.
(N.B.: The \textsc{Bib}\TeX\ field \textsf{annote} serves as an alias
for this.)

\mybigspace I \mymarginpar{\textbf{annotator}} have implemented this
\textsf{biblatex} field pretty much as that package's standard styles
do, even though the \emph{Manual} doesn't actually mention it.  It may
be useful for some purposes.  Cf.\ \textsf{commentator}.

%%\enlargethispage{\baselineskip}

\mybigspace For \mymarginpar{\textbf{author}} the most part, I have
implemented this field in a completely standard \textsc{Bib}\TeX\
fashion.  Remember that corporate or organizational authors need to
have an extra set of curly braces around them (e.g.,
\texttt{\{\{Associated Press\}\}}\,) to prevent \textsc{Bib}\TeX\ from
treating one part of the name as a surname (14.92, 14.212;
assocpress:gun, chicago:manual).  If there is no \textsf{author}, then
\textsf{biblatex-chicago-notes} will look, in sequence, for an
\textsf{editor}, \textsf{translator}, or \textsf{compiler} (actually
\textsf{namec}, currently) and use that name (or those names) instead,
followed by the appropriate identifying string (esp.\ 14.87, also
14.76, 14.126, 14.132, 14.189; boxer:china, brown:bremer,
harley:cartography, schellinger:novel, sechzer:women, silver:ga\-wain,
soltes:georgia).  Please note that when a \textsf{namec} appears at
the head of an entry, and you're not using \textsf{Biber}, you'll need
to assist \textsf{biblatex}'s sorting algorithms by providing a
\textsf{sortkey} field to ensure correct alphabetization in the
bibliography.  (See \cmd{DeclareSortingScheme} in
section~\ref{sec:formatopts}, below.)  A \textsf{shortauthor} entry is
no longer necessary to provide a \textsf{namec} at the head of the
short note form --- \textsf{biblatex-chicago} now takes care of this
automatically.

\mylittlespace In the rare cases when this substitution mechanism
isn't appropriate, you have two options: either you can
(chaucer:liferecords) put all the information into a \textsf{note}
field rather than individual fields, or you can use the
\textsf{biblatex} options \texttt{useauthor=false},
\texttt{useeditor=false}, \texttt{usetranslator=false}, and
\texttt{usecompiler=\\false} in the \textsf{options} field
(chaucer:alt).  If you look at the chaucer:alt entry in
\textsf{notes-test.bib}, you'll notice a peculiarity of this system of
toggles.  In order to ensure that the \textsf{title} of the book
appears at the head of the entry, you need to use \emph{all four} of
the toggles, even though the entry contains no \textsf{translator}.
Internally, \textsf{biblatex-chicago} is either searching for an
author-substitute, or it is skipping over elements of the ordered,
unidirectional chain \textsf{author -> editor -> translator ->
  compiler -> title}.  If you don't include
\texttt{usetranslator=false} in the \textsf{options} field, then the
package begins its search at \textsf{translator} and continues on to
\textsf{namec}, even though you have \texttt{usecompiler=false} in
\textsf{options}.  The result will be that the compilers' names will
appear at the head of the entry.  If you want to skip over parts of
the chain, you must turn off \emph{all} of the parts up to the one you
wish printed.  (Another peculiarity of the system, if you're using
\textsf{Biber}, is that setting the Chicago-specific
\texttt{usecompiler} option to \texttt{false} doesn't remove
\textsf{namec} from the sorting list, whereas the other standard
\textsf{biblatex} toggles \emph{do} remove their names from the
sorting list, so in the chaucer:alt case you need the \textsf{sortkey}
field.  See \cmd{DeclareSortingScheme} in
section~\ref{sec:formatopts}, below.)

\mylittlespace This system of toggles, then, can turn off
\textsf{biblatex-chicago-notes}'s mechanism for finding a name to
place at the head of an entry, but it also very usefully adds the
possibility of citing a work with an \textsf{author} by its editor,
compiler or translator instead (14.90; eliot:pound), something that
wasn't possible before.  For full details of how this works, see the
\textsf{editortype} documentation below.  (Of course, in
\textsf{collection} and \textsf{proceedings} entry types, an
\textsf{author} isn't expected, so there the \textsf{editor} is
required, as in standard \textsf{biblatex}.  Also, in \textsf{article}
or \textsf{review} entries with \textsf{entrysubtype}
\texttt{magazine}, the absence of an \textsf{author} triggers the use
of the \textsf{journaltitle} in its stead.  See those entry types for
further details.)

\mylittlespace \textbf{NB}: The \emph{Manual} provides specific
instructions for formatting the names of both anonymous and
pseudonymous authors (14.79--84).  In the former case, if no author is
known or guessed at, then it may simply be omitted
(virginia:plantation).  The use of \enquote{Anonymous} as the name is
\enquote{generally to be avoided,} but may in some cases be useful
\enquote{in a bibliography in which several anonymous works need to be
  grouped.}  If, on the other hand, \enquote{the authorship is known
  or guessed at but was omitted on the title page,} then you need to
use the \textsf{authortype} field to let
\textsf{biblatex-chicago-notes} know this fact.  If the author is
known (horsley:prosodies), then put \texttt{anon} in the
\textsf{authortype} field, if guessed at (cook:sotweed) put
\texttt{anon?}\ there.  (In both cases,
\textsf{biblatex-chicago-notes} tests for these \emph{exact} strings,
so check your typing if it doesn't work.)  This will have the effect
of enclosing the name in square brackets, with or without the question
mark indicating doubt.  As long as you have the right string in the
\textsf{authortype} field, \textsf{biblatex-chicago-notes} will also
do the right thing automatically in the short note form.

\mylittlespace In \textsf{nameaddon} most entry types (except
\textsf{customc} and \textsf{review}, which see), this field furnishes
the means to cope with the case of pseudonymous authorship.  If the
author's real name isn't known, simply put \texttt{pseud.} (or
\cmd{bibstring\{pseudonym\}}) in that field (centinel:letters).  If
you wish to give a pseudonymous author's real name, simply include it
there, formatted as you wish it to appear, as the contents of this
field won't be manipulated as a name by \textsf{biblatex}
(lecarre:quest).  If you have given the author's real name in the
\textsf{author} field, then the pseudonym goes in \textsf{nameaddon},
in the form \texttt{Firstname Lastname, pseud.}\ (creasey:ashe:blast,
creasey:morton:hide, creasey:york:death).  This latter method will
allow you to keep all references to one author's work under different
pseudonyms grouped together in the bibliography, as recommended by the
\emph{Manual}, though it is now recommended that, whichever system you
employ, you include a cross-reference from one name to the other in
the bibliography.  You can do this using a \textsf{customc} entry
(ashe:creasey, morton:creasey, york:creasey).

% %\enlargethispage{\baselineskip}

\mybigspace In \mymarginpar{\textbf{authortype}}
\textsf{biblatex-chicago}, this field serves a function very much in
keeping with the spirit of standard \textsf{biblatex}, if not with its
letter.  Instead of allowing you to change the string used to identify
an author, the field allows you to indicate when an author is
anonymous, that is, when his or her name doesn't appear on the title
page of the work you are citing.  As I've just detailed under
\textsf{author}, the \emph{Manual} generally discourages the use of
\enquote{Anonymous} as an author, preferring that you simply omit it.
If, however, the name of the author is known or guessed at, then
you're supposed to enclose that name within square brackets, which is
exactly what \textsf{biblatex-chicago} does for you when you put
either \texttt{anon} (author known) or \texttt{anon?} (author guessed
at) in the \textsf{authortype} field.  (Putting the square brackets in
yourself doesn't work right, hence this mechanism.)  The macros test
for these \emph{exact} strings, so check your typing if you don't see
the brackets.  Assuming the strings are correct,
\textsf{biblatex-chicago-notes} will also automatically do the right
thing in the short note form.  Cf.\ \textsf{author}.  (See 14.80--81;
cook:sotweed, horsley:prosodies.)

\mybigspace For \mymarginpar{\textbf{bookauthor}} the most part, as in
\textsf{biblatex}, a \textsf{bookauthor} is the author of a
\textsf{booktitle}, so that, for example, if one chapter in a book has
different authorship from the book as a whole, you can include that
fact in a reference (will:cohere).  Keep in mind, however, that the
entry type for introductions, forewords and afterwords
(\textsf{suppbook}) uses \textsf{bookauthor} as the author of
\textsf{title} (polakow:afterw, prose:intro).

\mybigspace This, \mymarginpar{\vspace{-12pt}\textbf{bookpagination}}
a standard \textsf{biblatex} field, allows you automatically to prefix
the appropriate string to information you provide in a \textsf{pages}
field.  If you leave it blank, the default is to print no identifying
string (the equivalent of setting it to \texttt{none}), as this is the
practice the \emph{Manual} recommends for nearly all page numbers.
Even if the numbers you cite aren't pages, but it is otherwise clear
from the context what they represent, you can still leave this blank.
If, however, you specifically need to identify what sort of unit the
\textsf{pages} field represents, then you can either hand-format that
field yourself, or use one of the provided bibstrings in the
\textsf{bookpagination} field.  These bibstrings currently are
\texttt{column,} \texttt{line,} \texttt{paragraph,} \texttt{page,}
\texttt{section,} and \texttt{verse}, all of which are used by
\textsf{biblatex's} standard styles.

\mylittlespace There are two points that may need explaining here.
First, all the bibstrings I have just listed follow the Chicago
specification, which may be confusing if they don't produce the
strings you expect.  Second, remember that \textsf{bookpagination}
applies only to the \textsf{pages} field --- if you need to format a
citation's \textsf{postnote} field, then you must use
\textsf{pagination}, which see (10.43--44, 14.154--163).

\mybigspace The \mymarginpar{\textbf{booksubtitle}} subtitle for a
\textsf{booktitle}.  See the next entry for further information.

\mybigspace In \mymarginpar{\textbf{booktitle}} the
\textsf{bookinbook}, \textsf{inbook}, \textsf{incollection},
\textsf{inproceedings}, and \textsf{letter} entry types, the
\textsf{booktitle} field holds the title of the larger volume in which
the \textsf{title} itself is contained as one part.  It is important
not to confuse this with the \textsf{maintitle}, which holds the more
general title of multiple volumes, e.g., \emph{Collected Works}.  It
is perfectly possible for one .bib file entry to contain all three
sorts of title (euripides:orestes, plato:republic:gr).  You may also
find a \textsf{booktitle} in other sorts of entries (e.g.,
\textsf{book} or \textsf{collection}), but there it will almost
invariably be providing information for the traditional
\textsc{Bib}\TeX\ cross-referencing apparatus, which I discuss below
(\textbf{crossref}).  This provision is unnecessary if you are using
\textsf{Biber}.

\mybigspace An \mymarginpar{\textbf{booktitleaddon}} annex to the
\textsf{booktitle}.  It will be printed in the main text font, without
quotation marks.  If your data begins with a word that would
ordinarily only be capitalized at the beginning of a sentence, then
simply ensure that that word is in lowercase, and
\textsf{biblatex-chicago-notes} will automatically do the right thing.

\mybigspace This \mymarginpar{\textbf{chapter}} field holds the
chapter number, mainly useful only in an \textsf{inbook} or an
\textsf{incollection} entry where you wish to cite a specific chapter
of a book (ashbrook:brain).

\mybigspace I \mymarginpar{\textbf{commentator}} have implemented this
\textsf{biblatex} field pretty much as that package's standard styles
do, even though the \emph{Manual} doesn't actually mention it.  It may
be useful for some purposes.  Cf.\ \textsf{annotator}.

%\enlargethispage{\baselineskip}

\paragraph*{\protect\colmarginpar{\textbf{crossref}}}
\label{sec:crossref}

This field is the standard \textsc{Bib}\TeX\ cross-referencing
mechanism, and \textsf{biblatex} has adopted it while also introducing
a modified one of its own (\textsf{xref}).  If you are using
\textsc{Bib}\TeX\ (or \textsf{bibtex8)} the \textsf{crossref} field
works exactly the same as it always has, while \textsf{xref} attempts
to remedy some of the deficiencies of the usual mechanism by ensuring
that child entries will inherit no data at all from their parents.
Section~2.4.1.1 of \textsf{biblatex.pdf} contains useful notes on the
intricacies of managing cross-referenced entries with these
traditional backends, and for the most part these backends are still
usable, if inconvenient.  New functionality, discussed below, for
abbreviating references in \textsf{book}, \textsf{bookinbook},
\textsf{collection}, and \textsf{proceedings} entries, and for using
the new \mycolor{\textsf{mv*}} entry types to do so, will prove
extremely difficult with the older backends, so if you plan on lots of
cross-referencing in \textsf{biblatex-chicago-notes} then I strongly
recommend you use \textsf{Biber}.

\mylittlespace (One reason for this is that when \textsf{Biber} is the
backend, \textsf{biblatex} defines a series of inheritance rules for
the \textsf{crossref} field which make it much more convenient to use.
Appendix B of \textsf{biblatex.pdf} explains the defaults, to which
\textsf{biblatex-chicago} has added several that I should mention
here: \textsf{incollection} entries can now inherit from \textsf{book}
and \textsf{mvbook} just as they do from \textsf{collection} and
\textsf{mvcollection} entries; \textsf{letter} entries now inherit
from \textsf{book}, \textsf{collection}, \textsf{mvbook}, and
\textsf{mvcollection} entries the same way an \textsf{inbook} or an
\textsf{incollection} entry would; the \textsf{namea}, \textsf{nameb},
\textsf{sortname}, \textsf{sorttitle}, and \textsf{sortyear} fields,
all highly single-entry specific, are no longer inheritable; and
\textsf{date} and \textsf{origdate} fields are not inheritable from
any of the new \textbf{mv*} entry types.)

\mylittlespace Turning now to the provision of abbreviated references
in \textsf{biblatex-chicago-notes}, the \emph{Manual} (14.113)
specifies that if you cite several contributions to the same
collection, all (including the collection itself) may be listed
separately in the bibliography, which the package does automatically,
using the default inclusion threshold of 2 in the case both of
\textsf{crossref}'ed and \textsf{xref}'ed entries.  (The familiar
\cmd{nocite} command may also help in some circumstances.)  In
footnotes the specification suggests that, after a citation of any one
contribution to the collection, all subsequent contributions may, even
in the first, long footnote, be cited using a slightly shortened form,
thus \enquote{avoiding clutter.}  In the bibliography the abbreviated
form is appropriate for all the child entries.  The
\textsf{biblatex-chicago-notes} package has always implemented these
instructions, but only if you use a \textsf{crossref} or an
\textsf{xref} field, and only in \textsf{incollection},
\textsf{inproceedings}, or \textsf{letter} entries (on the last named,
see just below).  With this release, I \colmarginpar{New!} have
considerably extended this functionality.

\mylittlespace First, I have added five new entry types ---
\mycolor{\textbf{book}}, \mycolor{\textbf{bookinbook}},
\mycolor{\textbf{collection}}, \mycolor{\textbf{inbook}}, and
\mycolor{\textbf{proceedings}} --- to the list of those which use
shortened cross references, and I have added two new options ---
\mycolor{\texttt{longcrossref}} and \mycolor{\texttt{booklongxref}},
on which more below --- which you can use in the preamble or in the
\textsf{options} field of an entry to enable or disable the automatic
provision of abbreviated references.  (The \textsf{crossref} or
\textsf{xref} field are still necessary for this provision, but they
are no longer sufficient on their own.)  The \textsf{inbook} type
works exactly like \textsf{incollection} or \textsf{inproceedings}; in
previous releases, you could use \textsf{inbook} instead of
\textsf{incollection} to avoid the automatic abbreviation, the two
types being otherwise identical.  Now that you can use an option to
turn off abbreviated references even in the presence of a
\textsf{crossref} or \textsf{xref} field, I have thought it sensible
to include this entry type alongside the others.  (Cf.\ ellet:galena,
keating:dearborn, lippincott:chicago, and prairie:state to see this
mechanism in action in both notes and bibliography.)

\mylittlespace The inclusion of \textbf{book}, \textbf{bookinbook},
\textbf{collection}, and \textbf{proceedings} entries fulfills a
request made by Kenneth L. Pearce, and allows you to obtain shortened
references to, for example, separate volumes within a multi-volume
work, or to different book-length works collected inside a single
volume.  Such references are not an explicit part of the
\emph{Manual's} specification, but they are a logical extension of it,
so the system of options for turning on this functionality behaves
differently for these four entry types than for the other 4 (see
below).  In \textsf{notes-test.bib} you can get a feel for how this
works by looking at bernhard:boris, bernhard:ritter,
bernhard:themacher, harley:ancient:cart, harley:cartography, and
har\-ley:hoc.

\mylittlespace Before discussing the new package options, I should say
a little about some subtleties involved in this mechanism.  First, and
especially for \textsf{book}, \textsf{bookinbook},
\textsf{collection}, and \textsf{proceedings} entries, it is much
simpler if your backend is \textsf{Biber}, which allows you to provide
\textsf{maintitles} by cross-referencing an \textbf{mv*} entry, and
\textsf{booktitles} by cross-referencing \textsf{book} or
\textsf{collection} entries.  Second, where and when to print
\textsf{volume} information in these references is extremely complex,
and I confess that I designed the tests primarily with \textsf{Biber}
in mind.  If you can't get it to work using \textsc{Bib}\TeX, or if
you find something that looks wrong to you, please let me know.
Third, Andrew Goldstone long ago identified some other difficulties in
the package's treatment of abbreviated citations, both in notes and
bibliography, difficulties exacerbated now by the extension of the
mechanism to book-like entries.  If you refer separately to chapters
in a single-author \textsf{book}, then the shortened part of the
reference, to the whole book, won't repeat the author's name before
the title of the whole.  If, however, you refer separately to parts of
a \textsf{collection} or \textsf{proceedings}, even when the
\textsf{editor} of the \textsf{collection} is the same as the
\textsf{author} of an essay in the collection, you will see the name
repeated before the abbreviated part referencing the whole parent
volume.

\mylittlespace Shortened references to book-like entries require, I
believe, a somewhat different treatment.  Here, repeated
\textsf{editors} are avoided if the abbreviated reference is to a
\textsf{collection} or \textsf{proceedings} entry, or to either of
their \textsf{mv*} versions, while for other entry types repeated
\textsf{authors} are avoided.  Because the code in these situations
tests for entry type, there may be corner cases where careful choice
of the parent entry type gets you what you want.  Likewise, judicious
use of the \textsf{editor} and \textsf{editortype} fields may also
help, in some circumstances, to clear names that are repeated
unnecessarily.  Also, because of the way dates are handled by the
\textsf{mv*} entry types, and by child entries cross-referenced to
such entry types, I thought it might help in these abbreviated
book-like entries to provide a date for the \textsf{title} when it's
part of a \textsf{maintitle}, though not when it's only part of a
\textsf{booktitle}.  If dates appear in shortened references where
you'd rather not have them, I have provided the new
\mycolor{\texttt{omitxrefdate}} option
\colmarginpar{\texttt{omitxrefdate}} to turn them off, either in the
preamble for the document as a whole or in the \textsf{options} field
of individual entries.  See \textbf{mvbook} in
section~\ref{sec:entrytypes} and \texttt{omitxrefdate} in
section~\ref{sec:useropts}.

\mylittlespace Finally, a published collection of letters also
requires different treatment (14.117).  If you cite more than one
letter from the same collection, then the \emph{Manual} specifies that
only the collection itself should appear in the bibliography.  In
footnotes, you can use the \textsf{letter} entry type, documented
above, for each individual letter, while the collection as a whole may
well require a \textsf{book} entry.  I have, after some consideration,
implemented the system of shortened references in \textsf{letter}
entries, even though the \emph{Manual} doesn't explicitly require it.
(See white:ross:memo, white:russ, and white:total, for examples of the
\textsf{crossref} and \textsf{xref} field in action in this way, and
please note that the second of these entries is entirely fictitious,
provided merely for the sake of example.)  How then to keep the
individual letters from appearing in the bibliography?  The simplest
mechanism is probably just to use \enquote{\texttt{skipbib}} in the
\textsf{options} field.

\mylittlespace If you look closely at the .bib entries for
white:ross:memo and white:russ, you'll see that, despite the latter
using \textsf{xref} instead of \textsf{crossref}, the first note
referring to it inherits data from the parent (white:total).  In the
abbreviated note and in abbreviated bibliography entries \emph{only},
the driver is making a separate call to the parent's .bib entry,
formatting the information there to fill out the bare data provided by
the child.  For the first white:ross:memo note, which contains the
full bibliographical information for the collection as a whole, I have
used \textsf{crossref} because this unabbreviated note no longer makes
a separate call to the parent's entry --- or, technically, it no
longer makes a call that prints anything at all.  This is a change
from all previous releases of \textsf{biblatex-chicago}, so if your
documents have come to rely on the side effects of this separate
citation for providing data that haven't been inherited by the child,
please be aware that it will no longer work as before.  (You could see
this by citing white:russ \emph{before} white:ross:memo.)  This change
only affects the eight entry types that provide the abbreviated
cross-references, a provision that is now dependent on the settings of
two new preamble and entry options.

\mylittlespace Those \colmarginpar{\texttt{longcrossref}} options
function asymmetrically.  The first, \mycolor{\texttt{longcrossref}},
generally controls the settings for the entry types more-or-less
authorized by the \emph{Manual}: \textsf{inbook},
\textsf{incollection}, \textsf{inproceedings}, and \textsf{letter}.

\begin{description}
\item[\qquad false:] This is the default.  If you use
  \textsf{crossref} or \textsf{xref} fields in the four mentioned
  entry types, you'll get the abbreviated references in both notes and
  bibliography.
\item[\qquad true:] You'll get no abbreviated references in these
  entry types, either in notes or in the bibliography.
\item[\qquad notes:] The abbreviated references will not appear in
  notes, but only in the bibliography.
\item[\qquad bib:] The abbreviated references will not appear in the
  bibliography, but only in notes.
\item[\qquad none:] This switch is special, allowing you with one
  setting to provide abbreviated references not just to the four entry
  types mentioned but also to \textsf{book}, \textsf{bookinbook},
  \textsf{collection}, and \textsf{proceedings} entries, both in notes
  and in the bibliography.
\end{description}

The \colmarginpar{\texttt{booklongxref}} second option,
\mycolor{\texttt{booklongxref}}, controls the settings for
\textsf{book}, \textsf{bookinbook}, \textsf{collection}, and
\textsf{proceedings} entries:

\begin{description}
\item[\qquad true:] This is the default.  If you use \textsf{crossref}
  or \textsf{xref} fields in these entry types, by default you will
  \emph{not} get any abbreviated references, either in notes or
  bibliography.
\item[\qquad false:] You'll get abbreviated references in these entry
  types both in notes and in the bibliography.
\item[\qquad notes:] The abbreviated references will not appear in
  notes, but only in the bibliography.
\item[\qquad bib:] The abbreviated references will not appear in the
  bibliography, but only in notes.
\end{description}

Please note that you can set both of these options either in the
preamble or in the \textsf{options} field of individual entries,
allowing you to change the settings on an entry-by-entry basis.

\mylittlespace Please \colmarginpar{New!} further note that in
previous releases of \textsf{biblatex-chicago} I recommended against
using \textsf{shorthand}, \textsf{reprinttitle} and/or \textsf{userf}
fields in combination with this abbreviated cross-referencing
mechanism.  I have, however, received a request from Alexandre Roberts
to allow the shorthand to appear in the place of the abbreviated
cross-reference as an additional space-saving measure, and one from
Kenneth Pearce to permit the combination of the other two fields with
\textsf{crossref}, as well.  The \textsf{userf} and
\textsf{reprinttitle} fields should now just work automatically in
such circumstances, but \colmarginpar{\texttt{inheritshort\-hand}} the
\textsf{shorthand} field in parent entries needs to be enabled by
setting the \mycolor{\texttt{inheritshorthand}} package option to
\texttt{true}.  There are, in addition, several other steps required
to make this function smoothly --- please see the documentation of the
\textbf{shorthand} field, below, for a full explanation.  (In case it
isn't clear, the combination of \textsf{userf}, \textsf{shorthand},
and \textsf{crossref} functionality in a single entry is now possible.
If you come across any problems or inaccuracies, please report them.)

\mybigspace This \colmarginpar{\textbf{date}} field may be used to
specify an item's complete date of publication, in \textsc{iso}8601
format, i.e., \texttt{yyyy-mm-dd}.  It may also be used to specify a
date range, according to Lehman's instructions in �~2.3.8 of
\textsf{biblatex.pdf}.  Please be aware, however, that \textsf{Biber}
is somewhat more exacting when parsing the \textsf{date} field than
\textsc{Bib}\TeX, so a field looking like \texttt{1968/75} will simply
be ignored --- you need \texttt{1968/1975} instead.  If you want to
present a more compressed year range, or more generally if only part
of a date is required, then the \textsf{month} and \textsf{year}
fields may be more convenient.  The latter may be particularly useful
in some entries because it can hold more than just numerical data, in
contrast to \textsf{date} itself.  Cf.\ the \textsf{misc} entry type
in section~\ref{sec:entrytypes} above for how to use this field to
distinguish between two classes of archival material.  See also
\textsf{origdate} and \textsf{urldate}.

\mylittlespace With \colmarginpar{New!} this release, you can now in
most entry types qualify a \textsf{date} with the \textsf{userd}
field, assuming that the entry contains no \textsf{urldate}.  For
\textsf{music} and \textsf{video} entries, there are several other
requirements --- please see the documentation of \textsf{userd},
below.

\mylittlespace (Users of the Chicago author-date style who wish to
minimize the labor needed to convert a .bib database for the notes \&\
bibliography style should be aware that, in this release, the latter
style includes compatibility code for the \texttt{cmsdate} (silently
ignored) and \texttt{switchdates} options, along with the mechanism
for reversing \textsf{date} and \textsf{origdate}.  This means that
you can, in theory, leave all of this alone in your .bib file when
making the conversion, though I'm retaining the right to revoke this
if the code in question demonstrably interferes with the functioning
of the notes \&\ bibliography style.)

\enlargethispage{\baselineskip}

\mybigspace This \mymarginpar{\textbf{day}} field, as of
\textsf{biblatex} 0.9, is obsolete, and will be ignored if you use it
in your .bib files.  Use \textsf{date} instead.

\mybigspace Standard \mymarginpar{\textbf{doi}} \textsf{biblatex}
field, providing the Digital Object Identifier of the work.  The 16th
edition of the \emph{Manual} specifies that, given their relative
permanence compared to URLs, \enquote{authors should include DOIs
  rather than URLs for sources that make them readily available}
(14.6).  (14.184; friedman:learn\-ing).  Cf.\ \textsf{url}.

\mybigspace Standard \mymarginpar{\textbf{edition}} \textsf{biblatex}
field.  If you enter a plain cardinal number, \textsf{biblatex} will
convert it to an ordinal (chicago:manual), followed by the appropriate
string.  Any other sort of edition information will be printed as is,
though if your data begins with a word (or abbreviation) that would
ordinarily only be capitalized at the beginning of a sentence, then
simply ensure that that word (or abbreviation) is in lowercase, and
\textsf{biblatex-chicago-notes} will automatically do the right thing
(babb:peru, times:guide).  In most situations, the \emph{Manual}
generally recommends the use of abbreviations in both bibliography and
notes, but there is room for the user's discretion in specific
citations (emerson:nature).

\mylittlespace In a previous release of
\textsf{biblatex-chicago-notes}, I introduced the \textsf{userd} field
to hold this non-numeric information, as \textsf{biblatex} only
accepted an integer in the \textsf{edition} field, but this changed in
version 0.8.  The \textsf{userd} field now has an entirely different
function --- please see its documentation below.

\mybigspace As \mymarginpar{\textbf{editor}} far as possible, I have
implemented this field as \textsf{biblatex}'s standard styles do, but
the requirements specified by the \emph{Manual} present certain
complications that need explaining.  Lehman points out in his
documentation that the \textsf{editor} field will be associated with a
\textsf{title}, a \textsf{booktitle}, or a \textsf{maintitle},
depending on the sort of entry.  More specifically,
\textsf{biblatex-chicago} associates the \textsf{editor} with the most
comprehensive of those titles, that is, \textsf{maintitle} if there is
one, otherwise \textsf{booktitle}, otherwise \textsf{title}, if the
other two are lacking.  In a large number of cases, this is exactly
the correct behavior (adorno:benj, centinel:letters,
plato:republic:gr, among others).  Predictably, however, there are
numerous cases that require, for example, an additional editor for one
part of a collection or for one volume of a multi-volume work.  For
these cases I have provided the \textsf{namea} field.  You should
format names for this field as you would for \textsf{author} or
\textsf{editor}, and these names will always be associated with the
\textsf{title} (donne:var).

\mylittlespace As you will see below, I have also provided a
\textsf{nameb} field, which holds the translator of a given
\textsf{title} (euripides:orestes).  If \textsf{namea} and
\textsf{nameb} are the same, \textsf{biblatex-chicago} will
concatenate them, just as \textsf{biblatex} already does for
\textsf{editor}, \textsf{translator}, and \textsf{namec} (i.e., the
compiler).  Furthermore, it is conceivable that a given entry will
need separate editors for each of the three sorts of title.  For this,
and for various other tricky situations, there is the \cmd{partedit}
macro (and its siblings), designed to be used in a \textsf{note} field
or in one of the \textsf{titleaddon} fields (chaucer:liferecords).
(Because the strings identifying an editor differ in notes and
bibliography, one can't simply write them out in such a field, hence
the need for a macro, which I discuss further in the commands section
below [\ref{sec:formatcommands}].)  Cf.\ \textsf{namea},
\textsf{nameb}, \textsf{namec}, and \textsf{translator}.

\mybigspace The \mymarginpar{\textbf{editora\\editorb\\editorc}} newer
releases of \textsf{biblatex} provide these fields as a means to
specify additional contributors to texts in a number of editorial
roles.  In the Chicago styles they seem most relevant for the
audiovisual types, especially \textsf{music} and \textsf{video}, where
they help to identify conductors, directors, producers, and
performers.  To specify the role, use the fields \textsf{editoratype},
\textsf{editorbtype}, and \textsf{editorctype}, which see.  (Cf.\
bernstein:shostakovich, handel:messiah.)

%\enlargethispage{\baselineskip}

\mybigspace Normally, \mymarginpar{\textbf{editortype}} with the
exception of the \textsf{article} and \textsf{review} types,
\textsf{biblatex-chicago-notes} will automatically find a name to put
at the head of an entry, starting with an \textsf{author}, and
proceeding in order through \textsf{editor}, \textsf{translator}, and
\textsf{namec} (the compiler).  If all four are missing, then the
\textsf{title} will be placed at the head.  (In \textsf{article} and
\textsf{review} entries with a \texttt{magazine}
\textsf{entrysubtype}, a missing author immediately prompts the use of
\textsf{journaltitle} at the head of an entry.  See above under
\textsf{article} for details.)  The \textsf{editortype} field, added
in \textsf{biblatex 0.7}, provides even greater flexibility, giving
you the ability to indicate any number of roles at the head of an
entry.  You can do this even though an author is named (eliot:pound
shows this mechanism in action for a standard editor, rather than for
an alternative role).  Two things are necessary for this to happen.
First, in the \textsf{options} field you need to set
\texttt{useauthor=false}, then you need to put the name you wish to
see at the head of your entry into the \textsf{editor} or the
\textsf{namea} field.  If the \enquote{editor} is in fact a compiler,
then you need to put \texttt{compiler} into the \textsf{editortype}
field, and \textsf{biblatex} will print the correct string after the
name in both the bibliography and in the long note form.

\mylittlespace There are a few details of which you need to be aware.
Because \textsf{biblatex-chicago} has added the \textsf{namea} field,
which gives you the ability to identify the editor specifically of a
\textsf{title} as opposed to a \textsf{maintitle} or a
\textsf{booktitle}, the \textsf{editortype} mechanism checks first to
see whether a \textsf{namea} is defined.  If it is, that name will be
used at the head of the entry, if it isn't it will go ahead and look
for an \textsf{editor}.  When the \textsf{editor} field is used,
\textsf{biblatex}'s sorting algorithms will work properly, and also
its \textsf{labelname} mechanism, meaning that a shortened form of the
\textsf{editor} will be used in the short note form.  If, however, the
\textsf{namea} field provides the name, and you are not using
\textsf{Biber}, then your .bib entry will need to have a
\textsf{sortkey} field to aid in alphabetizing, and it will also need
a \textsf{shorteditor} defined to help with the short note form, not a
\textsf{shortauthor}, ruled out because \texttt{useauthor=false}.

\mylittlespace In \textsf{biblatex} 0.9 Lehman reworked the string
concatenation mechanism, for reasons he outlined in his RELEASE file,
and I have followed his lead.  In short, if you define the
\textsf{editortype} field, then concatenation is turned off, even if
the name of the \textsf{editor} matches, for example, that of the
\textsf{translator}.  In the absence of an \textsf{editortype}, the
usual mechanisms remain in place, that is, if the \textsf{editor}
exactly matches a \textsf{translator} and/or a \textsf{namec}, or
alternatively if \textsf{namea} exactly matches a \textsf{nameb}
and/or a \textsf{namec}, then \textsf{biblatex} will print the
appropriate strings.  The \emph{Manual} specifically (14.87)
recommends not using these identifying strings in the short note form,
and \textsf{biblatex-chicago-notes} follows their recommendation.  If
you nevertheless need to provide such a string, you'll have to do it
manually in the \textsf{shorteditor} field, or perhaps, in a different
sort of entry, in a \textsf{shortauthor} field.

\mylittlespace It may also be worth noting that because of certain
requirements in the specification -- absence of an \textsf{author},
for example -- the \texttt{useauthor} mechanism won't work properly in
the following entry types: \textsf{collection}, \textsf{letter},
\textsf{patent}, \textsf{periodical}, \textsf{proceedings},
\textsf{review}, \textsf{suppbook}, \textsf{suppcollection}, and
\textsf{suppperiodical}.

\mybigspace These
\mymarginpar{\textbf{editoratype\\editorbtype\\editorctype}} fields
identify the exact role of the person named in the corresponding
\textsf{editor[a-c]} field.  Note that they are not part of the string
concatenation mechanism.  I have implemented them just as the standard
styles do, and they have now found a use particularly in
\textsf{music} and \textsf{video} entries.  Cf.\
bernstein:shostakovich, handel:messiah.

\mybigspace Standard \mymarginpar{\textbf{eid}} \textsf{biblatex}
field, providing a string or number some journals use uniquely to
identify a particular article.  Only applicable to the
\textsf{article} entry type.  Not typically required by the
\emph{Manual}.

\paragraph*{\protect\mymarginpar{\textbf{entrysubtype}}}
\label{sec:entrysub}

Standard and very powerful \textsf{biblatex} field, left undefined by
the standard styles.  In \textsf{biblatex-chicago-notes} it has four
very specific uses, the first three of which I have designed in order
to maintain, as much as possible, backward compatibility with the
standard styles.  First, in \textsf{article}, \textsf{periodical}, and
\textsf{review} entries, the field allows you to differentiate between
scholarly \enquote{journals,} on the one hand, and \enquote{magazines}
and \enquote{newspapers} on the other.  Usage is fairly simple: you
need to put the exact string \texttt{magazine} into the
\textsf{entrysubtype} field if you are citing one of the latter two
types of source, whereas if your source is a \enquote{journal,} then
you need do nothing.

\mylittlespace The second use involves references to works from
classical antiquity and, according to the \emph{Manual}, from the
Middle Ages, as well.  When you cite such a work using the traditional
divisions into books, sections, lines, etc., divisions which are
presumed to be the same across all editions, then you need to put the
exact string \texttt{classical} into the \textsf{entrysubtype} field.
This has no effect in long notes or in the bibliography, but it does
affect the formatting of short notes, where it suppresses some of the
punctuation.  Ordinarily, you will use this toggle in a \textsf{book}
or a \textsf{bookinbook} entry, but it is possible that a journal
might well also present an edition of such a work.  Given the
tradition of using italics for the titles of such works, this may
require using a \textsf{titleaddon} field (with hand formatting)
instead of a \textsf{title}.  If you wish to reference a classical or
medieval work by the page numbers of a particular, non-standard
edition, then you shouldn't use the \textsf{entrysubtype} toggle.
Also, and the specification is reasonably clear about this, works from
the Renaissance and later, even if cited by the traditional divisions,
have short notes formatted normally, and therefore don't need an
\textsf{entrysubtype} field.  (See \emph{Manual} 14.256--268;
aristotle:metaphy:gr, plato:republic:gr; euripides:orestes is an
example of a translation cited by page number in a modern edition.)

\mylittlespace The third use occurs in \textsf{misc} entries.  If such
an entry contains no \textsf{entrysubtype} field, then the citation
will be treated just as the standard \textsf{biblatex} styles would,
including the use of italics for the \textsf{title}.  Any string at
all in \textsf{entrysubtype} tells \textsf{biblatex-chicago-notes} to
treat the source as part of an unpublished archive.  A \textsf{misc}
entry with \textsf{entrysubtype} defined is the least formatted of all
those specified by the \emph{Manual} --- see
section~\ref{sec:entrytypes} above under \textbf{misc} for all the
details on how these citations work.

\mylittlespace Fourth, and finally, the field can be defined in the
new \textsf{artwork} entry type in order to refer to a work from
antiquity whose title you do not wish to be italicized.  Please see
the documentation of \textsf{artwork} above for the details.

\mybigspace Kazuo
\mymarginpar{\textbf{eprint}\\\textbf{eprintclass}\\\textbf{eprinttype}}
Teramoto suggested adding \textsf{biblatex's} excellent
\textsf{eprint} handling to \textsf{biblatex-chicago}, and he sent me
a patch implementing it.  With minor alterations, I have applied it to
this release, so these three fields now work more or less as they do
in standard \textsf{biblatex}.  They may prove helpful in providing
more abbreviated references to online content than conventional URLs,
though I can find no specific reference to them in the \emph{Manual}.

\mybigspace This \mymarginpar{\textbf{eventdate}} is a standard
\textsf{biblatex} field.  In the 15th edition it was barely used, but
in order to comply with changes in the 16th edition of the
\emph{Manual} it can now play a significant role in \textsf{music},
\textsf{review}, and \textsf{video} entries.  In \textsf{music}
entries, it identifies the recording or performance date of a
particular song (rather than of a whole disc, for which you would use
\textsf{origdate}), whereas in \textsf{video} entries it identifies
either the original broadcast date of a particular episode of a TV
series or the date of a filmed musical performance.  In both these
cases \textsf{biblatex-chicago} will automatically prepend a bibstring
--- \texttt{recorded} and \texttt{aired}, respectively --- to the
date, but you can change this string using the new \textsf{userd}
field, something you'll definitely want to do for filmed musical
performances (friends:leia, handel:messiah, holiday:fool).

\mylittlespace The field's use in \textsf{review} entries is somewhat
different.  There, it helps to identify a particular comment within an
online thread.  There isn't a particular string associated with it,
but you can further specify a comment by placing a time\-stamp in
parentheses in the \textsf{nameaddon} field, in case the date alone
isn't enough (ac:comment).

\mybigspace As \mymarginpar{\textbf{foreword}} with the
\textsf{afterword} field above, \textsf{foreword} will in general
function as it does in standard \textsf{biblatex}.  Like
\textsf{afterword} (and \textsf{introduction}), however, it has a
special meaning in a \textsf{suppbook} entry, where you simply need to
define it somehow (and leave \textsf{afterword} and
\textsf{introduction} undefined) to make a foreword the focus of a
citation.

\mybigspace A \mymarginpar{\textbf{holder}} standard \textsf{biblatex}
field for identifying a \textsf{patent}'s holder(s), if they differ
from the \textsf{author}.  The \emph{Manual} has nothing to say on the
subject, but \textsf{biblatex-chicago-notes} prints it (them), in
parentheses, just after the author(s).

%\enlargethispage{\baselineskip}

\mybigspace Standard \mymarginpar{\textbf{howpublished}}
\textsf{biblatex} field, mainly applicable in the \textsf{booklet}
entry type, where it replaces the \textsf{publisher}.  I have also
retained it in the \textsf{misc} and \textsf{unpublished} entry types,
for historical reasons.

\mybigspace Standard \mymarginpar{\textbf{institution}}
\textsf{biblatex} field.  In the \textsf{thesis} entry type, it will
usually identify the university for which the thesis was written,
while in a \textsf{report} entry it may identify any sort of
institution issuing the report.

\mybigspace As \mymarginpar{\textbf{introduction}} with the
\textsf{afterword} and \textsf{foreword} fields above,
\textsf{introduction} will in general function as it does in standard
\textsf{biblatex}.  Like those fields, however, it has a special
meaning in a \textsf{suppbook} entry, where you simply need to define
it somehow (and leave \textsf{afterword} and \textsf{foreword}
undefined) to make an introduction the focus of a citation.

\mybigspace Standard \mymarginpar{\textbf{isbn}} \textsf{biblatex}
field, for providing the International Standard Book Number of a
publication.  Not typically required by the \emph{Manual}.

\mybigspace Standard \mymarginpar{\textbf{isrn}} \textsf{biblatex}
field, for providing the International Standard Technical Report
Number of a report.  Only relevant to the \textsf{report} entry type,
and not typically required by the \emph{Manual}.

\mybigspace Standard \mymarginpar{\textbf{issn}} \textsf{biblatex}
field, for providing the International Standard Serial Number of a
periodical in an \textsf{article} or a \textsf{periodical} entry.  Not
typically required by the \emph{Manual}.

\mybigspace Standard \mymarginpar{\textbf{issue}} \textsf{biblatex}
field, designed for \textsf{article}, \textsf{periodical}, or
\textsf{review} entries identified by something like \enquote{Spring}
or \enquote{Summer} rather than by the usual \textsf{month} or
\textsf{number} fields (brown:bremer).

\mybigspace The \mymarginpar{\textbf{issuesubtitle}} subtitle for an
\textsf{issuetitle} --- see next entry.

\mybigspace Standard \mymarginpar{\textbf{issuetitle}}
\textsf{biblatex} field, intended to contain the title of a special
issue of any sort of periodical.  If the reference is to one article
within the special issue, then this field should be used in an
\textsf{article} entry (conley:fifthgrade), whereas if you are citing
the entire issue as a whole, then it would go in a \textsf{periodical}
entry, instead (good:wholeissue).  The \textsf{note} field is the
proper place to identify the type of issue, e.g.,\ \texttt{special
  issue}, with the initial letter lower-cased to enable automatic
contextual capitalization.

%%\enlargethispage{\baselineskip}

\mybigspace The \mymarginpar{\textbf{journalsubtitle}} subtitle for a
\textsf{journaltitle} --- see next entry.

\mybigspace Standard \mymarginpar{\textbf{journaltitle}}
\textsf{biblatex} field, replacing the standard \textsc{Bib}\TeX\
field \textsf{journal}, which, however, still works as an alias.  It
contains the name of any sort of periodical publication, and is found
in the \textsf{article} and \textsf{review} entry types.  In the case
where a piece in an \textsf{article} or \textsf{review}
(\textsf{entrysubtype} \texttt{magazine}) doesn't have an author,
\textsf{biblatex-chicago-notes} provides for this field to be used as
the author.  See above (section~\ref{sec:entrytypes}) under
\textbf{article} for details.  The lakeforester:pushcarts and
nyt:trevorobit entries in \textsf{notes-test.bib} will give you some
idea of how this works.

\mybigspace This \mymarginpar{\textbf{keywords}} field is
\textsf{biblatex}'s extremely powerful and flexible technique for
filtering bibliography entries, allowing you to subdivide a
bibliography according to just about any criteria you care to invent.
See \textsf{biblatex.pdf} (3.11.4) for thorough documentation.  In
\textsf{biblatex-chicago}, the field can provide a convenient means to
exclude certain entries from making their way into a bibliography.  We
have already seen (\textbf{letter}, above) how the \emph{Manual}
(14.117) requires, in the case of published collections of letters,
that when more than one letter from the same collected is cited, the
bibliography should contain only a reference to the collection as a
whole (white:ross:memo, white:russ, white:total).  Similarly, when
citing both an original text and its translation (see \textbf{userf},
below), the \emph{Manual} (14.109) suggests including the original at
the end of the translation's bibliography entry, a procedure which
requires that the original not also be printed as a separate
bibliography entry (furet:passing:eng, furet:passing:fr,
aristotle:metaphy:trans, aristotle:metaphy:gr).  Finally, citations of
well-known reference works (like the \emph{Encyclopaedia Britannica},
for example), need only be presented in notes, and not in the
bibliography (14.247--248; ency:britannica, wikiped:bibtex; see
\textsf{inreference}, above).  A \textsf{keywords} field can be a
convenient way to exclude all such entries from appearing in a
bibliography, though of course including \texttt{skipbib} in the
\textsf{options} field works, too.

\mybigspace A \mymarginpar{\textbf{language}} standard
\textsf{biblatex} field, designed to allow you to specify the
language(s) in which a work is written.  As a general rule, the
Chicago style doesn't require you to provide this information, though
it may well be useful for clarifying the nature of certain works, such
as bilingual editions, for example.  There is at least one situation,
however, when the \emph{Manual} does specify this data, and that is
when the title of a work is given in translation, even though no
translation of the work has been published, something that might
happen when a title is in a language deemed to be unparseable by a
majority of your expected readership (14.108, 14.110, 14.194;
pirumova, rozner:liberation).  In such a case, you should provide the
language(s) involved using this field, connecting multiple languages
using the keyword \texttt{and}.  (I have retained \textsf{biblatex's}
\cmd{bibstring} mechanism here, which means that you can use the
standard bibstrings or, if one doesn't exist for the language you
need, just give the name of the language, capitalized as it should
appear in your text.  You can also mix these two modes inside one
entry without apparent harm.)

\mylittlespace An alternative arrangement suggested by the
\emph{Manual} is to retain the original title of a piece but then to
provide its translation, as well.  If you choose this option, you'll
need to make use of the \textbf{usere} field, on which see below.  In
effect, you'll probably only ever need to use one of these two fields
in any given entry, and in fact \textsf{biblatex-chicago-notes} will
only print one of them if both are present, preferring \textsf{usere}
over \textsf{language} for this purpose (see kern and weresz).  Note
also that both of these fields are universally associated with the
\textsf{title} of a work, rather than with a \textsf{booktitle} or a
\textsf{maintitle}.  If you need to attach a language or a translation
to either of the latter two, you could probably manage it with special
formatting inside those fields themselves.

\mybigspace I \mymarginpar{\textbf{lista}} intend this field
specifically for presenting citations from reference works that are
arranged alphabetically, where the name of the item rather than a page
or volume number should be given.  The field is a \textsf{biblatex}
list, which means you should separate multiple items with the keyword
\texttt{and}.  Each item receives its own set of quotation marks, and
the whole list will be prefixed by the appropriate string
(\enquote{s.v.,} \emph{sub verbo}, pl.\ \enquote{s.vv.}).
\textsf{Biblatex-chicago-notes} will only print such a field in a
\textsf{book} or an \textsf{inreference} entry, and you should look at
the documentation of these entry types for further details.  (See
\emph{Manual} 14.247--248; ency:britannica, grove:sibelius,
times:guide, wikiped:bibtex.)

\mybigspace This \mymarginpar{\textbf{location}} is
\textsf{biblatex}'s version of the usual \textsc{Bib}\TeX\ field
\textsf{address}, though the latter is accepted as an alias if that
simplifies the modification of older .bib files.  According to the
\emph{Manual} (14.135), a citation usually need only provide the first
city listed on any title page, though a list of cities separated by
the keyword \enquote{\texttt{and}} will be formatted appropriately.
If the place of publication is unknown, you can use
\cmd{autocap\{n\}.p.}\ instead (14.138).  For all cities, you should
use the common English version of the name, if such exists (14.137).

%%\enlargethispage{\baselineskip}

\mylittlespace Three more details need explanation here.  In
\textsf{article}, \textsf{periodical}, and \textsf{review} entries,
there is usually no need for a \textsf{location} field, but
\enquote{if a journal might be confused with another with a similar
  title, or if it might not be known to the users of a bibliography,}
then this field can present the place or institution where it is
published (14.191, 14.203; lakeforester:pushcarts, kimluu:diethyl, and
garrett).  For blogs cited using \textsf{article} entries, this is a
good place to identify the nature of the source --- i.e., the word
\enquote{blog} --- letting the style automatically provide the
parentheses (14.246; ellis:blog).  Less predictably, it is here that
\emph{Manual} indicates that a particular book is a reprint edition
(14.119), so in such a case you can use the \textsf{biblatex-chicago}
macro \cmd{reprint}, followed by a comma, space, and the location
(aristotle:metaphy:gr, schweitzer:bach).  (You can also now, somewhat
more simply, just put the string \texttt{reprint} into the
\textsf{pubstate} field to achieve the same result.  See the
\textsf{pubstate} documentation below.)  The \textsf{origdate} field
may be used to give the original date of publication, and of course
more complicated situations should usually be amenable to inclusion in
the \textsf{note} field (emerson:nature).

\mybigspace The \mymarginpar{\textbf{mainsubtitle}} subtitle for a
\textsf{maintitle} --- see next entry.

\mybigspace The \mymarginpar{\textbf{maintitle}} main title for a
multi-volume work, e.g., \enquote{Opera} or \enquote{Collected Works.}
(See donne:var, euripides:orestes, harley:cartography, lach:asia,
pelikan:chris\-tian, and plato:republic:gr.)  When using a
\textsf{crossref} field and \textsf{Biber}, the \textsf{title} of
\textbf{mv*} entry types always becomes a \textsf{maintitle} in the
child entry.

%%\enlargethispage{\baselineskip}

\mybigspace An \mymarginpar{\textbf{maintitleaddon}} annex to the
\textsf{maintitle}, for which see previous entry.  Such an annex would
be printed in the main text font.  If your data begins with a word
that would ordinarily only be capitalized at the beginning of a
sentence, then simply ensure that that word is in lowercase, and
\textsf{biblatex-chicago-notes} will automatically do the right thing.

\mybigspace Standard \mymarginpar{\textbf{month}} \textsf{biblatex}
field, containing the month of publication.  This should be an
integer, i.e., \texttt{month=\{3\}} not \texttt{month=\{March\}}.  See
\textsf{date} for more information.

\mybigspace This \mymarginpar{\textbf{namea}} is one of the fields
\textsf{biblatex} provides for style writers to use, but which it
leaves undefined itself.  In \textsf{biblatex-chicago} it contains the
name(s) of the editor(s) of a \textsf{title}, if the entry has a
\textsf{booktitle} or \textsf{maintitle}, or both, in which situation
the \textsf{editor} would be associated with one of these latter
fields (donne:var).  (In \textsf{article} and \textsf{review} entries,
\textsf{namea} applies to the \textsf{title} instead of the
\textsf{issuetitle}, should the latter be present.)  You should
present names in the field exactly as you would those in an
\textsf{author} or \textsf{editor} field, and the package will
concatenate this field with \textsf{nameb} if they are identical.  See
under \textbf{editor} above for the full details.  Please note that,
as the field is highly single-entry specific, if you are using
\textsf{Biber} \textsf{namea} isn't inherited from a
\textsf{crossref}'ed parent entry.  Cf.\ also \textsf{nameb},
\textsf{namec}, \textsf{translator}, and the macros \cmd{partedit},
\cmd{parttrans}, \cmd{parteditandtrans}, \cmd{partcomp},
\cmd{parteditandcomp}, \cmd{parttransandcomp}, and
\cmd{partedittransand\-comp}, for which see
section~\ref{sec:formatcommands}.

\mybigspace This \mymarginpar{\textbf{nameaddon}} field is provided
by \textsf{biblatex}, though not used by the standard styles.  In
\textsf{biblatex-chicago}, it allows you, in most entry types, to
specify that an author's name is a pseudo\-nym, or to provide either
the real name or the pseudonym itself, if the other is being provided
in the \textsf{author} field.  The abbreviation
\enquote{\texttt{pseud}.}\ (always lowercase in English) is specified,
either on its own or after the pseudo\-nym (centinel:letters,
creasey:ashe:blast, creasey:morton:hide, creasey:york:death, and
le\-carre:quest); \cmd{bibstring\{pseudonym\}} does the work for you.
See under \textbf{author} above for the full details.

\mylittlespace In \textsf{review} entries, I have removed the
automatic provision of square brackets from the field, allowing it to
be used in at least two ways.  First, if you provide your own square
brackets, then it can have its standard function, as above.  Second,
and new to the 16th edition of the \emph{Manual}, you can further
specify comments to blogs and other online content using a timestamp
(in parentheses) that supplements the \textsf{eventdate}, particularly
when the latter is too coarse a specification to identify a comment
unambiguously.  Cf.\ ac:comment.

\mylittlespace In the \textsf{customc} entry type, finally, which is
used to create alphabetized cross-references to other bibliography
entries, the \textsf{nameaddon} field allows you to change the default
string linking the two parts of the cross-reference.  The code
automatically tests for a known bibstring, which it will italicize.
Otherwise, it prints the string as is.

\mybigspace Like \mymarginpar{\textbf{nameb}} \textsf{namea}, above,
this is a field left undefined by the standard \textsf{biblatex}
styles.  In \textsf{biblatex-chicago}, it contains the name(s) of the
translator(s) of a \textsf{title}, if the entry has a
\textsf{booktitle} or \textsf{maintitle}, or both, in which situation
the \textsf{translator} would be associated with one of these latter
fields (euripides:orestes).  (In \textsf{article} and \textsf{review}
entries, \textsf{nameb} applies to the \textsf{title} instead of the
\textsf{issuetitle}, should the latter be present.)  You should
present names in this field exactly as you would those in an
\textsf{author} or \textsf{translator} field, and the package will
concatenate this field with \textsf{namea} if they are identical.  See
under the \textbf{translator} field below for the full details.
Please note that, as the field is highly single-entry specific, if you
are using \textsf{Biber} \textsf{nameb} isn't inherited from a
\textsf{crossref}'ed parent entry.  Cf.\ also \textsf{namea},
\textsf{namec}, \textsf{origlanguage}, \textsf{translator},
\textsf{userf} and the macros \cmd{partedit}, \cmd{parttrans},
\cmd{parteditandtrans}, \cmd{partcomp}, \cmd{parteditandcomp},
\cmd{parttransandcomp}, and \cmd{partedittransandcomp} in
section~\ref{sec:formatcommands}.

\mybigspace The \mymarginpar{\textbf{namec}} \emph{Manual} (14.87)
specifies that works without an author may be listed under an editor,
translator, or compiler, assuming that one is available, and it also
specifies the strings to be used with the name(s) of compiler(s).  All
this suggests that the \emph{Manual} considers this to be standard
information that should be made available in a bibliographic
reference, so I have added that possibility to the many that
\textsf{biblatex} already provides, such as the \textsf{editor},
\textsf{translator}, \textsf{commentator}, \textsf{annotator}, and
\textsf{redactor}, along with writers of an \textsf{introduction},
\textsf{foreword}, or \textsf{afterword}.  Since \textsf{biblatex}
doesn't offer a \textsf{compiler} field, I have adopted for this
purpose the otherwise unused field \textsf{namec}.  It is important to
understand that, despite the analogous name, this field does not
function like \textsf{namea} or \textsf{nameb}, but rather like
\textsf{editor} or \textsf{translator}, and therefore if used will be
associated with whichever title field these latter two would be were
they present in the same entry.  Identical fields among these three
will be concatenated by the package, and concatenated too with the
(usually) unnecessary commentator, annotator and the rest.  Also
please note that I've arranged the concatenation algorithms to include
\textsf{namec} in the same test as \textsf{namea} and \textsf{nameb},
so in this particular circumstance you can, if needed, make
\textsf{namec} analogous to these two latter, \textsf{title}-only
fields.  (See above under \textbf{editortype} for details of how you
may, in certain circumstances, use that field to identify a compiler.
This method will be particularly useful if you don't need to
concatenate the \textsf{namec} with any other role, because if you use
the \textsf{editor} field \textsf{biblatex} will automatically attend
to alphabetization and name-replacement in the bibliography, and will
also provide a name for short notes.)

\enlargethispage{\baselineskip}

\mylittlespace It might conceivably be necessary at some point to
identify the compiler(s) of a \textsf{title} separate from the
compiler(s) of a \textsf{booktitle} or \textsf{maintitle}, but for the
moment I've run out of available \textsf{name} fields, so you'll have
to fall back on the \cmd{partcomp} macro or the related
\cmd{parteditandcomp}, \cmd{parttransandcomp}, and
\cmd{partedittransandcomp}, on which see Commands
(section~\ref{sec:formatcommands}) below.  (Future releases may be
able to remedy this.)  It may be as well to mention here too that of
the three names that can be substituted for the missing
\textsf{author} at the head of an entry,
\textsf{biblatex-chicago-notes} will choose an \textsf{editor} if
present, then a \textsf{translator} if present, falling back to
\textsf{namec} only in the absence of the other two, and assuming that
the fields aren't identical, and therefore to be concatenated.  In a
change from the previous behavior, these algorithms also now test for
\textsf{namea} or \textsf{nameb}, which will be used instead of
\textsf{editor} and \textsf{translator}, respectively, giving the
package the greatest likelihood of finding a name to place at the head
of an entry.  Please remember, however, that if this name is supplied
by any of the non-standard fields \textsf{name[a-c]}, and you're not
using \textsf{Biber}, then you will need to provide a \textsf{sortkey}
to assist with alphabetization in the bibliography (cf.\
\cmd{DeclareSortingScheme} in section~\ref{sec:formatopts}, below.)  A
\textsf{shortauthor} is no longer necessary for the short note form,
as the style will provide it automatically.

\mybigspace As \mymarginpar{\textbf{note}} in standard
\textsf{biblatex}, this field allows you to provide bibliographic data
that doesn't easily fit into any other field.  In this sense, it's
very like \textsf{addendum}, but the information provided here will be
printed just before the publication data.  (See chaucer:alt,
chaucer:liferecords, cook:sotweed, emerson:nature, and rodman:walk for
examples of this usage in action.)  It also has a specialized use in
all the periodical types (\textsf{article}, \textsf{periodical}, and
\textsf{review}), where it holds supplemental information about a
\textsf{journaltitle}, such as \enquote{special issue}
(conley:fifthgrade, good:wholeissue).  In all uses, if your data
begins with a word that would ordinarily only be capitalized at the
beginning of a sentence, then simply ensure that that word is in
lowercase, and \textsf{biblatex-chicago-notes} will automatically do
the right thing.  Cf.\ \textsf{addendum}.

\mybigspace This \mymarginpar{\textbf{number}} is a standard
\textsf{biblatex} field, containing the number of a
\textsf{journaltitle} in an \textsf{article} or \textsf{review} entry,
the number of a \textsf{title} in a \textsf{periodical} entry, the
volume/number of a book in a \textsf{series}, or the (generally
numerical) specifier of the \textsf{type} in a \textsf{report} entry.
Generally, in an \textsf{article}, \textsf{periodical}, or
\textsf{review} entry, this will be a plain cardinal number, but in
such entries \textsf{biblatex-chicago} now does the right thing if you
have a list or range of numbers (unsigned:ranke).  In any
\textsf{book}-like entry the field may well contain considerably more
information, including even a reference to \enquote{2nd ser.,} for
example, while the \textsf{series} field in such an entry will contain
the name of the series, rather than a number.  This field is also the
place for the patent number in a \textsf{patent} entry.  Cf.\
\textsf{issue} and \textsf{series}.  (See \emph{Manual} 14.128--132
and boxer:china, palmatary:pottery, wauchope:ceramics; 14.180--181 and
beattie:crime, conley:fifthgrade, friedman:learn\-ing, garrett,
gibbard, hlatky:hrt, mcmillen:antebellum, rozner:liberation,
warr:el\-lison.)

\mylittlespace \textbf{NB}: This may be an opportune place to point
out that the \emph{Manual} (14.154) prefers arabic to roman numerals
in most circumstances (chapters, volumes, series numbers, etc.), even
when such numbers might be roman in the work cited.  The obvious
exception is page numbers, in which roman numerals indicate that the
citation came from the front matter, and should therefore be retained.

\mybigspace A \mymarginpar{\textbf{options}} standard
\textsf{biblatex} field, for setting certain options on a per-entry
basis rather than globally.  Information about some of the more common
options may be found above under \textsf{author} and below in
section~\ref{sec:options}.  See chaucer:alt, eliot:pound,
herwign:office, lecarre:quest, and mla:style for examples of the field
in use.

\mybigspace A \mymarginpar{\textbf{organization}} standard
\textsf{biblatex} field, retained mainly for use in the \textsf{misc},
\textsf{online}, and \textsf{manual} entry types, where it may be of
use to specify a publishing body that might not easily fit in other
categories.  In \textsf{biblatex}, it is also used to identify the
organization sponsoring a conference in a \textsf{proceedings} or
\textsf{inproceedings} entry, and I have retained this as a
possibility, though the \emph{Manual} is silent on the matter.

\mybigspace This \mymarginpar{\textbf{origdate}} \textsf{biblatex}
field allows you to provide more than one full date specification for
those references which need it.  As with the analogous \textsf{date}
field, you provide the date (or range of dates) in \textsc{iso}8601
format, i.e., \texttt{yyyy-mm-dd}.  In most entry types, you would use
\textsf{origdate} to provide the date of first publication of a work,
most usually needed only in the case of reprint editions, but also
recommended by the \emph{Manual} for electronic editions of older
works (14.119, 14.166, 14.169; aristotle:metaphy:gr, emerson:nature,
james:ambassadors, schweitzer:bach).  In the \textsf{letter} and
\textsf{misc} (with \textsf{entrysubtype}) entry types, the
\textsf{origdate} identifies when a letter (or similar) was written.
In such \textsf{misc} entries, some \enquote{non-letter-like}
materials (like interviews) need the \textsf{date} field for this
purpose, while in \textsf{letter} entries the \textsf{date} applies to
the publication of the whole collection.  If such a published
collection were itself a reprint, improvisation in the
\textsf{location} field might be able to rescue the situation.  (See
jackson:paulina:letter, white:ross:memo, white:russ, and white:total
for how \textsf{letter} entries usually work; creel:house shows the
field in action in a \textsf{misc} entry, while spock:interview uses
\textsf{date}.)

\mylittlespace In \textsf{music} entries, you can use the
\textsf{origdate} in two separate but related ways.  First, it can
identify the recording date of an entire disc, rather than of one
track on that disc, which would go in \textsf{eventdate}.  (Compare
holiday:fool with nytrumpet:art.)  The style will automatically
prepend the bibstring \texttt{recorded} to the date, but you can
change it with the new \textsf{userd} field.  Be aware, however, that
if an entry also has an \textsf{eventdate}, then \textsf{userd} will
apply to that, instead, and you'll be forced to accept the default
string.  Second, the \textsf{origdate} can provide the original
release date of an album.  For this to happen, you need to put the
string \texttt{reprint} in the \textsf{pubstate} field, which is a
standard mechanism across many other entry types for identifying a
reprinted work.  (See floyd:atom.)

\mylittlespace Because the \textsf{origdate} field only accepts
numbers, some improvisation may be needed if you wish to include
\enquote{n.d.}\ (\cmd{bibstring\{nodate\}}) in an entry.  In
\textsf{letter} and \textsf{misc}, this information can be placed in
\textsf{titleaddon}, but in other entry types you may need to use the
\textsf{location} field.

%%\enlargethispage{\baselineskip}

\mybigspace In \mymarginpar{\textbf{origlanguage}} keeping with the
\emph{Manual}'s specifications, I have fairly thoroughly redefined
\textsf{biblatex}'s facilities for treating translations.  The
\textsf{origtitle} field isn't used, while the \textsf{language} and
\textsf{origdate} fields have been press-ganged for other duties.  The
\textsf{origlanguage} field, for its part, retains a dual role in
presenting translations in a bibliography.  The details of the
\emph{Manual}'s suggested treatment when both a translation and an
original are cited may be found below under \textbf{userf}.  Here,
however, I simply note that the introductory string used to connect
the translation's citation with the original's is \enquote{Originally
  published as,} which I suggest may well be inaccurate in a great
many cases, as for instance when citing a work from classical
antiquity, which will most certainly not \enquote{originally} have
been published in the Loeb Classical Library.  Although not, strictly
speaking, authorized by the \emph{Manual}, I have provided another way
to introduce the original text, using the \textsf{origlanguage} field,
which must be provided \emph{in the entry for the translation, not the
  original text} (aristotle:metaphy:trans).  If you put one of the
standard \textsf{biblatex} bibstrings there (enumerated below), then
the entry will work properly across multiple languages.  Otherwise,
just put the name of the language there, localized as necessary, and
\textsf{biblatex-chicago} will eschew \enquote{Originally published
  as} in favor of, e.g., \enquote{Greek edition:} or \enquote{French
  edition:}.  This has no effect in notes, where only the work cited
--- original or translation --- will be printed, but it may help to
make the \emph{Manual}'s suggestions for the bibliography more
palatable.

\mylittlespace That was the first usage, in keeping at least with the
spirit of the \emph{Manual}.  I have also, perhaps less in keeping
with that specification, retained some of \textsf{biblatex}'s
functionality for this field.  If an entry doesn't have a
\textsf{userf} field, and therefore won't be combining a text and its
translation in the bibliography, you can also use
\textsf{origlanguage} as Lehman intended it, so that instead of
saying, e.g., \enquote{translated by X,} the entry will read
\enquote{translated from the German by X.}  The \emph{Manual} doesn't
mention this, but it may conceivably help avoid certain ambiguities in
some citations.  As in \textsf{biblatex}, if you wish to use this
functionality, you have to provide \emph{not} the name of the
language, but rather a bibliography string, which may, at the time of
writing, be one of \texttt{american}, \texttt{brazilian},
\texttt{danish}, \texttt{dutch}, \texttt{english}, \texttt{french},
\texttt{german}, \texttt{greek}, \texttt{italian}, \texttt{latin},
\texttt{norwegian}, \texttt{portuguese}, \texttt{spanish}, or
\texttt{swedish}, to which I've added \texttt{russian}.

\mybigspace The \mymarginpar{\textbf{origlocation}} 16th edition of
the \emph{Manual} has somewhat clarified issues pertaining to the
documentation of reprint editions and their corresponding originals
(14.166).  Starting with this release of \textsf{biblatex-chicago},
you can provide both an \textsf{origlocation} and an
\textsf{origpublisher} to go along with the \textsf{origdate}, should
you so wish, and all of this information will be printed in long notes
and bibliography.  You can now also use this field in a
\textsf{letter} or \textsf{misc} (with \textsf{entrysubtype}) entry to
give the place where a published or unpublished letter was written
(14.117).  (Jonathan Robinson has suggested that the
\textsf{origlocation} may in some circumstances actually be necessary
for disambiguation, his example being early printed editions of the
same material printed in the same year but in different cities.  The
new functionality should make this simple to achieve.  Cf.\
\textsf{origdate}, \textsf{origpublisher} and \textsf{pubstate};
schweitzer:bach.)

\mybigspace As \mymarginpar{\textbf{origpublisher}} with the
\textsf{origlocation} field just above, the 16th edition of the
\emph{Manual} has clarified issues pertaining to reprint editions and
their corresponding originals (14.166).  You can now provide an
\textsf{origpublisher} and/or an \textsf{origlocation} in addition to
the \textsf{origdate}, and all will be presented in long notes and
bibliography.  (Cf.\ \textsf{origdate}, \textsf{origlocation}, and
\textsf{pubstate}; schweitzer:bach.)

\mybigspace This \colmarginpar{\textbf{pages}} is the standard
\textsf{biblatex} field for providing page references.  In many
\textsf{article} and \textsf{review} entries you'll find this contains
something other than a page number, e.g. a section name or edition
specification (14.203, 14.209; kozinn:review, nyt:obittrevor,
nyt:trevorobit).  Of course, the same may be true of almost any sort
of entry, though perhaps with less frequency.  Curious readers may
wish to look at brown:bremer (14.189) for an example of a
\textsf{pages} field used to facilitate reference to a two-part
journal article.  Cf.\ \textsf{number} for more information on the
\emph{Manual}'s preferences regarding the formatting of numerals;
\textsf{bookpagination} and \textsf{pagination} provide details about
\textsf{biblatex's} mechanisms for specifying what sort of division a
given \textsf{pages} field contains; and \textsf{usera} discusses a
different way to present the section information pertaining to a
newspaper article.

\mylittlespace David Gohlke has recently brought to my attention a
discussion that took place a couple of years ago on
\href{http://tex.stackexchange.com/questions/44492/biblatex-chicago-style-page-ranges}{Stackexchange}
regarding the automatic compression of page ranges, e.g., 101-{-}109
in the .bib file or in the \textsf{postnote} field would become 101--9
in the document.  \textsf{Biblatex} has long had the facilities for
providing this, and though the \emph{Manual's} rules (9.60) are fairly
complicated, Audrey Boruvka fortunately provided in that discussion
code that implements the specifications.  As some users may well be
accustomed to compressing page ranges themselves in their .bib files,
and in their \textsf{postnote} fields, I have made the activation of
this code a package option, so setting
\mycolor{\texttt{compresspages=true}} when loading
\textsf{biblatex-chicago} should automatically give you the
Chicago-recommended page ranges.

\mybigspace This, \mymarginpar{\textbf{pagination}} a standard
\textsf{biblatex} field, allows you automatically to prefix the
appropriate identifying string to information you provide in the
\textsf{postnote} field of a citation command, whereas
\textsf{bookpagination} allows you to prefix a string to the
\textsf{pages} field.  Please see \textbf{bookpagination} above for
all the details on this functionality, as aside from the difference
just mentioned the two fields are equivalent.

\mybigspace Standard \colmarginpar{\textbf{part}} \textsf{biblatex}
field, which identifies physical parts of a single logical volume in
\textsf{book}-like entries, not in periodicals.  It has the same
purpose in \textsf{biblatex-chicago-notes}, but because the
\emph{Manual} (14.126) calls such a thing a \enquote{book} and not a
\enquote{part,} the string printed in notes and bibliography will, at
least in English, be \enquote{\texttt{bk.}\hspace{-2pt}}\ instead of
the plain dot between volume number and part number
(harley:cartography, lach:asia).  If the field contains something
other than a number, \textsf{biblatex-chicago} will print it as is,
capitalizing it if necessary, rather than supplying the usual
bibstring, so this provides a mechanism for altering the string to
your liking.  The field will be printed in the same place in any entry
as would a \textsf{volume} number, and although it will most usually
be associated with such a number, it can also, as of this release,
function independently, allowing you to identify parts of works that
don't fit into the standard scheme.  If you need to identify
\enquote{parts} or \enquote{books} that are part of a published
\textsf{series}, for example, then you'll need to use a different
field, (which in this case would be \textsf{number}
[palmatary:pottery]).  Cf.\ \textsf{volume}.

\mybigspace Standard \mymarginpar{\textbf{publisher}}
\textsf{biblatex} field.  Remember that \enquote{\texttt{and}} is a
keyword for connecting multiple publishers, so if a publisher's name
contains \enquote{and,} then you should either use the ampersand (\&)
or enclose the whole name in additional braces.  (See \emph{Manual}
14.139--148; aristotle:metaphy:gr, cohen:schiff, creasey:ashe:blast,
dunn:revolutions.)

\mylittlespace There are, as one might expect, a couple of further
subtleties involved here.  Two publishers will be separated by a
forward slash in both notes and bibliography, and you no longer, in
the 16th edition, need to provide hand formatting if a company issues
\enquote{certain books through a special publishing division or under
  a special imprint,} as these, too, should be separated by a forward
slash.  If a book has two co-publishers, \enquote{usually in different
  countries,} (14.147) then the simplest thing to do is to choose one,
probably the nearest one geographically.  If you feel it necessary to
include both, then levistrauss:savage demonstrates one way of doing
so, using a combination of the \textsf{publisher} and
\textsf{location} fields.  Finally, if the publisher is unknown, then
the \emph{Manual} recommends (14.143) simply using the place (if
known) and the date.  If for some reason you need to indicate the
absence of a publisher, the abbreviation given by the \emph{Manual} is
\texttt{n.p.}, though this can also stand for \enquote{no place.}
Some style guides apparently suggest using \texttt{s.n.}\,(=
\emph{sine nomine}) to specify the lack of a publisher, but the
\emph{Manual} doesn't mention this.

\mybigspace Due \mymarginpar{\textbf{pubstate}} to specific
requirements in the author-date style, I have implemented this field
there as a way of providing accurate citations of reprinted books.  As
the functionality seemed useful, I have also included some of it in
\textsf{biblatex-chicago-notes}.  In previous releases you could
identify a reprint by placing \cmd{bibstring\{reprint\}} in the
\textsf{location} field, followed by a comma, and the style would
print the appropriate string in notes and bibliography.  Now, if it is
more convenient, easier to remember, or if you want to reuse your .bib
database for the author-date style, you can simply put the string
\texttt{reprint} into the \textsf{pubstate} field, and the package
will take care of everything for you.  Both of these methods will now
work just fine, but please choose only one per entry, otherwise the
string will be printed twice.

%\enlargethispage{\baselineskip}

\mylittlespace There are a couple of exceptions to this basic
functionality.  In \textsf{video} entries, no bibstring will be
printed, as it's not appropriate there, so in effect the
\textsf{pubstate} field will be ignored.  In \textsf{music} entries,
the mechanism transforms the \textsf{origdate} from a recording date
for an album into the original release date for that album.  Whereas a
recording date will be printed in the middle of the note or
bibliography entry, the original release date will be printed near the
end, preceded by the appropriate string.  (Cf.\ 14.276; floyd:atom.)
Please remember that, currently, if you put anything besides
\texttt{reprint} in the \textsf{pubstate} field it will silently be
ignored, but this may change in future releases.

\mybigspace I \mymarginpar{\textbf{redactor}} have implemented this
field just as \textsf{biblatex}'s standard styles do, even though the
\emph{Manual} doesn't actually mention it.  It may be useful for some
purposes.  Cf.\ \textsf{annotator} and \textsf{commentator}.

\mybigspace \textbf{NB:} \mymarginpar{\textbf{reprinttitle}}
\textbf{Please note that this feature is in an alpha state, and that
  I'm contemplating using a different field in the future for this
  functionality.  I include it here in the hope that it might receive
  some testing in the meantime.}  At the request of Will Small, I have
included a means of providing the original publication details of an
essay or a chapter that you are citing from a subsequent reprint,
e.g., a \emph{Collected Essays} volume.  In such a case, at least
according to the \emph{Manual} (14.115), such details needn't be
provided in notes, only in the bibliography, and then only if these
details are \enquote{of particular interest.}  The data would follow
an introductory phrase like \enquote{originally published as,} making
the problem strictly parallel to that of including details of a work
in the original language alongside the details of its translation.  I
have addressed the latter problem with the \textsf{userf} field, which
provides a sort of cross-referencing method for this purpose, and
\textsf{reprinttitle} works in \emph{exactly} the same way.  In the
.bib entry for the reprint you include a cross-reference to the cite
key of the original location using the \textsf{reprinttitle} field
(which it may help mnemonically to think of as a \enquote{reprinted
  title} field).  The main difference between the two forms is that
\textsf{userf} prints all but the \textsf{author} of the original
work, whereas \textsf{reprinttitle} suppresses both the
\textsf{author} and the \textsf{title} of the original, giving only
the more general details, beginning with, e.g., the
\textsf{journaltitle} or \textsf{booktitle} and continuing from there.
The string prefacing this information will be \enquote{Originally
  published in.}  Please see the documentation on \textsf{userf} below
for all the details on how to create .bib entries for presenting your
data.

\mybigspace A \mymarginpar{\textbf{series}} standard \textsf{biblatex}
field, usually just a number in an \textsf{article},
\textsf{periodical}, or \textsf{review} entry, almost always the name
of a publication series in \textsf{book}-like entries.  If you need to
attach further information to the \textsf{series} name in a
\textsf{book}-like entry, then the \textsf{number} field is the place
for it, whether it be a volume, a number, or even something like
\enquote{2nd ser.} or \enquote{\cmd{bibstring\{oldseries\}}.}  Of
course, you can also use \cmd{bibstring\{oldseries\}} or
\cmd{bibstring\{newseries\}} in an \textsf{article} entry, but there
you would place it in the \textsf{series} field itself.  (In fact, the
\textsf{series} field in \textsf{article}, \textsf{periodical}, and
\textsf{review} entries is one of the places where \textsf{biblatex}
allows you just to use the plain bibstring \texttt{oldseries}, for
example, rather than making you type \cmd{bibstring\{oldseries\}}.
The \textsf{type} field in \textsf{manual}, \textsf{patent},
\textsf{report}, and \textsf{thesis} entries also has this
auto-detection mechanism in place; see the discussion of
\cmd{bibstring} below for details.)  In whatever entry type, these
bibstrings produce the required abbreviation, which thankfully is the
same in both notes and bibliography.  (For books and similar entries,
see \emph{Manual} 14.128--132; boxer:china, browning:aurora,
palmatary:pottery, plato:republic:gr, wauchope:ceramics; for
periodicals, see 14.195; garaud:gatine, sewall:letter.)  Cf.\
\textsf{number} for more information on the \emph{Manual}'s
preferences regarding the formatting of numerals.

%\enlargethispage{\baselineskip}

\paragraph*{\protect\mymarginpar{\textbf{shortauthor}}}
\label{sec:shortauthor}

This is a standard \textsf{biblatex} field, but
\textsf{biblatex-chicago-notes} makes considerably grea\-ter use of it
than the standard styles.  For the purposes of the Chicago style, the
field provides the name to be used in the short form of a footnote.
In the vast majority of cases, you don't need to specify it, because
the \textsf{biblatex} system selects the author's last name from the
\textsf{author} field and uses it in such a reference, and if there is
no \textsf{author} it will search \textsf{namea}, \textsf{editor},
\textsf{nameb}, \textsf{translator}, and \textsf{namec}, in that
order.  (In the case of the non-standard names \textsf{name[a-c]}, you
will need to provide a \textsf{sortkey} if you aren't using
\textsf{Biber}.  Cf.\ \cmd{DeclareSortingScheme} and
\cmd{DeclareLabelname} in section~\ref{sec:formatopts}, below.)  In an
author-less \textsf{article} or \textsf{review} entry
(\textsf{entrysubtype} \texttt{magazine}), where
\textsf{biblatex-chicago-notes} will use the \textsf{journaltitle} as
the author, or in author-less \textsf{manual} entries, where the
\textsf{organization} will be so used, the style automatically
provides the same substitution in the short note form, though you'll
still need to help the alphabetization routines by providing a
\textsf{sortkey} field in such cases (dyna:browser, gourmet:052006,
lakeforester:pushcarts, nyt:trevorobit).

\mylittlespace As mentioned under \textsf{editortype}, the
\emph{Manual} (14.87) recommends against providing the identifying
string (e.g., ed.\ or trans.)\ in the short note form, and
\textsf{biblatex-chicago-notes} follows their recommendation.  If you
need to provide these strings in such a citation, then you'll have to
do so by hand in the \textsf{shortauthor} field, or in the
\textsf{shorteditor} field, whichever you are using.

\mybigspace Like \mymarginpar{\textbf{shorteditor}}
\textsf{shortauthor}, a field to provide a name for a short footnote,
in this case for, e.g., a \textsf{collection} entry that typically
lacks an author.  The \textsf{shortauthor} field works just as well in
most situations, but if you have set \texttt{useauthor=false} (and not
\texttt{useeditor=false}) in an entry's \textsf{options} field, then
only \textsf{shorteditor} will be recognized.  Cf.\
\textsf{editortype}, above.

\mybigspace This \colmarginpar{\textbf{shorthand}} is
\textsf{biblatex}'s mechanism for using abbreviations in place of the
usual short note form, and in previous releases I left it effectively
unmodified in \textsf{biblatex-chicago-notes}, apart from a few
formatting tweaks.  For this release, at the request of Kenneth Pearce
and following some hints in the \emph{Manual}, I have made the system
considerably more flexible, which I hope might be useful for those
with specialized formatting needs.  In the default configuration, any
entry which contains a \textsf{shorthand} field will produce a normal
first note, either long or short according to your package options,
informing the reader that the work will hereafter be cited by this
abbreviation.  As in \textsf{biblatex}, the \cmd{printshorthands}
command, now for \textsf{Biber} users at least an alias for
\cmd{printbib\-list\{shorthand\}}, will produce a formatted list of
abbreviations for reference purposes, a list which the \emph{Manual}
suggests should be placed either in the front matter (when using
footnotes) or before the endnotes, in case these are used.

\mylittlespace I have provided three options to alter these defaults.
First, there is a new citation command, \cmd{shorthandcite}, which
will print the \textsf{shorthand} even at the first citation.  I have
only provided the most general form of this command, so you'll need to
put it inside parentheses or in a \cmd{footnote} command yourself.
Second, I have included two \texttt{bibenvironments} for use with the
\texttt{env} option to the \cmd{printshorthands} command:
\texttt{losnotes} is designed to allow a list of shorthands to appear
inside footnotes, while \texttt{losendnotes} does the same for
endnotes.  Their main effect is to change the font size, and in the
latter case to clear up some spurious punctuation and white space that
I see on my system when using endnotes.  (You'll probably also want to
use the option \texttt{heading=none} in order to get rid of the
[oversized] default, providing your own within the \cmd{footnote}
command.)  Third, I have provided a package option,
\texttt{shorthandfull}, which prints entries in the list of shorthands
which contain full bibliographical information, effectively allowing
you to eschew the bibliography in favor of a fortified shorthand list.
(See 13.65, 14.54--55, and also \textsf{biblatex.pdf} for more
information.)

\mylittlespace Alexandre \colmarginpar{New!} Roberts has suggested a
further refinement to \textsf{shorthand} behavior, which allows for it
to appear in the place of the usual abbreviated citation of parent
entries cross-referenced by several different child entries.  In such
a case, instead of the usual \enquote{\ldots\,in Author, \emph{Title},
  24--38,} you would see instead \enquote{\ldots\,in \emph{ShrtHd},
  24--38.}  There are several steps required for enabling this
behavior.  First, you need to set the new package option
\mycolor{\texttt{inheritshorthand}} to \texttt{true}, which allows
child entries to inherit the necessary fields from their
cross-referenced parents.  Second, you'll probably want to use the
\textsf{shorthandintro} field somehow to clarify that the
\textsf{shorthand} applies to the \emph{parent} rather than to the
\emph{child}, as otherwise the reference will be ambiguous.  Third,
you'll need to put \texttt{skipbiblist}, formerly \texttt{skiplos}, in
the \textsf{options} field of the child entries so that the
\textsf{shorthand} itself appears in the list of shorthands
\emph{only} next to the parent entry, and not also next to all of its
children.

\mylittlespace As I mentioned above under \textbf{crossref}, I
formerly recommended against using shorthands with cross-references,
but this extension of their use makes sense as an extra space-saving
measure.  I'm not certain that I've identified all the possible
drawbacks to enabling the \mycolor{\texttt{inheritshorthand}} option,
so care is still needed, at least in the current state of
\textsf{biblatex-chicago-notes}.  Please report any problems you might
have with this functionality to the email address at the head of this
documentation.

\enlargethispage{-\baselineskip}

\mybigspace When \mymarginpar{\textbf{shorthandintro}} you include a
\textsf{shorthand} in an entry, it will ordinarily appear the first
time you cite the work, at the end of a long note, surrounded by
parentheses and prefaced by the phrase \enquote{hereafter cited as.}
With this standard \textsf{biblatex} field, you can change that
formatting and that phrase to suit your needs.  Please note, first,
that you need to include the shorthand in this field as you intend it
to appear and, second, that you still need the \textsf{shorthand}
field present in order to ensure the appropriate presentation of that
shorthand in later citations and in the list of shorthands.  Finally,
I've tried to allow for as many different styles of notification as
possible, so by default the only punctuation that will appear between
the rest of the citation and the \textsf{shorthandintro} is a space.
If you are not enclosing the whole phrase in parentheses, you may need
to provide additional punctuation in the field itself, e.g.,
\texttt{\{\textbackslash addperiod\textbackslash space Cited
  as\ldots\}}.

\mybigspace A \mymarginpar{\textbf{shorttitle}} standard
\textsf{biblatex} field, primarily used to provide an abbreviated
title for short notes.  In \textsf{biblatex-chicago-notes}, you need
to take particular care with \textsf{letter} entries, where, as
explained above, the \emph{Manual} requires a special format
(\enquote{\texttt{to Recipient}}).  (See 14.117;
jackson:paulina:letter, white:ross:memo, white:russ.)  Some
\textsf{misc} entries (with an \textsf{entrysubtype}) also need
special attention.  (See creel:house, where the full \textsf{title} is
used as the \textsf{shortauthor} + \textsf{shorttitle} by using
\cmd{headlesscite} commands.  Placing \cmd{isdot} into the
\textsf{shortauthor} field no longer works in \textsf{biblatex} 1.6,
so be sure to check your .bib files when you upgrade.)  Remember,
also, that the generic titles in \textsf{review} and \textsf{misc}
entries may not want capitalization in all contexts, so, as with the
\textsf{title} field, if you begin a \textsf{shorttitle} with a
lowercase letter the style will do the right thing (barcott:review,
bundy:macneil, Clemens:letter, kozinn:review, ratliff:review,
unsigned:ranke).

%%\enlargethispage{\baselineskip}

\mybigspace A \mymarginpar{\textbf{sortkey}} standard
\textsf{biblatex} field, designed to allow you to specify how you want
an entry alphabetized in a bibliography.  In general, if an entry
doesn't turn up where you expect or want it, this field should provide
the solution.  Entries with a corporate author can now omit the
definite or indefinite article, which should help (14.85;
cotton:manufacture, nytrumpet:art).  If you use \textsf{Biber} as your
backend, \textsf{biblatex-chicago} also now includes the three
supplemental name fields (\textsf{name[a-c]}) in the sorting
algorithm, so once again you should find that this field is needed
less than before.  Still, many entries without a name field of any
sort, particularly those with a definite or indefinite article
beginning the \textsf{title}, may require assistance (chaucer:alt,
dyna:browser, gourmet:052006, greek:filmstrip, grove:sibelius,
lakeforester:pushcarts, nyt:trevorobit, silver:ga\-wain,
un\-signed:ran\-ke, vir\-gin\-ia:plan\-tation).  Lehman also provides
\textbf{sortname}, \textbf{sorttitle}, and \textbf{sortyear} for more
fine-grained control.  Please consult \textsf{biblatex.pdf} and the
remarks on \cmd{DeclareSortingScheme} in section~\ref{sec:formatopts},
below.

\mybigspace The \mymarginpar{\textbf{subtitle}} subtitle for a
\textsf{title} --- see next entry.

\mybigspace In \mymarginpar{\textbf{title}} the vast majority of
cases, this field works just as it always has in \textsc{Bib}\TeX, and
just as it does in \textsf{biblatex}.  Nearly every entry will have
one, the most likely exceptions being \textsf{incollection} or
\textsf{online} entries with a merely generic title, instead of a
specific one (centinel:letters, powell:email).  The main source of
difficulties flows from the \emph{Manual}'s rules for formatting
\textsf{titles}, rules which also hold for \textsf{booktitles} and
\textsf{maintitles}.  The whole point of using a
\textsc{Bib}\TeX-based system is for it to do the formatting for you,
and in most cases \textsf{biblatex-chicago-notes} does just that,
surrounding titles with quotation marks, italicizing them, or
occasionally just leaving them alone.  When, however, a title is
quoted within a title, then you need to know some of the rules.  A
summary here should serve to clarify them, and help you to understand
when \textsf{biblatex-chicago-notes} might need your help in order to
comply with them.

\mylittlespace The internal rules of \textsf{biblatex-chicago-notes}
are as follows:

\begin{description}
\item[\qquad Italics:] \textsf{booktitle}, \textsf{maintitle}, and
  \textsf{journaltitle} in all entry types; \textsf{title} of
  \textsf{artwork}, \textsf{book}, \textsf{bookinbook},
  \textsf{booklet}, \textsf{collection}, \textsf{image},
  \textsf{inbook}, \textsf{manual}, \textsf{misc} (with no
  \textsf{entrysubtype}), \textsf{periodical}, \textsf{proceedings},
  \textsf{report}, \textsf{suppbook}, and \textsf{suppcollection}
  entry types.
\item[\qquad Quotation Marks:] \textsf{title} of \textsf{article},
  \textsf{incollection}, \textsf{inproceedings}, \textsf{online},
  \textsf{periodical}, \textsf{thesis}, and \textsf{unpublished} entry
  types, \textsf{issuetitle} in \textsf{article}, \textsf{periodical},
  and \textsf{review} entry types.
\item[\qquad Unformatted:] \textsf{booktitleaddon},
  \textsf{maintitleaddon}, and \textsf{titleaddon} in all entry types,
  \textsf{title} of \textsf{customc}, \textsf{letter}, \textsf{misc}
  (with an \textsf{entrysubtype}), \textsf{patent}, \textsf{review},
  and \textsf{suppperiodical} entry types.
\item[\qquad Italics or Quotation Marks:] All of the audiovisual entry
  types --- \textsf{audio}, \textsf{music}, and \textsf{video} ---
  have to serve as analogues both to \textsf{book} and to
  \textsf{inbook}.  Therefore, if there is both a \textsf{title} and a
  \textsf{booktitle}, then the \textsf{title} will be in quotation
  marks.  If there is no \textsf{booktitle}, then the \textsf{title}
  will be italicized.
\end{description}

Now, the rules for which entry type to use for which sort of work tend
to be fairly straightforward, but in cases of doubt you can consult
section \ref{sec:entrytypes} above, the examples in
\textsf{notes-test.bib}, or go to the \emph{Manual} itself,
8.154--195.  Assuming, then, that you want to present a title within a
title, and you know what sort of formatting each of the two would, on
its own, require, then the following rules apply:

\begin{enumerate}
\item Inside an italicized title, all other titles are enclosed in
  quotation marks and italicized, so in such cases all you need to do
  is provide the quotation marks using \cmd{mkbibquote}, which will
  take care of any following punctuation that needs to be brought
  within the closing quotation mark(s) (14.102; donne:var,
  mchugh:wake).
\item Inside a quoted title, you should present another title as it
  would appear if it were on its own, so in such cases you'll need to
  do the formatting yourself.  Within the double quotes of the title
  another quoted title would take single quotes --- the
  \cmd{mkbibquote} command does this for you automatically, and also,
  I repeat, takes care of any following punctuation that needs to be
  brought within the closing quotation mark(s).  (See 14.177; garrett,
  loften:hamlet, murphy:silent, white:callimachus.)
\item Inside a plain title (most likely in a \textsf{review} entry or
  a \textsf{titleaddon} field), you should present another title as it
  would appear on its own, once again formatting it yourself using
  \cmd{mkbibemph} or \cmd{mkbibquote}.  (barcott:review, gibbard,
  osborne:poison, ratliff:review, unsigned:ranke).
\end{enumerate}

The \emph{Manual} provides a few more rules, as well.  A word normally
italicized in text should also be italicized in a quoted or plain-text
title, but should be in roman (\enquote{reverse italics}) in an
italicized title.  A quotation used as a (whole) title (with or
without a subtitle) retains its quotation marks in an italicized title
\enquote{only if it appears that way in the source,} but always
retains them when the surrounding title is quoted or plain (14.104,
14.177; lewis).  A word or phrase in quotation marks, but that isn't a
quotation, retains those marks in all title types (kimluu:diethyl).

\mylittlespace Finally, please note that in all \textsf{review} (and
\textsf{suppperiodical}) entries, and in \textsf{misc} entries with an
\textsf{entrysubtype}, and only in those entries,
\textsf{biblatex-chicago-notes} will automatically capitalize the
first word of the \textsf{title} after sentence-ending punctuation,
assuming that such a \textsf{title} begins with a lowercase letter in
your .bib database.  See \textbf{\textbackslash autocap} below for
more details.

\mybigspace Standard \mymarginpar{\textbf{titleaddon}}
\textsf{biblatex} intends this field for use with additions to titles
that may need to be formatted differently from the titles themselves,
and \textsf{biblatex-chicago-notes} uses it in just this way, with the
additional wrinkle that it can, if needed, replace the \textsf{title}
entirely, and this in, effectively, any entry type, providing a fairly
powerful, if somewhat complicated, tool for getting \textsc{Bib}\TeX\
to do what you want (cf.\ centinel:letters, powell:email).  This field
will always be unformatted, that is, neither italicized nor placed
within quotation marks, so any formatting you may need within it
you'll need to provide manually yourself.  The single exception to
this rule is when your data begins with a word that would ordinarily
only be capitalized at the beginning of a sentence, in which case you
need then simply ensure that that word is in lowercase, and
\textsf{biblatex-chicago-notes} will automatically do the right thing.
See\ \textbf{\textbackslash autocap}, below.  (Cf.\ brown:bremer,
osborne:poison, reaves:rosen, and white:ross:memo for examples where
the field starts with a lowercase letter; morgenson:market provides an
example where the \textsf{titleaddon} field, holding the name of a
regular column in a newspaper, is capitalized, a situation that is
handled as you would expect.)

\enlargethispage{-\baselineskip}

\mybigspace As \mymarginpar{\textbf{translator}} far as possible, I
have implemented this field as \textsf{biblatex}'s standard styles do,
but the requirements specified by the \emph{Manual} present certain
complications that need explaining.  Lehman points out in his
documentation that the \textsf{translator} field will be associated
with a \textsf{title}, a \textsf{booktitle}, or a \textsf{maintitle},
depending on the sort of entry.  More specifically,
\textsf{biblatex-chicago} associates the \textsf{translator} with the
most comprehensive of those titles, that is, \textsf{maintitle} if
there is one, otherwise \textsf{booktitle}, otherwise \textsf{title},
if the other two are lacking.  In a large number of cases, this is
exactly the correct behavior (adorno:benj, centinel:letters,
plato:republic:gr, among others).  Predictably, however, there are
numerous cases that require, for example, an additional translator for
one part of a collection or for one volume of a multi-volume work.
For these cases I have provided the \textsf{nameb} field.  You should
format names for this field as you would for \textsf{author} or
\textsf{editor}, and these names will always be associated with the
\textsf{title} (euripides:orestes).

\mylittlespace I have also provided a \textsf{namea} field, which
holds the editor of a given \textsf{title} (euripides:orestes).  If
\textsf{namea} and \textsf{nameb} are the same,
\textsf{biblatex-chicago} will concatenate them, just as
\textsf{biblatex} already does for \textsf{editor},
\textsf{translator}, and \textsf{namec} (i.e., the compiler).
Furthermore, it is conceivable that a given entry will need separate
translators for each of the three sorts of title.  For this, and for
various other tricky situations, there is the \cmd{parttrans} macro
(and its siblings), designed to be used in a \textsf{note} field or in
one of the \textsf{titleaddon} fields (ratliff:review).  (Because the
strings identifying a translator differ in notes and bibliography, one
can't simply write them out in such a field, hence the need for a
macro, which I discuss further in the commands section below
[\ref{sec:formatcommands}].)

\mylittlespace Finally, as I detailed above under \textbf{author}, in
the absence of an \textsf{author} or an \textsf{editor}, the
\textsf{translator} will be used at the head of an entry
(silver:gawain), and the bibliography entry alphabetized by the
translator's name, behavior that can be controlled with the
\texttt{{usetranslator}} switch in the \textsf{options} field.  Cf.\
\textsf{author}, \textsf{editor}, \textsf{namea}, \textsf{nameb}, and
\textsf{namec}.

\mybigspace This \mymarginpar{\textbf{type}} is a standard
\textsf{biblatex} field, and in its normal usage serves to identify
the type of a \textsf{manual}, \textsf{patent}, \textsf{report}, or
\textsf{thesis} entry.  \textsf{Biblatex} 0.7 introduced the ability,
in some circumstances, to use a bibstring without inserting it in a
\cmd{bibstring} command, and in these entry types the \textsf{type}
field works this way, allowing you simply to input, e.g.,
\texttt{patentus} rather than \cmd{bibstring\{patentus\}}, though both
will work.  (See petroff:impurity; herwign:office, murphy:silent, and
ross:thesis all demonstrate how the \textsf{type} field may sometimes
be automatically set in such entries by using one of the standard
entry-type aliases).

\mylittlespace In the \textsf{suppbook} entry type, and in its alias
\textsf{suppcollection}, you can use the \textsf{type} field to
specify what sort of supplemental material you are citing, e.g.,
\enquote{\texttt{preface to}} or \enquote{\texttt{postscript to}.}
Cf.\ \textsf{suppbook} above for the details.  (See \emph{Manual}
14.116; polakow:afterw, prose:intro).

\mylittlespace You can also use the \textsf{type} field in
\textsf{artwork}, \textsf{audio}, \textsf{image}, \textsf{music}, and
\textsf{video} entries to identify the medium of the work, e.g.,
\texttt{oil on canvas}, \texttt{albumen print}, \texttt{compact disc}
or \texttt{MPEG}.  If the first word in this field would normally only
be capitalized at the beginning of a sentence, then leave it in
lowercase in your .bib file and \textsf{biblatex} will automatically
do the right thing in citations.  Cf.\ \textsf{artwork},
\textsf{audio}, \textsf{image}, \textsf{music}, and \textsf{video},
above, for all the details.  (See auden:reading, bedford:photo,
cleese:holygrail, leo:madonna, nytrumpet:art.)

%%\enlargethispage{-\baselineskip}

\mybigspace A standard \mymarginpar{\textbf{url}} \textsf{biblatex}
field, it holds the url of an online publication, though you can
provide one for all entry types.  The 16th edition of the
\textsf{Manual} expresses a strong preference for DOIs over URLs if
the former is available --- cf.\ \textsf{doi} above, and also
\textsf{urldate} just below.  The required \LaTeX\ package
\textsf{url} will ensure that your documents format such references
properly, in the text and in the reference apparatus.

\mybigspace A standard \mymarginpar{\textbf{urldate}}
\textsf{biblatex} field, it identifies exactly when you accessed a
given url, and is given in \textsc{iso}8601 format.  The 16th edition
of the \emph{Manual} prefers DOIs to URLs; in the latter case it
allows the use of access dates, particularly in contexts that require
it, but prefers that you use revision dates, if these are available.
To enable you to specify which date is at stake, I have provided the
\textbf{userd} field, documented below.  If an entry doesn't have a
\textsf{userd}, then the \textsf{urldate} will be treated, as before,
as an access date (14.6--8, 14.184; evanston:library, grove:sibelius,
hlatky:hrt, osborne:poison, sirosh:visualcortex, wikiped:bibtex).

\mybigspace A \mymarginpar{\textbf{usera}} supplemental
\textsf{biblatex} field which functions in \textsf{biblatex-chicago}
almost as a \enquote{\textsf{journaltitleaddon}} field.  In
\textsf{article}, \textsf{periodical}, and \textsf{review} entries
with \textsf{entrysubtype} \texttt{magazine}, the contents of this
field will be placed, unformatted and between commas, after the
\textsf{journaltitle} and before the date.  The main use is for
identifying the broadcast network when you cite a radio or television
program (14.221; bundy:macneil).

\mybigspace I \mymarginpar{\textbf{userc}} have now implemented this
supplemental \textsf{biblatex} field as part of Chicago's name
cross-referencing system.  (The \enquote{c} part is meant as a sort of
mnemonic for this function, though it's perfectly possible to use the
field in other contexts.)  If you use the \textbf{customc} entry type
to include alphabetized cross-references to other, separate entries in
a bibliography, it is unlikely that you will cite the \textsf{customc}
entry in the body of your text.  Therefore, in order for it to appear
in the bibliography, you have two choices.  You can either include the
entry key of the \textsf{customc} entry in a \cmd{nocite} command
inside your document, or you can place that entry key in the
\textsf{userc} field of another .bib entry that you will be citing.
In the latter case, \textsf{biblatex-chicago} will call \cmd{nocite}
for you, and this method should ensure that there will be at least one
entry in the bibliography to which the cross-reference will point.
(See 14.84, 14.86; creasey:ashe:blast, creasey:morton:hide,
creasey:york:death, lecarre:quest.)

\mybigspace The \colmarginpar{\textbf{userd}} \textsf{userd} field,
recently added to the package, acts as a sort of
\enquote{\textsf{datetype}} field, allowing you in most entry types to
identify whether a \textsf{urldate} is an access date or a revision
date.  The general usage is fairly simple.  If this field is absent,
then a \textsf{urldate} will be treated as an access date, as has long
been the default in \textsf{biblatex} and in
\textsf{biblatex-chicago}.  If you need to identify it in any other
way, what you include in \textsf{userd} will be printed \emph{before}
the \textsf{urldate}, so phrases like \enquote{\texttt{last modified}}
or \enquote{\texttt{last revised}} are what the field will typically
contain (14.7--8; wikiped:bibtex).  In \colmarginpar{New!} the absence
of a \textsf{urldate}, you can now, in most entry types, include a
\textsf{userd} field to qualify a \textsf{date} in the same way it
would have modified a \textsf{urldate}.

\mylittlespace Because of the rather specialized needs of some
audio-visual references, this basic schema changes for \textsf{music}
and \textsf{video} entries.  In \textsf{music} entries where an
\textsf{eventdate} is present, \textsf{userd} will modify that date
instead of any \textsf{urldate} that may also be present, and it will
modify an \textsf{origdate} if it is present and there is no
\textsf{eventdate}.  It will modify a \textsf{date} only in the
absence of the other three.  In \textsf{video} entries it will modify an
\textsf{eventdate} if it is present, and in its absence the
\textsf{urldate}.  In the absence of those two, it can modify a
\textsf{date}.  Please see the documentation of the \textsf{music}
and \textsf{video} entry types, and especially of the
\textsf{eventdate}, \textsf{origdate}, and \textsf{urldate} fields,
above (14.276--279; nytrumpet:art).

\mylittlespace In all cases, you can start the \textsf{userd} field
with a lowercase letter, and \textsf{biblatex} will take care of
automatic contextual capitalization for you.

\mybigspace Another \mymarginpar{\textbf{usere}} supplemental
\textsf{biblatex} field, which \textsf{biblatex-chicago} uses
specifically to provide a translated \textsf{title} of a work,
something that may be needed if you deem the original language
unparseable by a significant portion of your likely readership.  The
\emph{Manual} offers two alternatives in such a situation: either you
can translate the title and use that translation in your
\textsf{title} field, providing the original language in
\textsf{language}, or you can give the original title in
\textsf{title} and the translation in \textsf{usere}.  If you choose
the latter, you may need to provide a \textsf{shorttitle} so that the
short note form is also parseable.  Cf.\ \textbf{language}, above.
(See 14.108--110, 14.194; kern, weresz.)

%%\enlargethispage{\baselineskip}

\mybigspace This \mymarginpar{\textbf{userf}} is the last of the
supplemental fields which \textsf{biblatex} provides, and is used by
\textsf{biblatex-chicago} for a very specific purpose.  When you cite
both a translation and its original, the \emph{Manual} (14.109)
recommends that, in the bibliography at least, you combine references
to both texts in one entry, though the presentation in notes is pretty
much up to you.  In order to follow this specification, I have
provided a third cross-referencing system (the others being
\textsf{crossref} and \textsf{xref}), and have chosen the name
\textsf{userf} because it might act as a mnemonic for its function.

\mylittlespace In order to use this system, you should start by
entering both the original and its translation into your .bib file,
just as you normally would.  The mechanism works for any entry type,
and the two entries need not be of the same type.  In the entry for
the \emph{translation}, you put the cite key of the original into the
\textsf{userf} field.  In the \emph{original's} entry, you need to
include something that will prevent the entry from being printed
separately in the bibliography --- \texttt{skipbib} in the
\textsf{options} field will work, as would something in the
\textsf{keywords} field in conjunction with a \texttt{notkeyword=}
switch in the \cmd{printbibliography} command.  In this standard case,
the data for the translation will be printed first, followed by the
string \texttt{originally published as}, followed by the original,
author omitted, in what amounts to the same format that the
\emph{Manual} uses for long footnotes (furet:passing:eng,
furet:passing:fr).  As explained above (\textbf{origlanguage}), I have
also included a way to modify the string printed before the original.
In the entry for the \emph{translation}, you put the original's
language in \textsf{origlanguage}, and instead of \texttt{originally
  published as}, you'll get \texttt{French edition:} or \texttt{Latin
  edition:}, etc.\ (aristotle:metaphy:gr, aristotle:metaphy:trans).

\mybigspace Standard \mymarginpar{\textbf{venue}} \textsf{biblatex}
offers this field for use in \textsf{proceedings} and
\textsf{inproceedings} entries, but I haven't yet implemented it,
mainly because the \emph{Manual} has nothing to say about it.  Perhaps
the \textsf{organization} field could be used, for the moment,
instead.  Anything in a \textsf{venue} field will be ignored.

\mybigspace Standard \mymarginpar{\textbf{version}} \textsf{biblatex}
field, currently only available in \textsf{misc} and \textsf{patent}
entries in \textsf{biblatex-chicago-notes}.

\mybigspace Standard \colmarginpar{\textbf{volume}} \textsf{biblatex}
field.  It holds the volume of a \textsf{journaltitle} in
\textsf{article} (and some \textsf{review}) entries, and also the
volume of a multi-volume work in many other sorts of entry.  The
treatment and placement of \textsf{volume} information in
\textsf{book}-like entries is rather complicated in the \emph{Manual}
(14.121--27).  In bibliography entries, the \textsf{volume} appears
either before the \textsf{maintitle} or before the publication
information.  In long notes, the same applies, but with the additional
possibility of this information appearing \emph{after} the publication
data, just before page numbers.  In the past, if you wanted the volume
information to appear here, you had to leave that information out of
your .bib entry and give it in the \textsf{pages} or \textsf{postnote}
field.  Now, you can use the new \textsf{biblatex-chicago} option
\mycolor{\texttt{delayvolume}} \colmarginpar{\texttt{delayvolume}} in
your preamble or in the \textsf{options} field of an entry to ensure
that any \textsf{volume} information that would normally have appeared
just before the publication data in a long note appears after it.

\mylittlespace The \textsf{volume} information in both books and
periodicals, and in both the bibliography and long notes, can appear
\emph{immediately before} the page number(s).  In such a case, the
\emph{Manual} prescribes the same treatment for both sorts of sources,
that is, that \enquote{a colon separates the volume number from the
  page number with no intervening space.}  I have implemented this,
but at the request of Clea~F.\ Rees I have made this punctuation
customizable, using the new command \mycolor{\cmd{postvolpunct}}
\colmarginpar{\cmd{postvolpunct}}.  By default it prints
\cmd{addcolon}, but you can use \cmd{renewcommand\{\textbackslash
  postvolpunct\}\{\ldots\}} in your preamble to redefine it.  Cf.\
\textsf{part}, and the command documentation in
section~\ref{sec:formatcommands}.

%%\enlargethispage{\baselineskip}

\mybigspace Standard \colmarginpar{\textbf{volumes}} \textsf{biblatex}
field.  It holds the total number of volumes of a multi-volume work,
and in the 16th edition of the \emph{Manual} no longer triggers any
odd changes to the punctuation of short notes (14.159;
meredith:letters).  If both a \textsf{volume} and a \textsf{volumes}
field are present, as may occur particularly in cross-referenced
entries, then \textsf{biblatex-chicago} will ordinarily suppress the
\textsf{volumes} field in the list of references, except in some
instances when a \textsf{maintitle} is present.  In this latter case,
if the \textsf{volume} appears before the \textsf{maintitle}, the new
option \mycolor{\texttt{hidevolumes}},
\colmarginpar{\texttt{hidevolumes}} set to \texttt{true} by default,
controls whether to print the \textsf{volumes} field after that title
or not.  Set it to \texttt{false} either in the preamble or in the
\textsf{options} field of your entry to have it appear after the
\textsf{maintitle}.  See the option's documentation in
section~\ref{sec:chicpreset}, below.

\mybigspace A \mymarginpar{\textbf{xref}} modified \textsf{crossref}
field provided by \textsf{biblatex}, which prevents inheritance of any
data from the parent entry.  See \textbf{crossref}, above.

\mybigspace Standard \mymarginpar{\textbf{year}} \textsf{biblatex}
field.  It usually identifies the year of publication, though unlike
the \textsf{date} field it allows non-numeric input, so you can put
\enquote{n.d.}\ (or, to be language agnostic,
\cmd{bibstring\{nodate\}}) here if required, or indeed any other sort
of non-numerical date information.  If you can guess the date then you
can include that guess in square brackets instead of, or after, the
\enquote{n.d.}\ abbreviation.  Cf.\ bedford:photo, clark:mesopot,
leo:madonna, ross:thesis.

\subsection{Commands}
\label{sec:commands}

In this section I shall attempt to document all those commands you may
need when using \textsf{biblatex-chicago-notes} that I have either
altered with respect to the standard provided by \textsf{biblatex} or
that I have provided myself.  Some of these, unfortunately, will make
your .bib file incompatible with other \textsf{biblatex} styles, but
I've been unable to avoid this.  Any ideas for more elegant, and more
compatible, solutions will be warmly welcomed.

\subsubsection{Formatting Commands}
\label{sec:formatcommands}

These commands allow you to fine-tune the presentation of your
references in both notes and bibliography.  You can find many examples
of their usage in \textsf{notes-test.bib}, and I shall try to point
you toward a few such entries in what follows.  \textbf{NB:}
\textsf{biblatex's} \cmd{mkbibquote} command is now mandatory in some
situations.  See its entry below.

\mybigspace Version \mymarginpar{\textbf{\textbackslash autocap}} 0.8
of \textsf{biblatex} introduced the \cmd{autocap} command, which
capitalizes a word inside a note or bibliography entry if that word
follows sentence-ending punctuation, and leaves it lowercase
otherwise.  As this command is both more powerful and more elegant
than the kludge I designed for a previous version of
\textsf{biblatex-chicago-notes} (see\ \textbf{\textbackslash
  bibstring} below), you should be aware that the use of the
single-letter \cmd{bibstring} commands in your .bib file is obsolete.

\mylittlespace In order somewhat to reduce the burden on users even
further, I have, following Lehman's example, implemented a new system
which automatically tracks the capitalization of certain fields in
your .bib file.  I chose these fields after a non-scientific survey of
entries in my own databases, so of course if you have ideas for the
extension of this facility I would be most interested to hear them.
In order to take advantage of this functionality, all you need do is
begin the data in the appropriate field with a lowercase letter,
e.g.,\ \texttt{note = \{with the assistance of X\}}.  If the data
begins with a capital letter --- and this is not infrequent --- that
capital will always be retained.  (cf., e.g., creel:house,
morgenson:market.)  If, on the other hand, you for some reason need
such a field always to start with a lowercase letter, then you can try
using the \cmd{isdot} macro at the start, which turns off the
mechanism without printing anything itself.  Here, then, is the
complete list of fields where this functionality is active:

\begin{enumerate}
\item The \textbf{addendum} field in all entry types.
\item The \textbf{booktitleaddon} field in all entry types.
\item The \textbf{edition} field in all entry types.  (Numerals work
  as you expect them to here.)
\item The \textbf{maintitleaddon} field in all entry types.
\item The \textbf{note} field in all entry types.
\item The \mycolor{\textbf{part}} field in entry types that use it.
\item The \mycolor{\textbf{prenote}} field prefixed to citation
  commands.
\item The \textbf{shorttitle} field in the \textsf{review}
  (\textsf{suppperiodical}) entry type and in the \textsf{misc} type,
  in the latter case, however, only when there is an
  \textsf{entrysubtype} defined, indicating that the work cited is
  from an archive.
\item The \textbf{title} field in the \textsf{review}
  (\textsf{suppperiodical}) entry type and in the \textsf{misc} type,
  in the latter case, however, only when there is an
  \textsf{entrysubtype} defined, indicating that the work cited is
  from an archive.
\item The \textbf{titleaddon} field in all entry types.
\item The \textbf{type} field in \textsf{artwork}, \textsf{audio},
  \textsf{image}, \textsf{music}, \textsf{suppbook},
  \textsf{suppcollection}, and \textsf{video} entry types.
\end{enumerate}

In any other cases --- and there are only two examples of this in
\textsf{notes-test.bib} (centinel:letters, powell:email) --- you'll
need to provide the \cmd{autocap} command yourself.  Indeed, if you
accidentally do so in one of the above fields, it shouldn't matter at
all, and you'll still get what you want, but taking advantage of the
automatic provisions should at least save some typing.

\mybigspace This \mymarginpar{\textbf{\textbackslash bibstring}} is
Lehman's very powerful mechanism to allow \textsf{biblatex}
automatically to provide a localized version of a string, and to
determine whether that string needs capitalization, depending on where
it falls in an entry.  In the first release of
\textsf{biblatex-chicago-notes}, the style relied very heavily on this
macro, particularly on an extension I provided by defining all 26
letters of the (ASCII) alphabet as \texttt{bibstrings}
(\cmd{bibstring\{a\}}, \cmd{bibstring\{b\}}, etc.)  While you should
continue to use the standard, whole-word bibstrings, \textbf{all use
  of the single-letter variants I formerly provided is obsolete, and
  will generate an error}.  This functionality has been replaced by
the \cmd{autocap} command, which does the same thing, only more
elegantly.  This command was designed by Philipp Lehman, and has now
been included in version 0.8 of \textsf{biblatex}.  For yet greater
convenience I have implemented, following Lehman's example, a system
automating this functionality in all of the entry fields where its use
was, by my reckoning, most frequent.  This means that, when you
require this functionality, all you need do is input the data in such
a field starting with a lowercase letter, and
\textsf{biblatex-chicago-notes} will do the rest with no further
assistance.  In my \textsf{notes-test.bib} file, this new mechanism in
effect eliminated all need for the single-letter \texttt{bibstrings}
and very nearly all need for the \cmd{autocap} command ---
centinel:letters and powell:email being the only exceptions.  Please
see \textbf{\textbackslash autocap} above for full details.

%%\enlargethispage{\baselineskip}

\mylittlespace I should also mention here that \textsf{biblatex 0.7}
introduced a new functionality which sometimes allows you simply to
input, for example, \texttt{newseries} instead of
\cmd{bib\-string\{newseries\}}, the package auto-detecting when a
bibstring is involved and doing the right thing, though in all such
cases either form will work.  This functionality is available in the
\textsf{series} field of \textsf{article}, \textsf{periodical}, and
\textsf{review} entries; in the \textsf{type} field of
\textsf{manual}, \textsf{patent}, \textsf{report}, and \textsf{thesis}
entries; in the \textsf{location} field of \textsf{patent} entries; in
the \textsf{language} field in all entry types; and in the
\textsf{nameaddon} field in \textsf{customc} entries.  These are the
places, as far as I can make out, where \textsf{biblatex's} standard
styles support this feature, and I have added the last,
style-specific, one.  If Lehman generalizes it still further in a
future release, I shall do the same, if possible.

\mybigspace In \mymarginpar{\textbf{\textbackslash custpunct} \\
  \textbf{\textbackslash custpunctb}} common with other American
citation styles, the \emph{Manual} requires that the commas and
periods separating units of a reference go inside any quotation marks
that happen to be present.  As of version 0.8c, \textsf{biblatex}
contains truly remarkable code that handles this situation in very
nearly complete generality, detecting punctuation after the closing
quotation mark and moving it inside when necessary, and also
controlling which punctuation marks can be printed after which other
punctuation marks, whether quotation marks intervene or not.  This
functionality is now mature, and \textsf{biblatex-chicago-notes}
relies on this code to place punctuation in the \enquote{American
  style,} rather than on complicated \cmd{DeclareFieldFormat}
instructions that attempt to anticipate all possible permutations.
One result of this, thankfully, is that both \cmd{custpunct} and
\cmd{custpunctb} are now basically unnecessary, as their only purpose
was to supply context-appropriate punctuation inside any quotation
marks that users themselves provided as part of various entry fields.
A second consequence, and I've already recommended this in previous
releases anyway, is that users now \emph{must} use \cmd{mkbibquote}
instead of \cmd{enquote} or the usual \LaTeX\ mechanisms inside their
.bib files.  For further details, please see the \cmd{mkbibquote}
entry below.

\mylittlespace I have retained the code for the \cmd{custpunct}
commands in \textsf{chicago-notes.cbx}, in case a particularly gnarly
entry might still require them, but I have already started to re-use
the \textsf{type} field, which formerly served as a switch for
\cmd{custpunct}, in other contexts (see \textbf{artwork},
\textbf{image}, and \textbf{suppbook} above).

\mybigspace These \colmarginpar{\textbf{\textbackslash
    foottextcite\\\textbackslash foottextcites}} two commands look
like citation commands, but are in fact wrappers for customizing the
behavior of the \cmd{textcite} and \cmd{textcites} citation commands
when they are used inside a foot- or an endnote.  By default, in such
a context these commands print the name of the \textsf{author(s)}
followed by the \emph{short} citation or citations, i.e., usually
\textsf{title} only, enclosed within parentheses.  You can change the
way the citation part is presented by using \cmd{renewcommand} in your
preamble.  The default definitions are: \texttt{\{\textbackslash
  addspace\textbackslash headlessparenshortcite\}} and
\texttt{\{\textbackslash addspace\textbackslash
  headlessparenshortcites\}}.  If you wanted to return to the default
behavior of the previous release of \textsf{biblatex-chicago}, for
example, you could change the first to: \texttt{\{\textbackslash
  newcunit\textbackslash bibstring\{in\}\textbackslash
  addspace\textbackslash headlesscite\}}, and the second similarly,
only using \cmd{headlesscites}.  (There is also, by the way, a
\cmd{headlessparencite(s)} command if you want to retain the long
citations inside the parentheses.)

\enlargethispage{\baselineskip}

\mybigspace This \mymarginpar{\textbf{\textbackslash isdot}} is a
standard \textsf{biblatex} macro, which in previous releases of
\textsf{biblatex-chicago} could function as a convenient placeholder
in entry fields that, for one reason or another, you may have wanted
to have defined and yet to print nothing.  With the release of
\textsf{biblatex} 1.6, this no longer works as before, a situation
which has revealed a number of inconsistencies and bugs in my code,
the rectification of which may therefore require some changes to your
.bib files, assuming you've taken advantage of this mechanism.  I
believe that all the situations formerly calling for this specific use
of the macro can now be addressed by more standard means, i.e., the
\cmd{headlesscite} commands and the \texttt{useauthor=false}
declaration in the \textsf{options} field.  (See creel:house,
nyt:obittrevor, sewall:letter, unsigned:ranke, and white:total.)

\mybigspace I \mymarginpar{\textbf{\textbackslash letterdatelong}}
have provided this macro mainly for use in the optional postnote field
of the various citation commands.  When citing a letter (published or
unpublished, \textsf{letter} or \textsf{misc}), it may be useful to
append the date to the usual short note form in order to disambiguate
references.  This macro simply prints the date of a letter, or indeed
of any other sort of correspondence.  (If your main document language
isn't English, it's better just to use the standard \textsf{biblatex}
command \cmd{printorigdate}.)

\mybigspace This \mymarginpar{\textbf{\textbackslash mkbibquote}} is
the standard \textsf{biblatex} command, which requires attention here
because it is a crucial part of the mechanism of Lehman's
\enquote{American} punctuation system.  If you look in
\textsf{chicago-notes.cbx} you'll see that the quoted fields, e.g., an
\textsf{article} or \textsf{incollection title}, have this command in
their formatting, which does most of the work for you.  If, however,
you need to provide additional quotation marks in a field --- a quoted
title within a title, for example --- then you may need to use this
command so that any following period or comma will be brought within
the closing quotation marks.  Its use is \emph{required} when the
quoted material comes at the end of a field, and I recommend always
using it in your .bib database, as it does no harm even when that
condition is not fulfilled.  A few examples from
\textsf{notes-test.bib} should help to clarify this.

\mylittlespace In an \textsf{article} entry, the \textsf{title}
contains a quoted phrase:

\begin{quotation}
  \noindent\texttt{title = \{Diethylstilbestrol and Media Coverage of the \\
    \indent\cmd{mkbibquote}\{Morning After\} Pill\}}
\end{quotation}

Here, because the quoted text doesn't come at the end of title, and no
punctuation will ever need to be drawn within the closing quotation
mark, you could instead use \texttt{\cmd{enquote}\{Morning After\}} or
even \texttt{`Morning After'}. (Note the single quotation marks here
--- the other two methods have the virtue of taking care of nesting
for you.)  All of these will produce the formatted
\enquote{Diethylstilbestrol and Media Coverage of the \enquote{Morning
    After} Pill.}  Here, by contrast, is a \textsf{book title}:

\begin{quotation}
  \noindent \texttt{title = \{Annotations to
    \cmd{mkbibquote}\{Finnegans Wake\}\}}
\end{quotation}

Because the quoted title within the title comes at the end of the
field, and because this bibliographical unit will be separated from
what follows by a period in the bibliography, then the
\cmd{mkbibquote} command is necessary to bring that period within the
final quotation marks, like so: \emph{Annotations to
  \enquote{Finnegans Wake.}}

\mylittlespace Let me also add that this command interacts well with
Lehman's \textsf{csquotes} package, which I highly recommend, though
the latter isn't strictly necessary in texts using an American style,
to which \textsf{biblatex} defaults when \textsf{csquotes} isn't
loaded.

\mybigspace The \colmarginpar{\textbf{\textbackslash postvolpunct}}
\emph{Manual} (14.121) unequivocally prescribes that when a
\textsf{volume} number appears immediately before a page number,
\enquote{the abbreviation \emph{vol.}\ is omitted and a colon
  separates the volume number from the page number with no intervening
  space.}  The treatment is basically the same whether the citation is
of a book or of a periodical, and it appears to be a surprising and
unwelcome feature for many users, conflicting as it may do with
established typographic traditions in a number of contexts.  Clea~ F.\
Rees has requested a way to customize this, so I have provided the new
\cmd{postvolpunct} command, which prints the punctuation between a
\textsf{volume} number and a page number.  It is set to \cmd{addcolon}
by default, except when the current language of the entry is French,
in which case it defaults to \cmd{addcolon\textbackslash addspace}.
You can use \cmd{renewcommand\{\textbackslash
  postvolpunct\}\{\ldots\}} in your preamble to redefine it, but
please note that the command only applies in this limited context, not
more generally to the punctuation that appears between, e.g., a
\textsf{volume} and a \textsf{part} field.

\enlargethispage{-\baselineskip}

\mybigspace This \mymarginpar{\textbf{\textbackslash reprint}} and the
following 7 macros all help \textsf{biblatex-chicago-notes} cope with
the fact that many bibstrings in the Chicago system differ between
notes and bibliography, the former sometimes using abbreviated forms
when the latter prints them in full.  In the current case, if a book
is a reprint, then the macro \cmd{reprint}, followed by a comma,
should go in the \textsf{location} field before the city of
publication (aristotle:metaphy:gr, schweitzer:bach).  See
\textbf{location}, above.

\mylittlespace \textbf{NB:} The rules for employing abbreviated or
full bibstrings in the \emph{Manual} are remarkably complex, but I
have attempted to make them as transparent for users as possible.  In
\textsf{biblatex-chicago-notes}, if you don't see it mentioned in this
section, then in theory you should always provide an abbreviated
version, using the \cmd{bibstring} mechanism, if necessary
(babb:peru).  The standard \textsf{biblatex} bibstrings should also
work (palmatary:pottery), and any that won't should be covered by the
series of macros beginning here with \cmd{reprint} and ending below
with \cmd{parttransandcomp}.

\mybigspace Since \mymarginpar{\textbf{\textbackslash partcomp}} the
\emph{Manual} specifies that the strings \texttt{editor},
\texttt{translator}, and \texttt{compiler} all require different forms
in notes and bibliography, and since it mentions these three apart
from all the others \textsf{biblatex} provides (\textsf{annotator},
\textsf{commentator}, et al.), and further since it may indeed happen
that the available fields (\textsf{editor}, \textsf{namea},
\textsf{translator}, \textsf{nameb}, and \textsf{namec}) aren't
adequate for presenting some entries, I have provided 7 macros to
allow you to print the correct strings for these functions in both
notes and bibliography.  Their names all begin with \cmd{part}, as
originally I intended them for use when a particular name applied only
to a specific \textsf{title}, rather than to a \textsf{maintitle} or
\textsf{booktitle} (cf.\ \textbf{namea} and \textbf{nameb}, above).

\mylittlespace In the present instance, you can use \cmd{partcomp} to
identify a compiler when \textsf{namec} won't do, e.g., in a
\textsf{note} field or the like.  In such a case,
\textsf{biblatex-chicago-notes} will print the appropriate string in
your references.

\mybigspace Use \mymarginpar{\textbf{\textbackslash partedit}} this
macro when identifying an editor whose name doesn't conveniently fit
into the usual fields (\textsf{editor} or \textsf{namea}).  (N.B.: If
you are writing in French and using \textsf{cms-french.lbx}, then
currently you'll need to add either \texttt{de} or \texttt{d'} after
this command in your .bib files to make the references come out right.
I'm working on this.)  See chaucer:liferecords.

\enlargethispage{-\baselineskip}

\mybigspace As \mymarginpar{\textbf{\textbackslash
    partedit-\\andcomp}} before, but for use when an editor is also a
compiler.

\vspace{1.3\baselineskip} As \mymarginpar{\textbf{\textbackslash
    partedit-\\andtrans}} before, but for when when an editor is also a
translator (ratliff:review).

\vspace{1.3\baselineskip} As \mymarginpar{\textbf{\textbackslash
    partedit-\\transandcomp}} before, but for when an editor is also a
translator and a compiler.

\vspace{1.4\baselineskip} As \mymarginpar{\textbf{\textbackslash
    parttrans-\\andcomp}} before, but for when a translator is also a
compiler.

\vspace{1.3\baselineskip} As \mymarginpar{\textbf{\textbackslash
    parttrans}} before, but for use when identifying a translator
whose name doesn't conveniently fit into the usual fields
(\textsf{translator} and \textsf{nameb}).

\subsubsection{Citation Commands}
\label{sec:citecommands}

The \textsf{biblatex} package is particularly rich in citation
commands, some of which (e.g., \cmd{supercite(s)}, \cmd{citeyear})
provide functionality that isn't really needed by the Chicago notes
and bibliography style offered here.  If you are getting unexpected
behavior when using them please have a look in your .log file ---
there may be warnings there that alert you to undefined citation
commands.  Other \textsf{biblatex}-provided commands, though I haven't
tested them extensively, should pretty much work out of the box.  What
remains are the commands I have found most useful and necessary for
following the \emph{Manual}'s specifications, and I document in this
section any alterations I have made to these.  As always, if there are
standard commands that don't work for you, or new commands that would
be useful, please let me know, and it should be possible to fix or add
them.

\mylittlespace A number of users have run into a problem that appears
when they've used a command like \cmd{cite} inside a \cmd{footnote}
macro.  In this situation, the automatic capitalization routines will
not be in operation at the start of the footnote, so instead of
\enquote{Ibid.,} for example, you'll see \enquote{ibid.}  If you need
to use the \cmd{cite} command within a \cmd{footnote} command, the
solution is to use \cmd{Cite} instead.  Alternatively, don't use a
\cmd{footnote} macro at all, rather try \cmd{footcite} or
\cmd{autocite} with the optional prenote and postnote arguments.  Cf.\
\cmd{Citetitle} below, and also section~3.7 of \textsf{biblatex.pdf}.

\mybigspace I \mymarginpar{\textbf{\textbackslash autocite}} haven't
adapted this in the slightest, but I thought it worth pointing out
that \textsf{biblatex-chicago-notes} sets this command to use
\cmd{footcite} as the default option. It is, in my experience, much
the most common citation command you will use, and also works fine in
its multicite form, \textbf{\textbackslash autocites}.

\mybigspace While \mymarginpar{\textbf{\textbackslash cite*}} the
\cmd{cite} command works just as you would expect it to, I have also
provided a starred version for the rare situations when you might need
to turn off the ibidem tracking mechanism.  \textsf{Biblatex} provides
very sophisticated algorithms for using \enquote{Ibid} in notes, so in
general you won't find a need for this command, but in case you'd
prefer a longer citation where you might automatically find
\enquote{Ibid,} I've provided this.  Of course, you'll need to put it
inside a \cmd{footnote} command manually.  (See also section
\ref{sec:useropts}, below.)

\mybigspace I \mymarginpar{\textbf{\textbackslash citeauthor}} have
adapted this standard \textsf{biblatex} command only very slightly to
bring it into line with \textsf{biblatex-chicago's} needs.  Its main
usage will probably be for references to works from classical
antiquity, when an \textsf{author's} name (abbreviated or not)
sometimes suffices in the absence of a \textsf{title}, e.g.,
Thuc.\ 2.40.2--3 (14.258).  You'll need to put it inside a
\cmd{footnote} command manually.  (Cf.\ also \textsf{entrysubtype} in
section~\ref{sec:entryfields}, above.)

\mybigspace This \mymarginpar{\textbf{\textbackslash citejournal}}
command provides an alternative short form when citing journal
\textsf{articles}, giving the \textsf{journaltitle} and
\textsf{volume} number instead of the article \textsf{title} after the
\textsf{author's} name.  The \emph{Manual} suggests that this format
might be helpful \enquote{in the absence of a full bibliography}
(14.196).  It may also prove useful when you want to provide
parenthetical references to newspaper articles within the text rather
than in the bibliography, a style endorsed by the \emph{Manual}
(14.206).  In such a case, an article's author, if there is one, could
form part of the running text.  As usual with these general citation
commands, if you want the reference to appear in a footnote you need
to put it inside a \cmd{footnote} command manually.

\mybigspace This \mymarginpar{\textbf{\textbackslash Citetitle}}
simply prepends \cmd{bibsentence} to the usual \cmd{citetitle}
command.  Some titles may need this for the automatic contextual
capitalization facility to work correctly.  (Included as standard from
\textsf{biblatex} 0.8d.)

\mybigspace Joseph \mymarginpar{\textbf{\textbackslash citetitles}}
Reagle noticed that, because of the way
\textsf{biblatex-chicago-notes} formats titles in quotation marks,
using the \cmd{citetitle} command will often get you punctuation you
don't want, especially when presenting a list of titles.  I've
included this multicite command to enable you to present such a list,
if the need arises.  Remember that you'll have to put it inside a
\cmd{footnote} command manually.

%\enlargethispage{\baselineskip}

\mybigspace Another \mymarginpar{\textbf{\textbackslash footfullcite}}
standard \textsf{biblatex} command, modified to work properly with
\textsf{biblatex-chicago-notes}, and provided in case you find
yourself in a situation where you really need the full citation in a
footnote, but where \cmd{autocite} would print a short note or even
\enquote{Ibid.}  This may be particularly useful if you've chosen to use all
short notes by setting the \texttt{short} option in the arguments to
\cmd{usepackage\{biblatex\}}, yet still feel the need for the
occasional full citation.

\mybigspace This, \mymarginpar{\textbf{\textbackslash fullcite}} too,
is a standard command, and it too provides a full citation, but unlike
the previous command it doesn't automatically place it in a footnote.
It may be useful within long textual notes.

\mybigspace Matthew \mymarginpar{\textbf{\textbackslash headlesscite}}
Lundin requested a more generalized \cmd{headlesscite} macro,
suppressing the author's name in specific contexts while allowing
users not to worry about whether a particular citation needs the long
or short form, a responsibility thereby handed over to
\textsf{biblatex's} tracking mechanisms.  This citation command
attempts to fulfill this request.  Please note that, in the short
form, the result will be rather like a \cmd{citetitle} command, which
may or may not be what you want.  Note, also, that as I have provided
only the most flexible form of the command, you'll have to wrap it in
a \cmd{footnote} yourself.  Please see the next entry for further
discussion of some of the needs this command might help address.

\mybigspace I \mymarginpar{\textbf{\textbackslash
    headless-\\fullcite}} have provided this command in case you want
to print a full citation without the author's name.  The \emph{Manual}
(14.78, 14.88) suggests this for brevity's sake in cases where that
name is already obvious enough from the title, and where repetition
might seem awkward (creel:house, feydeau:farces, meredith:letters, and
sewall:letter).  \textsf{Letter} entries --- and only such entries ---
do this for you automatically, and of course the repetition is
tolerated in bibliographies for the sake of alphabetization, but in
notes this command may help achieve greater elegance, even if it isn't
strictly necessary.  As I've provided only the most flexible form of
the command, you'll have to wrap it in a \cmd{footnote} yourself.

\mybigspace I \mymarginpar{\textbf{\textbackslash shortcite}} have
provided this command in case, for any reason, you specifically
require the short form of a note, and \textsf{biblatex} thinks you
want something else.  Again, I've provided only the most flexible form
of the command, so you'll have to wrap it in a \cmd{footnote}
manually.

\mybigspace At \mymarginpar{\textbf{\textbackslash shorthandcite}}
the request of Kenneth Pearce, I have included this command which
always prints the \textsf{shorthand}, even at the first citation of a
given work.  Again, I've only provided the most flexible form of the
command, so you'll need to place it inside parentheses or wrap it in a
\cmd{footnote} manually.

\mybigspace This \mymarginpar{\textbf{\textbackslash surnamecite}}
command is analogous to \cmd{headlesscite}, but whereas the latter
allows you to omit an \textsf{author's} name when that name is obvious
from the \textsf{title} of a work, \cmd{surnamecite} allows you to
shorten a full note citation in contexts where the full name(s) of the
\textsf{author} have already been provided in the text.  In short
notes this falls back to the standard format, but in long notes it
simply omits the given names of the \textsf{author} and provides only
the surname, along with the full data of the entry.  (Cf.\ 14.52.)

\mybigspace Norman \colmarginpar{\textbf{\textbackslash textcite}}
Gray started a discussion on
\href{http://tex.stackexchange.com/questions/67837/citations-as-nouns-in-biblatex-chicago}{Stackexchange}
which established both that \textsf{biblatex} had begun including a
\cmd{textcite} command in its verbose styles and that
\textsf{biblatex-chicago-notes} hadn't kept up.  In that thread Audrey
Boruvka provided some code, adapted from \textsf{verbose.cbx}, to
provide such a command for the Chicago notes \&\ bibliography style.
More recently, Rasmus Pank Rouland pointed out some changes in
\textsf{biblatex} that made the \cmd{textcites} command fit more
elegantly into the flow of text.  I've adapted this solution in this
release.  I'm still not entirely certain how best to accommodate this
request within the package, but there are now at least commands
(\cmd{textcite} and \cmd{textcites}) for users to test.  Their
functionality is a little complicated.  In the main text, they will
provide an \textsf{author's} name(s), followed immediately by a foot-
or endnote which contains the full (or short) reference, following the
usual rules.  If you use \cmd{textcite} inside \colmarginpar{New!} a
foot- or endnote, then the default behavior has changed for this
release, but I've made that behavior configurable, so you can tailor
it to your needs.  Now, by default, for both \cmd{textcite} and
\cmd{textcites}, you'll get the \textsf{author's} name(s) followed by
a headless \emph{short} citation (or citations) placed within
parentheses.  Such parentheses are generally discouraged by the
\emph{Manual} (14.33), but are nonetheless somewhat better than other
solutions for smoothing the syntax of sentences that include such a
citation.  I have made the citation short, i.e., \textsf{title} only,
because this again seems likely to be the least awkward solution
syntactically.  If you want to configure this behavior for either
citation command, please see \mycolor{\cmd{foottextcite}} and
\mycolor{\cmd{foottextcites}} in section~\ref{sec:formatcommands}.

\mylittlespace If you look at \textsf{chicago-notes.cbx}, you'll see a
number of other citation commands, but those are intended for internal
use only, mainly in cross-references of various sorts.  Use at your
own risk.

\subsection{Package Options}
\label{sec:options}

\subsubsection{Pre-Set \textsf{biblatex} Options}
\label{sec:presetopts}

Although a quick glance through \textsf{biblatex-chicago.sty} will
tell you which \textsf{biblatex} options the package sets for you, I
thought I might gather them here also for your perusal.  These
settings are, I believe, consistent with the specification, but you
can alter them in the options to \textsf{biblatex-chicago} in your
preamble or by loading the package via
\cmd{usepackage[style=chicago-notes]\{biblatex\}}, which gives you the
\textsf{biblatex} defaults unless you redefine them yourself inside
the square brackets.

\mylittlespace By \mymarginpar{\texttt{abbreviate=\\false}} default,
\textsf{biblatex-chicago-notes} prints the longer bibstrings, mainly
for use in the bibliography, but since notes require the shorter forms
of many of them, I've had to define many new strings for use there.

\mylittlespace \textsf{Biblatex-chicago-notes}
\mymarginpar{\texttt{autocite=\\footnote}} places references in
footnotes by default.

\mybigspace The \mymarginpar{\texttt{citetracker=\\true}} citetracker
for the \cmd{ifciteseen} test is enabled globally.

\mybigspace The \mymarginpar{\texttt{alldates=comp}} specification calls
for the long format when presenting dates, slightly shortened when
presenting date ranges.

\mylittlespace The \mymarginpar{\texttt{dateabbrev=\\false}}
\emph{Manual} prefers full month names in the notes \&\ bibliography
style.

\mybigspace This \mymarginpar{\texttt{ibidtracker=\\constrict}}
enables the use of \enquote{Ibid} in notes, but only in the most
strictly-defined circumstances.  Whenever there might be any
ambiguity, \textsf{biblatex} should default to printing a more
informative reference.  Remember also that you can use the \cmd{cite*}
command to disable this functionality in any given reference, or
indeed one of the \texttt{fullcite} commands if you need the long note
form for any reason.

\mylittlespace This \mymarginpar{\texttt{loccittracker\\=constrict}}
allows the package to determine whether two consecutive citations of
the same source also cite the same page of that source.  In such a
case, \texttt{Ibid} alone will be printed, without the page reference,
following the specification (14.29).

\mylittlespace These \mymarginpar{\textsf{\texttt{maxbibnames\\=10\\
      minbibnames\\=7}}} two options are new, and control the number
of names printed in the bibliography when that number exceeds 10.
These numbers follow the recommendations of the \emph{Manual} (14.76),
and they are different from those for use in notes.  With
\textsf{biblatex} 1.6 you can no longer redefine \texttt{maxnames} and
\texttt{minnames} in the \cmd{printbibliography} command at the bottom
of your document, so \textsf{biblatex-chicago} now does this
automatically for you, though of course you can change them in your
document preamble.

\mylittlespace This \mymarginpar{\texttt{pagetracker=\\true}} enables
page tracking for the \cmd{iffirstonpage} and \cmd{ifsamepage}
commands for controlling, among other things, the printing of
\enquote{Ibid.}  It tracks individual pages if \LaTeX\ is in oneside
mode, or whole spreads in twoside mode.

\mylittlespace This \mymarginpar{\texttt{sortcase=\\false}} turns off
the sorting of uppercase and lowercase letters separately, a practice
which the \emph{Manual} doesn't appear to recommend.

\mylittlespace This \mymarginpar{\texttt{sorting=\\}\cmd{cms@choose}}
new setting tests whether you are using \textsf{Biber} as your
backend, and if so enables a custom \textsf{biblatex-chicago} sorting
scheme for the bibliography (\texttt{cms}).  If you are using any
other backend, it reverts to the \textsf{biblatex} default
(\texttt{nty}).  Please see the discussion of
\cmd{DeclareSortingScheme} just below.

\mylittlespace This \mymarginpar{\texttt{usetranslator\\=true}}
enables automatic use of the \textsf{translator} at the head of
entries in the absence of an \textsf{author} or an \textsf{editor}.
In the bibliography, the entry will be alphabetized by the
translator's surname.  You can disable this functionality on a
per-entry basis by setting \texttt{usetranslator=false} in the
\textsf{options} field.  Cf.\ silver:gawain.

\subsubsection*{Other \textsf{biblatex} Formatting Options}
\label{sec:formatopts}

I've chosen defaults for many of the general formatting commands
provided by \textsf{biblatex}, including the vertical space between
bibliography items and between items in the list of shorthands
(\cmd{bibitemsep} and \cmd{lositemsep}).  I define many of these in
\textsf{biblatex-chicago.sty}, and of course you may want to redefine
them to your own needs and tastes.  It may be as well you know that
the \emph{Manual} does state a preference for two of the formatting
options I've implemented by default: the 3-em dash as a replacement
for repeated names in the bibliography (14.63--67, and just below);
and the formatting of note numbers, both in the main text and at the
bottom of the page / end of the essay (superscript in the text,
in-line in the notes; 14.19).  The code for this last formatting is
also in \textsf{biblatex-chicago.sty}, and I've wrapped it in a test
that disables it if you are using the \textsf{memoir} class, which I
believe has its own commands for defining these parameters.  You can
also disable it by using the \texttt{footmarkoff} package option, on
which see below.

\mylittlespace Gildas Hamel pointed out that my default definition, in
\textsf{biblatex-chicago.sty}, of \textsf{biblatex's}
\cmd{bibnamedash} didn't work well with many fonts, leaving a line of
three dashes separated by gaps.  He suggested an alternative, which
I've adopted, with a minor tweak to make the dash thicker, though you
can toy with all the parameters to find what looks right with your
chosen font.  The default definition is:
\cmd{renewcommand*\{\textbackslash bibnamedash\}\{\textbackslash
  rule[.4ex]\{3em\}\{.6pt\}\}}.

\mylittlespace At \mymarginpar{\texttt{losnotes}
  \&\\\texttt{losendnotes}} the request of Kenneth Pearce, I have
added two new \texttt{bibenvironments} to \textsf{chicago-notes.bbx},
for use with the \texttt{env} option to the \cmd{printshorthands}
command.  The first, \texttt{losnotes}, is designed to allow a list of
shorthands to appear inside footnotes, while \texttt{losendnotes} does
the same for endnotes.  Their main effect is to change the font size,
and in the latter case to clear up some spurious punctuation and white
space that I see on my system when using endnotes.  (You'll probably
also want to use the option \texttt{heading=none} in order to get rid
of the [oversized] default, providing your own within the
\cmd{footnote} command.)  Please see the documentation of
\textsf{shorthand} in section~\ref{sec:entryfields} above for further
options available to you for presenting and formatting the list of
shorthands.

\mylittlespace The \mymarginpar{\cmd{Declare-}\\\texttt{Labelname}}
next-generation backend \textsf{Biber} offers enhanced functionality
in many areas, two of which I've implemented in this release.
\cmd{DeclareLabelname} allows you to add name fields for consideration
when \textsf{biblatex} is attempting to find a shortened name for
short notes.  This, for example, allows a compiler (=\textsf{namec})
to appear at the head of short notes without any other intervention
from the user, rather than requiring a \textsf{shortauthor} field as
previous releases of \textsf{biblatex-chicago} did.  In point of fact,
I have implemented this functionality in such a way as to make it
available even to users of other backends, but this required reducing
its flexibility considerably.  When \textsf{biblatex} reaches version
2.0, \textsf{Biber} will become a requirement, so I recommend getting
to know it sooner rather than later.

\mylittlespace The
\mymarginpar{\cmd{Declare-}\\\texttt{SortingScheme}} second
\textsf{Biber} enhancement I have implemented allows you to include
almost any field whatsoever in \textsf{biblatex's} sorting algorithms
for the bibliography, so that a great many more entries will be sorted
correctly automatically rather than requiring manual intervention in
the form of a \textsf{sortkey} field or the like.  Code in
\textsf{biblatex-chicago.sty} detects whether you are using
\textsf{Biber}, and if and only if this is the case changes the
sorting scheme to a custom one (\texttt{cms}), a Chicago-specific
variant of the default \texttt{nty}.  (You can find its definition in
\textsf{chicago-notes.cbx}.)  Users of all other backends will still
be using \texttt{nty}.

%%\enlargethispage{\baselineskip}

\mylittlespace The advantages of this scheme are, specifically, that
any entry headed by one of the supplemental name fields
(\textsf{name[a-c]}), a \textsf{manual} entry headed by an
\textsf{organization}, or an \textsf{article} or \textsf{review} entry
headed by a \textsf{journaltitle} will no longer need a
\textsf{sortkey} set.  The main disadvantage should only occur very
rarely, and appears because the supplemental name fields are treated
differently from the standard name fields by \textsf{biblatex}.
Ordinarily, you can set, for example, \texttt{useauthor=false} in the
\textsf{options} field to remove the \textsf{author's} name from
consideration for sorting purposes.  The Chicago-specific option
\textsf{usecompiler=false}, however, doesn't remove \textsf{namec}
from such consideration, so in an entry like chaucer:alt you \emph{do}
need a \textsf{sortkey} or else it will be alphabetized by
\textsf{namec} rather than by \textsf{title}.

\subsubsection{{Pre-Set \textsf{chicago} Options}}
\label{sec:chicpreset}

At \mymarginpar{\texttt{bookpages=\\true}} the request
of Scot Becker, I have included this rather specialized option, which
controls the printing of the \textsf{pages} field in \textsf{book}
entries.  Some bibliographic managers, apparently, place the total
page count in that field by default, and this option allows you to
stop the printing of this information in notes and bibliography.  It
defaults to true, which means the field is printed, but it can be set
to false either in the preamble, for the whole document, or on a
per-entry basis in the \textsf{options} field (though rather than use
this latter method it would make sense to eliminate the \textsf{pages}
field from the affected entries).

\mylittlespace This \mymarginpar{\texttt{doi=true}} option controls
whether any \textsf{doi} fields present in the .bib file will be
printed in notes and bibliography.  At the request of Daniel
Possenriede, and keeping in mind the \emph{Manual's} preference for
this field instead of a \textsf{url} (14.6), I have added a third
switch, \texttt{only}, which prints the \textsf{doi} if it is present
and the \textsf{url} only if there is no \textsf{doi}.  The package
default remains the same, however --- it defaults to true, which will
print both \textsf{doi} and \textsf{url} if both are present.  The
option can be set to \texttt{only} or to \texttt{false} either in the
preamble, for the whole document, or on a per-entry basis in the
\textsf{options} field.  In \textsf{online} entries, the \textsf{doi}
field will always be printed, but the \texttt{only} switch will still
eliminate any \textsf{url}.

% %\enlargethispage{\baselineskip}

\mylittlespace This \mymarginpar{\texttt{eprint=true}} option controls
whether any \textsf{eprint} fields present in the .bib file will be
printed in notes and bibliography.  It defaults to true, and can be
set to false either in the preamble, for the whole document, or on a
per-entry basis, in the \textsf{options} field.  In \textsf{online}
entries, the \textsf{eprint} field will always be printed.

\mylittlespace This \mymarginpar{\texttt{isbn=true}} option controls
whether any \textsf{isan}, \textsf{isbn}, \textsf{ismn},
\textsf{isrn}, \textsf{issn}, and \textsf{iswc} fields present in the
.bib file will be printed in notes and bibliography.  It defaults to
true, and can be set to false either in the preamble, for the whole
document, or on a per-entry basis, in the \textsf{options} field.

\mylittlespace Once \mymarginpar{\texttt{numbermonth=\\true}} again at
the request of Scot Becker, I have included this option, which
controls the printing of the \textsf{month} field in all the
periodical-type entries when a \textsf{number} field is also present.
Some bibliographic software, apparently, always includes the month of
publication even when a \textsf{number} is present.  When all this
information is available the \emph{Manual} (14.180, 14.185) prints
everything, so this option defaults to true, which means the field is
printed, but it can be set to false either in the preamble, for the
whole document, or on a per-entry basis in the \textsf{options} field.

\mylittlespace This \mymarginpar{\texttt{url=true}} option controls
whether any \textsf{url} fields present in the .bib file will be
printed in notes and bibliography.  It defaults to true, and can be
set to false either in the preamble, for the whole document, or on a
per-entry basis, in the \textsf{options} field.  Please note that, as
in standard \textsf{biblatex}, the \textsf{url} field is always
printed in \textsf{online} entries, regardless of the state of this
option.

\mylittlespace This \mymarginpar{\texttt{includeall=\\true}} is the
one option that rules the six preceding, either printing all the
fields under consideration --- the default --- or excluding all of
them.  It is set to \texttt{true} in \textsf{chicago-notes.cbx}, but
you can change it either in the preamble for the whole document or,
for specific fields, in the \textsf{options} field of individual
entries.  The rationale for all of these options is the availability
of bibliographic managers that helpfully present as much data as
possible, in every entry, some of which may not be felt to be entirely
necessary.  Setting \texttt{includeall} to \texttt{true} probably
works just fine for those compiling their .bib databases by hand, but
others may find that some automatic pruning helps clear things up, at
least to a first approximation.  Some per-entry work afterward may
then polish up the details.

\mylittlespace At \mymarginpar{\texttt{addendum=\\true}} the request
of Roger Hart, I have included this option, which controls the
printing of the \textsf{addendum} field, but \emph{only} in long
notes.  It defaults to true, and can be set to false either in the
preamble, for the whole document, or on a per-entry basis, in the
\textsf{options} field.

\mylittlespace According \mymarginpar{\texttt{bookseries=\\true}} to
the \emph{Manual} (14.128), the \textsf{series} field in book-like
entries \enquote{may be omitted to save space (especially in a
  footnote).}  This option allows you to control the printing of that
field in long notes.  It defaults to true, and can be set to false
either in the preamble, for the whole document, or on a per-entry
basis, in the \textsf{options} field.  Several entry types don't use
this field, so the option will have no effect in them, and it is also
ignored in \textsf{article}, \textsf{misc}, \textsf{music},
\textsf{periodical}, and \textsf{review} entries.

\mylittlespace As \mymarginpar{\texttt{notefield=\\true}} with the
previous two options, Roger Hart requested an option to control the
printing of the \textsf{note} field in long notes.  It defaults to
true, and can be set to false either in the preamble, for the whole
document, or on a per-entry basis, in the \textsf{options} field.  The
option will be ignored in \textsf{article}, \textsf{misc},
\textsf{periodical}, and \textsf{review} fields.

\mylittlespace This
\mymarginpar{\vspace{-1\baselineskip}\texttt{completenotes=}%
  \\\texttt{true}} is the one option that rules
the three preceding, either printing all the fields under
consideration --- the default --- or excluding all of them from long
notes.  It is set to \texttt{true} in \textsf{chicago-notes.cbx}, but
you can change it either in the preamble for the whole document or,
for specific fields, in the \textsf{options} field of individual
entries.

\mylittlespace At \colmarginpar{\texttt{booklongxref=\\true}} the
request of Bertold Schweitzer, I have included two new options for
controlling whether and where \textsf{biblatex-chicago} will print
abbreviated references when you cite more than one part of a given
collection or series.  This option controls whether multiple
\textsf{book}, \textsf{bookinbook}, \textsf{collection}, and
\textsf{proceedings} entries which are part of the same collection
will appear in this space-saving format.  The parent collection itself
will usually be presented in, e.g., a \textsf{book},
\textsf{bookinbook}, \textsf{mvbook}, \textsf{mvcollection}, or
\textsf{mvproceedings} entry, and using \textsf{crossref} or
\textsf{xref} in the child entries will allow such presentation
depending on the value of the option:

\begin{description}
\item[\qquad true:] This is the default.  If you use \textsf{crossref}
  or \textsf{xref} fields in these entry types, by default you will
  \emph{not} get any abbreviated references, either in notes or
  bibliography.
\item[\qquad false:] You'll get abbreviated references in these entry
  types both in notes and in the bibliography.
\item[\qquad notes:] The abbreviated references will not appear in
  notes, but only in the bibliography.
\item[\qquad bib:] The abbreviated references will not appear in the
  bibliography, but only in notes.
\end{description}

This option can be set either in the preamble or in the
\textsf{options} field of individual entries.  For controlling the
behavior of \textsf{inbook}, \textsf{incollection},
\textsf{inproceedings}, and \textsf{letter} entries, please see
\mycolor{\texttt{longcrossref}}, below, and also the documentation of
\textsf{crossref} in section~\ref{sec:entryfields}.

\mylittlespace If \colmarginpar{\texttt{hidevolumes=\\true}} both a
\textsf{volume} and a \textsf{volumes} field are present, as may occur
particularly in cross-referenced entries, then
\textsf{biblatex-chicago} will ordinarily suppress the
\textsf{volumes} field.  In some instances, when a \textsf{maintitle}
is present, this may not be the desired result.  In this latter case,
if the \textsf{volume} appears before the \textsf{maintitle}, this new
option, set to \texttt{true} by default, controls whether to print the
\textsf{volumes} field after that title or not.  Set it to
\texttt{false} either in the preamble or in the \textsf{options} field
of your entry to have it appear after the \textsf{maintitle}.

\mylittlespace This \colmarginpar{\texttt{longcrossref=\\false}} is
the second option, requested by Bertold Schweitzer, for controlling
whether and where \textsf{biblatex-chicago} will print abbreviated
references when you cite more than one part of a given collection or
series.  It controls the settings for the entry types more-or-less
authorized by the \emph{Manual}, i.e., \textsf{inbook},
\textsf{incollection}, \textsf{inproceedings}, and \textsf{letter}.
The mechanism itself is enabled by multiple \textsf{crossref} or
\textsf{xref} references to the same parent, whether that be, e.g., a
\textsf{collection}, an \textsf{mvcollection}, a \textsf{proceedings},
or an \textsf{mvproceedings} entry.  Given these multiple cross
references, the presentation in the reference apparatus will be
governed by the following options:

\begin{description}
\item[\qquad false:] This is the default.  If you use
  \textsf{crossref} or \textsf{xref} fields in the four mentioned
  entry types, you'll get the abbreviated references in both notes and
  bibliography.
\item[\qquad true:] You'll get no abbreviated references in these
  entry types, either in notes or in the bibliography.
\item[\qquad notes:] The abbreviated references will not appear in
  notes, but only in the bibliography.
\item[\qquad bib:] The abbreviated references will not appear in the
  bibliography, but only in notes.
\item[\qquad none:] This switch is special, allowing you with one
  setting to provide abbreviated references not just to the four entry
  types mentioned but also to \textsf{book}, \textsf{bookinbook},
  \textsf{collection}, and \textsf{proceedings} entries, both in notes
  and in the bibliography.
\end{description}

This option can be set either in the preamble or in the
\textsf{options} field of individual entries.  For controlling the
behavior of \textsf{book}, \textsf{bookinbook}, \textsf{collection},
and \textsf{proceedings} entries, please see
\mycolor{\texttt{booklongxref}}, above, and also the documentation of
\textsf{crossref} in section~\ref{sec:entryfields}.

\mylittlespace This \mymarginpar{\texttt{usecompiler=\\true}} option
enables automatic use of the name of the compiler (in the
\textsf{namec} field) at the head of an entry, usually in the absence
of an \textsf{author}, \textsf{editor}, or \textsf{translator}, in
accordance with the specification (\emph{Manual} 14.87).  It may also,
like \texttt{useauthor}, \texttt{useeditor}, and
\texttt{usetranslator}, be disabled on a per-entry basis by setting
\texttt{usecompiler=false} in the \textsf{options} field.  Please
remember that, because \textsf{namec} isn't a standard
\textsf{biblatex} field, it may take a little extra effort to get it
to work smoothly.  The package should now automatically take care of
finding a name for short notes, but it will alphabetize by this name
in the bibliography only if you use \textsf{Biber}, failing which
you'll need to provide a \textsf{sortkey} for this purpose.  (These
rules don't apply when you modify the \textsf{editor's} identifying
string using the \textsf{editortype} field, which is the procedure I
recommend if the entry-heading compiler is only a compiler, and not
also, e.g., an editor or a translator.)  Cf.\
\cmd{DeclareSortingScheme} and \cmd{DeclareLabelname} in
section~\ref{sec:formatopts}, above; also, chaucer:alt for an entry
where, because none of the names provided appear at the head of the
reference, you will need to provide a \textsf{sortkey} to stop
\textsf{Biber} using the \textsf{namec} --- because it's not a
standard name field, you can stop it being printed at the head of the
entry, but you can't stop it turning up in the sorting algorithms.

\subsubsection{Style Options -- Preamble}
\label{sec:useropts}

These are parts of the specification that not everyone will wish to
enable.  All except the fourth can be used even if you load the
package in the old way via a call to \textsf{biblatex}, but most users
can just place the appropriate string(s) in the options to the
\cmd{usepackage\{biblatex-chicago\}} call in your preamble.

\mylittlespace At \mymarginpar{\texttt{annotation}} the request of
Emil Salim, I included in \textsf{biblatex-chicago} the ability to
produce annotated bibliographies.  If you turn this option on then the
contents of your \textsf{annotation} (or \textsf{annote}) field will
be printed after the bibliographical reference.  (You can also use
external files to store annotations -- please see
\textsf{biblatex.pdf} �~3.11.8 for details on how to do this.)  This
functionality is currently in a beta state, so before you use it
please have a look at the documentation for the \textsf{annotation}
field, on page~\pageref{sec:annote} above.

\mylittlespace When \colmarginpar{\texttt{compresspages}} set to
\texttt{true}, any page ranges in your .bib file or in the
\textsf{postnote} field of your citation commands will be compressed
in accordance with the \emph{Manual's} specifications (9.60).
Something like 321-{-}328 in your .bib file would become 321--28 in
your document.  See the \textsf{pages} field in
section~\ref{sec:entryfields}, above.

\mylittlespace The \colmarginpar{\texttt{delayvolume}} presentation of
\textsf{volume} information in the notes \&\ bibliography style is
complicated (\emph{Manual}, 14.121--27).  Depending on entry type and
on the presence or absence of a \textsf{booktitle} or a
\textsf{maintitle}, \textsf{volume} data will be presented, in the
bibliography, either before a \textsf{maintitle} or after a
\textsf{booktitle} or \textsf{maintitle}, that is, just before
publication information.  This, so far, is handled for you
automatically by \textsf{biblatex-chicago-notes}.  In long notes, the
same options apply, but it is also sometimes better to place
\textsf{volume} information \emph{after} the publication information
and just before any page numbers, so I have included this option,
which you can set either for the whole document or on a per-entry
basis, to allow you to move \textsf{volume} data to the end of a long
note.  Please note that this doesn't affect any \textsf{volume} data
printed \emph{before} a \textsf{maintitle}, but only data that would,
without this option, be printed \emph{after} a \textsf{booktitle} or
\textsf{maintitle}.  Cf.\ also \mycolor{\cmd{postvolpunct}}, below.

\mylittlespace Although \mymarginpar{\texttt{footmarkoff}} the
\emph{Manual} (14.19) recommends specific formatting for footnote (and
endnote) marks, i.e., superscript in the text and in-line in foot- or
endnotes, Charles Schaum has brought it to my attention that not all
publishers follow this practice, even when requiring Chicago style.  I
have retained this formatting as the default setup, but if you include
the \texttt{footmarkoff} option, \textsf{biblatex-chicago-notes} will
not alter \LaTeX 's (or the \textsf{endnote} package's) defaults in
any way, leaving you free to follow the specifications of your
publisher.  I have placed all of this code in
\textsf{biblatex-chicago.sty}, so if you load the package with a call
to \textsf{biblatex} instead, then once again footnote marks will
revert to the \LaTeX\ default, but of course you also lose a fair
amount of other formatting, as well.  See section~\ref{sec:loading},
below.

\mylittlespace Setting \colmarginpar{\texttt{inheritshort\-hand}} this
option to \texttt{true} allows child entries to inherit the
\textsf{shorthand} and \textsf{shorthandintro} fields from
cross-referenced parent entries.  This in turn allows abbreviated
references to the parent entry to use the \textsf{shorthand} instead
of the usual and merely short citation, thus allowing for extra space
savings.  There are several other steps required to make this all
function smoothly, so please see the documentation of the
\textbf{shorthand} field in section~\ref{sec:entryfields}, above.

\mylittlespace The \mymarginpar{\texttt{juniorcomma}} \emph{Manual}
(6.47) states that \enquote{commas are not required around \emph{Jr.}\
  and \emph{Sr.},} so by default \textsf{biblatex-chicago} has
followed standard \textsf{biblatex} in using a simple space in names
like \enquote{John Doe Jr.}  Charles Schaum has pointed out that
traditional \textsc{Bib}\TeX\ practice was to include the comma, and
since the \emph{Manual} has no objections to this, I have provided an
option which allows you to turn this behavior back on, either for the
whole document or on a per-entry basis.  Please note, first, that
numerical suffixes (John Doe III) never take the comma.  The code
tests for this situation, and detects cardinal numbers well, but if
you are using ordinals you may need to set this to \texttt{false} in
the \textsf{options} field of some entries.  Second, I have fixed a
bug in older releases which always printed the \enquote{Jr.}\ part of
the name immediately after the surname, even when the surname came
before the given names (as in a bibliography).  The package now
correctly puts the \enquote{Jr.}\ part at the end, after the given
names, and in this position it always takes a comma, the presence of
which is unaffected by this option.

\mylittlespace This \mymarginpar{\texttt{natbib}} may look like the
standard \textsf{biblatex} option, but to keep the coding of
\textsf{biblatex-chicago.sty} simpler for the moment I have
reimplemented it there, from whence it is merely passed on to
\textsf{biblatex}.  If you load the Chicago style with
\cmd{usepackage\{biblatex-chicago\}}, then the option should simply
read \texttt{natbib}, rather than \texttt{natbib=true}.  The shorter
form also works if you load the style using
\cmd{usepackage[style=chicago-notes]\{biblatex\}}, so I hope this
requirement isn't too onerous.

\mylittlespace At \mymarginpar{\texttt{noibid}} the request of an
early tester, I have included this option to allow you globally to
turn off the \texttt{ibidem} mechanism that
\textsf{biblatex-chicago-notes} uses by default.  Some publishers, it
would appear, require this.  Setting this option will mean that all
possible instances of \emph{ibid.}\ will be replaced by the short note
form.  For more fine-grained control of individual citations you'll
probably want to use specialized citation commands, instead.  See
section \ref{sec:citecommands}.

\mylittlespace As \colmarginpar{\texttt{omitxrefdate}} part of the new
abbreviated cross-referencing functionality for \textsf{book},
\textsf{bookinbook}, \textsf{collection}, and \textsf{proceedings}
entries, I have thought it helpful to include, in the abbreviated
references only, a date for any \textsf{title} that's part of a
\textsf{maintitle}, though not for those that are only part of
\textsf{booktitle}.  If these dates annoy you, you can use this option
to turn them off, either in the preamble for the document as a whole
or in the \textsf{options} field of individual entries.  Cf.\
harley:ancient:cart, harley:cartography, and harley:hoc; and
\textsf{crossref} in section~\ref{sec:entryfields}, above.

\mylittlespace Several
\colmarginpar{\texttt{postnotepunct}\\(experimental)} users, most
recently David Gohlke, have requested a way to alter the punctuation
that appears just before the \textsf{postnote} argument of citation
commands, usually, but perhaps not always, to allow citations to fit
better into the flow of text.  This punctuation is a complex issue in
the \emph{Manual}, and I've attempted to make
\textsf{biblatex-chicago} follow the specifications closely.  Still,
as a first stab at enabling the greater flexibility in punctuation
that some have requested, I have introduced the
\mycolor{\texttt{postnotepunct}} package option.  Set to
\texttt{true}, it allows you to start the \textsf{postnote} field with
a punctuation mark (.\,,\,;\,:) and have it appear as the
\cmd{postnotedelim} in place of whatever the package might otherwise
automatically have chosen.  Please note that this functionality relies
on a very nifty macro by Philipp Lehman which I haven't extensively
tested, so I'm labeling this option \mycolor{experimental}.  Note
also that the option only affects the \textsf{postnote} field of
citation commands, not the \textsf{pages} field in your .bib file.

% %\enlargethispage{\baselineskip}

\mylittlespace This \mymarginpar{\texttt{short}} option means that
your text will only use the short note form, even in the first
citation of a particular work.  The \emph{Manual} (14.14) recommends
this space-saving format only when you provide a \emph{full}
bibliography, though even with such a bibliography you may feel it
easier for your readers to present long first citations.  If you do
use the \texttt{short} option, remember that there are several
citation commands which allow you to present the full reference in
specific cases (see section \ref{sec:citecommands}).  If your
bibliography is not complete, then you should not use this option.

\mylittlespace Kenneth Pearce \mymarginpar{\texttt{shorthandfull}} has
suggested that, in some fields of study, a list of shorthands
providing full bibliographical information may replace the
bibliography itself.  This option prints this full information in the
list of shorthands, though of course you should remember that any .bib
entry not containing a \textsf{shorthand} field won't appear in such a
list.  Please see the documentation of the \textsf{shorthand} field in
section~\ref{sec:entryfields} above for information on further options
available to you for presenting and formatting the list of shorthands.

\mylittlespace Chris Sparks \mymarginpar{\texttt{shorthandibid}}
pointed out that \textsf{biblatex-chicago-notes} would never use
\emph{ibid.}\ in the case of entries containing a \textsf{shorthand}
field, but rather that consecutive references to such an entry
continued to provide the shorthand, instead.  The \emph{Manual} isn't,
as far as I can tell, completely clear on this question.  In 14.258,
discussing references to works from classical antiquity, it states
that \enquote{when abbreviations are used, these rather than
  \emph{ibid.}\ should be used in succeeding references to the same
  work,} but I can't make out whether this rule is specific to
classical references or has more general scope.  Given this ambiguity,
I don't think it unreasonable to provide an option to allow printing
of \emph{ibid.}\ instead of the shorthand in such circumstances,
though the default behavior remains the same as it always has.

\mylittlespace This \mymarginpar{\texttt{strict}} still-experimental
option attempts to follow the \emph{Manual}'s recommendations (14.36)
for formatting footnotes on the page, using no rule between them and
the main text unless there is a run-on note, in which case a short
rule intervenes to emphasize this continuation.  I haven't tested this
code very thoroughly, and it's possible that frequent use of floats
might interfere with it.  Let me know if it causes problems.

\subsection{General Usage Hints}
\label{sec:hints}

\subsubsection{Loading the Style}
\label{sec:loading}

With the addition of the author-date styles to the package, I have
provided three keys for choosing which style to load, \texttt{notes},
\texttt{authordate}, and \textsf{authordate-trad}, one of which you
put in the options to the \cmd{usepackage} command.  The default way
of loading the notes + bibliography style has therefore slightly
changed.  With early versions of \textsf{biblatex-chicago-notes}, the
standard way of loading the package was via a call to
\textsf{biblatex}, e.g.:
\begin{quote}
  \cmd{usepackage[style=chicago-notes,strict,backend=bibtex8,\%\\
    babel=other,bibencoding=inputenc]\{biblatex\}}
\end{quote}
Now, the default way to load the style, and one that will in the
vast majority of standard cases produce the same results as the old
invocation, will look like this:
\begin{quote}
  \cmd{usepackage[notes,strict,backend=biber,autolang=other,\%\\
    bibencoding=inputenc]\{biblatex-chicago\}}
\end{quote}

(In point of fact, the previous \textsf{biblatex-chicago} loading
method without the \texttt{notes} option will still work, but only
because I've made the notes \&\ bibliography style the default if no
style is explicitly requested.)  If you read through
\textsf{biblatex-chicago.sty}, you'll see that it sets a number of
\textsf{biblatex} options aimed at following the Chicago
specification, as well as setting a few formatting variables intended
as reasonable defaults (see section~\ref{sec:presetopts}, above).
Some parts of this specification, however, are plainly more
\enquote{suggested} than \enquote{required,} and indeed many
publishers, while adopting the main skeleton of the Chicago style in
citations, nonetheless maintain their own house styles to which the
defaults I have provided do not conform.

\mylittlespace If you only need to change one or two parameters, this
can easily be done by putting different options in the call to
\textsf{biblatex-chicago} or redefining other formatting variables in
the preamble, thereby overriding the package defaults.  If, however,
you wish more substantially to alter the output of the package,
perhaps to use it as a base for constructing another style altogether,
then you may want to revert to the old style of invocation above.
You'll lose all the definitions in \textsf{biblatex-chicago.sty},
including those to which I've already alluded and also the code that
sets the note number in-line rather than superscript in endnotes or
footnotes.  Also in this file is the code that calls
\textsf{cms-american.lbx}, which means that you'll lose all the
Chicago-specific bibstrings I've defined unless you provide, in your
preamble, a \cmd{DeclareLanguageMapping} command adapted for your
setup, on which see section~\ref{sec:international} below and also
��~4.9.1 and 4.11.8 in Lehman's \textsf{biblatex.pdf}.

%%\enlargethispage{-\baselineskip}

\mylittlespace What you \emph{will not} lose is the ability to call
the package options \texttt{annotation, strict, short,} and
\texttt{noibid} (section~\ref{sec:useropts}, above), in case these
continue to be useful to you when constructing your own modifications.
There's very little code, therefore, actually in
\textsf{biblatex-chicago.sty}, but I hope that even this minimal
separation will make the package somewhat more adaptable.  Any
suggestions on this score are, of course, welcome.

\subsubsection{Other Hints}
\label{sec:otherhints}

One useful rule, when you are having difficulty creating a .bib entry,
is to ask yourself whether all the information you are providing is
strictly necessary.  The Chicago specification is a very full one, but
the \emph{Manual} is actually, in many circumstances, fairly relaxed
about how much of the data from a work's title page you need to fit
into a reference.  Authors of introductions and afterwords, multiple
publishers in different countries, the real names of authors more
commonly known under pseudonyms, all of these are candidates for
exclusion if you aren't making specific reference to them, and if you
judge that their inclusion won't be of particular interest to your
readers.  Of course, any data that may be of such interest, and
especially any needed to identify and track down a reference, has to
be present, but sometimes it pays to step back and reevaluate how much
information you're providing.  I've tried to make
\textsf{biblatex-chicago-notes} robust enough to handle the most
complex, data-rich citations, but there may be instances where you can
save yourself some typing by keeping it simple.

\mylittlespace Scot Becker has pointed out to me that the inverse
problem not only exists but may well become increasingly common, to
wit, .bib database entries generated by bibliographic managers which
helpfully provide as much information as is available, including
fields that users may well wish not to have printed (ISBN, URL, DOI,
\textsf{pagetotal}, inter alia).  The standard \textsf{biblatex}
styles contain a series of options, detailed in \textsf{biblatex.pdf}
�3.1.2.2, for controlling the printing of some of these fields, and
with this release I have implemented the ones that are relevant to
\textsf{biblatex-chicago}, along with a couple that Scot requested and
that may be of more general usefulness.  There is also a general
option to excise with one command all the fields under consideration
-- please see section~\ref{sec:chicpreset} above.

\mylittlespace If you are having problems with the interaction of
punctuation and quotation marks in notes or bibliography, first please
check that you've used \cmd{mkbibquote} in the relevant part of your
.bib file.  If you are still getting errors, please let me know, as it
may well be a bug.

\mylittlespace For the \textsf{biblatex-chicago-notes} style, I have
fully adopted \textsf{biblatex's} system for providing punctuation at
the end of entries.  Several users noted insufficiencies in previous
releases of \textsf{biblatex-chicago}, sometimes related to the
semicolon between multiple citations, sometimes to ineradicable
periods after long notes, bugs that were byproducts of my attempt to
fix other end-of-entry errors.  One of the side effects of this older
code was (wrongly) to put a period after a long note produced, e.g.,
by a command like \cmd{footnote\{\textbackslash headlessfullcite\}},
whereas only the \enquote{foot} cite commands (including
\cmd{autocite} in the default \textsf{biblatex-chicago-notes} set up)
should do so.  If you came to rely on this side effect, please note
now that you'll have to put the period in yourself when explicitly
calling \cmd{footnote}, like so: \cmd{footnote\{\textbackslash
  headlessfullcite\{key\}.\}}

\mylittlespace When you use abbreviations at the ends of fields in
your .bib file (e.g., \enquote{\texttt{n.d.}} or
\enquote{\texttt{Inc.},}) \textsf{biblatex-chicago-notes} should deal
automatically with adding (or suppressing) appropriate punctuation
after the final dot.  This includes retaining periods after such dots
when a closing parenthesis intervenes, as in (n.d.).  Merely entering
the abbreviation without informing \textsf{biblatex} that the final
dot is a dot and not a period should always work, though you do have
to provide manual formatting in those rare cases when you need a comma
after the author's initials in a bibliography, usually in a
\textsf{misc} entry (see house:papers).  If you find you need to
provide such formatting elsewhere, please let me know.

\mylittlespace Finally, allow me to reiterate what Philipp Lehman says
in \textsf{biblatex.pdf}, to wit, use \textsf{Biber} if you can.  It's
not absolutely required for the notes \&\ bibliography style, but it is
required for an increasing amount of very useful functionality in all
\textsf{biblatex} styles.  In this release, for example, the new
\mycolor{\textbf{mv*}} entry types can help streamline your .bib
database, particularly when used as \textsf{crossref} targets, but
this utility is severely limited if you are using one of the older
backends.  If you have to use one of these older backends, then Lehman
advises using \textsf{bibtex8}, rather than standard \textsc{Bib}\TeX,
to avoid the cryptic errors that ensue when your .bib file gets to a
certain size.

\section{The Specification: Author-Date}
\label{sec:authdate}

The \textsf{biblatex-chicago} package now contains \emph{two}
different author-date styles.  The first,
\textsf{biblatex-chicago-authordate}, implements the specifications of
the 16th edition of the \emph{Chicago Manual of Style}.  Numbers in
parentheses refer to sections of the \emph{Manual}, though as this
latest edition now recommends \enquote{a uniform treatment for the
  main elements of citation in both of its systems of documentation}
(15.2), many of these references will in fact be to the chapter on the
notes \&\ bibliography style (chap.\ 14), which chapter is, by design,
considerably more detailed than that devoted to the author-date style.
The second, \textsf{biblatex-chicago-authordate-trad}, implements that
same specification but with a markedly different style of title
presentation, including sentence-style capitalization and an absence
of quotation marks around the (plain-text) titles of \textsf{article}
or \textsf{incollection} entries, \emph{inter alia}.  The
\textsf{trad} style is so named because older versions of the
\emph{Manual}, up to and including the 15th edition, recommended this
plainer style for author-date titles, and the 16th edition itself
suggests the possibility, when needed, of retaining such title
presentation in combination with its own recommendations for other
parts of the reference apparatus (15.45).  In practice, the
differences between the two styles necessitate separate discussions of
the \textsf{title} field and one extra package option
(\texttt{headline}), and that's about it.

\mylittlespace Generally, then, the following documentation covers
both Chicago author-date styles, and attempts to explain all the parts
of the specification that might be considered somehow \enquote{non
  standard,} at least with respect to the styles included with
\textsf{biblatex} itself.  In the section on entry fields I admit I
have also duplicated a lot of the information in
\textsf{biblatex.pdf}, which I hope won't badly annoy expert users of
the system.  As usual, headings in \mycolor{green}
\colmarginpar{\textsf{New in this release}} indicate material new to
this release, or occasionally old material that has undergone
significant revision.  The file \textsf{dates-test.bib} contains many
examples from the \emph{Manual} which, when processed using
\textsf{biblatex-chicago-authordate}, should produce the same output
as you see in the \emph{Manual} itself, or at least compliant output,
where the specifications are vague or open to interpretation, a state
of affairs which does sometimes occur.  If you are using
\textsf{biblatex-chicago-authordate-trad} the same basically holds,
but you'd have to keep one eye on the 15th edition of the
\emph{Manual} (chap.\ 17) for the titles.  I have provided
\textsf{cms-dates-sample.pdf} and \textsf{cms-trad-sample.pdf}, which
show how my system processes \textsf{dates-test.bib}, and I have also
included the reference keys from the latter file below in parentheses.

\subsection{Entry Types}
\label{sec:types:authdate}

The complete list of entry types currently available in
\textsf{biblatex-chicago-authordate} and \textsf{authordate-trad},
minus the odd \textsf{biblatex} alias, is as follows:
\textbf{article}, \textbf{artwork}, \textbf{audio},
\mycolor{\textbf{book}}, \mycolor{\textbf{bookinbook}},
\textbf{booklet}, \mycolor{\textbf{collection}}, \textbf{customc},
\textbf{image}, \textbf{inbook}, \textbf{incollection},
\textbf{inproceedings}, \textbf{inreference}, \textbf{letter},
\textbf{manual}, \textbf{misc}, \textbf{music},
\mycolor{\textbf{mvbook}}, \mycolor{\textbf{mvcollection}},
\mycolor{\textbf{mvproceedings}}, \mycolor{\textbf{mvreference}},
\textbf{online} (with its alias \textbf{www}), \textbf{patent},
\textbf{periodical}, \mycolor{\textbf{proceedings}},
\textbf{reference}, \textbf{report} (with its alias
\textbf{techreport}), \textbf{review}, \textbf{suppbook},
\textbf{suppcollection}, \textbf{suppperiodical}, \textbf{thesis}
(with its aliases \textbf{mastersthesis} and \textbf{phdthesis}),
\textbf{unpublished}, and \textbf{video}.

\mylittlespace What follows is an attempt to specify all the
differences between these types and the standard provided by
\textsf{biblatex}.  If an entry type isn't discussed here, then it is
safe to assume that it works as it does in the standard styles.  In
general, I have attempted not to discuss specific entry fields here,
unless such a field is crucial to the overall operation of a given
entry type.  As a general and important rule, most entry types require
very few fields when you use \textsf{biblatex-chicago-authordate}, so
it seemed to me better to gather information pertaining to fields in
the next section.

\mybigspace The \mymarginpar{\textbf{article}} \emph{Chicago Manual of
  Style} (14.170) recognizes three different sorts of periodical
publication, \enquote{journals,} \enquote{magazines,} and
\enquote{newspapers.}  The first (14.172) includes \enquote{scholarly
  or professional periodicals available mainly by subscription,} while
the second refers to \enquote{weekly or monthly} publications that are
\enquote{available either by subscription or in individual issues at
  bookstores or newsstands.}  \enquote{Magazines} will tend to be
\enquote{more accessible to general readers,} and typically won't have
a volume number.  The following paragraphs detail how to construct
your .bib entries for all these sorts of periodical publication.

\mylittlespace For articles in \enquote{journals} you can simply use
the traditional \textsc{Bib}\TeX\ --- and indeed \textsf{biblatex} ---
\textsf{article} entry type, which will work as expected and set off
the page numbers with a colon in the list of references, as required
by the \emph{Manual}.  If, however, you wish to cite a
\enquote{magazine} or a \enquote{newspaper}, then you need to add an
\textsf{entrysubtype} field containing the exact string
\texttt{magazine}.  The main formatting differences between a
\texttt{magazine} (which includes both \enquote{magazines} and
\enquote{newspapers}) and a plain \textsf{article} are that time
specifications (month, day, season) aren't placed within parentheses,
and that page numbers are set off by a comma rather than a colon.
Otherwise, the two sorts of reference have much in common.  (For
\textsf{article}, see \emph{Manual} 14.175--198, 15.9, 15.43--46;
batson, beattie:crime, chu:panda, connell:chronic, conway:evolution,
friedman:learning, garaud:gatine, garrett, hlatky:hrt, kern, lewis,
loften:hamlet, loomis:structure, rozner:liberation,
schneider:mittelpleistozaene, terborgh:pre\-ser\-vation,
wall:ra\-di\-o, warr:ellison, white:callimachus. With
\textsf{entrysubtype} \texttt{maga\-zine}, cf.\ 14.181, 14.199--202,
15.47; assocpress:gun, lakeforester:pushcarts, mor\-genson:\-mar\-ket,
reaves:ro\-sen, stenger:privacy.)

\mylittlespace The \emph{Manual} now suggests that, no matter which
citation style you are using, it is \enquote{usually sufficient to
  cite newspaper and magazine articles entirely within the text}
(15.47).  This involves giving the title of the journal and the full
date of publication in a parenthetical reference, including any other
information in the main text (14.206), thereby obviating the need to
present such an entry in the list of references.  To utilize this
method in the author-date styles, in addition to a \texttt{magazine}
\textsf{entrysubtype}, you'll need to place \texttt{cmsdate=full} into
the \textsf{options} field, including \texttt{skipbib} there as well
to stop the entry printing in the list of references.  If the entry
only contains a \textsf{date} and \textsf{journaltitle} that's enough,
but if it's a fuller entry also containing an \textsf{author} then
you'll also need \texttt{useauthor=false} in the \textsf{options}
field.  Other surplus fields will be ignored.  (See osborne:poison.)

\mylittlespace If you are familiar with the notes \&\ bibliography
style, you'll know that the \emph{Manual} treats reviews (of books,
plays, performances, etc.) as a sort of recognizable subset of
\enquote{journals,} \enquote{magazines,} and \enquote{newspapers,}
distinguished mainly by the way one formats the title of the review
itself.  With the 16th edition's changes to the way titles are
presented in the \textsf{authordate} style, users need to learn how to
present this sort of material, which involves using an entry type
(\textsf{review}) that wasn't necessary in the 15th edition.  The key
rule is this: if a review has a separate, non-generic title (gibbard;
osborne:poison) in addition to something that reads like
\enquote{review of \ldots,} then you need an \textsf{article} entry,
with or without the \texttt{magazine} \textsf{entrysubtype}, depending
on the sort of publication containing the review.  If the only title
is the generic \enquote{review of \ldots,} for example, then you'll
need the \textsf{review} entry type, with or without this same
\textsf{entrysubtype} toggle using \texttt{magazine}.  On
\textsf{review} entries, see below.  (The curious reader will no doubt
notice that the code for formatting any sort of review still exists in
\textsf{article}, as it was initially designed for \textsf{biblatex
  0.6}, but this new arrangement is somewhat simpler and therefore, I
hope, better.)

\mylittlespace In the case of a review with a specific as well as a
generic title, the former goes in the \textsf{title} field, and the
latter in the \textsf{titleaddon} field.  Standard \textsf{biblatex}
intends this field for use with additions to titles that may need to
be formatted differently from the titles themselves, and
\textsf{biblatex-chicago-authordate} uses it in just this way, with
the additional wrinkle that it can, if needed, replace the
\textsf{title} entirely, and this in, effectively, any entry type,
providing a fairly powerful, if somewhat complicated, tool for getting
\textsc{Bib}\TeX\ to do what you want.  Here, however, if all you need
is a generic title like \enquote{review of \ldots,} then you want to
switch to the \textsf{review} type, where you can simply use the
\textsf{title} field for it.

%\enlargethispage{\baselineskip}

\mylittlespace No less than eight more things need explication under
this heading.  First, since the \emph{Manual} specifies that what goes
into the \textsf{titleaddon} field of \textsf{article} entries stays
unformatted --- no italics, no quotation marks --- this plain style is
the default for such text, which means that you'll have to format any
titles within \textsf{titleaddon} yourself, e.g., with
\cmd{mkbibemph\{\}}.  Second, the \emph{Manual} specifies a similar
plain style for the titles of other sorts of material found in
\enquote{magazines} and \enquote{newspapers,} e.g., obituaries,
letters to the editor, interviews, the names of regular columns, and
the like.  References may contain both the title of an individual
article and the name of the regular column, in which case the former
should go, as usual, in a \textsf{title} field, and the latter in
\textsf{titleaddon}.  As with reviews proper, if there is only the
generic title, then you want the \textsf{review} entry type.  (See
14.203, 14.205, 14.208; morgenson:market, reaves:rosen.)

\mylittlespace Third, the 16th edition of the \emph{Manual} suggests
that \enquote{unsigned newspaper articles or features are best dealt
  with in text \ldots} (14.207).  As with newspaper or magazine
articles in general, you can place \texttt{cmsdate=full} and
\texttt{skipbib} into the \textsf{options} field to produce an
augmented in-text citation whilst keeping this material out of the
reference list.  If you do use the reference list, then the standard
shorter citation will be sufficient, and in both cases the name of the
periodical (in the \textsf{journaltitle} field) will be used in place
of the missing author.  Just to clarify: in \textsf{article} or
\textsf{review} entries, \textsf{entrysubtype} \texttt{magazine}, a
missing \textsf{author} field results in the name of the periodical
(in the \textsf{journaltitle} field) being used as the missing author.
Without an \textsf{entrysubtype}, and assuming that no name whatsoever
can be found to put at the head of the entry, the \textsf{title} will
be used, not the \textsf{journaltitle}, or so I interpret the
\emph{Manual} (14.175).  The new default sorting scheme in
\textsf{biblatex-chicago-authordate} considers the
\textsf{journaltitle} before the \textsf{title}, so if the latter
heads an entry you'll need a \textsf{sortkey}, just as you will if you
retain the definite or indefinite article at the beginning of the
\textsf{journaltitle} in author-less entries with an
\textsf{entrysubtype}.  If you want to abbreviate the
\textsf{journaltitle} for use in citations, but give the full name in
the list of references, then the \textsf{shortauthor} field, somewhat
surprisingly, is the place for it.  A shortened \textsf{title} should
go, as usual, in \textsf{shorttitle}.  (See
section~\ref{sec:authformopts}, below; lakeforester:pushcarts,
nyt:trevorobit, unsigned:ranke.)

\mylittlespace Fourth, Bertold Schweitzer has pointed out, following
the \emph{Manual} (14.192), that while an \textsf{issuetitle} often
has an \textsf{editor}, it is not too unusual for a \textsf{title} to
have, e.g., an \textsf{editor} and/or a \textsf{translator}.  In order
to allow as many permutations as possible on this theme, I have
brought the \textsf{article} entry type into line with most of the
other types in allowing the use of the \textsf{namea} and
\textsf{nameb} fields in order to associate an editor or a translator
specifically with the \textsf{title}.  The \textsf{editor} and
\textsf{translator} fields, in strict homology with other entry types,
are associated with the \textsf{issuetitle} if one is present, and
with the \textsf{title} otherwise.  The usual string concatenation
rules still apply --- cf.\ \textsf{editor} and \textsf{editortype} in
section~\ref{sec:fields:authdate}, below.

\enlargethispage{\baselineskip}

\mylittlespace Fifth, in certain fields, just beginning your data with
a lowercase letter activates the mechanism for capitalizing that
letter depending on its context within a list of references entry.
This is less important in the author-date styles, where this
information only turns up in the reference list and not in citations,
but you can consult \textbf{\textbackslash autocap} below for all the
details.  Both the \textsf{titleaddon} and \textsf{note} fields are
among those treating their data this way, and since both appear
regularly in \textsf{article} entries, I thought the problem merited a
preliminary mention here.

\mylittlespace Sixth, if you need to cite an entire issue of any sort
of periodical, rather than one article in an issue, then the
\textsf{periodical} entry type, once again with or without the
\texttt{magazine} toggle in \textsf{entrysubtype,} is what you'll
need.  (You can also use the \textsf{article} type, placing what would
normally be the \textsf{issuetitle} in the \textsf{title} field and
retaining the usual \textsf{journaltitle} field, but this arrangement
isn't compatible with standard \textsf{biblatex}.)  The \textsf{note}
field is where you place something like \enquote{special issue} (with
the small \enquote{s} enabling the automatic capitalization routines),
whether you are citing one article or the whole issue
(conley:fifthgrade, good:wholeissue).  Indeed, this is a somewhat
specialized use of \textsf{note}, and if you have other sorts of
information you need to include in an \textsf{article} or
\textsf{periodical} entry, then you shouldn't put it in the
\textsf{note} field, but rather in \textsf{titleaddon} or perhaps
\textsf{addendum} (brown:bremer).

\mylittlespace Seventh, I would suggest that if you wish to cite a
television or radio broadcast, the \textsf{article} type,
\textsf{entrysubtype} \texttt{magazine} is the place for it.  The name
of the program would go in \textsf{journaltitle}, with the name of the
episode in \textsf{title}.  The network's name goes into the
\textsf{usera} field.  (8.185, 14.221; see bundy:macneil for an
example of how this all might look in a .bib file.  Commercial
recordings of such material would need one of the audiovisual entry
types, probably \textsf{audio} or \textsf{video} [friends:leia], while
recordings from archives fit best into \textsf{misc} entries with an
\textsf{entrysubtype} [coolidge:speech, roosevelt:speech].)

\mylittlespace Finally, the 16th edition of the \emph{Manual}
(14.243--6) specifies that blogs and other, similar online material
should be presented like \textsf{articles}, with \texttt{magazine}
\textsf{entrysubtype} (ellis:blog).  The title of the specific entry
goes in \textsf{title}, the general title of the blog goes in
\textsf{journaltitle}, and the word \enquote{\texttt{blog}} in the
\textsf{location} field (though you could just use special formatting
in the \textsf{journaltitle} field itself, which may sometimes be
necessary).  Comments on blogs, with generic titles like
\enquote{comment on} or \enquote{reply to,} need a \textsf{review}
entry with the same \textsf{entrysubtype}.  Such comments make
particular use of the \textsf{eventdate} and of the \textsf{nameaddon}
fields; please see the documentation of \textbf{review}, below.

% %\enlargethispage{-\baselineskip}

\mylittlespace If you're still with me, allow me to recommend that you
browse through \textsf{dates-test.bib} to get a feel for just how many
of the \emph{Manual}'s complexities the \textsf{article},
\textsf{periodical}, and \textsf{review} types attempt to address.  It
may be that in future releases of \textsf{biblatex-chicago} I'll be
able to simplify these procedures somewhat, but with any luck the vast
majority of sources won't require knowledge of these onerous details.

\mybigspace Arne \mymarginpar{\textbf{artwork}} Kjell Vikhagen has
pointed out to me that none of the standard entry types were
straightforwardly adaptable when referring to visual artworks.  The
\emph{Manual} doesn't give any thorough specifications for such
references, and indeed it's unclear that it believes it necessary to
include them in the reference apparatus at all.  Still, it's easy to
conceive of contexts in which a list of artworks studied might be
desirable, and \textsf{biblatex} includes entry types for just this
purpose, though the standard styles leave them undefined.  The two I
chose to include in previous releases were \textsf{artwork} and
\textsf{image}, the former intended for paintings, sculptures,
etchings, and the like, the latter for photographs.  The 16th edition
of the \emph{Manual} has modified its specifications for presenting
photographs so that they are the same as for works in all other media.
The \textsf{image} type, therefore, is now merely a clone of the
\textsf{artwork} type, maintained mainly to provide backward
compatibility for users migrating from the old specification to the
current one.

\mylittlespace As one might expect, the artist goes in \textsf{author}
and the name of the work in \textsf{title}.  The \textsf{type} field
is intended for the medium --- e.g., oil on canvas, charcoal on paper
--- and the \textsf{version} field might contain the state of an
etching.  You can place the dimensions of the work in \textsf{note},
and the current location in \textsf{organization},
\textsf{institution}, and/or \textsf{location}, in ascending order of
generality.  The \textsf{type} field, as in several other entry types,
uses \textsf{biblatex's} automatic capitalization routines, so if the
first word only needs a capital letter at the beginning of a sentence,
use lowercase in the .bib file and let \textsf{biblatex} handle it for
you.  (See \emph{Manual} 3.22, 8.193; leo:madonna, bedford:photo.)

\mylittlespace As a final complication, the \emph{Manual} (8.193) says
that \enquote{the names of works of antiquity \ldots\,are usually set
  in roman.}  If you should need to include such a work in the
reference apparatus, you can either define an \textsf{entrysubtype}
for an \textsf{artwork} entry --- anything will do --- or you could
use the \textsf{image} type, or you could try the \textsf{misc} entry
type with an \textsf{entrysubtype}.  Fortunately, in this instance the
other fields in a \textsf{misc} entry function pretty much as in
\textsf{artwork} or \textsf{image}.

\mybigspace Following \mymarginpar{\textbf{audio}} the request of
Johan Nordstrom, I have included three entry types, all undefined by
the standard styles, designed to allow users to present audiovisual
sources in accordance with the Chicago specifications.  The
\emph{Manual's} presentation of such sources (14.263--273, 15.53),
though admirably brief, seems to me somewhat inconsistent; the
proliferation of online sources has made the task yet more complex.
For the 15th edition I attempted to condense all the requirements into
two new entry types, but ended up relying on three.  For the 16th
edition, in particular, I also need to include the \textbf{online} and
even the \textbf{misc} entry types, which see, under the audiovisual
rubric.  I shall attempt to delineate the main differences here, and
though there are likely to be occasions when your choice of entry type
is not obvious, at the very least \textsf{biblatex-chicago} should
help you maintain consistency.

\mylittlespace For users of the author-date styles, the 16th edition
of the \emph{Manual} (15.53) \enquote{recommends a more comprehensive
  approach to dating audiovisual materials than in previous editions,}
meaning that nearly all such entries will have some sort of dating
information and will therefore fit better stylistically with other
references.  In particular, \enquote{the date of the original
  recording should be privileged in the citation.}  Guidance for
supplying dates for this class of material will be found below under
the different entry types in use, though it will also be worthwhile to
look at the documentation of \textsf{date}, \textsf{eventdate},
\textsf{origdate}, and \textsf{urldate}, in
section~\ref{sec:fields:authdate}, below.  The \emph{Manual} continues
to suggest, also, that \enquote{it is often more appropriate to list
  such materials in running text and group them in a separate section
  or discography}.

\mylittlespace The \textbf{music} type is intended for all musical
recordings that do not have a video component.  This means, for
example, digital media (whether on CD or hard drive), vinyl records,
and tapes.  The \textbf{video} type includes most visual media,
whether it be films, TV shows, tapes and DVDs of the preceding or of
any sort of performance (including music), or online multimedia.  The
\emph{Manual's} treatment (14.280) of the latter suggests that online
video excerpts, short pieces, and interviews should generally use the
\textbf{online} type (harwood:biden, horowitz:youtube, pollan:plant).
The \textbf{audio} type, our current concern, fills gaps in the
others, and presents its sources in a more \enquote{book-like} manner.
Published musical scores need this type --- unpublished ones would use
\textsf{misc} with an \textsf{entrysubtype} (shapey:partita) --- as do
such favorite educational formats as the slideshow and the filmstrip
(greek:filmstrip, schubert:muellerin, verdi:corsaro).  The
\emph{Manual} (14.277--280) sometimes uses a similar format for audio
books (twain:audio), though, depending on the sorts of publication
facts you wish to present, this sort of material may fall under
\textsf{music} (auden:reading).  Dated audio recordings that are part
of an archive, online or no, may best be presented in a \textbf{misc}
entry with an \textsf{entrysubtype} (coolidge:speech,
roosevelt:speech).

\mylittlespace Once you've accepted the analogy of composer to
\textsf{author}, constructing an \textsf{audio} entry should be fairly
straightforward, since many of the fields function just as they do in
\textsf{book} or \textsf{inbook} entries.  Indeed, please note that I
compare it to both these other types as, in common with the other
audiovisual types, \textsf{audio} has to do double duty as an analogue
for both books and collections, so while there will normally be an
\textsf{author}, a \textsf{title}, a \textsf{publisher}, a
\textsf{date}, and a \textsf{location}, there may also be a
\textsf{booktitle} and/or a \textsf{maintitle} --- see
schubert:muellerin for an entry that uses all three in citing one song
from a cycle.  If the medium in question needs specifying, the
\textsf{type} field is the place for it.  Finally, the
\textsf{titleaddon} field can specify functions for which
\textsf{biblatex-chicago} provides no automated handling, e.g., a
librettist (verdi:corsaro).

\mybigspace This \colmarginpar{\textbf{book}} is the standard
\textsf{biblatex} and \textsc{Bib}\TeX\ entry type, but with this
release the package can now provide automatically abbreviated
references in the reference list when you use a \textsf{crossref} or
an \textsf{xref} field.  The functionality is not enabled by default,
but you can enable it in the preamble or in the \textsf{options} field
using the new \mycolor{\texttt{booklongxref}} option.  Please see
\textbf{crossref} in section~\ref{sec:fields:authdate} and
\texttt{booklongxref} in section~\ref{sec:authpreset}, below.  Also,
cf.\ harley:ancient:cart, harley:cartography, and harley:hoc for how
this might look.

\mybigspace This \colmarginpar{\textbf{bookinbook}} type provides the
means of referring to parts of books that are considered, in other
contexts, themselves to be books, rather than chapters, essays, or
articles.  Such an entry can have a \textsf{title} and a
\textsf{maintitle}, but it can also contain a \textsf{booktitle}, all
three of which will be italicized in the reference matter.  In general
usage it is, therefore, rather like the traditional \textsf{inbook}
type, only with its \textsf{title} in italics rather than in quotation
marks.  As with the \textsf{book} type, you can now enable
automatically abbreviated references in the reference list, though
this isn't the default.  Please see \textbf{crossref} in
section~\ref{sec:fields:authdate} and \mycolor{\texttt{booklongxref}}
in section~\ref{sec:preset:authdate}, below.  (Cf.\ \emph{Manual}
14.114, 14.127, 14.130; bernhard:boris, bernhard:ritter, and
bernhard:themacher for the new abbreviating functionality; also
euripides:orestes, plato:republic:gr.)

\mylittlespace \textbf{NB}: The Euripides play receives slightly
different presentations in 14.127 and 14.130.  Although the
specification is very detailed, it doesn't eliminate all choice or
variation.  Using a system like \textsc{Bib}\TeX\ should help to
maintain consistency.

\mybigspace This \mymarginpar{\textbf{booklet}} is the first of two
entry types --- the other being \textsf{manual}, on which see below
--- which are traditional in \textsc{Bib}\TeX\ styles, but which the
\emph{Manual} (14.249) suggests may well be treated basically as
books.  In the interests of backward compatibility,
\textsf{biblatex-chicago-authordate} will so format such an entry,
which uses the \textsf{howpublished} field instead of a standard
\textsf{publisher}, though of course if you do decide just to use a
\textsf{book} entry then any information you might have given in a
\textsf{howpublished} field should instead go in \textsf{publisher}.
(See clark:mesopot.)

\mybigspace This \colmarginpar{\textbf{collection}} is the standard
\textsf{biblatex} entry type, but with this release the package can
now provide automatically abbreviated references in the reference list
when you use a \textsf{crossref} or an \textsf{xref} field.  The
functionality is not enabled by default, but you can enable it in the
preamble or in the \textsf{options} field using the new
\mycolor{\texttt{booklongxref}} option.  Please see \textbf{crossref}
in section~\ref{sec:fields:authdate} and \texttt{booklongxref} in
section~\ref{sec:authpreset}, below.  See harley:ancient:cart,
harley:cartography, and harley:hoc for how this might look.

\mybigspace This \mymarginpar{\textbf{customa}} entry type is now
obsolete, and any such entries in your .bib file will trigger an
error.  Please use the standard \textsf{biblatex} \textbf{letter} type
instead.

%%\enlargethispage{-\baselineskip}

\mybigspace This \mymarginpar{\textbf{customb}} entry type is now
obsolete, and any such entries in your .bib file will trigger an
error.  Please use the standard \textsf{biblatex} \textbf{bookinbook}
type instead.

\mybigspace This \mymarginpar{\textbf{customc}} entry type allows you
to include alphabetized cross-references to other, separate entries in
the bibliography, particularly to other names or pseudonyms, as
recommended by the \emph{Manual}.  (This is different from the usual
\textsf{crossref}, \textsf{xref}, and \textsf{userf} mechanisms, all
primarily designed to include cross-references to other works.  Cf.\
14.84,86).  In the 15th edition's specification of the author-date
style, it allowed you, in particular, to include the expansions of
abbreviations and shorthands --- usually of corporate \textsf{authors}
--- \emph{inside} the list of references itself, rather than in the
list of shorthands.  The 16th edition of the \emph{Manual} (15.36),
however, has a new specification for such corporate authors.  As in
the old specification, the shorthand appears in citations and at the
head of the entry in the list of references, but its expansion now
appears within parentheses \emph{directly after} the shorthand, i.e.,
\emph{within} the same entry.  This means you no longer need the
\textsf{customc} entry for shorthands of this sort.  (See
\textsf{shorthand}, below; bsi:abbreviation, iso:electrodoc.)

\mylittlespace I should add immediately that, as I read the
specification (14.84,86, 15.34), the alphabetized cross-references
provided by \textsf{customc} are particularly encouraged, bordering on
required, when a reference list \enquote{includes two or more works
  published by the same author but under different pseudonyms.}  The
following entries in \textsf{dates-test.bib} show one way of
addressing this: crea\-sey:ashe:blast, crea\-sey:york:death,
crea\-sey:mor\-ton:hide, ashe:crea\-sey, york:crea\-sey and
mor\-ton:crea\-sey.  In these latter cases, you would need merely to
place the pseudo\-nym in the \textsf{author} field, and the author's
real name, under which his or her works are presented in the
bibliography, in the \textsf{title} field.  To make sure the
cross-reference also appears in the bibliography, you can either
manually include the entry key in a \cmd{nocite} command, or you can
put that entry key in the \textbf{userc} field in the main .bib entry,
in which case \textsf{biblatex-chicago} will print the expanded
abbreviation if and only if you cite the main entry.  (Cf.\
\textsf{userc}, below.)

\mylittlespace Under ordinary circumstances, \textsf{biblatex-chicago}
will connect the two parts of the cross-reference with the word
\enquote{\emph{See}} --- or its equivalent in the document's language
--- in italics.  If you wish to present the cross-reference
differently, you can put the connecting word(s) into the
\textsf{nameaddon} field.

\mylittlespace Finally, you may need to use this entry type if you
wish to include a comment inside the parentheses of a citation, as
specified by the \emph{Manual} (15.23).  If you have a
\textsf{postnote}, then you can manually provide the punctuation and
comment there, e.g., \cmd{autocite[4; the unrevised
  trans.]\{stendhal:parma\}}.  Without a \textsf{postnote}, you now
have two choices.  You can enable the new
\mycolor{\texttt{postnotepunct}} option, which allows you simply to
type \cmd{autocite[; the unrevised trans.]\\\{stendhal:parma\}}, or you
can continue to use a separate \textsf{misc} or \textsf{customc} entry
containing just the text of the comment in the \textsf{title} field,
\textsf{entrysubtype} \texttt{classical}, and \textsf{options}
\texttt{skipbib}.  An \cmd{autocites} command calling both the main
text and the comment then does the trick, e.g.,
\cmd{autocites\{chicago:manual\}\{chicago:\\comment\}}.  Cf.\
\mycolor{\texttt{postnotepunct}} in section~\ref{sec:authuseropts},
below.

\mybigspace This \mymarginpar{\textbf{image}} entry type, left
undefined in the standard styles, was in previous releases of
\textsf{biblatex-chicago} intended for referring to photographs, but
the 16th edition of the \emph{Manual} has changed its specifications
for such works, which are now treated the same as works in all other
media.  This means that this entry type is now a clone of the
\textsf{artwork} type, which see.  I retain it here as a convenience
for users migrating from the old to the new specification.  (See 3.22,
8.193; bedford:photo.)

\mybigspace These
\mymarginpar{\mycolor{\textbf{inbook}}\\\textbf{incollection}} two
standard \textsf{biblatex} types have very nearly identical formatting
requirements as far as the Chicago specification is concerned, but I
have retained both of them for compatibility.  \textsf{Biblatex.pdf}
(�~2.1.1) intends the first for \enquote{a part of a book which forms
  a self-contained unit with its own title,} while the second would
hold \enquote{a contribution to a collection which forms a
  self-contained unit with a distinct author and its own title.}  The
\textsf{title} of both sorts will be placed within quotation marks,
and in general you can use either type for most material falling into
these categories.  There was an important difference between them, as
in previous releases of \textsf{biblatex-chicago} it was only in
\textsf{incollection} entries that I implemented the \emph{Manual's}
recommendations for space-saving abbreviations in the reference list
when you cite multiple pieces from the same \textsf{collection}.
These abbreviations are now activated by default when you use the
\textsf{crossref} or \textsf{xref} field in \textsf{incollection}
entries \emph{and} in \textsf{inbook} entries, because although the
\emph{Manual} (14.113) here specifies a \enquote{multiauthor book,} at
least for the notes \&\ bibliography style, I believe the distinction
between the two is fine enough, and the author-date discussion in
15.37 general enough, to encourage similar treatments.  (For more on
this mechanism see \textbf{crossref} in
section~\ref{sec:fields:authdate}, below, and the new option
\mycolor{\texttt{longcrossref}} in section~\ref{sec:authpreset}.
Please note that it is also active by default in \textsf{letter} and
\textsf{inproceedings} entries.)  If the part of a book to which you
are referring has had a separate publishing history as a book in its
own right, then you may wish to use the \textsf{bookinbook} type,
instead, on which see above.  (See \emph{Manual} 14.111--114, 15.37;
\textsf{inbook}: ashbrook:brain, phibbs:diary, will:cohere;
\textsf{incollection}: centinel:letters, contrib:contrib,
sirosh:visualcortex; ellet:galena, keating:dearborn, and
lippincott:chicago [and the \textsf{collection} entry prairie:state]
demonstrate the use of the \textsf{crossref} field with its attendant
abbreviations in the list of references.)

\mylittlespace \textbf{NB}: The \emph{Manual} suggests that, when
referring to a chapter, one use either a chapter number or the
inclusive page numbers, not both.  In-text citations, of course,
require any \textsf{postnote} field to specify if it is a whole
chapter to which you are referring.

\enlargethispage{-\baselineskip}

\mybigspace This \mymarginpar{\textbf{inproceedings}} entry type works
pretty much as in standard \textsf{biblatex}.  Indeed, the main
differences between it and \textsf{incollection} are the lack of an
\textsf{edition} field and the possibility that an
\textsf{organization} may be cited alongside the \textsf{publisher},
even though the \emph{Manual} doesn't specify its use (14.226).
Please note, also, that the \textsf{crossref} and \textsf{xref}
mechanism for shortening citations of multiple pieces from the same
\textsf{proceedings} is operative here, just as it is in
\textsf{incollection} and \textsf{inbook} entries.  See
\textbf{crossref} in section~\ref{sec:fields:authdate} and the new
option \mycolor{\texttt{longcrossref}} in
section~\ref{sec:authpreset}, below, for more details.

\mybigspace This \mymarginpar{\textbf{inreference}} entry type is
aliased to \textsf{incollection} in the standard styles, but the
\emph{Manual's} requirements for the notes \&\ bibliography style
prompted a thoroughgoing revision.  Unfortunately, instructions for
the author-date style are considerably less copious, so parts of what
follows are my best guess at following the specification
(14.247--248).

\mylittlespace One thing, at least, seems clear.  If your reference
work can easily or conveniently be presented like a regular book, that
is, with an author or editor, a year of publication, and a title, and
if you you will be citing it by page or section number, then you
should almost certainly simply choose the \textsf{book} entry type for
your .bib entry. (Cf.\ mla:style, schellinger:novel, times:guide.  The
latter was presented as an \textsf{inreference} entry for the notes
\&\ bibliography style, but because the \textsf{book} entry type can
also present references to alphabetized headings [see below], at least
in the list of references, then it seemed better just to choose a
\textsf{book} entry for the author-date styles.)

\mylittlespace If you simply cannot make your source fit the template
for a \textsf{book}, then you may need to use the \textsf{inreference}
type, the main feature of which is the \textsf{lista} field, which you
use to present citations from \enquote{alphabetically arranged} works
by named article rather than by page number.  You should present these
article names just as they appear in the work, separated by the
keyword \enquote{\texttt{and}} if there is more than one, and
\textsf{biblatex-chicago-authordate} will provide the appropriate
prefatory string (\texttt{s.v.}, plural \texttt{s.vv.}), and enclose
each in its own set of quotation marks (times:guide).  More relevant
to the author-date styles is the fact that the \textsf{postnote} field
works the same way in \textsf{inreference} entries, the only
limitation on this system being that this field, unlike
\textsf{lista}, is not a list, and therefore for the formatting to
work correctly you can only put one article name in it.  In the case
of \enquote{[w]ell-known reference books, such as major dictionaries
  and encyclopedias,} you are encouraged not to include them in the
list of references, so the \textsf{lista} field actually may be of
less use than this special formatting of \textsf{postnote}.  You may
want to look at ency:britannica, where only a (carefully-formatted)
\textsf{shorttitle} and an \textsf{options} field are necessary to
allow you to produce in-text citations that look like (\emph{Ency.\
  Brit.}\ 15th ed., s.v. \enquote{Article}).

\mylittlespace If it seems appropriate to include such a work in the
list of references, perhaps because the work is not so well known that
a short citation will be parseable by your readers, or perhaps because
it is an online work, which requires you to provide a \textsf{urldate}
(see below), be aware that the contents of the \textsf{lista} field
will also be presented there, which may not be what you want.  A
separate \textsf{inreference} or \textsf{reference} entry might solve
this problem, but you may also need a \textsf{sortkey} field to ensure
proper alphabetization, as \textsf{biblatex} will attempt to use an
\textsf{editor} or \textsf{author} name, if either is present.  In a
typical \textsf{inreference} entry, very few fields are needed, as
\enquote{the facts of publication are often omitted, but the edition
  (if not the first) must be specified.}  In practice, this means a
\textsf{title} and possibly an \textsf{edition} field.  The
\textsf{author} field holds the author of the specific article (in
\textsf{lista}), not the author of the \textsf{title} as a whole.
This name will be printed in parentheses after the entry's name
(grove:sibelius).

\mylittlespace All of these rules apply to online reference works, as
well, for which you need to provide not only a \textsf{url} but also,
always, a \textsf{urldate}, as these sources are in constant flux
(wikiped:bibtex, grove:sibelius).  The author-date styles will
automatically use this as the identifying date in citations and the
list of references, assuming a more conventional \textsf{date} isn't
available.  Please note, however, that the automatic provision of the
\enquote{n.d.} abbreviation, in the absence of any sort of date
whatsoever, has been turned off for \textsf{inreference} entries, as
for \textsf{misc} and \textsf{reference} entries.

\mybigspace This \mymarginpar{\textbf{letter}} entry type was designed
to be used for citing letters, memoranda, or similar texts, but
\emph{only} when they appear in a published collection.  (Unpublished
material of this nature needs a \textsf{misc} entry, for which see
below.)  The author-date specification (15.40), however, recommends
against individual letters appearing in a list of references,
suggesting instead that you put the whole published collection in a
\textsf{book} entry and use a notice in the text to specify the letter
(white:total).

\mylittlespace If you absolutely must include individual letters in
the list of references, for whatever reason, then the instructions
above for the notes \&\ bibliography style in
section~\ref{sec:entrytypes}, s.v.\ \enquote{\textsf{letter,}} should
get you started.  There are a few wrinkles, related to date
specifications, that I shall attempt to clarify here.  If you look at
white:ross:memo and white:russ, you'll see two letters from the same
published collection, both written in the same year.  You can now
simply use the \textsf{origdate} field in both of them, because in the
absence of a \textsf{date} (or an \textsf{eventdate}) \textsf{Biber}
and \textsf{biblatex} will use the \textsf{origyear} as the
\textsf{labelyear}, putting it at the head of the entry and in the
citation, and also ensuring that the letters \texttt{a,b,c} are
appended to disambiguate the two sources.  You no longer need anything
in the \textsf{options} field at all, thanks to the way
\cmd{DeclareLabeldate} works through the possibilities and finds a
date to head the entry.  In this case, it works because we are using
the \textsf{xref} mechanism to refer to the whole published collection
(white:total), so a separate citation of that entry provides the
\textsf{date} for the shortened cross-reference included in the list
of references, and the \textsf{letter} entry never sees that
\textsf{date} at all.

%\enlargethispage{\baselineskip}

\mylittlespace If this all seems clear as mud, I'm not surprised, but
let me suggest that you experiment with the different date settings to
see what kinds of effects they have on the final result, and also read
the documentation of the \textsf{date} field in
section~\ref{sec:fields:authdate} below.

\mybigspace This \mymarginpar{\textbf{manual}} is the second of two
traditional \textsc{Bib}\TeX\ entry types that the \emph{Manual}
suggests formatting as books, the other being \textsf{booklet}. As
with this latter, I have retained it in
\textsf{biblatex-chicago-authordate} for backward compatibility, its
main peculiarity being that, in the absence of a named author, the
\textsf{organization} producing the manual will be provided both as
author and as publisher.  (You can give a shortened form of the
\textsf{organization} in the \textsf{shortauthor} field for text
citations, if needed.)  Of course, if you were to use a \textsf{book}
entry for such a reference, then you would need to define both
\textsf{author} and \textsf{publisher} using the name you here might
have put in \textsf{organization}.  (See 14.92; chicago:manual,
dyna:browser, natrecoff:camera.)

\mybigspace As \mymarginpar{\textbf{misc}} its name suggests, the
\textsf{misc} entry type was designed as a hold-all for citations that
didn't quite fit into other categories.  In \textsf{biblatex-chicago},
I have somewhat extended its applicability, while retaining its
traditional use.  Put simply, with no \textsf{entrysubtype} field, a
\textsf{misc} entry will retain backward compatibility with the
standard styles, so the usual \textsf{howpublished}, \textsf{version},
and \textsf{type} fields are all available for specifying an otherwise
unclassifiable text, and the \textsf{title} will be italicized.  (The
\emph{Manual}, you may wish to note, doesn't give specific
instructions on how such citations should be formatted, so when using
the Chicago style I would recommend you have recourse to this
traditional entry type as sparingly as possible.)

\mylittlespace If you do provide an \textsf{entrysubtype} field, the
\textsf{misc} type provides a means for citing unpublished letters,
memoranda, private contracts, wills, interviews, and the like, making
it something of an unpublished analogue to the \textsf{letter},
\textsf{article}, and \textsf{review} entry types (which see).
Typically, such an entry will cite part of an archive, and equally
typically the text cited won't have a specific title, but only a
generic one, whereas an \textsf{unpublished} entry will ordinarily
have a specific author and title, and won't come from a named archive.
The \textsf{misc} type with an \textsf{entrysubtype} defined is the
least formatted of all those specified by the \emph{Manual}, so titles
are in plain text by default.  It is quite possible, though somewhat
unusual, for archival material to have a specific title, rather than a
generic one.  In these cases, you will need to enclose the title
inside a \cmd{mkbibquote} command manually.  Cf.\ coolidge:speech,
roosevelt:speech, shapey:partita.  As a rule, and as with the
\textsf{letter} type, the \emph{Manual} (15.49) suggests that the list
of references will usually contain only the name of the whole archived
collection, with more specific information about individual items
provided in the text, outside the parentheses.  If, on the other hand,
\enquote{only one item from a collection has been mentioned in text,
  the entry may begin with the writer's name (if known).}  (See
14.219-220, 14.231, 14.232-242; house:papers cites a whole archive,
while creel:house, dinkel:agassiz, and spock:interview cite individual
pieces.)

\mylittlespace As far as constructing your .bib entry goes, you should
first know that, like the \textsf{inreference} and \textsf{reference}
types, the absence of any date will not result in the \enquote{n.d.}
abbreviation automatically being provided.  As for presenting the
date, the \emph{Manual} draws a distinction between archival material
that is \enquote{letter-like} (letters, memoranda, reports, telegrams)
and that which isn't (interviews, wills, contracts, or even personal
communications you've received and which you wish to cite).  This may
not always be the easiest distinction to draw, and in previous
releases of \textsf{biblatex-chicago} I have been ignoring it, but
once you've decided to classify it one way or the other you put the
date in the \textsf{origdate} field for letters, etc.\ (creel:house),
and into the \textsf{date} field for the others (spock:interview).
Like with the \textsf{letter} type, if the only date present is an
\textsf{origdate}, you no longer need to set the \texttt{cmsdate}
option in your .bib entry to make sure that that year appears at the
head of the entry (and in citations) --- this now happens
automatically.  (Cf.\ particularly the documentation in
section~\ref{sec:fields:authdate} below, s.v.\ \enquote{date}, and
also the \textsf{letter} type above for some of the date-related
complications that can arise, and how you can address them with
judicious use of the \textsf{options}, \textsf{date}, and
\textsf{origdate} fields.)

\mylittlespace As in \textsf{letter} entries, the titles of
unpublished letters are of the form \texttt{Author to Recipient},
further information can be given in the \textsf{titleaddon} field,
while the \textsf{origlocation} field can hold the place where the
letter was written.  Interviews or similar pieces will have a
different sort of title, but all types will use the \textsf{note},
\textsf{organization}, \textsf{institution}, and \textsf{location}
fields (in ascending order of generality) to identify the archive,
though the \emph{Manual} specifies (14.238) that well-known
depositories don't usually need a city, state or country specified.
(The traditional \textsf{misc} fields are all still available, also.)

\mylittlespace When your .bib entry refers to an entire archived
collection, then you may wish to use the word
\enquote{\texttt{classical}} as your \textsf{entrysubtype}, which will
have no effect on the list of references but will change the look of
the in-text citations (house:papers).  Instead of any date, the
citation will include the \textsf{title}, separated from the
\textsf{author's} name by a space, e.g., (House Papers).  This same
arrangement, happily, allows you easily to cite individual books of
the Bible, and also certain other sacred texts (14.252--55; genesis).
Please see under \textsf{entrysubtype} in
section~\ref{sec:fields:authdate} below for all the details of the
\texttt{classical} toggle.

\mylittlespace In all this class of archived material, the
\emph{Manual} (14.232) quite specifically requires more consistency
within your own work than conformity to some external standard, so it
is the former which you should pursue.  I hope that
\textsf{biblatex-chicago} proves helpful in this regard.

\mybigspace The \mymarginpar{\textbf{music}} 16th edition of the
manual has revised its recommendations more for this type than for any
other, so the notes which follow present several large changes that
you'll need to make to your .bib files.  The good news is that some,
though by no means all, of those changes involve considerable
simplifications.  \textbf{Music} is one of three audiovisual entry
types, and is intended primarily to aid in the presentation of musical
recordings that do not have a video component, though it can also
include audio books (auden:reading).  A DVD or VHS of an opera or
other performance, by contrast, should use the \textbf{video} type
instead (handel:messiah).  Because \textsf{biblatex} --- and
\textsc{Bib}\TeX\ before it --- were designed primarily for citing
book-like objects, some choices needed to be made in assigning the
various roles found on the back of a CD to the fields in a typical
.bib entry.  I have also implemented several bibstrings to help in
identifying these roles within entries.  If you can think of a simpler
way to distribute the roles, please let me know, so that I can
consider making changes before anyone gets used to the current
equivalences.

\mylittlespace These equivalences, in summary form, are:

{\renewcommand{\descriptionlabel}[1]{\qquad\textsf{#1}}
\begin{description}
\item[author =] composer, songwriter, or performer(s),
  depending on whom you wish to emphasize by placing them at the head
  of the entry.
\item[editor, editora, editorb =] conductor, director or
  performer(s).  These will ordinarily follow the \textsf{title} of
  the work, though the usual \texttt{useauthor} and \texttt{useeditor}
  options can alter the presentation within an entry.  Because these
  are non-standard roles, you will need to identify them using the
  following:
\item[editortype, editoratype, editorbtype:] The most common roles,
  all associated with specific bibstrings (or their absence), will be
  \texttt{conductor}, \texttt{director}, \texttt{producer}, and,
  oddly, \texttt{none}.  The last is particularly useful when
  identifying the group performing a piece, as it usually doesn't need
  further specifying and this role prevents \textsf{biblatex} from
  falling back on the default \texttt{editor} bibstring.
\item[title, booktitle, maintitle:] As with the other audiovisual
  types, \textsf{music} serves as an analogue both to books and to
  collections, so the title will either be, e.g., the album title or a
  song title, in which latter case the album title would go into
  \textsf{booktitle}.  The \textsf{maintitle} might be necessary for
  something like a box set of \emph{Complete Symphonies}.
\item[publisher, series, number:] These three closely- associated
  fields are intended for presenting the catalog information provided
  by the music publisher.  The 16th edition generally only requires
  the \textsf{series} and \textsf{number} fields (nytrumpet:art),
  which hold the record label and catalog number, respectively.
  Alternatively, \textsf{publisher} would function as a synonym for
  \textsf{series} (holiday:fool), but there may be cases when you need
  or want to specify a publisher in addition to a label, as was the
  general requirement in the 15th edition.  (This might happen, for
  example, when a single publisher oversees more than one label.)  You
  can certainly put all of this information into one of the above
  fields, but separating it may help make the .bib entry more
  readable.
\item[howpublished/pubstate:] The 16th edition of the \emph{Manual}
  (14.276, 15.53) has rather helpfully eliminated any reference to the
  specialized symbols (\texttt{\textcircledP} \&\
  \texttt{\textcopyright}) found in the 15th edition for presenting
  publishing information for musical recordings.  This means that the
  \textsf{howpublished} field is now obsolete, and you can remove it
  from \textsf{music} entries in your .bib files.  The
  \textsf{pubstate} field, therefore, can revert to its standard use
  for identifying reprints.  In \textsf{music} entries, putting
  \texttt{reprint} here will transform the \textsf{origdate} from a
  recording date for an entire album into an original release date for
  that album, notice of which will be printed towards the end of a
  reference list entry, always assuming that the \textsf{origdate}
  hasn't already appeared at the head of the entry and in citations.
\item[date, eventdate, origdate:] As though to compensate for the
  simplification I've just mentioned, the \textsf{Manual} now
  \enquote{recommends a more comprehensive approach to dating
    audiovisual materials than in previous editions} (15.53).  Indeed,
  \enquote{citations without a date are generally unacceptable}
  (14.276), while if there is more than one date \enquote{the date of
    the original recording should be privileged} (15.53).  Finding
  these dates may take some research, but they will basically fall
  into two types, i.e., the date of the recording or the copyright /
  publishing date.  Recording dates go either in \textsf{origdate}
  (for complete albums) or \textsf{eventdate} (for individual tracks).
  The current copyright or publishing date goes in the \textsf{date}
  field, while the original release date goes in \textsf{origdate}.
  You may have noticed that the \textsf{origdate} has two slightly
  different uses --- you can tell \textsf{biblatex-chicago} which sort
  you intend by using the string \texttt{reprint} in the
  \textsf{pubstate} field, which transforms the \textsf{origdate} from
  a recording date into an original release date.  The style will
  automatically use the \textsf{eventdate} or the \textsf{origdate} in
  citations and at the head of the list of references, falling back on
  a \textsf{date} or even a \textsf{urldate} in their absence.  It
  will also prepend the bibstring \texttt{recorded} to any part of the
  \textsf{eventdate} that doesn't appear at the head of the list of
  references or, in the absence of the \textsf{pubstate} mechanism, to
  the \textsf{origdate}, or indeed to both.  You can modify what is
  printed here using the new \mycolor{\textsf{userd}} field, which
  acts as a sort of date type modifier.  In \textsf{music} entries,
  \textsf{userd} will be prepended to an \textsf{eventdate} if there
  is one, barring that to the \textsf{origdate}, barring that to a
  \textsf{urldate}, and absent those three to a \textsf{date}.  (See
  holiday:fool, nytrumpet:art.)
\item[type:] As in all the audiovisual entry types, the \textsf{type}
  field holds the medium of the recording, e.g., vinyl, 33 rpm,
  8-track tape, cassette, compact disc, mp3, ogg vorbis.
\end{description}}

The entries in \textsf{dates-test.bib} should at least give you a good
idea of how this all works, and that file also contains an example of
an audio book presented in a \textsf{music} entry.  If you browse the
examples in the \emph{Manual} you will see some variations in the
formatting choices there, from which I have made selections for
\textsf{biblatex-chicago}.  It wasn't always clear to me that these
variations were rules as opposed to possibilities, so I've ignored
some of them in the code.  Arguments as to why I'm wrong will, of
course, be entertained.  (Cf. 14.276--77, 15.53; \textsf{eventdate},
\textsf{origdate}, \textsf{userd}; \cmd{DeclareLabeldate} in
section~\ref{sec:authformopts} and \texttt{avdate} in
section~\ref{sec:authpreset}; auden:reading, beethoven:sonata29,
bernstein:shostakovich, floyd:atom, holiday:fool, nytrumpet:art,
rubinstein:chopin.)

\mybigspace All \colmarginpar{\textbf{mvbook}\\\textbf{mvcollection}%
  \\\textbf{mvproceedings}\\\textbf{mvreference}} four of these entry
types are new to \textsf{biblatex-chicago}, and all function more or
less as in standard \textsf{biblatex}.  I would like, however, to
emphasize a couple of things.  First, each is aliased to the entry
type that results from removing the \enquote{\textbf{mv}} from their
names.  Second, each has an important role as the target of
cross-references from other entries, the \textsf{title} of the
\textbf{mv*} entry \emph{always} providing a \textsf{maintitle} for
the entry referencing it.  If you want to provide a \textsf{booktitle}
for the referencing entry, please use another entry type, e.g.,
\textbf{collection} for \textbf{incollection} or \textbf{book} for
\textbf{inbook}.  (These distinctions are particularly important to
the correct functioning of the abbreviated references that
\textsf{biblatex-chicago}, in various circumstances, provides.  Please
see the documentation of the \textbf{crossref} field in
section~\ref{sec:fields:authdate}, below.)

\enlargethispage{\baselineskip}

\mylittlespace On the same subject, when multi-volume works are
presented in the reference apparatus, the \emph{Manual} (14.121--27,
15.39) requires that any dates presented should be appropriate to the
specific nature of the citation.  In short, this means that a date
range that is right for the presentation of a multi-volume work in its
entirety isn't right for citing, e.g., a single volume of that work
which appeared in one of the years contained in the date range.
Because child entries will by default inherit all the date fields from
their parent (including the \textsf{endyear} of a date range), I have
turned off the inheritance of \textsf{date} and \textsf{origdate}
fields from all of the \textbf{mv*} entry types to any other entry
type.  When the dates of the parent and of the child in such a
situation are exactly the same, then this unfortunately requires an
extra field in the child's .bib entry.  When they're not the same, as
will, I believe, often be the case, this arrangement saves a lot of
annoying work in the child entry to suppress wrongly-inherited fields.
Other sorts of parent entries aren't affected by this.  See
harley:ancient:cart, harley:cartography, and harley:hoc for how this
might look.

\mybigspace The \mymarginpar{\textbf{online}} \emph{Manual}'s
scattered instructions (14.4--13, 14.166--169, 14.184--185, 14.200,
14.223, 14.243--246, 15.4, 15.9) for citing online materials are
slightly different from those suggested by standard \textsf{biblatex}.
Indeed, this is a case where complete backward compatibility with
other \textsf{biblatex} styles may be impossible, because as a general
rule the \emph{Manual} considers relevant not only where a source is
found, but also the nature of that source, e.g., if it's an online
edition of a book (james:ambassadors), then it calls for a
\textsf{book} entry.  Even if you cite an intrinsically online source,
if that source is structured more or less like a conventional printed
periodical, then you'll probably want to use \textsf{article} or
\textsf{review} instead of \textsf{online} (stenger:privacy, which
cites \emph{CNN.com}).  The 16th edition's suggestions for blogs lend
themselves well to the \textsf{article} type, too, while comments
become, logically, \textsf{reviews} (14.243--6; ellis:blog,
ac:comment).  Otherwise, for online documents not \enquote{formally
  published,} the \textsf{online} type is usually the best choice
(evanston:library, powell:email).  Online videos, in particular short
pieces or those that present excerpts of some longer event or work,
and also online interviews, usually require this type, too.  (See
harwood:biden, horowitz:youtube, pollan:plant, but cp.\ weed:flatiron,
a complete film, which requires a \textsf{video} entry.  Online audio
pieces, particularly dated ones from an archive, work best as
\textsf{misc} entries with an \textsf{entrysubtype}: coolidge:speech,
roosevelt:speech.)  Some online materials will, no doubt, make it
difficult to choose an entry type, but so long as all locating
information is present, then perhaps that is enough to fulfill the
specification, or at least so I'd like to hope.

\mylittlespace Constructing an \textsf{online} .bib file entry is much
the same as in \textsf{biblatex}.  The \textsf{title} field would
contain the title of the page, the \textsf{organization} field could
hold the title or owner of the whole site.  If there is no specific
title for a page, but only a generic one (powell:email), then such a
title should go in \textsf{titleaddon}, not forgetting to begin that
field with a lowercase letter so that capitalization will work out
correctly.  It is worth remarking here, too, that the 16th edition of
the \emph{Manual} (14.7--8) prefers, if they're available, revision
dates to access dates when documenting online material.  See
\textsf{urldate} and \textsf{userd}, below.

%%\enlargethispage{\baselineskip}

\mybigspace The \mymarginpar{\textbf{patent}} \emph{Manual} is very
brief on the subject of patents (15.50), but very clear about which
information it wants you to present, so such entries may not work well
with other \textsf{biblatex} styles.  In a change to previous
practice, the 16th edition of Chicago's author-date style prefers the
\emph{later} of the two possible dates to appear in citations and at
the head of the entry in the list of references.  If a patent has been
filed but not yet granted, then you can place the filing date in
either the \textsf{date} field or the \textsf{origdate} field, and
\textsf{biblatex-chicago-authordate} will automatically prepend the
bibstring \texttt{patentfiled} to it.  If the patent has been granted,
then you put the filing date in the \textsf{origdate} field, and you
put the date it was issued in the \textsf{date} field, to which the
bibstring \texttt{patentissued} will automatically be prepended, and
it is this later date that will head the entry and appear in
citations.  The patent number goes in the \textsf{number} field, and
you should use the standard \textsf{biblatex} bibstrings in the
\textsf{type} field.  Though it isn't mentioned by the \emph{Manual},
\textsf{biblatex-chicago-authordate} will print the \textsf{holder}
after the \textsf{author}, if you provide one.  Finally, the 16th
edition of the \emph{Manual} capitalizes the \textsf{title}
sentence-style, which seems to be the generally-accepted convention,
across both Chicago styles.  As I've removed all of the automatic
down-casing code from previous editions, you may need manually to
revise the \textsf{title} field to provide the lowercase letters.  See
petroff:impurity.

\mybigspace This \mymarginpar{\textbf{periodical}} is the standard
\textsf{biblatex} entry type for presenting an entire issue of a
periodical, rather than one article within it.  It has the same
function in \textsf{biblatex-chicago}, and in the main uses the same
fields, though in keeping with the system established in the
\textsf{article} entry type (which see) you'll need to provide
\textsf{entrysubtype} \texttt{magazine} if the periodical you are
citing is a \enquote{newspaper} or \enquote{magazine} instead of a
\enquote{journal.}  Also, remember that the \textsf{note} field is the
place for identifying strings like \enquote{special issue,} with its
initial lowercase letter to activate the automatic capitalization
routines, though this isn't strictly necessary in the author-date
styles.  (See \emph{Manual} 14.187; good:wholeissue.)

\mybigspace This \colmarginpar{\textbf{proceedings}} is the standard
\textsf{biblatex} and \textsc{Bib}\TeX\ entry type, but with this
release the package can now provide automatically abbreviated
references in the reference list when you use a \textsf{crossref} or
an \textsf{xref} field.  The functionality is not enabled by default,
but you can enable it in the preamble or in the \textsf{options} field
using the new \mycolor{\texttt{booklongxref}} option.  Please see
\textbf{crossref} in section~\ref{sec:fields:authdate} and
\texttt{booklongxref} in section~\ref{sec:authpreset}, below.

\mybigspace This \mymarginpar{\textbf{reference}} entry type is
aliased to \textsf{collection} by the standard \textsf{biblatex}
styles, but I intend it to be used in cases where you need to cite a
reference work but not an alphabetized article or articles in that
work.  This could be because it doesn't contain such articles, and yet
you still want the entry in the list of references to start with the
\textsf{title}.  Indeed, the only differences between it and
\textsf{inreference} are the lack of a \textsf{lista} field to present
an alphabetized entry, and the fact that any \textsf{postnote} field
will be printed verbatim, rather than formatted as an alphabetized
entry.  (Cf.\ \textsf{inreference}, above.)

\mybigspace This \mymarginpar{\textbf{report}} entry type is a
\textsf{biblatex} generalization of the traditional \textsc{Bib}\TeX\
type \textsf{techreport}.  Instructions for such entries are rather
thin on the ground in the \emph{Manual} (8.183, 14.249), so I have
followed the generic advice about formatting it like a book, and hope
that the results conform to the specification.  Its main peculiarities
are the \textsf{institution} field in place of a \textsf{publisher},
the \textsf{type} field for identifying the kind of report in
question, and the \textsf{isrn} field containing the International
Standard Technical Report Number of a technical report.  As in
standard \textsf{biblatex}, if you use a \textsf{techreport} entry,
then the \textsf{type} field automatically defaults to
\cmd{bibstring\{techreport\}}.  As with \textsf{booklet} and
\textsf{manual}, you can also use a \textsf{book} entry, putting the
report type in \textsf{note} and the \textsf{institution} in
\textsf{publisher}.  (See herwign:office.)

\mybigspace The \mymarginpar{\textbf{review}} \textsf{review} entry
type wasn't, strictly speaking, necessary for the 15th edition
author-date specification.  With the major changes to the presentation
of the title fields in the 16th edition, however, it has now become
necessary for \textsf{authordate} users, if not
\textsf{authordate-trad} users, to familiarize themselves with it as a
means of coping with the \emph{Manual}'s complicated requirements for
citing periodicals of all sorts.  As its name suggests, this entry
type was designed for reviews published in periodicals, and if you've
already read the \textsf{article} instructions above --- if you
haven't, I recommend doing so now --- you'll know that \textsf{review}
serves as well for citing other sorts of material with generic titles,
like letters to the editor, obituaries, interviews, online comments
and the like.  The primary rule is that any piece that has only a
generic title, like \enquote{review of \ldots,} \enquote{interview
  with \ldots,} or \enquote{obituary of \ldots,} calls for the
\textsf{review} type.  Any piece that also has a specific title, e.g.,
\enquote{\enquote{Lost in \textsc{Bib}\TeX,} an interview with
  \ldots,} requires an \textsf{article} entry.  (This assumes the text
is found in a periodical of some sort.  Were it found in a book, then
the \textsf{incollection} type would serve your needs, and you could
use \textsf{title} and \textsf{titleaddon} there.  While we're on the
topic of exceptions, the \emph{Manual} includes an example --- 14.221
--- where the \enquote{Interview} part of the title is considered a
subtitle rather than a titleaddon, said part therefore being included
inside the quotation marks and capitalized accordingly.  Not having
the journal in front of me I'm not sure what prompted that decision,
but \textsf{biblatex-chicago} would obviously have no difficulty
coping with such a situation.)

\mylittlespace Once you've decided to use \textsf{review}, then you
need to determine which sort of periodical you are citing, the rules
for which are the same as for an \textsf{article} entry.  If it is a
\enquote{magazine} or a \enquote{newspaper}, then you need an
\textsf{entrysubtype} \texttt{magazine}.  The generic title goes in
\textsf{title} and the other fields work just as as they do in an
\textsf{article} entry with the same \textsf{entrysubtype}, including
the substitution of the \textsf{journaltitle} for the \textsf{author}
if the latter is missing. (See 14.202--203, 14.205, 14.208,
14.214--217, 14.221, 15.47; barcott:review, bundy:macneil,
Clemens:letter, gourmet:052006, kozinn:review, nyt:trevorobit,
unsigned:ranke, wallraff:word.)  If, on the other hand, the piece
comes from a \enquote{journal,} then you don't need an
\textsf{entrysubtype}.  The generic title goes in \textsf{title}, and
the remaining fields work just as they do in a plain \textsf{article}
entry.  (See 14.215; ratliff:review.)

\mylittlespace The \emph{Manual} now suggests that, no matter which
citation style you are using, it is \enquote{usually sufficient to
  cite newspaper and magazine articles entirely within the text}
(15.47).  This involves giving the title of the journal and the full
date of publication in a parenthetical reference, including any other
information in the main text (14.206), thereby obviating the need to
present such an entry in the list of references.  To utilize this
method in the author-date styles, in addition to a \texttt{magazine}
\textsf{entrysubtype}, you'll need to place \texttt{cmsdate=full} into
the \textsf{options} field, including \texttt{skipbib} there as well
to stop the entry printing in the list of references.  If the entry
only contains a \textsf{date} and \textsf{journaltitle} that's enough,
but if it's a fuller entry also containing an \textsf{author} then
you'll also need \texttt{useauthor=false} in the \textsf{options}
field.  Other surplus fields will be ignored.  (See osborne:poison.)

\mylittlespace Most of the onerous details are the same as I described
them in the \textbf{article} section above, but I'll repeat some of
them briefly here.  If anything in the \textsf{title} needs
formatting, you need to provide those instructions yourself, as the
default is completely plain.  \textsf{Author}-less \textsf{reviews}
are treated just like similar \textsf{articles} --- with an
\textsf{entrysubtype}, the \textsf{journaltitle} replaces the author
in citations and heads the entry in the list of references, without an
\textsf{entrysubtype} the \textsf{title} does the same.  In the former
case, \textsf{Biber} handles the sorting for you, but in the latter
you'll need a \textsf{sortkey} because \textsf{journaltitle} comes
before \textsf{title} in the sorting scheme.  (14.175, 14.217;
gourmet:052006, nyt:trevorobit, unsigned:ranke, and see
\cmd{DeclareSortingScheme} in section~\ref{sec:authformopts}, below.).
As in \textsf{misc} entries with an \textsf{entrysubtype}, words like
\enquote{interview,} \enquote{review,} and \enquote{letter} only need
capitalization after a full stop, so you can start the \textsf{title}
field with a lowercase letter and let the automatic field formatting
with \cmd{autocap} do its work, though this isn't strictly necessary
with \textsf{biblatex-chicago-authordate}.

\mylittlespace One detail of the \textsf{review} type is new to both
specifications, and responds to the needs of the 16th edition of the
\emph{Manual}.  As I mentioned above, blogs are best treated as
\textsf{articles} with \texttt{magazine} \textsf{entrysubtype},
whereas comments on those blogs --- or on any similar sort of online
content --- need the \textsf{review} type with the same
\textsf{entrysubtype}.  What they will frequently also need is a date
of some sort closely associated with the comment (14.246; ac:comment),
so I have included the \textsf{eventdate} in \textsf{review} entries
for just this purpose.  The \textsf{eventyear} will appear in
citations and at the head of the reference list entry, while the
remainder of the \textsf{eventdate} will be printed just after the
\textsf{title}.  If, in addition, you need an identifying timestamp,
then the \textsf{nameaddon} field is the place for it, but you'll have
to provide your own parentheses, in order to preserve the possibility
of providing pseudonyms in square brackets that is the standard
function of this field in all other entry types, and possibly in the
the \textsf{review} type as well.  (Cf.\ the documentation of
\textsf{eventdate} in section~\ref{sec:fields:authdate},
\cmd{DeclareLabeldate} in section~\ref{sec:authformopts}, and
\texttt{avdate} in section~\ref{sec:authpreset}.)

\mylittlespace For the reasons I explained in the \textsf{article}
docs above, I have brought the \textsf{article} and \textsf{review}
entry types into line with most of the other types in allowing the use
of the \textsf{namea} and \textsf{nameb} fields in order to associate
an editor or a translator specifically with the \textsf{title}.  The
\textsf{editor} and \textsf{translator} fields, in strict homology
with other entry types, are associated with the \textsf{issuetitle} if
one is present, and with the \textsf{title} otherwise.  The usual
string concatenation rules still apply --- cf.\ \textsf{editor} and
\textsf{editortype} in section~\ref{sec:fields:authdate}, below.

%%\enlargethispage{\baselineskip}

\mybigspace This \mymarginpar{\textbf{suppbook}} is the entry type to
use if the main focus of a reference is supplemental material in a
book or in a collection, e.g., an introduction, afterword, or forward,
either by the same or by a different author.  There are two mechanisms
in \textsf{biblatex-chicago} for producing such a citation.  First,
these three just-mentioned types of material, and only these three
types, can be referenced using the \textsf{introduction},
\textsf{afterword}, or \textsf{foreword} fields, a system that
requires you simply to define one of them in any way and leave the
others undefined.  The macros don't use the text provided by such an
entry, they merely check to see if one of them is defined, in order to
decide which sort of pre- or post-matter is at stake, and to print the
appropriate string before the \textsf{title} in the list of
references, and possibly also in the list of shorthands.  This
mechanism works without modification across multiple languages, but I
have also provided functionality which allows you to cite any sort of
supplemental material whatever, using the \textsf{type} field.  Under
this second system, simply put the nature of the material, including
the relevant preposition, in that field, beginning with a lowercase
letter so \textsf{biblatex} can decide whether it needs capitalization
depending on the context.  Examples might be \enquote{\texttt{preface
    to}} or \enquote{\texttt{colophon of}.}  (Please note, however,
that unless you use a \cmd{bibstring} command in the \textsf{type}
field, the resultant entry will not be portable across languages.)

% %\enlargethispage{\baselineskip}

\mylittlespace The other rules for constructing your .bib entry remain
the same.  The \textsf{author} field refers to the author of the
introduction or afterword, while \textsf{bookauthor} refers to the
author of the main text of the work, if the two differ.  For the 16th
edition, the \emph{Manual} requires that you include the page range
for the cited part in the list of references.  As ever, if the focus
of the reference is the main text of the book, but you want to mention
the name of the writer of an introduction or afterword for
completeness, then the normal \textsf{biblatex} rules apply, and you
can just put their name in the appropriate field of a \textsf{book}
entry, that is, in the \textsf{foreword}, \textsf{afterword}, or
\textsf{introduction} field.  (See \emph{Manual} 14.116;
friedman:intro, polakow:afterw, prose:intro).

\mybigspace This \mymarginpar{\textbf{suppcollection}} fulfills a
function analogous to \textsf{suppbook}.  Indeed, I believe the
\textbf{suppbook} type can serve to present supplemental material in
both types of work, so this entry type is an alias to
\textsf{suppbook}, which see.

\mybigspace This \mymarginpar{\textbf{suppperiodical}} type is
intended to allow reference to generically-titled works in
periodicals, such as regular columns or letters to the editor.
\textsf{Biblatex} also provides the \textsf{review} type for this
purpose, so in both Chicago styles \textsf{suppperiodical} is an alias
of \textsf{review}.  In the 16th edition of the \textsf{authordate}
style, as discussed above, the use of this latter entry type has
become necessary, so please see its documentation for instructions on
how to construct a .bib entry for such works.

\mybigspace This \mymarginpar{\textbf{video}} is the last of the three
audiovisual entry types, and as its name suggests it is intended for
citing visual media, be it films of any sort or TV shows, broadcast,
on the Net, on VHS, DVD, or Blu-ray.  As with the \textsf{music} type
discussed above, certain choices had to be made when associating the
production roles found, e.g., on a DVD, to those bookish ones provided
by \textsf{biblatex}.  Here are the main correspondences:

{\renewcommand{\descriptionlabel}[1]{\qquad\textsf{#1}}
\begin{description}
\item[author:] This will not infrequently be left undefined, as the
  director of a film should be identified as such and therefore placed
  in the \textsf{editor} field with the appropriate
  \textsf{editortype} (see below).  You will need it, however, to
  identify the composer of, e.g., an oratorio on VHS (handel:messiah),
  or perhaps the provider of commentaries or other extras on a film
  DVD (cleese:holygrail).
\item[editor, editora, editorb:] The director or producer, or possibly
  the performer or conductor in recorded musical performances.  These
  will ordinarily follow the \textsf{title} of the work, though the
  usual \texttt{useauthor} and \texttt{useeditor} options can alter
  the presentation within an entry.  Because these are non-standard
  roles, you will need to identify them using the following:
\item[editortype, editoratype, editorbtype:] The most common roles,
  all associated with specific bibstrings (or their absence), will
  likely be \texttt{director}, \texttt{produ\-cer}, and, oddly,
  \texttt{none}.  The last is particularly useful if you want to
  identify performers, as they usually don't need further specifying
  and this role prevents \textsf{biblatex} from falling back on the
  default \texttt{editor} bibstring.
\item[title, titleaddon, booktitle, booktitleaddon, maintitle:] As
  with the other audiovisual types, \textsf{video} serves as an
  analogue both to books and to collections, so the \textsf{title} may
  be of a whole film DVD or of a TV series, or it may identify one
  episode in a series or one scene in a film.  In the latter cases,
  the title of the whole would go in \textsf{booktitle}.  The
  \textsf{booktitleaddon} field, in a change from the 15th edition,
  may be useful for specifying the season and/or episode number of a
  TV series, while the \textsf{titleaddon} is for any information that
  needs to come between the \textsf{title} and the \textsf{booktitle}
  (cleese:holygrail, episode:tv, handel:messiah).  As in the
  \textsf{music} type, \textsf{maintitle} may be necessary for a boxed
  set or something similar.
\item[date, eventdate, origdate, pubstate:] The 16th edition of the
  \textsf{Manual} now \enquote{recommends a more comprehensive
    approach to dating audiovisual materials than in previous
    editions} (15.53).  Indeed, \enquote{citations without a date are
    generally unacceptable} (14.276), while if there is more than one
  date \enquote{the date of the original recording should be
    privileged} (15.53).  As with \textsf{music} entries, in order to
  follow these specifications I have had to provide three separate
  date fields for citing \textsf{video} sources, but their uses differ
  somewhat between the two types.  In both, the \textsf{date} will
  generally provide the publishing or copyright date of the medium you
  are referencing.  More specific to this entry type, the
  \textsf{origdate} will generally hold the date of the original
  theatrical release of a film, while the \textsf{eventdate} will most
  commonly present either the broadcast date of a particular TV
  program, or the recording/performance date of, for example, an opera
  on DVD.  The style will automatically prepend the bibstring
  \texttt{broadcast} to such a date, though you can use the
  \mycolor{\textsf{userd}} field to change the string printed there.
  (Absent an \textsf{eventdate}, the \textsf{userd} field in
  \textsf{video} entries will modify the \textsf{urldate}, and absent
  those two it will modify the \textsf{date}.)  Typically, any given
  \textsf{video} entry will only need an \textsf{eventdate} \emph{or}
  an \textsf{origdate}, and it is this date that will appear in
  citations and at the head of the entry in the reference list.  It's
  conceivable that you may need all three dates, in which case you can
  also use the standard \textsf{pubstate} field with \texttt{reprint}
  in it to control the printing of the \textsf{origdate} at the end of
  the entry, though I have altered the string that is printed there.
  Cf.\ friends:leia, handel:messiah, hitchcock:nbynw;
  \textsf{pubstate}, below.
\item[entrysubtype:] With the changes to the date fields detailed just
  above, this field is no longer needed for \textsf{video} entries,
  and will be ignored.
\item[type:] As in all the audiovisual entry types, the \textsf{type}
  field holds the medium of the \textsf{title}, e.g., 8 mm, VHS, DVD,
  Blu-ray, MPEG.
\end{description}}

As with the \textsf{music} type, entries in \textsf{dates-test.bib}
should at least give you a good idea of how all this works.  (Cf.\
14.279--80; \textsf{eventdate}, \textsf{origdate}, \textsf{userd};
\cmd{DeclareLabeldate} in section~\ref{sec:authformopts}, and
\texttt{avdate} in section~\ref{sec:authpreset}; cleese:holygrail,
friends:leia, handel:messiah, hitchcock:nbynw, loc:city.)

\subsection{Entry Fields}
\label{sec:fields:authdate}

The following discussion presents, in alphabetical order, a complete
list of the entry fields you will need to use
\textsf{biblatex-chicago-authordate}.  As in
section~\ref{sec:types:authdate}, I shall include references to the
numbered paragraphs of the \emph{Chicago Manual of Style}, and also to
the entries in \textsf{dates-test.bib}.  Many fields are most easily
understood with reference to other, related fields.  In such cases,
cross references should allow you to find the information you need.

\mybigspace As \mymarginpar{\textbf{addendum}} in standard
\textsf{biblatex}, this field allows you to add miscellaneous
information to the end of an entry, after publication data but before
any \textsf{url} or \textsf{doi} field.  In the \textsf{patent} entry
type (which see), it will be printed in close association with the
filing and issue dates.  In any entry type, if your data begins with a
word that would ordinarily only be capitalized at the beginning of a
sentence, then simply ensure that that word is in lowercase, and the
style will take care of the rest.  Cf.\ \textsf{note}. (See
\emph{Manual} 14.119, 14.166--68; davenport:attention,
natrecoff:camera.)

\mybigspace In most \mymarginpar{\textbf{afterword}} circumstances,
this field will function as it does in standard \textsf{biblatex},
i.e., you should include here the author(s) of an afterword to a given
work.  The \emph{Manual} suggests that, as a general rule, the
afterword would need to be of significant importance in its own right
to require mentioning in the reference apparatus, but this is clearly
a matter for the user's judgment.  As in \textsf{biblatex}, if the
name given here exactly matches that of an editor and/or a translator,
then \textsf{biblatex-chicago} will concatenate these fields in the
formatted references.

\mylittlespace As noted above, however, this field has a special
meaning in the \textsf{suppbook} entry type, used to make an
afterword, foreword, or introduction the main focus of a citation.  If
it's an afterword at issue, simply define \textsf{afterword} any way
you please, leave \textsf{foreword} and \textsf{introduction}
undefined, and \textsf{biblatex-chicago} will do the rest. Cf.\
\textsf{foreword} and \textsf{introduction}. (See \emph{Manual} 14.91,
14.116; polakow:afterw.)

% %\enlargethispage{\baselineskip}

\mybigspace At \mymarginpar{\textbf{annotation}} the request of Emil
Salim, \textsf{biblatex-chicago} has, as of version 0.9, added a
package option (see \texttt{annotation} below, section
\ref{sec:useropts}) to allow you to produce annotated lists of
references.  The formatting of such a list is currently fairly basic,
though it conforms with the \emph{Manual's} minimal guidelines
(14.59).  The default in \textsf{chicago-authordate.cbx} is to define
\cmd{DeclareFieldFormat\{an\-notation\}} using \cmd{par}\cmd{nobreak}
\cmd{vskip} \cmd{bibitemsep}, though you can alter it by re-declaring
the format in your preamble.  The page-breaking algorithms don't
always give perfect results here, but the default formatting looks, to
my eyes, fairly decent.  In addition to tweaking the field formatting
you can also insert \cmd{par} (or even \cmd{vadjust\{\cmd{eject}\}})
commands into the text of your annotations to improve the appearance.
Please consider the \texttt{annotation} option a work in progress, but
it is usable now.  (N.B.: The \textsc{Bib}\TeX\ field \textsf{annote}
serves as an alias for this.)

\mybigspace I \mymarginpar{\textbf{annotator}} have implemented this
\textsf{biblatex} field pretty much as that package's standard styles
do, even though the \emph{Manual} doesn't actually mention it.  It may
be useful for some purposes.  Cf.\ \textsf{commentator}.

\mybigspace For \mymarginpar{\textbf{author}} the most part, I have
implemented this field in a completely standard \textsc{Bib}\TeX\
fashion.  Remember that corporate or organizational authors need to
have an extra set of curly braces around them (e.g.,
\texttt{\{\{Associated Press\}\}}\,) to prevent \textsc{Bib}\TeX\ from
treating one part of the name as a surname (14.92, 14.212, 15.36;
assocpress:gun, chicago:manual).  If there is no \textsf{author}, then
\textsf{biblatex-chicago} will look, in sequence, for an
\textsf{editor}, \textsf{translator}, or \textsf{compiler} (actually
\textsf{namec}, currently) and use that name (or those names) instead,
followed by the appropriate identifying string (esp.\ 15.35, also
14.76, 14.87, 14.126, 14.132, 14.189; boxer:china, brown:bremer,
harley:cartography, schellinger:novel, sechzer:women, silver:ga\-wain,
soltes:georgia).  \textsf{Biber} now takes care of alphabetizing
entries no matter which name appears at their head, and the package
also automatically provides a name for citations.

\mylittlespace If you wish to emphasize the activity of an editor or a
translator, you can use the \textsf{biblatex} and
\textsf{biblatex-chicago} options \texttt{useauthor=false},
\texttt{useeditor=false}, \texttt{usetranslator=false}, and
\texttt{usecompiler=false} in the \textsf{options} field to choose
which one appears at the head of an entry.  A peculiarity of this
system of toggles is that in order to ensure that the \textsf{title}
of a book appears at the head of an entry, you would need to use
\emph{all four} of the toggles, even though the hypothetical entry
contains no \textsf{translator}.  Internally,
\textsf{biblatex-chicago} is either searching for an
author-substitute, or it is skipping over elements of the ordered,
unidirectional chain \textsf{author -> editor -> translator ->
  compiler -> title}.  If you don't include
\texttt{usetranslator=false} in the \textsf{options} field, then the
package begins its search at \textsf{translator} and continues on to
\textsf{namec}, even though you have \texttt{usecompiler=false} in
\textsf{options}.  The result will be that the compilers' names will
appear at the head of the entry.  If you want to skip over parts of
the chain, you must turn off \emph{all} of the parts up to the one you
wish printed.  Another peculiarity of the system is that setting the
Chicago-specific \texttt{usecompiler} option to \texttt{false} doesn't
remove \textsf{namec} from the sorting list, whereas the other
standard \textsf{biblatex} toggles \emph{do} remove their names from
the sorting list, so in some corner cases you may need the
\textsf{sortkey} field.  See \cmd{DeclareSortingScheme} in
section~\ref{sec:authformopts}, below.

\mylittlespace This system of toggles, then, can turn off
\textsf{biblatex-chicago}'s mechanism for finding a name to place at
the head of an entry, but it also very usefully adds the possibility
of citing a work with an \textsf{author} by its editor, compiler or
translator instead (14.90; eliot:pound), something that wasn't
possible before.  For full details of how this works, see the
\textsf{editortype} documentation below.  (Of course, in
\textsf{collection} and \textsf{proceedings} entry types, an
\textsf{author} isn't expected, so there the \textsf{editor} is
required, as in standard \textsf{biblatex}.  Also, in \textsf{article}
and \textsf{review} entries with \textsf{entrysubtype}
\texttt{magazine}, the absence of an \textsf{author} triggers the use
of the \textsf{journaltitle} in its stead.  Without an
\textsf{entrysubtype}, the \textsf{title} will be used.  See the next
paragraph, and those entry types, for further details.)

\mylittlespace As its name suggests, the author-date style very much
wants to have a name of some sort present both for the entries in the
list of references and for the in-text citations.  The \emph{Manual}
is nothing if not flexible, however, so with unsigned articles or
encyclopedia entries the \textsf{journaltitle} or \textsf{title} may
take the place of the \textsf{author} (gourmet:052006,
lakeforester:pushcarts, nyt:trevorobit, unsigned:ranke,
wikipedia:bibtex).  Even in such entries, however, it may be
advantageous to provide either a standard \textsf{shorttitle} or, for
abbreviating a \textsf{journaltitle}, a (formatted)
\textsf{shortauthor} field, thereby keeping the in-text citations to a
reasonable length, though not at the expense of making it hard to find
the relevant entries in the reference list.

% %\enlargethispage{\baselineskip}

\mylittlespace Recommendations concerning anonymous authors in other
kinds of references have changed somewhat in the 16th edition of the
\emph{Manual} (15.32), placing greater emphasis on using the
\textsf{title} in citations and at the head of reference list entries,
rather than \enquote{Anonymous.}  The latter may still in some cases
be useful \enquote{in a bibliography in which several anonymous works
  need to be grouped} (14.79), but even with a source like
virginia:plantation, \enquote{the reference list entry should normally
  begin with the title\ldots\ Text citations may refer to a short form
  of the title but must include the first word (other than an initial
  article)} (15.32).  The \textsf{shorttitle} field is the place for
the short form, and you'll also need a \textsf{sortkey} of some sort
if the full title begins with an article that is to be ignored when
alphabetizing.

\mylittlespace If \enquote{the authorship is known or guessed at but
  was omitted on the title page,} then you need to use the
\textsf{authortype} field to let \textsf{biblatex-chicago} know this
fact (15.33).  If the author is known (horsley:prosodies), then put
\texttt{anon} in the \textsf{authortype} field, if guessed at
(cook:sotweed) put \texttt{anon?}\ there.  (In both cases,
\textsf{biblatex-chicago} tests for these \emph{exact} strings, so
check your typing if it doesn't work.)  This will have the effect of
enclosing the name in square brackets, with or without the question
mark indicating doubt.  As long as you have the right string in the
\textsf{authortype} field, \textsf{biblatex-chicago-authordate} will
also do the right thing automatically in text citations.

\mylittlespace The \textsf{nameaddon} field furnishes the means to
cope with the case of pseudonymous authorship.  If the author's real
name isn't known, simply put \texttt{pseud.}\ (or
\cmd{bibstring\{pseudonym\}}) in that field (centinel:letters).  If
you wish to give a pseudonymous author's real name, simply include it
there, formatted as you wish it to appear, as the contents of this
field won't be manipulated as a name by \textsf{biblatex}
(lecarre:quest, stendhal:parma).  If you have given the author's real
name in the \textsf{author} field, then the pseudonym goes in
\textsf{nameaddon}, in the form \texttt{Firstname Lastname,\,pseud.}\
(creasey:ashe:blast, creasey:morton:hide, creasey:\\york:death).  This
latter method will allow you to keep all references to one author's
work under different pseudonyms grouped together in the list of
references, a method recommended by the \emph{Manual}.  The 16th
edition of the \emph{Manual} (14.84) has now strengthened its policies
about cross-references from author to pseudonym or vice versa, so in
these latter examples I have included such references from the various
pseudonyms back to the author's name, using the \textsf{customc} entry
type, which see (ashe:creasey, morton:creasey, york:creasey).

\mylittlespace One final piece of advice.  An institutional author's
name, or a journal's name being used in place of an author, can be
rather too long for in-text citations.  In unsigned:ranke I placed an
abbreviated form of the \textsf{journaltitle} into
\textsf{shortauthor}, adapting for a periodical the practice
recommended for books in 15.32.  In iso:electrodoc, I provided a
\textsf{shorthand} field, which by default in
\textsf{biblatex-chicago-authordate} will appear in text citations.
Pursuant to the 16th edition's specifications, this \textsf{shorthand}
will now also appear at the head of the entry in the list of
references, followed, within the entry, by its expansion, this latter
placed within parentheses.  You no longer, therefore, need to use a
\textsf{customc} entry to provide the expansion --- please see
\textsf{shorthand} below for the details.  (You can also still utilize
the list of shorthands to clarify the abbreviation, if you wish.)

\mybigspace In \mymarginpar{\textbf{authortype}}
\textsf{biblatex-chicago}, this field serves a function very much in
keeping with the spirit of standard \textsf{biblatex}, if not with its
letter.  Instead of allowing you to change the string used to identify
an author, the field allows you to indicate when an author is
anonymous, that is, when his or her name doesn't appear on the title
page of the work you are citing.  As I've just detailed under
\textsf{author}, the \emph{Manual} generally discourages the use of
\enquote{Anonymous} (or \enquote{Anon.} as an author, though in some
cases it may well be your best option.  If, however, the name of the
author is known or guessed at, then you're supposed to enclose that
name within square brackets, which is exactly what
\textsf{biblatex-chicago} does for you when you put either
\texttt{anon} (author known) or \texttt{anon?} (author guessed at) in
the \textsf{authortype} field.  (Putting the square brackets in
yourself doesn't work right, hence this mechanism.)  The macros test
for these \emph{exact} strings, so check your typing if you don't see
the brackets.  Assuming the strings are correct,
\textsf{biblatex-chicago} will also automatically do the right thing
in citations.  (See the \textsf{author} docs just above.  Also
\emph{Manual} 15.33; cook:sotweed, horsley:prosodies.)

%%\enlargethispage{-\baselineskip}

\mybigspace For \mymarginpar{\textbf{bookauthor}} the most part, as in
\textsf{biblatex}, a \textsf{bookauthor} is the author of a
\textsf{booktitle}, so that, for example, if one chapter in a book has
different authorship from the book as a whole, you can include that
fact in a reference (will:cohere).  Keep in mind, however, that the
entry type for introductions, forewords and afterwords
(\textsf{suppbook}) uses \textsf{bookauthor} as the author of
\textsf{title} (polakow:afterw, prose:intro).

\mybigspace This, \mymarginpar{\vspace{-12pt}\textbf{bookpagination}}
a standard \textsf{biblatex} field, allows you automatically to prefix
the appropriate string to information you provide in a \textsf{pages}
field.  If you leave it blank, the default is to print no identifying
string (the equivalent of setting it to \texttt{none}), as this is the
practice the \emph{Manual} recommends for nearly all page numbers.
Even if the numbers you cite aren't pages, but it is otherwise clear
from the context what they represent, you can still leave this blank.
If, however, you specifically need to identify what sort of unit the
\textsf{pages} field represents, then you can either hand-format that
field yourself, or use one of the provided bibstrings in the
\textsf{bookpagination} field.  These bibstrings currently are
\texttt{column,} \texttt{line,} \texttt{paragraph,} \texttt{page,}
\texttt{section,} and \texttt{verse}, all of which are used by
\textsf{biblatex's} standard styles.

\mylittlespace There are two points that may need explaining here.
First, all the bibstrings I have just listed follow the Chicago
specification, which may be confusing if they don't produce the
strings you expect.  Second, remember that \textsf{bookpagination}
applies only to the \textsf{pages} field --- if you need to format a
citation's \textsf{postnote} field, then you must use
\textsf{pagination}, which see (10.43--44, 14.154--163).

\mybigspace The \mymarginpar{\textbf{booksubtitle}} subtitle for a
\textsf{booktitle}.  See the next entry for further information.

\mybigspace In \mymarginpar{\textbf{booktitle}} the
\textsf{bookinbook}, \textsf{inbook}, \textsf{incollection},
\textsf{inproceedings}, and \textsf{letter} entry types, the
\textsf{booktitle} field holds the title of the larger volume in which
the \textsf{title} itself is contained as one part.  It is important
not to confuse this with the \textsf{maintitle}, which holds the more
general title of multiple volumes, e.g., \emph{Collected Works}.  It
is perfectly possible for one .bib file entry to contain all three
sorts of title (euripides:orestes, plato:republic:gr).  You may also
find a \textsf{booktitle} in other sorts of entries (e.g.,
\textsf{book} or \textsf{collection}), but there it will almost
invariably be providing information for the traditional
\textsc{Bib}\TeX\ cross-referencing apparatus (prairie:state), which I
discuss below (\textbf{crossref}).  Such provision is unnecessary when
using \textsf{Biber}.  The \textsf{booktitle} no longer takes
sentence-style capitalization in \textsf{authordate}, though it does
in \textsf{authordate-trad}.

%%\enlargethispage{\baselineskip}

\mybigspace An \mymarginpar{\textbf{booktitleaddon}} annex to the
\textsf{booktitle}.  It will be printed in the main text font, without
quotation marks.  If your data begins with a word that would
ordinarily only be capitalized at the beginning of a sentence, then
simply ensure that that word is in lowercase, and
\textsf{biblatex-chicago} will automatically do the right thing.

\mybigspace This \mymarginpar{\textbf{chapter}} field holds the
chapter number, mainly useful only in an \textsf{inbook} or an
\textsf{incollection} entry where you wish to cite a specific chapter
of a book (ashbrook:brain).

\mybigspace I \mymarginpar{\textbf{commentator}} have implemented this
\textsf{biblatex} field pretty much as that package's standard styles
do, even though the \emph{Manual} doesn't actually mention it.  It may
be useful for some purposes.  Cf.\ \textsf{annotator}.

\mybigspace \textsf{Biblatex} \colmarginpar{\textbf{crossref}} uses
the standard \textsc{Bib}\TeX\ cross-referencing mechanism, and has
also introduced a modified one of its own (\textsf{xref}).  The latter
works as it always has, attempting to remedy some of the deficiencies
of the traditional mechanism by ensuring that child entries will
inherit no data at all from their parents.  For the \textsf{crossref}
field, when \textsf{Biber} is the backend, \textsf{biblatex} defines a
series of inheritance rules which make it much more convenient to use.
Appendix B of \textsf{biblatex.pdf} explains the defaults, to which
\textsf{biblatex-chicago} has added several that I should mention
here: \textsf{incollection} entries can now inherit from \textsf{book}
and \textsf{mvbook} just as they do from \textsf{mvcollection}
entries; \textsf{letter} entries now inherit from \textsf{book},
\textsf{collection}, \textsf{mvbook}, and \textsf{mvcollection}
entries the same way an \textsf{inbook} or an \textsf{incollection}
entry would; the \textsf{namea}, \textsf{nameb}, \textsf{sortname},
\textsf{sorttitle}, and \textsf{sortyear} fields, all highly
single-entry specific, are no longer inheritable; and \textsf{date}
and \textsf{origdate} fields are not inheritable from any of the new
\textbf{mv*} entry types.

\mylittlespace Aside from these inheritance questions, the other main
function of the \textsf{crossref} and \textsf{xref} fields in
\textsf{biblatex-chicago} is as a trigger for the provision of
abbreviated entries in the list of references.  The \emph{Manual}
(15.37) specifies that if you cite several contributions to the same
collection, all (including the collection itself) may be listed
separately in the list, which the package does automatically, using
the default inclusion threshold of 2 in the case both of
\textsf{crossref}'ed and \textsf{xref}'ed entries.  (The familiar
\cmd{nocite} command may also help in some circumstances.)  In the
reference list an abbreviated form will be appropriate for all the
child entries.  The \textsf{biblatex-chicago-authordate} package has
always implemented these instructions, but only if you use a
\textsf{crossref} or an \textsf{xref} field, and only in
\textsf{incollection}, \textsf{inproceedings}, or \textsf{letter}
entries (on the last named, see just below).  With this release, I
\colmarginpar{New!} have considerably extended this functionality.

\mylittlespace First, I have added five new entry types ---
\mycolor{\textbf{book}}, \mycolor{\textbf{bookinbook}},
\mycolor{\textbf{collection}}, \mycolor{\textbf{inbook}}, and
\mycolor{\textbf{proceedings}} --- to the list of those which use
shortened cross references, and I have added two new options ---
\mycolor{\texttt{longcrossref}} and \mycolor{\texttt{booklongxref}},
on which more below --- which you can use in the preamble or in the
\textsf{options} field of an entry to enable or disable the automatic
provision of abbreviated references.  (The \textsf{crossref} or
\textsf{xref} field are still necessary for this provision, but they
are no longer sufficient on their own.)  The \textsf{inbook} type
works exactly like \textsf{incollection} or \textsf{inproceedings}; in
previous releases, you could use \textsf{inbook} instead of
\textsf{incollection} to avoid the automatic abbreviation, the two
types being otherwise identical.  Now that you can use an option to
turn off abbreviated references even in the presence of a
\textsf{crossref} or \textsf{xref} field, I have thought it sensible
to include this entry type alongside the others.  (Cf.\ ellet:galena,
keating:dearborn, lippincott:chicago, and prairie:state to see this
mechanism in action in the reference list.)

\mylittlespace The inclusion of \textbf{book}, \textbf{bookinbook},
\textbf{collection}, and \textbf{proceedings} entries fulfills a
request made by Kenneth L. Pearce, and allows you to obtain shortened
references to, for example, separate volumes within a multi-volume
work, or to different book-length works collected inside a single
volume.  Such references are not part of the \emph{Manual's}
specification, but they are a logical extension of it, so the system
of options for turning on this functionality behaves differently for
these four entry types than for the other 4 (see below).  In
\textsf{dates-test.bib} you can get a feel for how this works by
looking at bernhard:boris, bernhard:ritter, bernhard:themacher,
harley:ancient:cart, harley:cartography, and harley:hoc.

%%\enlargethispage{\baselineskip}

\mylittlespace A published collection of letters requires a somewhat
different treatment (15.40).  In the author-date style, the
\emph{Manual} discourages individual letters from appearing in the
list of references at all, preferring that the \enquote{dates of
  individual correspondence should be woven into the text.}  If you
have special reason to do so, however, you can still present
individual published letters there (using the \textsf{letter} entry
type), and they too can use the system of shortened references just
outlined, even though the \emph{Manual} doesn't explicitly require it.
As with \textsf{book}, \textsf{bookinbook}, \textsf{collection},
\textsf{inbook}, \textsf{incollection}, \textsf{inproceedings}, and
\textsf{proceedings} entries, the use of a \textsf{crossref} or
\textsf{xref} field will activate this mechanism, assuming the new
preamble and entry options are set to enable it.  (See
white:ross:memo, white:russ, and white:total, for examples of the
\textsf{xref} field in action in this way, and please note that the
second of these entries is entirely fictitious, provided merely for
the sake of example.)

\mylittlespace These \colmarginpar{\texttt{longcrossref}} new options
function asymmetrically.  The first, \mycolor{\texttt{longcrossref}},
generally controls the settings for the entry types more-or-less
authorized by the \emph{Manual}: \textsf{inbook},
\textsf{incollection}, \textsf{inproceedings}, and \textsf{letter}.

\begin{description}
\item[\qquad false:] This is the default.  If you use
  \textsf{crossref} or \textsf{xref} fields in the four mentioned
  entry types, you'll get the abbreviated entries in the reference
  list.
\item[\qquad true:] You'll get no abbreviated citations of these entry
  types in the reference list.
\item[\qquad none:] This switch is special, allowing you with one
  setting to provide abbreviated citations not just of the four entry
  types mentioned but also of \textsf{book}, \textsf{bookinbook},
  \textsf{collection}, and \textsf{proceedings} entries.
\item[\qquad notes,bib:] These two options are carried over from the
  notes \&\ bibliography style; here they are synonymous with
  \texttt{false} and \textsf{true}, respectively.
\end{description}

The \colmarginpar{\texttt{booklongxref}} second option,
\mycolor{\texttt{booklongxref}}, controls the settings for
\textsf{book}, \textsf{bookinbook}, \textsf{collection}, and
\textsf{proceedings} entries:

\begin{description}
\item[\qquad true:] This is the default.  If you use \textsf{crossref}
  or \textsf{xref} fields in these entry types, by default you will
  \emph{not} get any abbreviated citations in the reference list.
\item[\qquad false:] You'll get abbreviated citations in these entry
  types in the reference list.
\item[\qquad notes,bib:] These two options are carried over from the
  notes \&\ bibliography style; here they are synonymous with
  \texttt{false} and \textsf{true}, respectively.
\end{description}

Please note that you can set both of these options either in the
preamble or in the \textsf{options} field of individual entries,
allowing you to change the settings on an entry-by-entry basis.

\mylittlespace Please \colmarginpar{New!} further note that in
previous releases of \textsf{biblatex-chicago} I recommended against
using \textsf{shorthand}, \textsf{reprinttitle} and/or \textsf{userf}
fields in combination with this abbreviated cross-referencing
mechanism.  I have, however, received a request from Alexandre Roberts
to allow the shorthand to appear in the place of the abbreviated
cross-reference as an additional space-saving measure, and one from
Kenneth Pearce to permit the combination of the other two fields with
\textsf{crossref}, as well.  All three of these fields, in any
combination, should now just work in such circumstances in
\textsf{biblatex-chicago-authordate}, though if you are using a list
of shorthands then you may need to include \texttt{skipbiblist} in the
\textsf{options} field of some entries to avoid duplicates.  If you
come across any problems or inaccuracies, please report them.

\mybigspace Predictably, \colmarginpar{\textbf{date}} this is one of
the key fields for the author-date styles, and one which, as a general
rule, every .bib entry designed for this system ought to contain.  So
important is it, that \textsf{biblatex-chicago-authordate} will, in
most entry types, supply a missing \cmd{bibstring\{nodate\}} if there
is no date otherwise provided (15.41); citations will look like
(Author n.d.), and entries in the list of references will begin:
Author, Firstname.\ n.d.  This seems simple enough, but there are a
surprising number of complications which require attention.

\mylittlespace To start, in each entry, \textsf{Biber} attempts to
find something which it can designate a \textsf{labeldate}, which
will, in general and ideally, be the year printed both in citations
and at the head of the entry in the list of references.  The search
for the \textsf{labeldate} is governed by instances of the declaration
\cmd{DeclareLabeldate}, which cannot be set on an entry-by-entry
basis, but rather only in a document preamble (or in files used by
\textsf{biblatex} or its styles, like \textsf{biblatex-chicago}).  The
declaration can set a different search order according to entry type,
but other differentiations are not currently possible.  In all cases,
guided by the instructions given by the \cmd{DeclareLabeldate}
instances, \textsf{Biber} will search each entry in the declared
order, and the first match will provide the \textsf{labeldate}.  Only
when it finds no match at all will it fall back on
\cmd{bibstring\{nodate\}}.  Now, the entry types in which this
automatic provision is turned off are \textsf{inreference},
\textsf{misc}, and \textsf{reference}, none of which may be expected
in the standard case to have a date provided.  In all other entry
types \enquote{\texttt{n.d.}}\ will appear if no date is provided,
though you can turn this off throughout the document in all entry
types with the option \texttt{nodates=false} when loading
\textsf{biblatex-chicago} in your preamble.  (See
section~\ref{sec:authpreset}, below.  If you wish to provide the
\enquote{\texttt{n.d.}}\ yourself in the \textsf{year} field, please
instead put \cmd{bibstring\{nodate\}} there, as otherwise the
punctuation in citations will come out [subtly] wrong.)

\mylittlespace The thing to keep in mind is that \emph{only} for a
\textsf{labelyear} will \textsf{biblatex} provide what it calls the
\textsf{extrayear} field, which means the alphabetical suffix
(1978\textbf{a}) to differentiate entries with the same author and
year.  A style can print any year it wants in a citation, but only the
\textsf{labelyear} comes equipped with an \textsf{extrayear}.  (It is
also, by the way, the field that the sorting algorithm will use for
ordering the list of references.)  So the challenge, in a style
wherein entries can contain more than one date, is to allow different
dates to appear in citations and at the head of reference list
entries, but to ensure that, as often as is possible, that date
\emph{is} the \textsf{labeldate}.  In previous releases of
\textsf{biblatex-chicago-authordate}, the search for a
\textsf{labeldate} could occur in two possible orders: in
\textsf{music}, \textsf{review}, and \textsf{video} entries, the
default order was \textsf{eventdate, origdate, date, urldate}, while
in all other entry types the order was \textsf{date, eventdate,
  origdate, urldate}.  This, I believe, still works well for reference
lists that contain relatively few entries with multiple dates, where
using, e.g., an \textsf{origdate} at the head of an entry, even if it
wasn't the \textsf{labeldate}, would rarely cause problems, because it
was unlikely that there would be another entry by the same author with
the same date (or \textsf{origdate}) requiring the \textsf{extrayear}
field.  Judging from the feedback I've received, I significantly
overestimated the likelihood that most reference lists would be so
cooperative.  Users could always eliminate some of these dates from
the running, or change the search order, using \cmd{DeclareLabeldate}
in their preamble, but I had to hard-code the default order(s) into
the author-date styles in order to cope with some tricky corners of
the specification.  If users modified \cmd{DeclareLabeldate}, and
their references entered these tricky corners, the results could be
surprising.

\mylittlespace With \colmarginpar{New!} this release, therefore, I
have included a new means of coping with multiple dates in database
entries, while retaining the old mechanisms for those for whom they
work well.  These old mechanisms include the \texttt{avdate} option,
set to \texttt{true} by default, which treats as \emph{sui generis}
\textsf{music}, \textsf{review}, and \textsf{video} entries.  They
have their own rules, and their own version of \cmd{DeclareLabeldate}
(\textsf{eventdate, origdate, date, urldate}), so please see their
documentation above in section~\ref{sec:types:authdate} for the
details of how multiple dates will be treated in such entries, and
also see \texttt{avdate} in section~\ref{sec:authpreset}, below.  If
you don't alter the \textsf{avdate} settings, the other settings I
describe below don't apply to such entries.

\mylittlespace For other entry types, the 16th edition of the
\emph{Manual} (15.38) presents a fairly simple scheme for when a
particular entry has more than one date, but I have been unable to
make its implementation quite as straightforward.  If a reprinted
book, say, has both a \textsf{date} of publication for the reprint
edition and an \textsf{origdate} for the original edition, then by
default \textsf{biblatex-chicago-authordate} will use the
\textsf{date} in citations and at the head of the entry in the
reference list.  If you inform \textsf{biblatex-chicago} that the book
is a reprint by putting the string \texttt{reprint} in the
\textsf{pubstate} field, then a notice will be printed at the end of
the entry saying \enquote{First published 1898.}  With no
\textsf{pubstate} field (and no \texttt{cmsdate} option), the
algorithms will ignore the \textsf{origdate}.

\mylittlespace If, \colmarginpar{\texttt{cmsdate}\\\emph{in entry}}
for any reason, you wish the \textsf{origdate} to appear at the head
of the entry, then your first option is to use the \texttt{cmsdate}
toggle in the \textsf{options} field of the entry itself.  This has 3
possible states relevant to this context, though there is a fourth
state (\texttt{full}) which I shall discuss below:

\begin{enumerate}
\item \texttt{cmsdate=both} prints both the \textsf{origdate} and the
  \textsf{date}, using the \emph{Manual's}\ standard format: (Author
  [1898] 1952) in parenthetical citations, Author (1898) 1952 outside
  parentheses, e.g., in the reference list.
\item \texttt{cmsdate=off} is the default, discussed above:
  (Author 1952).
\item \texttt{cmsdate=on} prints the \textsf{origdate} at the head of
  the entry in the list of references and in citations: (Author 1898).
  NB: The \emph{Manual} no longer includes this among the approved
  options.  If you want to present the \textsf{origdate} at the head
  of an entry, then generally speaking you should probably use
  \texttt{cmsdate=both}.  I have nevertheless retained this option for
  certain cases where it has proved useful.  The 15th-edition options
  \texttt{new} and \texttt{old} now work like \texttt{both}.
\end{enumerate}

In the first and third cases, if you put the string \texttt{reprint}
in the \textsf{pubstate} field, then the publication data in the list
of references will include a notice, formatted according to the
specifications, that the modern edition is a reprint.  In the third
case, since the \textsf{date} hasn't yet been printed, this
publication data will also include the date of the modern reprint.

\mylittlespace Let us imagine, however, that your list of references
contains another book by the same author, also a reprint edition:
(Author [1896] 1974).  How will these two works be ordered in the list
of references?  By the \textsf{labelyear}, in this case the
\textsf{year} field, which appears first in the default definition
(\textsf{date, eventdate, origdate, urldate}) of
\cmd{DeclareLabeldate}, and which in this case will be wrong, because
the entries should always be ordered by the \emph{first} date to
appear there, in this case the contents of \textsf{origdate}.  In this
example, the solution can be as simple as a \textsf{sortyear} field
set to something earlier than the date of the other work, e.g.,
\texttt{1951}.

\mylittlespace And if the reprint dates --- in the \textsf{date} field
--- of the two works were the same?  Just as when it is ordering
entries, \textsf{biblatex} will always first process the contents of
the \textsf{labelyear} field when it is deciding whether to add the
\textsf{extrayear} alphabetical suffix (\texttt{a,b,c} etc.)\ to the
year to distinguish different works by the same author published in
the same year.  Our current hypothetical examples would look like
this: ([1896] 1974a) and ([1898] 1974b), with the suffixes
unnecessary, strictly-speaking, either for ordering or for
disambiguating the entries.  If the original publication dates --- in
the \textsf{origdate} field --- are the same, and the reprint dates
different, you may prefer citations of the two works to read, e.g.,
(Author [1898a] 1952) and (Author [1898b] 1974), when they in fact
read (Author [1898] 1952) and (Author [1898] 1974).  These latter
forms aren't ambiguous, and even if the reprints themselves appeared
in the same year then the alphabetical suffix would appear attached to
the \textsf{date} --- (Author [1898] 1974a) and (Author [1898] 1974b)
--- again avoiding ambiguity.

\mylittlespace The \emph{Manual} doesn't give clear instructions for
how to cope with these situations, but
\textsf{biblatex-chicago-authordate} provides help.  You can't
manually put the alphabetical suffix on an \textsf{origdate} yourself
because that field only accepts numerical data.  Instead, we can now
choose between two solutions.  The old way
\colmarginpar{\texttt{cmsdate}\\\emph{in entry}
  \\+ \texttt{switchdates}} is an unusual expedient, which amounts to
switching the two date fields, placing the earlier date in
\textsf{date} and the later one in \textsf{origdate}.  The style tests
for this condition using a simple arithmetical comparison between the
two years, then prints the two dates according to the state of the
\texttt{cmsdate} toggle.  The three relevant states of this toggle are
the same as before, but there are only two possible outcomes, as
follows:

\begin{enumerate}
\item \texttt{cmsdate=off} (the default) and \texttt{cmsdate=on}
  \emph{both} print the \textsf{date} at the head of the entry in the
  list of references and in citations: (Author 1898a), (Author 1898b).
  As noted above, this style is no longer recommended by the 16th
  edition of the \emph{Manual}, but it is still useful in some cases.
\item \texttt{cmsdate=both} prints both the \textsf{date} and the
  \textsf{origdate}, using the \emph{Manual's}\ preferred format:
  (Author [1898a] 1952), (Author [1898b] 1974).  The 15th-edition
  options \texttt{old} and \texttt{new} are now synonyms for this.
\end{enumerate}

If, for some reason, the automatic switching of the dates cannot be
achieved, perhaps in crossref'd \textsf{letter} entries that you
really want to have in your list of references (white:ross:memo,
white:russ), or perhaps in a reprint edition that hasn't yet appeared
in print (preventing the comparison between a year and the word
\enquote{forthcoming}), then you can use the per-entry option
\texttt{switchdates} in the \textsf{options} field to achieve the
required effects.

\mylittlespace The \mycolor{new}
\colmarginpar{\texttt{cmsdate}\\\emph{in preamble}} method of
simplifying the creation of databases with a great many multi-date
entries is to use the \texttt{cmsdate} option \emph{in the preamble}.
Despite warnings in previous releases, users have plainly already been
setting this option in their preambles, so I thought I might at least
attempt to make it work as \enquote{correctly} as I can.  The switches
for this option are the same as for the entry-only option, that is:

\begin{enumerate}
\item \texttt{cmsdate=off} is the default: (Author 1952).
\item \texttt{cmsdate=both} prints both the \textsf{origdate} and the
  \textsf{date}, using the \emph{Manual's}\ standard format: (Author
  [1898] 1952) in parenthetical citations, Author (1898) 1952 outside
  parentheses, e.g., in the reference list.
\item \texttt{cmsdate=on} prints the \textsf{origdate} at the head of
  the entry in the list of references and in citations: (Author 1898).
  NB: The \emph{Manual} no longer includes this among the approved
  options.  If you want to present the \textsf{origdate} at the head
  of an entry, then generally speaking you should probably use
  \texttt{cmsdate=both}.  I have nevertheless retained this option for
  certain cases where it has proved useful.  The 15th-edition options
  \texttt{new} and \texttt{old} now work like \texttt{both}.
\end{enumerate}

The important change for the user is that, when you set this option in
your preamble to \texttt{on} or \texttt{both} (or to the 15th-edition
synonyms for the latter, \texttt{new} or \texttt{old}), then
\textsf{biblatex-chicago-authordate} will now change the default
\cmd{DeclareLa\-bel\-date} definition so that the \textsf{labelyear}
search order will be \mycolor{\textsf{origdate, date, eventdate,
    urldate}}.  This means that for entry types not covered by the
\texttt{avdate} option, and for those types as well if you turn off
that option, the \textsf{labelyear} will, in any entry containing an
\textsf{origdate}, be that very date.  If you want \emph{every} such
entry to present its \textsf{origdate} in citations and at the head of
reference list entries, then setting the option this way makes sense,
as you should automatically get the proper \textsf{extrayear} and the
correct sorting, without having to switch dates around
counter-intuitively in your .bib file.  A few clarifications may yet
be in order.

\mylittlespace Obviously, any entry with only a \textsf{date} should
behave as usual.  Also, since \textsf{patent} entries have fairly
specialized needs, I have exempted them from this change to
\cmd{DeclareLabeldate}.  Third, the per-entry \texttt{cmsdate} options
will still affect which dates are printed in citations and at the head
of reference list entries, but they cannot change the search order for
the \textsf{labeldate}.  This will be fixed by the preamble option.
Fourth, if you have been used to switching the \textsf{date} and the
\textsf{origdate} to get the correct results, then you should be aware
that this mechanism may actually still be useful when using the
\texttt{on} switch to \texttt{cmsdate} in the preamble, but it
produces incorrect results when the \texttt{cmsdate} option is
\texttt{both} in the preamble and the individual entry.  The preamble
option is designed to make the need for this switching as rare as
possible, so some editing of existing databases may be necessary.

\mylittlespace Finally, Bertold Schweitzer has brought to my attention
certain difficult corner cases involving cross-referenced works with
more than one date.  In order to facilitate the accurate presentation
of such sources, I made a slight change to the way the entry-only
\texttt{cmsdate=on} and \texttt{cmsdate=both} work.  If, and only if,
a work has only one date, and there is no \texttt{switchdates} in the
\textsf{options} field, then \texttt{cmsdate=on} and
\texttt{cmsdate=both} will both result in the suppression of the
\textsf{extrayear} field in that entry, that is, the year will no
longer be printed with its following lowercase letter used to
distinguish works by the same \textsf{author} published in the same
year.  Obviously, if the same options are set in the preamble, this
behavior is turned off, so that single-date entries will still work
properly without manual intervention.

\mylittlespace There are several more general remarks about the
\textsf{date} field that may be helpful to users.  First, for most
entry types, only a year is really necessary, and in most situations
only the year --- or year range --- will be printed in text citations
and at the head of entries in the list of references.  More specific
\textsf{date} fields are often present, however, in \textsf{article},
\textsf{misc}, \textsf{music}, \textsf{online}, \textsf{patent},
\textsf{unpublished}, and \textsf{video} entries, for all of which any
day or month provided will be printed later in the reference list
entry.  If you follow the recommendations of the 16th edition of the
\emph{Manual} and present newspaper and magazine articles
\enquote{entirely within the text} (15.47), then the citations need to
contain the complete \textsf{date} along with the
\textsf{journaltitle}.  Placing \mymarginpar{\texttt{cmsdate=full}}
\texttt{cmsdate=full} (and \texttt{skipbib}) in the \textsf{options}
field of an \textsf{article} or a \textsf{review} entry, alongside a
possible \texttt{useauthor=false}, should allow you to achieve this.
While we're on this subject, the \emph{Manual} is flexible (in both
specifications) on abbreviating the names of months (14.180).  By
default, \textsf{biblatex-chicago-authordate} uses the full names,
which you can change by setting the option \texttt{dateabbrev=true} in
your document preamble.  (Cf.\ assocpress:gun, barcott:review, batson,
creel:house, friends:leia, holiday:fool, nass:address,
petroff:\-impurity, powell:email.)

\mylittlespace Second, when you need to indicate that a work is
\enquote{\texttt{forthcoming},} the \textsf{year} field, instead of
the \textsf{date} field, is the place for it, though you should use
the \cmd{autocap} macro there to make sure the word comes out
correctly in both citations and the list of references.  The reason
for the field switch is that the \textsf{date} field accepts only
numerical data, in \textsc{iso}8601 format (\texttt{yyyy-mm-dd}),
whereas \textsf{year} can, conveniently, hold just about anything.
Third, it may be worth noting here that \textsf{Biber} is somewhat
more exacting when parsing the \textsf{date} field than
\textsc{Bib}\TeX, so a field looking like \texttt{1968/75} will simply
be ignored, producing \enquote{\texttt{n.d.}}\ in the output --- you
need \texttt{1968/1975} instead.  If you want a more compressed year
range, then you'll want to use the \textsf{year} field.

%%\enlargethispage{\baselineskip}

\mylittlespace Fourth, in the \textsf{misc} entry type the
\textsf{date} field can help to distinguish between two classes of
archival material, letters and \enquote{letter-like} sources using
\textsf{origdate} while others (interviews, wills, contracts) use
\textsf{date}.  (See \textsf{misc} in section~\ref{sec:types:authdate}
for the details.)  If such an entry, as may well occur, contains only
an \textsf{origdate}, as can also be the case in \textsf{letter}
entries, then \textsf{Biber} and either \cmd{DeclareLabeldate}
definition will make it work without further intervention.  Fifth, and
finally, with \colmarginpar{New!} this release, you can now in most
entry types qualify a \textsf{date} with the \textsf{userd} field,
assuming that the entry contains no \textsf{urldate}.  For
\textsf{music} and \textsf{video} entries, there are several other
requirements --- please see the documentation of \textsf{userd},
below.

\mylittlespace I recommend that you have a look through
\textsf{dates-test.bib} to see how all these complications will affect
the construction of your .bib database, especially at the following:
aristotle:metaphy:gr, creel:house, emerson:nature,
james:ambas\-sa\-dors, mait\-land:canon, mait\-land:equity,
schweit\-zer:bach, spock:in\-terview, white:\\ross:me\-mo, and
white:russ.  Cf.\ also \textsf{origdate} and \textsf{year}, below; the
\texttt{cmsdate}, \texttt{nodates}, and \texttt{switchdates} options
in sections~\ref{sec:preset:authdate}, \ref{sec:authuseropts}, and
\ref{sec:authentryopts}; and section~4.5.8 in \textsf{biblatex.pdf},
and section~\ref{sec:authformopts}, below, for the
\cmd{DeclareLabeldate} command


\mybigspace This \mymarginpar{\textbf{day}} field, as of
\textsf{biblatex} 0.9, is obsolete, and will be ignored if you use it
in your .bib files.  Use \textsf{date} instead.

\mybigspace Standard \mymarginpar{\textbf{doi}} \textsf{biblatex}
field, providing the Digital Object Identifier of the work.  The 16th
edition of the \emph{Manual} specifies that, given their relative
permanence compared to URLs, \enquote{authors should include DOIs
  rather than URLs for sources that make them readily available}
(14.6; cf.\ 15.9).  (14.184; friedman:learn\-ing).  Cf.\ \textsf{url}.

%%\enlargethispage{-\baselineskip}

\mybigspace Standard \mymarginpar{\textbf{edition}} \textsf{biblatex}
field.  If you enter a plain cardinal number, \textsf{biblatex} will
convert it to an ordinal (chicago:manual), followed by the appropriate
string.  Any other sort of edition information will be printed as is,
though if your data begins with a word (or abbreviation) that would
ordinarily only be capitalized at the beginning of a sentence, then
simply ensure that that word (or abbreviation) is in lowercase, and
\textsf{biblatex-chicago} will automatically do the right thing
(babb:peru, times:guide).  In most situations, the \emph{Manual}
generally recommends the use of abbreviations in the list of
references, but there is room for the user's discretion in specific
citations (emerson:nature).

\mybigspace As \mymarginpar{\textbf{editor}} far as possible, I have
implemented this field as \textsf{biblatex}'s standard styles do, but
the requirements specified by the \emph{Manual} present certain
complications that need explaining.  Lehman points out in his
documentation that the \textsf{editor} field will be associated with a
\textsf{title}, a \textsf{booktitle}, or a \textsf{maintitle},
depending on the sort of entry.  More specifically,
\textsf{biblatex-chicago} associates the \textsf{editor} with the most
comprehensive of those titles, that is, \textsf{maintitle} if there is
one, otherwise \textsf{booktitle}, otherwise \textsf{title}, if the
other two are lacking.  In a large number of cases, this is exactly
the correct behavior (adorno:benj, centinel:letters,
plato:republic:gr, among others).  Predictably, however, there are
numerous cases that require, for example, an additional editor for one
part of a collection or for one volume of a multi-volume work.  For
these cases I have provided the \textsf{namea} field.  You should
format names for this field as you would for \textsf{author} or
\textsf{editor}, and these names will always be associated with the
\textsf{title} (donne:var).

\mylittlespace As you will see below, I have also provided a
\textsf{nameb} field, which holds the translator of a given
\textsf{title} (euripides:orestes).  If \textsf{namea} and
\textsf{nameb} are the same, \textsf{biblatex-chicago} will
concatenate them, just as \textsf{biblatex} already does for
\textsf{editor}, \textsf{translator}, and \textsf{namec} (i.e., the
compiler).  Furthermore, it is conceivable that a given entry will
need separate editors for each of the three sorts of title.  For this,
and for various other tricky situations, there is the \cmd{partedit}
macro (and its siblings), designed to be used in a \textsf{note}
field, in one of the \textsf{titleaddon} fields, or even in a
\textsf{number} field (howell:marriage).  (Because the strings
identifying an editor differ in notes and bibliography, one can't
simply write them out in such a field when using the notes \&\
bibliography style, but you can certainly do so in the author-date
styles, if you wish.  Using the macros will make your .bib file more
portable across both Chicago specifications, and also across multiple
languages, but they are otherwise unnecessary.
Cf. section~\ref{sec:international}, and also \textsf{namea},
\textsf{nameb}, \textsf{namec}, and \textsf{translator}.)

\mybigspace The \mymarginpar{\textbf{editora\\editorb\\editorc}} newer
releases of \textsf{biblatex} provide these fields as a means to
specify additional contributors to texts in a number of editorial
roles.  In the Chicago styles they seem most relevant for the
audiovisual types, especially \textsf{music} and \textsf{video}, where
they help to identify conductors, directors, producers, and
performers.  To specify the role, use the fields \textsf{editoratype},
\textsf{editorbtype}, and \textsf{editorctype}, which see.  (Cf.\
bernstein:shostakovich, handel:messiah.)

\mybigspace Normally, \mymarginpar{\textbf{editortype}} with the
exception of the \textsf{article} and \textsf{review} types with a
\texttt{magazine} \textsf{entrysubtype},
\textsf{biblatex-chicago-authordate} will automatically find a name to
put at the head of an entry, starting with an \textsf{author}, and
proceeding in order through \textsf{editor}, \textsf{translator}, and
\textsf{namec} (the compiler).  If all four are missing, then the
\textsf{title} will be placed at the head.  (In \textsf{article} and
\textsf{review} entries with a \texttt{magazine}
\textsf{entrysubtype}, a missing author immediately prompts the use of
\textsf{journaltitle} at the head of an entry.  See above under
\textsf{article} for details.)  The \textsf{editortype} field provides
even greater flexibility, allowing you to choose from a variety of
editorial roles while only using the \textsf{editor} field.  You can
do this even though an author is named (eliot:pound shows this
mechanism in action for a standard editor, rather than for some other
role).  Two things are necessary for this to happen.  First, in the
\textsf{options} field you need to set \texttt{useauthor=false} (if
there is an \textsf{author}), then you need to put the name you wish
to see at the head of your entry into the \textsf{editor} or the
\textsf{namea} field.  If the \enquote{editor} is in fact, e.g., a
compiler, then you need to put \texttt{compiler} into the
\textsf{editortype} field, and \textsf{biblatex} will print the
correct string after the name in the list of references.

%\enlargethispage{\baselineskip}

\mylittlespace There are a few details of which you need to be aware.
Because \textsf{biblatex-chicago} has added the \textsf{namea} field,
which gives you the ability to identify the editor specifically of a
\textsf{title} as opposed to a \textsf{maintitle} or a
\textsf{booktitle}, the \textsf{editortype} mechanism checks first to
see whether a \textsf{namea} is defined.  If it is, that name will be
used at the head of the entry, if it isn't it will go ahead and look
for an \textsf{editor}.  \textsf{Biblatex}'s sorting algorithms, and
also its \textsf{labelname} mechanism, should both work properly no
matter sort of name you provide, thanks to \textsf{Biber} and the
(default) Chicago-specific definitions of \cmd{DeclareLabelname} and
\cmd{DeclareSortingScheme}.  (Cf.\ section~\ref{sec:authformopts},
below).  If, however, the \textsf{namea} field provides the name, and
that name isn't automatically shortened properly by \textsf{biblatex},
then your .bib entry will need to have a \textsf{shorteditor} defined
to help with in-text citations, not a \textsf{shortauthor}, possibly
ruled out because \texttt{useauthor=false}.

\mylittlespace In \textsf{biblatex} 0.9 Lehman reworked the string
concatenation mechanism, for reasons he outlines in his RELEASE file,
and I have followed his lead.  In short, if you define the
\textsf{editortype} field, then concatenation is turned off, even if
the name of the \textsf{editor} matches, for example, that of the
\textsf{translator}.  In the absence of an \textsf{editortype}, the
usual mechanisms remain in place, that is, if the \textsf{editor}
exactly matches a \textsf{translator} and/or a \textsf{namec}, or
alternatively if \textsf{namea} exactly matches a \textsf{nameb}
and/or a \textsf{namec}, then \textsf{biblatex} will print the
appropriate strings.  The \emph{Manual} specifically (15.7) recommends
not using these identifying strings in citations, and
\textsf{biblatex-chicago-authordate} follows that recommendation.  If
you nevertheless need to provide such a string, you'll have to do it
manually in the \textsf{shorteditor} field, or perhaps, in a different
sort of entry, in a \textsf{shortauthor} field.

\mylittlespace It may also be worth noting that because of certain
requirements in the specification -- absence of an \textsf{author},
for example -- the \texttt{useauthor} mechanism won't work properly in
the following entry types: \textsf{collection}, \textsf{letter},
\textsf{patent}, \textsf{periodical}, \textsf{proceedings},
\textsf{suppbook}, \textsf{suppcollection}, and
\textsf{suppperiodical}.

\mybigspace These
\mymarginpar{\textbf{editoratype\\editorbtype\\editorctype}} fields
identify the exact role of the person named in the corresponding
\textsf{editor[a-c]} field.  Note that they are not part of the string
concatenation mechanism.  I have implemented them just as the standard
styles do, and they have now found a use particularly in
\textsf{music} and \textsf{video} entries.  Cf.\
bernstein:shostakovich, handel:messiah.

\mybigspace Standard \mymarginpar{\textbf{eid}} \textsf{biblatex}
field, providing a string or number some journals use uniquely to
identify a particular article.  Only applicable to the
\textsf{article} entry type.  Not typically required by the
\emph{Manual}.

\mybigspace Standard \mymarginpar{\textbf{entrysubtype}} and very
powerful \textsf{biblatex} field, left undefined by the standard
styles.  In \textsf{biblatex-chicago-authordate} it has four very
specific uses, the first three of which I have designed in order to
maintain, as much as possible, backward compatibility with the
standard styles.  First, in \textsf{article} and \textsf{periodical}
entries, the field allows you to differentiate between scholarly
\enquote{journals,} on the one hand, and \enquote{magazines} and
\enquote{newspapers} on the other.  Usage is fairly simple: you need
to put the exact string \texttt{magazine} into the
\textsf{entrysubtype} field if you are citing one of the latter two
types of source, whereas if your source is a \enquote{journal,} then
you need do nothing.

\mylittlespace The second use involves references to works from
classical antiquity and, according to the \emph{Manual}, from the
Middle Ages, as well.  When you cite such a work using the traditional
divisions into books, sections, lines, etc., divisions which are
presumed to be the same across all editions, then you need to put the
exact string \texttt{classical} into the \textsf{entrysubtype} field.
This has no effect in the list of references, which will still present
the particular edition you are using, but it does affect the
formatting of in-text citations, in two ways.  First, it suppresses
some of the punctuation.  Second, and more importantly, it suppresses
the \textsf{date} field in favor of the \textsf{title}, so that
citations look like (Aristotle \emph{Metaphysics} 3.2.996b5--8)
instead of (Aristotle 1997, 3.2.996b5--8).  This mechanism may also
prove useful in \textsf{misc} entries for citations from the Bible or
other sacred texts (cf.\ genesis), and for citing archival collections
(house:papers), where it produces citations of the form (House
Papers).  (Cf.\ the next but one paragraph.)

\mylittlespace If you wish to reference a classical or medieval work
by the page numbers of a particular, non-standard edition, then you
shouldn't use the \texttt{classical} \textsf{entrysubtype} toggle.
Also, and the specification isn't entirely clear about this, works
from the Renaissance and later, even if cited by the traditional
divisions, seem to have citations formatted normally, and therefore
don't need an \textsf{entrysubtype} field.  (See \emph{Manual}
14.256--268; aristotle:metaphy:gr, plato:republic:gr;
euripides:orestes is an example of a translation cited by page number
in a modern edition.)

\mylittlespace The third use of the \textsf{entrysubtype} field occurs
in \textsf{misc} entries.  If such an entry contains no such field,
then the citation will be treated just as the standard
\textsf{biblatex} styles would, including the use of italics for the
\textsf{title}.  Any string at all in \textsf{entrysubtype} tells
\textsf{biblatex-chicago} to treat the source as part of an
unpublished archive.  Please see section~\ref{sec:types:authdate}
above under \textbf{misc} for all the details on how these citations
work.

% %\enlargethispage{\baselineskip}

\mylittlespace Fourth, the field can be defined in the
\textsf{artwork} entry type in order to refer to a work from antiquity
whose title you do not wish to be italicized.  Please see the
documentation of \textsf{artwork} above for the details.  (In previous
releases, there was a special \texttt{tv} \textsf{entrysubtype} for
\textsf{video} entries.  This is no longer necessary.  Please see the
documentation of \textsf{video} in section~\ref{sec:types:authdate}
above, and that of \textsf{userd} below.)

\mybigspace Kazuo
\mymarginpar{\textbf{eprint}\\\textbf{eprintclass}\\\textbf{eprinttype}}
Teramoto suggested adding \textsf{biblatex's} excellent
\textsf{eprint} handling to \textsf{biblatex-chicago}, and he sent me
a patch implementing it.  With minor alterations, I have applied it to
this release, so these three fields now work more or less as they do
in standard \textsf{biblatex}.  They may prove helpful in providing
more abbreviated references to online content than conventional URLs,
though I can find no specific reference to them in the \emph{Manual}.

\mybigspace This \mymarginpar{\textbf{eventdate}} is a standard
\textsf{biblatex} field.  In the 15th edition it was barely used, but
in order to comply with changes in the 16th edition of the
\emph{Manual} it will now play a significant role in \textsf{music},
\textsf{review}, and \textsf{video} entries.  In \textsf{music}
entries, it identifies the recording or performance date of a
particular song (rather than of a whole disc, for which you would use
\textsf{origdate}), whereas in \textsf{video} entries it identifies
either the original broadcast date of a particular episode of a TV
series or the date of a filmed musical performance.  In both these
cases \textsf{biblatex-chicago} will automatically prepend a bibstring
--- \texttt{recorded} and \texttt{aired}, respectively --- to the
date, but you can change this string using the new \textsf{userd}
field, something you'll definitely want to do for filmed musical
performances (friends:leia, handel:messiah, holiday:fool).

\mylittlespace In the default configuration of \cmd{DeclareLabeldate},
dates for citations and for the head of reference list entries are
searched for in the order \textsf{date, eventdate, origdate, urldate}.
This suits the Chicago author-date styles very well, except for
\textsf{music} and \textsf{video} entries, where the general rule is
to emphasize the earliest date, whether that be, for example, the
recording date or original release date (15.53).  For these two entry
types, then, \cmd{DeclareLabeldate} uses the order \textsf{eventdate,
  origdate, date, urldate}.  (See the \texttt{avdate} option in
section~\ref{sec:authpreset}, below.)

\mylittlespace For \textsf{review} entries I use the same, custom
definition of \cmd{DeclareLabeldate}, but for somewhat different
reasons.  In general, such an entry will only have a \textsf{date},
but an \textsf{eventdate} can be used to identify a particular comment
within an online thread.  The year of the comment will therefore
appear at the head of the entry and in citations, while the remainder
of the \textsf{eventdate} will appear just after the \textsf{title},
and the \textsf{date} after the \textsf{journaltitle}.  There isn't a
particular string associated with the \textsf{eventdate}, but you can
further specify a comment by placing a time\-stamp in parentheses in
the \textsf{nameaddon} field, in case the date alone isn't enough
(14.246; ac:comment, ellis:blog).

\mybigspace As \mymarginpar{\textbf{foreword}} with the
\textsf{afterword} field above, \textsf{foreword} will in general
function as it does in standard \textsf{biblatex}.  Like
\textsf{afterword} (and \textsf{introduction}), however, it has a
special meaning in a \textsf{suppbook} entry, where you simply need to
define it somehow (and leave \textsf{afterword} and
\textsf{introduction} undefined) to make a foreword the focus of a
citation.

\mybigspace A \mymarginpar{\textbf{holder}} standard \textsf{biblatex}
field for identifying a \textsf{patent}'s holder(s), if they differ
from the \textsf{author}.  The \emph{Manual} has nothing to say on the
subject, but \textsf{biblatex-chicago} prints it (them), in
parentheses, just after the author(s).

%%\enlargethispage{\baselineskip}

\mybigspace Standard \mymarginpar{\textbf{howpublished}}
\textsf{biblatex} field, mainly applicable in the \textsf{booklet}
entry type, where it replaces the \textsf{publisher}.  I have also
retained it in the \textsf{misc} and \textsf{unpublished} entry types,
for historical reasons.

\mybigspace Standard \mymarginpar{\textbf{institution}}
\textsf{biblatex} field.  In the \textsf{thesis} entry type, it will
usually identify the university for which the thesis was written,
while in a \textsf{report} entry it may identify any sort of
institution issuing the report.

\mybigspace As \mymarginpar{\textbf{introduction}} with the
\textsf{afterword} and \textsf{foreword} fields above,
\textsf{introduction} will in general function as it does in standard
\textsf{biblatex}.  Like those fields, however, it has a special
meaning in a \textsf{suppbook} entry, where you simply need to define
it somehow (and leave \textsf{afterword} and \textsf{foreword}
undefined) to make an introduction the focus of a citation.

\mybigspace Standard \mymarginpar{\textbf{isbn}} \textsf{biblatex}
field, for providing the International Standard Book Number of a
publication.  Not typically required by the \emph{Manual}.

\mybigspace Standard \mymarginpar{\textbf{isrn}} \textsf{biblatex}
field, for providing the International Standard Technical Report
Number of a report.  Only relevant to the \textsf{report} entry type,
and not typically required by the \emph{Manual}.

\mybigspace Standard \mymarginpar{\textbf{issn}} \textsf{biblatex}
field, for providing the International Standard Serial Number of a
periodical in an \textsf{article} or a \textsf{periodical} entry.  Not
typically required by the \emph{Manual}.

\mybigspace Standard \mymarginpar{\textbf{issue}} \textsf{biblatex}
field, designed for \textsf{article} or \textsf{periodical} entries
identified by something like \enquote{Spring} or \enquote{Summer}
rather than by the usual \textsf{month} or \textsf{number} fields
(brown:bremer).

\mybigspace The \mymarginpar{\textbf{issuesubtitle}} subtitle for an
\textsf{issuetitle} --- see next entry.

\mybigspace Standard \mymarginpar{\textbf{issuetitle}}
\textsf{biblatex} field, intended to contain the title of a special
issue of any sort of periodical.  If the reference is to one article
within the special issue, then this field should be used in an
\textsf{article} entry (conley:fifthgrade), whereas if you are citing
the entire issue as a whole, then it would go in a \textsf{periodical}
entry, instead (good:wholeissue).  The \textsf{note} field is the
proper place to identify the type of issue, e.g.,\ \texttt{special
  issue}, with the initial letter lower-cased to enable automatic
contextual capitalization.

\mybigspace The \mymarginpar{\textbf{journalsubtitle}} subtitle for a
\textsf{journaltitle} --- see next entry.

\mybigspace Standard \mymarginpar{\textbf{journaltitle}}
\textsf{biblatex} field, replacing the standard \textsc{Bib}\TeX\
field \textsf{journal}, which, however, still works as an alias.  It
contains the name of any sort of periodical publication, and is found
in the \textsf{article} and \textsf{review} entry types.  In the case
where a piece in an \textsf{article} or \textsf{review}
(\textsf{entrysubtype} \texttt{magazine}) doesn't have an author,
\textsf{biblatex-chicago} provides for this field to be used as the
author.  See above (section~\ref{sec:fields:authdate}) under
\textbf{article} for details.  The lakeforester:pushcarts and
nyt:trevorobit entries in \textsf{dates-test.bib} will give you some
idea of how this works.

\mybigspace This \mymarginpar{\textbf{keywords}} field is
\textsf{biblatex}'s extremely powerful and flexible technique for
filtering entries in a list of references, allowing you to subdivide
it according to just about any criteria you care to invent.  See
\textsf{biblatex.pdf} (3.11.4) for thorough documentation.  In
\textsf{biblatex-chicago}, the field provides one convenient means to
exclude certain entries from making their way into a list of
references, though the toggle \texttt{skipbib} in the \textsf{options}
field works just as well, and perhaps more simply.  There are a few
reasons for so excluding entries.  When citing both an original text
and its translation (see \textbf{userf}, below), the \emph{Manual}
(14.109) suggests including the original at the end of the
translation's reference list entry, a procedure which requires that
the original not also be printed as a separate entry
(furet:passing:eng, furet:passing:fr, aristotle:metaphy:trans,
aristotle:metaphy:gr).  Well-known reference works (like the
\emph{Encyclopaedia Britannica}, for example) and many sacred texts
need only be presented in citations, and not in the list of references
(14.247--248; ency:britannica, genesis, wikiped:bibtex; see
\textsf{inreference} and \textsf{misc}, above).

\mybigspace A \mymarginpar{\textbf{language}} standard
\textsf{biblatex} field, designed to allow you to specify the
language(s) in which a work is written.  As a general rule, the
Chicago style doesn't require you to provide this information, though
it may well be useful for clarifying the nature of certain works, such
as bilingual editions, for example.  There is at least one situation,
however, when the \emph{Manual} does specify this data, and that is
when the title of a work is given in translation, even though no
translation of the work has been published, something that might
happen when a title is in a language deemed to be unparseable by a
majority of your expected readership (14.108, 14.110, 14.194;
chu:panda, pirumova, rozner:liberation).  In such a case, you should
provide the language(s) involved using this field, connecting multiple
languages using the keyword \texttt{and}.  (I have retained
\textsf{biblatex's} \cmd{bibstring} mechanism here, which means that
you can use the standard bibstrings or, if one doesn't exist for the
language you need, just give the name of the language, capitalized as
it should appear in your text.  You can also mix these two modes
inside one entry without apparent harm.)

\mylittlespace An alternative arrangement suggested by the
\emph{Manual} is to retain the original title of a piece but then to
provide its translation, as well.  If you choose this option, you'll
need to make use of the \textbf{usere} field, on which see below.  In
effect, you'll probably only ever need to use one of these two fields
in any given entry, and in fact \textsf{biblatex-chicago} will only
print one of them if both are present, preferring \textsf{usere} over
\textsf{language} for this purpose (see kern, pirumova:russian, and
weresz).  Note also that both of these fields are universally
associated with the \textsf{title} of a work, rather than with a
\textsf{booktitle} or a \textsf{maintitle}.  If you need to attach a
language or a translation to either of the latter two, you could
probably manage it with special formatting inside those fields
themselves.

%%\enlargethispage{\baselineskip}

\mybigspace I \mymarginpar{\textbf{lista}} intend this field
specifically for presenting citations from reference works that are
arranged alphabetically, where the name of the article rather than a
page or volume number should be given.  The field is a
\textsf{biblatex} list, which means you should separate multiple items
with the keyword \texttt{and}.  Each item receives its own set of
quotation marks, and the whole list will be prefixed by the
appropriate string (\enquote{s.v.,} \emph{sub verbo}, pl.\
\enquote{s.vv.}).  \textsf{Biblatex-chicago} will only print such a
field in a \textsf{book} or an \textsf{inreference} entry, and you
should look at the documentation of these entry types for further
details.  (See \emph{Manual} 14.247--248; grove:sibelius, times:guide,
wikiped:bibtex.)

\mybigspace This \mymarginpar{\textbf{location}} is
\textsf{biblatex}'s version of the usual \textsc{Bib}\TeX\ field
\textsf{address}, though the latter is accepted as an alias if that
simplifies the modification of older .bib files.  According to the
\emph{Manual} (14.135), a citation usually need only provide the first
city listed on any title page, though a list of cities separated by
the keyword \enquote{\texttt{and}} will be formatted appropriately.
If the place of publication is unknown, you can use
\cmd{autocap\{n\}.p.}\ instead (14.138).  For all cities, you should
use the common English version of the name, if such exists (14.137).

%%\enlargethispage{\baselineskip}

\mylittlespace Two other uses need explanation here.  In
\textsf{article}, \textsf{periodical}, and \textsf{review} entries,
there is usually no need for a \textsf{location} field, but
\enquote{if a journal might be confused with another with a similar
  title, or if it might not be known to the users of a bibliography,}
then this field can present the place or institution where it is
published (14.191, 14.203; garrett, kimluu:diethyl, and
lakeforester:pushcarts).  For blogs cited using \textsf{article}
entries, this is a good place to identify the nature of the source ---
i.e., the word \enquote{blog} --- letting the style automatically
provide the parentheses (14.246; ellis:blog).

\mybigspace The \mymarginpar{\textbf{mainsubtitle}} subtitle for a
\textsf{maintitle} --- see next entry.

\mybigspace The \mymarginpar{\textbf{maintitle}} main title for a
multi-volume work, e.g., \enquote{Opera} or \enquote{Collected Works.}
It no longer takes sentence-style capitalization in
\textsf{authordate}, though it does in \textsf{authordate-trad}.  In
cross references produced using the \textsf{crossref} field, the
\textsf{title} of \textbf{mv*} entry types always becomes a
\textsf{maintitle} in the child entry.  (See donne:var,
euripides:\-orestes, harley:cartography, lach:asia,
pelikan:chris\-tian, and plato:re\-public:gr.)

\mybigspace An \mymarginpar{\textbf{maintitleaddon}} annex to the
\textsf{maintitle}, for which see previous entry.  Such an annex would
be printed in the main text font.  If your data begins with a word
that would ordinarily only be capitalized at the beginning of a
sentence, then simply ensure that that word is in lowercase, and
\textsf{biblatex-chicago} will automatically do the right thing.

\mybigspace Standard \mymarginpar{\textbf{month}} \textsf{biblatex}
field, containing the month of publication.  This should be an
integer, i.e., \texttt{month=\{3\}} not \texttt{month=\{March\}}.  See
\textsf{date} for more information.

\mybigspace This \mymarginpar{\textbf{namea}} is one of the fields
\textsf{biblatex} provides for style writers to use, but which it
leaves undefined itself.  In \textsf{biblatex-chicago} it contains the
name(s) of the editor(s) of a \textsf{title}, if the entry has a
\textsf{booktitle} or \textsf{maintitle}, or both, in which situation
the \textsf{editor} would be associated with one of these latter
fields (donne:var).  (In \textsf{article} and \textsf{review} entries,
\textsf{namea} applies to the \textsf{title} instead of the
\textsf{issuetitle}, should the latter be present.)  You should
present names in this field exactly as you would those in an
\textsf{author} or \textsf{editor} field, and the package will
concatenate this field with \textsf{nameb} if they are identical.  See
under \textbf{editor} and \textbf{editortype} above for the full
details.  Please note that, as the field is highly single-entry
specific, \textsf{namea} isn't inherited from a \textsf{crossref}'ed
parent entry.  Cf.\ also \textsf{nameb}, \textsf{namec},
\textsf{translator}, and the macros \cmd{partedit}, \cmd{parttrans},
\cmd{parteditandtrans}, \cmd{partcomp}, \cmd{parteditandcomp},
\cmd{parttransandcomp}, and \cmd{partedittransand\-comp}, for which
see section~\ref{sec:formatting:authdate}.

\mybigspace This \mymarginpar{\textbf{nameaddon}} field is provided
by \textsf{biblatex}, though not used by the standard styles.  In
\textsf{biblatex-chicago}, it allows you to specify that an author's
name is a pseudo\-nym, or to provide either the real name or the
pseudonym itself, if the other is being provided in the
\textsf{author} field.  The abbreviation
\enquote{\texttt{pseud.}\hspace{-2pt}}\ (always lowercase in English)
is specified, either on its own or after the pseudonym
(centinel:letters, creasey:ashe:blast, creasey:morton:hide,
creasey:york:death, and le\-carre:quest); \cmd{bibstring\{pseudonym\}}
does the work for you.  See under \textbf{author} above for the full
details.

\mylittlespace In \textsf{review} entries, I have removed the
automatic provision of square brackets from the field, allowing it to
be used in at least two ways.  First, if you provide your own square
brackets, then it can have its standard function, as above.  Second,
and new to the 16th edition of the \emph{Manual}, you can further
specify comments to blogs and other online content using a timestamp
(in parentheses) that supplements the \textsf{eventdate}, particularly
when the latter is too coarse a specification to identify a comment
unambiguously.  Cf.\ ac:comment.

\mylittlespace In the \textsf{customc} entry type, finally, which is
used to create alphabetized cross-references to other entries in the
reference list, the \textsf{nameaddon} field allows you to change the
default string linking the two parts of the cross-reference.  The code
automatically tests for a known bibstring, which it will italicize.
Otherwise, it prints the string as is.

\mybigspace Like \mymarginpar{\textbf{nameb}} \textsf{namea}, above,
this is a field left undefined by the standard \textsf{biblatex}
styles.  In \textsf{biblatex-chicago}, it contains the name(s) of the
translator(s) of a \textsf{title}, if the entry has a
\textsf{booktitle} or \textsf{maintitle}, or both, in which situation
the \textsf{translator} would be associated with one of these latter
fields (euripides:orestes).  (In \textsf{article} and \textsf{review}
entries, \textsf{nameb} applies to the \textsf{title} instead of the
\textsf{issuetitle}, should the latter be present.)  You should
present names in this field exactly as you would those in an
\textsf{author} or \textsf{translator} field, and the package will
concatenate this field with \textsf{namea} if they are identical.  See
under the \textbf{translator} field below for the full details.
Please note that, as the field is highly single-entry specific,
\textsf{nameb} isn't inherited from a \textsf{crossref}'ed parent
entry.  Cf.\ also \textsf{namea}, \textsf{namec},
\textsf{origlanguage}, \textsf{translator}, \textsf{userf} and the
macros \cmd{partedit}, \cmd{parttrans}, \cmd{parteditandtrans},
\cmd{partcomp}, \cmd{parteditandcomp}, \cmd{parttransandcomp}, and
\cmd{partedittransandcomp} in section~\ref{sec:formatting:authdate}.

\mybigspace The \mymarginpar{\textbf{namec}} \emph{Manual} (15.35)
specifies that works without an author may be listed under an editor,
translator, or compiler, assuming that one is available, and it also
specifies the strings to be used with the name(s) of compiler(s).  All
this suggests that the \emph{Manual} considers this to be standard
information that should be made available in a bibliographic
reference, so I have added that possibility to the many that
\textsf{biblatex} already provides, such as the \textsf{editor},
\textsf{translator}, \textsf{commentator}, \textsf{annotator}, and
\textsf{redactor}, along with writers of an \textsf{introduction},
\textsf{foreword}, or \textsf{afterword}.  Since \textsf{biblatex.bst}
doesn't offer a \textsf{compiler} field, I have adopted for this
purpose the otherwise unused field \textsf{namec}.  It is important to
understand that, despite the analogous name, this field does not
function like \textsf{namea} or \textsf{nameb}, but rather like
\textsf{editor} or \textsf{translator}, and therefore if used will be
associated with whichever title field these latter two would be were
they present in the same entry.  Identical fields among these three
will be concatenated by the package, and concatenated too with the
(usually) unnecessary commentator, annotator and the rest.  Also
please note that I've arranged the concatenation algorithms to include
\textsf{namec} in the same test as \textsf{namea} and \textsf{nameb},
so in this particular circumstance you can, if needed, make
\textsf{namec} analogous to these two latter, \textsf{title}-only
fields.  (See above under \textbf{editortype} for details of how you
can use that field to identify a compiler.)

\mylittlespace It might conceivably be necessary at some point to
identify the compiler(s) of a \textsf{title} separate from the
compiler(s) of a \textsf{booktitle} or \textsf{maintitle}, but for the
moment I've run out of available \textsf{name} fields, so you'll have
to fall back on the \cmd{partcomp} macro or the related
\cmd{parteditandcomp}, \cmd{parttransandcomp}, and
\cmd{partedittransandcomp}, on which see Commands
(section~\ref{sec:formatting:authdate}) below.  (Future releases may
be able to remedy this.)  It may be as well to mention here too that
of the three names that can be substituted for the missing
\textsf{author} at the head of an entry, \textsf{biblatex-chicago}
will choose an \textsf{editor} if present, then a \textsf{translator}
if present, falling back to \textsf{namec} only in the absence of the
other two, and assuming that the fields aren't identical, and
therefore to be concatenated.  In a change from the previous behavior,
these algorithms also now test for \textsf{namea} or \textsf{nameb},
which will be used instead of \textsf{editor} and \textsf{translator},
respectively, giving the package the greatest likelihood of finding a
name to place at the head of an entry.  \textsf{Biblatex}'s sorting
algorithms, and also its \textsf{labelname} mechanism, should both
work properly no matter sort of name you provide, thanks to
\textsf{Biber} and the (default) Chicago-specific definitions of
\cmd{DeclareLabelname} and \cmd{DeclareSortingScheme}.  (Cf.\
section~\ref{sec:authformopts}, below).

\mybigspace As \mymarginpar{\textbf{note}} in standard
\textsf{biblatex}, this field allows you to provide bibliographic data
that doesn't easily fit into any other field.  In this sense, it's
very like \textsf{addendum}, but the information provided here will be
printed just before the publication data.  (See chaucer:alt,
cook:sotweed, emerson:nature, and rodman:walk for examples of this
usage in action.)  It also has a specialized use in the periodical
types (\textsf{article}, \textsf{periodical}, and \textsf{review}),
where it holds supplemental information about a \textsf{journaltitle},
such as \enquote{special issue} (conley:fifthgrade, good:wholeissue).
In all uses, if your data begins with a word that would ordinarily
only be capitalized at the beginning of a sentence, then simply ensure
that that word is in lowercase, and \textsf{biblatex-chicago} will
automatically do the right thing.  Cf.\ \textsf{addendum}.

\mybigspace This \mymarginpar{\textbf{number}} is a standard
\textsf{biblatex} field, containing the number of a
\textsf{journaltitle} in an \textsf{article} or \textsf{review} entry,
the number of a \textsf{title} in a \textsf{periodical} entry, the
volume/number of a book in a \textsf{series}, or the (generally
numerical) specifier of the \textsf{type} in a \textsf{report} entry.
Generally, in an \textsf{article}, \textsf{periodical}, or
\textsf{review} entry, this will be a plain cardinal number, but in
such entries \textsf{biblatex-chicago} now does the right thing if you
have a list or range of numbers (unsigned:ranke).  In any
\textsf{book}-like entry it may well contain considerably more
information, including even a reference to \enquote{2nd ser.,} for
example, while the \textsf{series} field in such an entry will contain
the name of the series, rather than a number.  This field is also the
place for the patent number in a \textsf{patent} entry.  Cf.\
\textsf{issue} and \textsf{series}.  (See \emph{Manual} 14.128--132
and boxer:china, palmatary:pottery, wauchope:ceramics; 14.180--181 and
beattie:crime, conley:fifthgrade, friedman:learn\-ing, garrett,
gibbard, hlatky:hrt, mcmillen:antebellum, rozner:liberation,
warr:el\-lison.)

\mylittlespace \textbf{NB}: This may be an opportune place to point
out that the \emph{Manual} (14.154) prefers arabic to roman numerals
in most circumstances (chapters, volumes, series numbers, etc.), even
when such numbers might be roman in the work cited.  The obvious
exception is page numbers, in which roman numerals indicate that the
citation came from the front matter, and should therefore be retained.

\mybigspace A \mymarginpar{\textbf{options}} standard
\textsf{biblatex} field, for setting certain options on a per-entry
basis rather than globally.  Information about some of the more common
options may be found above under \textsf{author} and \textsf{date},
and below in section~\ref{sec:authuseropts}.  See creel:house,
eliot:pound, emerson:nature, ency:britannica, herwign:office,
lecarre:quest, and maitland:canon for examples of the field in use.

% %\enlargethispage{\baselineskip}

\mybigspace A \mymarginpar{\textbf{organization}} standard
\textsf{biblatex} field, retained mainly for use in the \textsf{misc},
\textsf{online}, and \textsf{manual} entry types, where it may be of
use to specify a publishing body that might not easily fit in other
categories.  In \textsf{biblatex}, it is also used to identify the
organization sponsoring a conference in a \textsf{proceedings} or
\textsf{inproceedings} entry, and I have retained this as a
possibility, though the \emph{Manual} is silent on the matter.

\mybigspace This \colmarginpar{\textbf{origdate}} is a standard
\textsf{biblatex} field which allows more than one full date
specification for those references which need to provide more than
just one.  As with the analogous \textsf{date} field, you provide the
date (or range of dates) in \textsc{iso}8601 format, i.e.,
\texttt{yyyy-mm-dd}.  In most entry types, you would use
\textsf{origdate} to provide the date of first publication of a work,
most usually needed only in the case of reprint editions, but also
recommended by the \emph{Manual} for electronic editions of older
works (15.38, 14.119, 14.166, 14.169; aristotle:metaphy:gr,
emerson:nature, james:ambassadors, schweitzer:bach).  In both the
\textsf{letter} and \textsf{misc} (with \textsf{entrysubtype)} entry
types, the \textsf{origdate} identifies when a letter (or similar) was
written.  In such \textsf{misc} entries, some
\enquote{non-letter-like} materials (like interviews) need the
\textsf{date} field for this purpose, while in \textsf{letter} entries
the \textsf{date} applies to the publication of the whole collection.
If such a published collection were itself a reprint, judicious use of
the \textsf{pubstate} field or perhaps improvisation in the
\textsf{location} field might be able to rescue the situation.  (See
white:ross:memo, white:russ, and white:total for how \textsf{letter}
entries can work; creel:house shows the field in action in a
\textsf{misc} entry, while spock:interview uses \textsf{date}
instead.)

\mylittlespace Because of the importance of date specifications in the
author-date styles, \textsf{bibla\-tex-chicago-authordate} and
\textsf{authordate-trad} provide options and automated behaviors that
allow you to emphasize the \textsf{origdate} in citations and at the
head of entries in the list of references.  In entries which have
\emph{only} an \textsf{origdate} --- usually \textsf{misc} with an
\textsf{entrysubtype} --- \textsf{Biber} and the default
\cmd{DeclareLabeldate} configuration make it possible to do without a
\texttt{cmsdate} option, as the \textsf{origdate} will automatically
appear where and as it should.  In \textsf{book}-like entries with
both a \textsf{date} and an \textsf{origdate}, the 16th edition of the
\emph{Manual} recommends that you present, in citations and at the
head of reference list entries, only the \textsf{date} or both dates
together.  The latter is accomplished using the \texttt{cmsdate} entry
option.  In some cases it may even be necessary to reverse the two
date fields, putting the earlier year in \textsf{date} and the later
in \textsf{origdate}.  If your reference apparatus contains many such
instances, it may well be convenient for you instead to use the new
\colmarginpar{\texttt{cmsdate}\\\emph{in preamble}} \texttt{cmsdate}
preamble option, which I have designed in an attempt to reduce the
amount of manual intervention needed to present lots of entries with
multiple dates.  In short, setting \texttt{cmsdate} to \texttt{both}
or \texttt{on} in the preamble promotes the \textsf{origdate} to the
top of the search for a \textsf{labeldate} to use in citations and at
the head of entries in the reference list.  This can solve many
problems with the \textsf{extrayear} field --- 1978\textbf{a} --- and
also with sorting in the reference list.  Please see above under
\textbf{date} for all the details on how these options interact.

\mylittlespace In the default configuration of \cmd{DeclareLabeldate},
dates for citations and for the head of reference list entries are
searched for in the order \textsf{date, eventdate, origdate, urldate}.
If you set the \texttt{cmsdate} preamble options I've just mentioned,
this changes to \textsf{origdate, date, eventdate, urldate}.  These
generally cover the needs of the Chicago author-date styles well,
except for \textsf{music} and \textsf{video} entries, and,
exceptionally, some \textsf{review} entries.  Here the general rule is
to emphasize the earliest date.  For these three entry types, then,
\cmd{DeclareLabeldate} uses the order \textsf{eventdate, origdate,
  date, urldate}.  In \textsf{music} entries, you can use the
\textsf{origdate} in two separate but related ways.  First, it can
identify the recording date of an entire disc, rather than of one
track on that disc, which would go in \textsf{eventdate}.  (Compare
holiday:fool with nytrumpet:art.)  Second, the \textsf{origdate} can
provide the original release date of an album.  For this to happen,
you need to put the string \texttt{reprint} in the \textsf{pubstate}
field, which is the standard mechanism across many other entry types
for identifying a reprinted work.  (See floyd:atom.)  In
\textsf{video} entries, the \textsf{origdate} is intended for the
original release date of a film, whereas the \textsf{eventdate} would
hold the original broadcast date of, e.g., an episode of a TV series.
In both these two entry types, the style will, depending on the
context, automatically prepend appropriate bibstrings to the
\textsf{origdate}.  You can, assuming you've not activated the
\textsf{pubstate} mechanism in a \textsf{music} entry, choose a
different string using the \textsf{userd} field, but please be aware
that if an entry also has an \textsf{eventdate}, then \textsf{userd}
will apply to that, instead, and you'll be forced to accept the
default string.  (Compare friends:leia with hitchcock:nbynw; 15.53,
14.279-280; cf.\ \texttt{cmsdate} in sections~\ref{sec:authuseropts}
and \ref{sec:authentryopts}, \cmd{DeclareLabeldate} in
section~\ref{sec:authformopts}, and \texttt{avdate} in
section~\ref{sec:authpreset}.)

\mylittlespace Because the \textsf{origdate} field only accepts
numbers, some improvisation may be needed if you wish to include
\enquote{n.d.}\ (\cmd{bibstring\{nodate\}}) in an entry.  In
\textsf{letter} and \textsf{misc}, this information can be placed in
\textsf{titleaddon}, but in other entry types you may need to use the
\textsf{location} field.

%\enlargethispage{\baselineskip}

\mybigspace In \mymarginpar{\textbf{origlanguage}} keeping with the
\emph{Manual}'s specifications, I have fairly thoroughly redefined
\textsf{biblatex}'s facilities for treating translations.  The
\textsf{origtitle} field isn't used, while the \textsf{language} and
\textsf{origdate} fields have been press-ganged for other duties.  The
\textsf{origlanguage} field, for its part, retains a dual role in
presenting translations in a list of references.  The details of the
\emph{Manual}'s suggested treatment when both a translation and an
original are cited may be found below under \textbf{userf}.  Here,
however, I simply note that the introductory string used to connect
the translation's citation with the original's is \enquote{Originally
  published as,} which I suggest may well be inaccurate in a great
many cases, as for instance when citing a work from classical
antiquity, which will most certainly not \enquote{originally} have
been published in the Loeb Classical Library.  Although not, strictly
speaking, authorized by the \emph{Manual}, I have provided another way
to introduce the original text, using the \textsf{origlanguage} field,
which must be provided \emph{in the entry for the translation, not the
  original text} (aristotle:metaphy:trans).  If you put one of the
standard \textsf{biblatex} bibstrings there (enumerated below), then
the entry will work properly across multiple languages.  Otherwise,
just put the name of the language there, localized as necessary, and
\textsf{biblatex-chicago} will eschew \enquote{Originally published
  as} in favor of, e.g., \enquote{Greek edition:} or \enquote{French
  edition:}.  This has no effect in citations, where only the work
cited --- original or translation --- will be printed, but it may help
to make the \emph{Manual}'s suggestions for the list of references
more palatable.

\mylittlespace That was the first usage, in keeping at least with the
spirit of the \emph{Manual}.  I have also, perhaps less in keeping
with that specification, retained some of \textsf{biblatex}'s
functionality for this field.  If an entry doesn't have a
\textsf{userf} field, and therefore won't be combining a text and its
translation in the list of references, you can also use
\textsf{origlanguage} as Lehman intended it, so that instead of
saying, e.g., \enquote{translated by X,} the entry will read
\enquote{translated from the German by X.}  The \emph{Manual} doesn't
mention this, but it may conceivably help avoid certain ambiguities in
some citations.  As in \textsf{biblatex}, if you wish to use this
functionality, you have to provide \emph{not} the name of the
language, but rather a bibstring, which may, at the time of writing,
be one of \texttt{american}, \texttt{brazilian}, \texttt{danish},
\texttt{dutch}, \texttt{english}, \texttt{french}, \texttt{german},
\texttt{greek}, \texttt{italian}, \texttt{latin}, \texttt{norwegian},
\texttt{portuguese}, \texttt{spanish}, or \texttt{swedish}, to which
I've added \texttt{russian}.

\mybigspace The \mymarginpar{\textbf{origlocation}} 16th edition of
the \emph{Manual} has somewhat clarified issues pertaining to the
documentation of reprint editions and their corresponding originals
(14.166, 15.38).  In \textsf{biblatex-chicago} you can now provide
both an \textsf{origlocation} and an \textsf{origpublisher} to go
along with the \textsf{origdate}, should you so wish, and all of this
information will be printed in the reference list.  You can now also
use this field in a \textsf{letter} or \textsf{misc} (with
\textsf{entrysubtype}) entry to give the place where a published or
unpublished letter was written (14.117).  (Jonathan Robinson has
suggested that the \textsf{origlocation} may in some circumstances
actually be helpful for disambiguation, his example being early
printed editions of the same material printed in the same year but in
different cities.  The new functionality should make this simple to
achieve.  Cf.\ \textsf{origdate}, \textsf{origpublisher} and
\textsf{pubstate}; schweitzer:bach.)

\mybigspace As \mymarginpar{\textbf{origpublisher}} with the
\textsf{origlocation} field just above, the 16th edition of the
\emph{Manual} has clarified issues pertaining to reprint editions and
their corresponding originals (14.166, 15.38).  You can now provide an
\textsf{origpublisher} and/or an \textsf{origlocation} in addition to
the \textsf{origdate}, and all will be presented in long notes and
bibliography.  (Cf.\ \textsf{origdate}, \textsf{origlocation}, and
\textsf{pubstate}; schweitzer:bach.)

\mybigspace This \colmarginpar{\textbf{pages}} is the standard
\textsf{biblatex} field for providing page references.  In many
\textsf{article} entries you'll find this contains something other
than a page number, e.g. a section name or edition specification
(14.203, 14.209; kozinn:review, nyt:trevorobit).  Of course, the same
may be true of almost any sort of entry, though perhaps with less
frequency.  Curious readers may wish to look at brown:bremer (14.189)
for an example of a \textsf{pages} field used to facilitate reference
to a two-part journal article.  Cf.\ \textsf{number} for more
information on the \emph{Manual}'s preferences regarding the
formatting of numerals; \textsf{bookpagination} and
\textsf{pagination} provide details about \textsf{biblatex's}
mechanisms for specifying what sort of division a given \textsf{pages}
field contains; and \textsf{usera} discusses a different way to
present the section information pertaining to a newspaper article.

\mylittlespace David Gohlke has recently brought to my attention a
discussion that took place a couple of years ago on
\href{http://tex.stackexchange.com/questions/44492/biblatex-chicago-style-page-ranges}{Stackexchange}
regarding the automatic compression of page ranges, e.g., 101-{-}109
in the .bib file or in the \textsf{postnote} field would become 101--9
in the document.  \textsf{Biblatex} has long had the facilities for
providing this, and though the \emph{Manual's} rules (9.60) are fairly
complicated, Audrey Boruvka fortunately provided in that discussion
code that implements the specifications.  As some users may well be
accustomed to compressing page ranges themselves in their .bib files,
and in their \textsf{postnote} fields, I have made the activation of
this code a package option, so setting
\mycolor{\texttt{compresspages=true}} when loading
\textsf{biblatex-chicago} should automatically give you the
Chicago-recommended page ranges.

\mybigspace This, \mymarginpar{\textbf{pagination}} a standard
\textsf{biblatex} field, allows you automatically to prefix the
appropriate identifying string to information you provide in the
\textsf{postnote} field of a citation command, whereas
\textsf{bookpagination} allows you to prefix a string to the
\textsf{pages} field.  Please see \textbf{bookpagination} above for
all the details on this functionality, as aside from the difference
just mentioned the two fields are equivalent.

\mybigspace Standard \colmarginpar{\textbf{part}} \textsf{biblatex}
field, which identifies physical parts of a single logical volume in
\textsf{book}-like entries, not in periodicals.  It has the same
purpose in \textsf{biblatex-chicago}, but because the \emph{Manual}
(14.126) calls such a thing a \enquote{book} and not a \enquote{part,}
the string printed in the list of references will, at least in
English, be \enquote{\texttt{bk.}\hspace{-2pt}}\ instead of the plain
dot between volume number and part number (harley:cartography,
lach:asia).  If the field contains something other than a number,
\textsf{biblatex-chicago} will print it as is, capitalizing it if
necessary, rather than supplying the usual bibstring, so this provides
a mechanism for altering the string to your liking.  The field will be
printed in the same place in any entry as would a \textsf{volume}
number, and although it will most usually be associated with such a
number, it can also, as of this release, function independently,
allowing you to identify parts of works that don't fit into the
standard scheme.  If you need to identify \enquote{parts} or
\enquote{books} that are part of a published \textsf{series}, for
example, then you'll need to use a different field, (which in the case
of a series would be \textsf{number} [palmatary:pottery]).  Cf.\
\textsf{volume}; iso:electrodoc.

\mybigspace Standard \mymarginpar{\textbf{publisher}}
\textsf{biblatex} field.  Remember that \enquote{\texttt{and}} is a
keyword for connecting multiple publishers, so if a publisher's name
contains \enquote{and,} then you should either use the ampersand (\&)
or enclose the whole name in additional braces.  (See \emph{Manual}
14.139--148; aristotle:metaphy:gr, cohen:schiff, creasey:ashe:blast,
dunn:revolutions.)

\mylittlespace There are, as one might expect, a couple of further
subtleties involved here.  Two publishers will be separated by a
forward slash in the list of references, and you no longer, in the
16th edition, need to provide hand formatting if a company issues
\enquote{certain books through a special publishing division or under
  a special imprint,} as these, too, should be separated by a forward
slash.  If a book has two co-publishers, \enquote{usually in different
  countries,} (14.147) then the simplest thing to do is to choose one,
probably the nearest one geographically.  If you feel it necessary to
include both, then levistrauss:savage demonstrates one way of doing
so, using a combination of the \textsf{publisher} and
\textsf{location} fields.  Finally, if the publisher is unknown, then
the \emph{Manual} recommends (14.143) simply using the place (if
known) and the date.  If for some reason you need to indicate the
absence of a publisher, the abbreviation given by the \emph{Manual} is
\texttt{n.p.}, though this can also stand for \enquote{no place.}
Some style guides apparently suggest using \texttt{s.n.}\,(=
\emph{sine nomine}) to specify the lack of a publisher, but the
\emph{Manual} doesn't mention this.

\mybigspace A \mymarginpar{\textbf{pubstate}} standard
\textsf{biblatex} field, introduced in version 0.9.  Because the
author-date specification has fairly complicated rules about
presenting reprinted editions, I have adopted this field as a means of
simplifying the problem for users.  Instead of hand-formatting in the
\textsf{location} field, you can now simply put the string
\texttt{reprint} into the \textsf{pubstate} field, and depending on
which date(s) you have chosen to appear at the head of the entry,
\textsf{biblatex-chicago-authordate} will either print the (localized)
string \texttt{Reprint} in the proper place or otherwise provide a
parenthesized notice at the end of the entry detailing the original
publication date.  See under \textbf{date} above for the available
permutations.  (Cf.\ aristotle:metaphy:gr, maitland:canon,
maitland:equity, schweitzer:bach.)  If the field contains something
other than the word \texttt{reprint}, then it will be treated as in
the standard styles, and printed after the publication information.

\mylittlespace There is one subtlety of which you ought to be aware.
In \textsf{music} entries, the \textsf{pubstate} mechanism transforms
the \textsf{origdate} from a recording date for an album into the
original release date for that album.  If that date appears in
citations and at the head of reference-list entries, then this
mechanism won't generally make much difference, but if it appears
elsewhere then a recording date will be printed in the middle of the
reference list entry, the original release date will be printed near
the end, preceded by the appropriate string.

%\enlargethispage{\baselineskip}

\mybigspace I \mymarginpar{\textbf{redactor}} have implemented this
field just as \textsf{biblatex}'s standard styles do, even though the
\emph{Manual} doesn't actually mention it.  It may be useful for some
purposes.  Cf.\ \textsf{annotator} and \textsf{commentator}.

\mybigspace \textbf{NB:} \mymarginpar{\textbf{reprinttitle}}
\textbf{Please note that this feature is in an alpha state, and that
  I'm contemplating using a different field in the future for this
  functionality.  I include it here in the hope that it might receive
  some testing in the meantime.}  At the request of Will Small, I have
included a means of providing the original publication details of an
essay or a chapter that you are citing from a subsequent reprint,
e.g., a \emph{Collected Essays} volume.  In such a case, at least
according to the \emph{Manual} (14.115), such details need be provided
only if they are \enquote{of particular interest.}  The data would
follow an introductory phrase like \enquote{originally published as,}
making the problem strictly parallel to that of including details of a
work in the original language alongside the details of its
translation.  I have addressed the latter problem with the
\textsf{userf} field, which provides a sort of cross-referencing
method for this purpose, and \textsf{reprinttitle} works in
\emph{exactly} the same way.  In the .bib entry for the reprint you
include a cross-reference to the cite key of the original location
using the \textsf{reprinttitle} field (which it may help mnemonically
to think of as a \enquote{reprinted title} field).  The main
difference between the two forms is that \textsf{userf} prints all but
the \textsf{author} of the original work, whereas
\textsf{reprinttitle} suppresses both the \textsf{author} and the
\textsf{title} of the original, giving only the more general details,
beginning with, e.g., the \textsf{journaltitle} or \textsf{booktitle}
and continuing from there.  The string prefacing this information will
be \enquote{Orig.\ pub.\ in.}  Please see the documentation on
\textsf{userf} below for all the details on how to create .bib entries
for presenting your data.

%%\enlargethispage{\baselineskip}

\mybigspace A \mymarginpar{\textbf{series}} standard \textsf{biblatex}
field, usually just a number in an \textsf{article},
\textsf{periodical}, or \textsf{review} entry, almost always the name
of a publication series in \textsf{book}-like entries.  If you need to
attach further information to the \textsf{series} name in a
\textsf{book}-like entry, then the \textsf{number} field is the place
for it, whether it be a volume, a number, or even something like
\enquote{2nd ser.} or \enquote{\cmd{bibstring\{oldseries\}}.}  Of
course, you can also use \cmd{bibstring\{oldseries\}} or
\cmd{bibstring\{newseries\}} in an \textsf{article} entry, but there
you would place it in the \textsf{series} field itself.  (In fact, the
\textsf{series} field in \textsf{article} and \textsf{periodical}
entries is one of the places where \textsf{biblatex} allows you just
to use the plain bibstring \texttt{oldseries}, for example, rather
than making you type \cmd{bibstring\{oldseries\}}.  The \textsf{type}
field in \textsf{manual}, \textsf{patent}, \textsf{report}, and
\textsf{thesis} entries also has this auto-detection mechanism in
place; see the discussion of \cmd{bibstring} below for details.)  In
whatever entry type, these bibstrings produce the required
abbreviation.  (For books and similar entries, see \emph{Manual}
14.128--132; boxer:china, browning:aurora, palmatary:pottery,
plato:republic:gr, wauchope:ceramics; for periodicals, see 14.195;
garaud:gatine, sewall:letter.)  Cf.\ \textsf{number} for more
information on the \emph{Manual}'s preferences regarding the
formatting of numerals.

\mybigspace This \mymarginpar{\textbf{shortauthor}} is a standard
\textsf{biblatex} field, but \textsf{biblatex-chicago} makes
considerably grea\-ter use of it than the standard styles.  For the
purposes of the author-date specification, the field provides the name
to be used in text citations.  In the vast majority of cases, you
don't need to specify it, because the \textsf{biblatex} system selects
the author's last name from the \textsf{author} field and uses it in
such a reference, and if there is no \textsf{author} it will search
\textsf{namea}, \textsf{editor}, \textsf{nameb}, \textsf{translator},
and \textsf{namec}, in that order.  The current versions of
\textsf{biblatex} and \textsf{Biber} will now automatically
alphabetize by any of these names if they appear at the head of an
entry.  If, in an author-less \textsf{article} entry
(\textsf{entrysubtype} \texttt{magazine}), you allow
\textsf{biblatex-chicago} to use the title of the periodical as the
author --- the default behavior --- then your \textsf{shortauthor}
field can optionally contain an abbreviated form of the periodical
name, formatted appropriately, which usually means something like
\enquote{\cmd{mkbibemph\{Abbrev.\ Period.\ Title\}}} (gourmet:052006,
lakeforester:pushcarts, nyt:trevorobit, unsigned:ranke).  Indeed, with
long, institutional authors, a shortened version in
\textsf{shortauthor} may save space in the running text
(evanston:library).  See just below under \textbf{shorthand} for
another method of saving space.

\mylittlespace As mentioned under \textsf{editortype}, the
\emph{Manual} (15.21) recommends against providing the identifying
string (e.g., ed.\ or trans.)\ in text citations, and
\textsf{biblatex-chicago} follows their recommendation.  If you need
to provide these strings in such a citation, then you'll have to do so
by hand in the \textsf{shortauthor} field, or in the
\textsf{shorteditor} field, whichever you are using.

\mybigspace Like \mymarginpar{\textbf{shorteditor}}
\textsf{shortauthor}, a field to provide a name for a text citation,
in this case for, e.g., a \textsf{collection} entry that typically
lacks an author.  The \textsf{shortauthor} field works just as well in
most situations, but if you have set \texttt{useauthor=false} (and not
\texttt{useeditor=false}) in an entry's \textsf{options} field, then
only \textsf{shorteditor} will be recognized.  Cf.\
\textsf{editortype}, above.

% %\enlargethispage{\baselineskip}

\mybigspace This \mymarginpar{\textbf{shorthand}} is
\textsf{biblatex}'s mechanism for using abbreviations in citations.
For \textsf{biblatex-chicago-authordate} I have modified it somewhat
to conform to the needs of the specification, though there is a
package option to revert the behavior to something closer to the
\textsf{biblatex} standard --- see below and under \texttt{cmslos} in
section~\ref{sec:authpreset}.  The main problem when presenting
readers with an abbreviation is to ensure that they know how to expand
it.  In the notes \&\ bibliography style this is accomplished with a
notice in the first footnote citing a given work, which explains that
henceforth the abbreviation will be used instead, and also, if needed,
with a list of shorthands that summarizes all the abbreviations used
in a particular text.  The first part of this system isn't available
in the author-date style of citation, and indeed these citations are
in themselves already highly-abbreviated keys to the fuller
information to be found in the list of references.  There are cases,
however, particularly when institutions or \textsf{journaltitles}
appear as authors, when you may feel the need to provide a shortened
version for citations.  I have already discussed one option available
to you just above (cf.\ \textbf{shortauthor}), but for this to work
the abbreviation must either be instantly recognizable to your
readership or at least easily parseable by them.

\mylittlespace The \emph{Manual's} recommendation (15.36), and this
has changed for the 16th edition, involves using an abbreviation for
long institutional names, an abbreviation which will appear not only
in citations but also at the head of the entry in the list of
references.  Such an entry should therefore be alphabetized by the
abbreviation, which will be expanded within the same entry and placed
(inside parentheses) between the abbreviation and the date.  This new
formatting can be produced in one of two ways: either you can provide
a specially-formatted \textsf{author} field (for the reference list,
and including both the abbreviation and the parenthesized expansion) +
a \textsf{shortauthor} (for the citations), or you can use a normal
\textsf{author} field + a \textsf{shorthand}, in which case
\textsf{biblatex-chicago-authordate} will automatically use the
\textsf{shorthand} in text citations and also place it at the head of
the reference list entry, followed by the \textsf{author} within
parentheses.  This method is simpler and more compatible with other
styles, though you do need a \textsf{sortkey} when you use the
\textsf{shorthand} field this way.  (Cf.\ bsi:abbreviation,
iso:electrodoc.)

\mylittlespace I should clarify here that this automatic placement of
the \textsf{shorthand} at the head of the entry will \emph{not} occur
if you set the package option \texttt{cmslos=false} in your preamble.
This allows you to implement other systems of shorthand expansion
using either a list of shorthands (via \cmd{printshorthands}, which is
always available no matter what the state of \texttt{cmslos}) or
cross-references (via \textsf{customc}) within the reference list
itself.  You can place \texttt{skiplos} in the \textsf{options} field
to exclude a particular entry from the list of shorthands if you do
decide to print that list, giving maximum flexibility.

\mylittlespace Indeed, I have provided two new options to add to this
flexibility.  First, I have included two new \texttt{bibenvironments}
for use with the \texttt{env} option to the \cmd{printshorthands}
command: \texttt{losnotes} is designed to allow a list of shorthands
to appear inside footnotes, while \texttt{losendnotes} does the same
for endnotes.  Their main effect is to change the font size, and in
the latter case to clear up some spurious punctuation and white space
that I see on my system when using endnotes.  (You'll probably also
want to use the option \texttt{heading=none} in order to get rid of
the [oversized] default, providing your own within the \cmd{footnote}
command.)  Second, I have provided a new package option,
\texttt{short\-handfull}, which prints entries in the list of shorthands
which contain full bibliographical information, effectively allowing
you to eschew the list of references in favor of a fortified shorthand
list.  This option will only work if used in tandem with
\texttt{cmslos=false}, as otherwise the shorthand will be printed
twice.  (See 15.36, 13.65, 14.54--55, and also \textsf{biblatex.pdf}
for more information.)

%%\enlargethispage{-\baselineskip}

\mylittlespace As I mentioned above under \textbf{crossref}, I believe
it is now safe to use shorthands in parent entries, as this, in the
standard configuration, gives you the shorthand itself in the child
entry's abbreviated cross-reference, which may well save space in the
list of references.

\mybigspace A \mymarginpar{\textbf{shorttitle}} standard
\textsf{biblatex} field, primarily used to provide an abbreviated
title for citation styles that need one.  In
\textsf{biblatex-chicago-authordate} such a field will be necessary
only very rarely (unlike in the notes \&\ bibliography style), and is
most likely to turn up in \textsf{inreference} or \textsf{reference}
entries (where the \textsf{title} takes the place of the
\textsf{author}), or in any sort of entry with a \texttt{classical}
\textsf{entrysubtype}.  This latter toggle makes citations use
\textsf{author} and \textsf{title} instead of \textsf{author} and
\textsf{year}, and if an abbreviated version of that title would save
space in your running text this is the field where you can provide it.
(Cf.\ ency:britannica, grove:sibelius, aristotle:metaphy:gr.)

\mybigspace A \mymarginpar{\textbf{sortkey}} standard
\textsf{biblatex} field, designed to allow you to specify how you want
an entry alphabetized in a list of references.  In general, if an
entry doesn't turn up where you expect or want it, this field should
provide the solution.  Entries with a corporate author can now omit
the definite or indefinite article, which should help (14.85;
cotton:manufacture, nytrumpet:art).  The default settings of
\cmd{DeclareSortingScheme} now include the three supplemental name
fields (\textsf{name[a-c]}) and also the \textsf{journaltitle} in the
sorting algorithm, so once again you should find those algorithms
needing less help than before.  Entries using a \textsf{shorthand},
and entries headed by a \textsf{title} beginning with the definite or
indefinite article, may well now require such assistance
(bsi:abbreviation, grove:sibelius, iso:electrodoc).  There may be
circumstances --- several reprinted books by the same author, for
example --- when the \textbf{sortyear} field is more appropriate, on
which see below.  Lehman also provides \textbf{sortname} and
\textbf{sorttitle} for equally fine-grained control.  Please consult
\textsf{biblatex.pdf} for the details.

\mybigspace A \mymarginpar{\textbf{sortyear}} standard
\textsf{biblatex} field, provided by Lehman for more fine-grained
control over the sorting of entries in a list of references, and
possibly useful in \textsf{biblatex-chicago-authordate} to help
present several reprinted books by the same author.  See
\textsf{sortkey} and \textsf{date} above.

\mybigspace The \mymarginpar{\textbf{subtitle}} subtitle for a
\textsf{title} --- see next entry.

%\enlargethispage{\baselineskip}

\mybigspace This \mymarginpar{\textbf{title}} release of
\textsf{biblatex-chicago} now includes the \textsf{authordate-trad}
style, designed as a kind of hybrid style according to indications
contained in the 16th edition of the \emph{Manual} (14.45).  This
\textsf{trad} style differs \emph{only} in the way it treats the
\textsf{title} and related fields, which retain the forms they have
traditionally had in the Chicago author-date specifications prior to
the latest edition.  Where the new edition uses headline-style
capitalization, the older editions used sentence-style; where the new
edition places \textsf{article} or \textsf{incollection}
\textsf{titles} within quotation marks, the older editions presented
them in plain text.  If you have been using the 15th-edition
author-date style, then your \textsf{title} fields won't need any
changes for \textsf{authordate-trad}, but I shall include just below,
under a separate rubric, full documentation for \textsf{trad}
\textsf{title} fields for those just coming to the package.  First,
though, I document the same field(s) for the standard author-date
style.

\mylittlespace In the vast majority of cases, this field works just as
it always has in \textsc{Bib}\TeX, and just as it does in
\textsf{biblatex}.  In a major change to previous editions of the
\emph{Manual}, the 16th edition now recommends that \textsf{titles} be
treated more or less identically across both its systems of
documentation (15.2, 15.6, 15.13).  This means that users of the
author-date style no longer need to worry about sentence-style
capitalization when compiling their .bib databases, and so can eschew
the extra curly braces needed to preserve uppercase letters in this
context.  The other new rules, however, mean that a few new
complications, familiar to users of the notes \&\ bibliography style,
will arise.  First, although nearly every entry will have a
\textsf{title}, there are some exceptions, particularly
\textsf{incollection} or \textsf{online} entries with a merely generic
title, instead of a specific one (centinel:letters, powell:email).
Second, the \emph{Manual}'s rules for formatting \textsf{titles},
which also hold for \textsf{booktitles} and \textsf{maintitles},
require additional attention.  The whole point of using a
\textsc{Bib}\TeX-based system is for it to do the formatting for you,
and in most cases \textsf{biblatex-chicago-authordate} does just that,
surrounding titles with quotation marks, italicizing them, or
occasionally just leaving them alone.  When, however, a title is
quoted within a title, then you need to know some of the rules.  A
summary here should serve to clarify them, and help you to understand
when \textsf{biblatex-chicago-authordate} might need your help in
order to comply with them.

\mylittlespace The internal rules of
\textsf{biblatex-chicago-authordate} are as follows:

\begin{description}
\item[\qquad Italics:] \textsf{booktitle}, \textsf{maintitle}, and
  \textsf{journaltitle} in all entry types; \textsf{title} of
  \textsf{artwork}, \textsf{book}, \textsf{bookinbook},
  \textsf{booklet}, \textsf{collection}, \textsf{image},
  \textsf{inbook}, \textsf{manual}, \textsf{misc} (with no
  \textsf{entrysubtype}), \textsf{periodical}, \textsf{proceedings},
  \textsf{report}, \textsf{suppbook}, and \textsf{suppcollection}
  entry types.
\item[\qquad Quotation Marks:] \textsf{title} of \textsf{article},
  \textsf{incollection}, \textsf{inproceedings}, \textsf{online},
  \textsf{periodical}, \textsf{thesis}, and \textsf{unpublished} entry
  types, \textsf{issuetitle} in \textsf{article}, \textsf{periodical},
  and \textsf{review} entry types.
\item[\qquad Unformatted:] \textsf{booktitleaddon},
  \textsf{maintitleaddon}, and \textsf{titleaddon} in all entry types,
  \textsf{title} of \textsf{customc}, \textsf{letter}, \textsf{misc}
  (with an \textsf{entrysubtype}), \textsf{patent}, \textsf{review},
  and \textsf{suppperiodical} entry types.
\item[\qquad Italics or Quotation Marks:] All of the audiovisual entry
  types --- \textsf{audio}, \textsf{music}, and \textsf{video} ---
  have to serve as analogues both to \textsf{book} and to
  \textsf{inbook}.  Therefore, if there is both a \textsf{title} and a
  \textsf{booktitle}, then the \textsf{title} will be in quotation
  marks.  If there is no \textsf{booktitle}, then the \textsf{title}
  will be italicized.
\end{description}

Now, the rules for which entry type to use for which sort of work tend
to be fairly straightforward, but in cases of doubt you can consult
section~\ref{sec:types:authdate} above, the examples in
\textsf{dates-test.bib}, or go to the \emph{Manual} itself,
8.154--195.  Assuming, then, that you want to present a title within a
title, and you know what sort of formatting each of the two would, on
its own, require, then the following rules apply:

\begin{enumerate}
\item Inside an italicized title, all other titles are enclosed in
  quotation marks and italicized, so in such cases all you need to do
  is provide the quotation marks using \cmd{mkbibquote}, which will
  take care of any following punctuation that needs to be brought
  within the closing quotation mark(s) (14.102; donne:var,
  mchugh:wake).
\item Inside a quoted title, you should present another title as it
  would appear if it were on its own, so in such cases you'll need to
  do the formatting yourself.  Within the double quotes of the title
  another quoted title would take single quotes --- the
  \cmd{mkbibquote} command does this for you automatically, and also,
  I repeat, takes care of any following punctuation that needs to be
  brought within the closing quotation mark(s).  (See 14.177; garrett,
  loften:hamlet, murphy:silent, white:callimachus.)
\item Inside a plain title (most likely in a \textsf{review} entry or
  a \textsf{titleaddon} field), you should present another title as it
  would appear on its own, once again formatting it yourself using
  \cmd{mkbibemph} or \cmd{mkbibquote}.  (barcott:review, gibbard,
  osborne:poison, ratliff:review, unsigned:ranke).
\end{enumerate}

The \emph{Manual} provides a few more rules, as well.  A word normally
italicized in text should also be italicized in a quoted or plain-text
title, but should be in roman (\enquote{reverse italics}) in an
italicized title.  A quotation used as a (whole) title (with or
without a subtitle) retains its quotation marks in an italicized title
\enquote{only if it appears that way in the source,} but always
retains them when the surrounding title is quoted or plain (14.104,
14.177; lewis).  A word or phrase in quotation marks, but that isn't a
quotation, retains those marks in all title types (kimluu:diethyl).

\mylittlespace Finally, please note that in all \textsf{review} (and
\textsf{suppperiodical}) entries, and in \textsf{misc} entries with an
\textsf{entrysubtype}, and only in those entries,
\textsf{biblatex-chicago-authordate} will automatically capitalize the
first word of the \textsf{title} after sentence-ending punctuation,
assuming that such a \textsf{title} begins with a lowercase letter in
your .bib database.  See \textbf{\textbackslash autocap} below for
more details.

% %\enlargethispage{\baselineskip}

\mybigspace When \mymarginpar{\textbf{title (trad)}} you choose the
new \textsf{authordate-trad} style, your \textsf{title} and related
fields will need extra care, familiar to users of the 15th-edition
author-date style.  The whole point of using a \textsc{Bib}\TeX-based
system is for it to do the formatting for you, and in most cases
\textsf{biblatex-chicago-authordate-trad} does just that, capitalizing
them sentence-style, italicizing them, and sometimes both.  There are
two situations that require user intervention.  First, in titles that
take sentence-style capitalization, you need, as always in traditional
\textsc{Bib}\TeX, to assist the algorithms by placing anything that
needs to remain capitalized within an extra pair of curly braces.
Second, when a title is quoted within a title, you need to know some
of the rules of the Chicago style.  A summary here should serve to
clarify them, and help you to understand when
\textsf{biblatex-chicago-authordate-trad} might need your help in
order to comply with them.

\mylittlespace With regard to sentence-style capitalization, the rules
of the Chicago \textsf{authordate-trad} style are fairly simple:

\begin{description}
\item[\qquad Headline Style:] \textsf{journaltitle} in all types,
  \textsf{series} in all \textsf{book}-like entries (i.e., not in
  \textsf{articles}), and \textsf{title} in \textsf{periodical}
  entries.
\item[\qquad Sentence Style:] every other \textsf{title},
  \emph{except} in \textsf{letter} entries, \textsf{review} entries,
  and in \textsf{misc} entries with an \textsf{entrysubtype}.  Also,
  the \textsf{booktitle}, \textsf{issuetitle}, and \textsf{maintitle}
  in all entry types use sentence style.
\item[\qquad Contextual Capitalization of First Word:]
  \textsf{titleaddon}, \textsf{booktitleaddon},
  \textsf{maintitleaddon} in all entry types, also the \textsf{title}
  of \textsf{review} entries and of \textsf{misc} entries with an
  \textsf{entrysubtype}.
\item[\qquad Plain:] \textsf{title} in \textsf{letter} entries.
\end{description}

What this means in practice is that to get a title like \emph{The
  Chicago manual of style}, your .bib entry needs to have a field that
looks something like this:
\begin{quote}
  \texttt{title = \{The \{Chicago\} Manual of Style\}}
\end{quote}

This is completely straightforward, but remember that if an
\textsf{article} has a title like: Review of \emph{The Chicago manual
  of style}, then the curly braces enclosing material to be formatted
in italics will cause the capitalization algorithm to stop and leave
all of that material as it is, so your .bib entry would need to have a
field something like this:

\begin{quote}
  \texttt{title = \{}\cmd{bibstring\{reviewof\}} \cmd{mkbibemph\{The
    Chicago manual of style\}\}}
\end{quote}

(As an aside, the use of the \texttt{reviewof} bibstring isn't
strictly necessary here, but it helps with portability across
languages and across the two Chicago styles.  If you've noticed a lot
of lowercase letters starting fields in \textsf{dates-test.bib},
they're present because in the notes \&\ bibliography style
capitalization is complicated by notes using commas where the
bibliography uses periods, and words like \enquote{review} start in
uppercase only if the context demands it.  There's considerably less
of this in the author-date styles [note the \textsf{*titleaddon}
fields], but it still pays to be aware of the issue.)

\mylittlespace With regard to italics, the rules of
\textsf{biblatex-chicago-authordate-trad} are as follows:

\begin{description}
\item[\qquad Italics:] \textsf{booktitle}, \textsf{maintitle}, and
  \textsf{journaltitle} in all entry types; \textsf{title} of
  \textsf{artwork}, \textsf{book}, \textsf{bookinbook},
  \textsf{booklet}, \textsf{collection}, \textsf{inbook},
  \textsf{manual}, \textsf{misc} (with no \textsf{entrysubtype}),
  \textsf{periodical}, \textsf{proceedings}, \textsf{report},
  \textsf{suppbook}, and \textsf{suppcollection} entry types.
\item[\qquad Main Text Font (Roman):] \textsf{title} of
  \textsf{article}, \textsf{image}, \textsf{incollection},
  \textsf{inproceedings}, \textsf{letter}, \textsf{misc} (with an
  \textsf{entrysubtype}), \textsf{online}, \textsf{patent},
  \textsf{periodical}, \textsf{review}, \textsf{suppperiodical},
  \textsf{thesis}, and \textsf{unpublished} entry types,
  \textsf{issuetitle} in \textsf{article} and \textsf{periodical}
  entry types.  \textsf{booktitleaddon}, \textsf{maintitleaddon}, and
  \textsf{titleaddon} in all entry types.
\item[\qquad Italics or Roman:] All of the audiovisual entry types ---
  \textsf{audio}, \textsf{music}, and \textsf{video} --- have to serve
  as analogues both to \textsf{book} and to \textsf{inbook}.
  Therefore, if there is both a \textsf{title} and a
  \textsf{booktitle}, then the \textsf{title} will be in the main text
  font.  If there is no \textsf{booktitle}, then the \textsf{title}
  will be italicized.
\end{description}

Now, the rules for which entry type to use for which sort of work tend
to be fairly straightforward, but in cases of doubt you can consult
section~\ref{sec:types:authdate} above, the examples in
\textsf{dates-test.bib}, or go to the \emph{Manual} itself,
8.154--195.  Assuming, then, that you want to present a title within a
title, and you know what sort of formatting each of the two would, on
its own, require, then the following rules apply:

\begin{enumerate}
\item Inside an italicized title, all other titles are enclosed in
  quotation marks and italicized, so in such cases all you need to do
  is provide the quotation marks using \cmd{mkbibquote}, which will
  take care of any following punctuation that needs to be brought
  within the closing quotation mark(s) (14.102; donne:var,
  mchugh:wake).
\item Inside a plain-text title, you should set off other plain-text
  titles with quotation marks, while italicized titles should appear
  as they would if they were on their own.  In such cases you'll need
  to do the formatting yourself, using \cmd{mkbibemph} or
  \cmd{mkbibquote}.  (See barcott:review, garrett, gibbard,
  loften:hamlet, loomis:structure, murphy:silent, osborne:poi\-son,
  ratliff:review, unsigned:ranke, white:callimachus.)
\end{enumerate}

The \emph{Manual} provides a few more rules, as well.  A word normally
italicized in text should also be italicized in a plain-text title,
but should be in roman (\enquote{reverse italics}) in an italicized
title.  A quotation used as a (whole) title (with or without a
subtitle) retains its quotation marks when it is plain, but loses them
when it is italicized, unless it specifically retains them in the
source (14.104, 14.177; lewis).  A word or phrase in quotation marks,
but that isn't a quotation, retains those marks in all title types
(kimluu:diethyl).

\mylittlespace Finally, please note that there is also a preamble
option --- \texttt{headline} --- that disables the automatic
sentence-style capitalization routines in \textsf{authordate-trad}.
If you set this option, the word case in your title fields will not be
changed in any way, that is, this doesn't automatically transform your
titles into headline-style, but rather allows the .bib file to
determine capitalization.  It works by redefining the command
\cmd{MakeSentenceCase}, so in the unlikely event you are using the
latter anywhere in your document please be aware that it will also be
turned off there.  See section~\ref{sec:authuseropts}, below.

\mybigspace Standard \mymarginpar{\textbf{titleaddon}}
\textsf{biblatex} intends this field for use with additions to titles
that may need to be formatted differently from the titles themselves,
and \textsf{biblatex-chicago} uses it in just this way, with the
additional wrinkle that it can, if needed, replace the \textsf{title}
entirely, and this in, effectively, any entry type, providing a fairly
powerful, if somewhat complicated, tool for getting \textsc{Bib}\TeX\
to do what you want (cf.\ centinel:letters).  This field will always
be unformatted, that is, neither italicized nor placed within
quotation marks, so any formatting you may need within it you'll need
to provide manually yourself.  The single exception to this rule is
when your data begins with a word that would ordinarily only be
capitalized at the beginning of a sentence, in which case you need
then simply ensure that that word is in lowercase, and
\textsf{biblatex-chicago} will automatically do the right thing.  See\
\textbf{\textbackslash autocap}, below.  (Cf.\ brown:bremer,
osborne:poison, reaves:rosen, and white:ross:memo for examples where
the field starts with a lowercase letter; morgenson:market provides an
example where the \textsf{titleaddon} field, holding the name of a
regular column in a newspaper, is capitalized, a situation that is
handled as you would expect.)

\mybigspace As \mymarginpar{\textbf{translator}} far as possible, I
have implemented this field as \textsf{biblatex}'s standard styles do,
but the requirements specified by the \emph{Manual} present certain
complications that need explaining.  Lehman points out in his
documentation that the \textsf{translator} field will be associated
with a \textsf{title}, a \textsf{booktitle}, or a \textsf{maintitle},
depending on the sort of entry.  More specifically,
\textsf{biblatex-chicago} associates the \textsf{translator} with the
most comprehensive of those titles, that is, \textsf{maintitle} if
there is one, otherwise \textsf{booktitle}, otherwise \textsf{title},
if the other two are lacking.  In a large number of cases, this is
exactly the correct behavior (adorno:benj, centinel:letters,
plato:republic:gr, among others).  Predictably, however, there are
numerous cases that require, for example, an additional translator for
one part of a collection or for one volume of a multi-volume work.
For these cases I have provided the \textsf{nameb} field.  You should
format names for this field as you would for \textsf{author} or
\textsf{editor}, and these names will always be associated with the
\textsf{title} (euripides:orestes).

\mylittlespace I have also provided a \textsf{namea} field, which
holds the editor of a given \textsf{title} (euripides:orestes).  If
\textsf{namea} and \textsf{nameb} are the same,
\textsf{biblatex-chicago} will concatenate them, just as
\textsf{biblatex} already does for \textsf{editor},
\textsf{translator}, and \textsf{namec} (i.e., the compiler).
Furthermore, it is conceivable that a given entry will need separate
translators for each of the three sorts of title.  For this, and for
various other tricky situations, there is the \cmd{parttrans} macro
(and its siblings), designed to be used in a \textsf{note} field or in
one of the \textsf{titleaddon} fields (ratliff:review).  (Because the
strings identifying a translator differ in notes and bibliography, one
can't simply write them out in such a field when using the notes \&\
bibliography style, but you can certainly do so in the author-date
styles, if you wish.  Using the macros will make your .bib file more
portable across both Chicago specifications, and also across multiple
languages, but they are otherwise unnecessary.  [See
section~\ref{sec:international}].)

\mylittlespace Finally, as I detailed above under \textbf{author}, in
the absence of an \textsf{author} or an \textsf{editor}, the
\textsf{translator} will be used at the head of an entry
(silver:gawain), and the reference list entry alphabetized by the
translator's name, behavior that can be controlled with the
\texttt{{usetranslator}} switch in the \textsf{options} field.  Cf.\
\textsf{author}, \textsf{editor}, \textsf{namea}, \textsf{nameb}, and
\textsf{namec}.

%%\enlargethispage{\baselineskip}

\mybigspace This \mymarginpar{\textbf{type}} is a standard
\textsf{biblatex} field, and in its normal usage serves to identify
the type of a \textsf{manual}, \textsf{patent}, \textsf{report}, or
\textsf{thesis} entry.  \textsf{Biblatex} implements the possibility,
in some circumstances, to use a bibstring without inserting it in a
\cmd{bibstring} command, and in these entry types the \textsf{type}
field works this way, allowing you simply to input, e.g.,
\texttt{patentus} rather than \cmd{bibstring\{patentus\}}, though both
will work.  (See petroff:impurity; herwign:office, murphy:silent, and
ross:thesis all demonstrate how the \textsf{type} field may sometimes
be automatically set in such entries by using one of the standard
entry-type aliases).

\mylittlespace Another use for the field is to generalize the
functioning of the \textsf{suppbook} entry type, and of its alias
\textsf{suppcollection}.  In such entries, the \textsf{type} field can
specify what sort of supplemental material you are citing, e.g.,
\enquote{\texttt{preface to}} or \enquote{\texttt{postscript to}.}
Cf.\ \textsf{suppbook} above for the details.  (See \emph{Manual}
17.74--75; polakow:afterw, prose:intro).

\mylittlespace You can also use the \textsf{type} field in
\textsf{artwork}, \textsf{audio}, \textsf{image}, \textsf{music}, and
\textsf{video} entries to identify the medium of the work, e.g.,
\texttt{oil on canvas}, \texttt{albumen print}, \texttt{compact disc},
or \texttt{MPEG}.  If the first word in this field would normally only
be capitalized at the beginning of a sentence, then leave it in
lowercase in your .bib file and \textsf{biblatex} will automatically
do the right thing in citations.  Cf.\ \textsf{artwork},
\textsf{audio}, \textsf{image}, \textsf{music}, and \textsf{video},
above, for all the details.  (See auden:reading, bedford:photo,
cleese:holygrail, leo:madonna, nytrumpet:art.)

\mybigspace Standard \mymarginpar{\textbf{url}} \textsf{biblatex}
field, it holds the url of an online publication, though you can
provide one for all entry types.  The 16th edition of the
\textsf{Manual} expresses a strong preference for DOIs over URLs if
the former is available --- cf.\ \textsf{doi} above, and also
\textsf{urldate} just below.  The required \LaTeX\ package
\textsf{url} will ensure that your documents format such references
properly, in the text and in the reference apparatus.

\mybigspace Standard \mymarginpar{\textbf{urldate}} \textsf{biblatex}
field, it identifies exactly when you accessed a given url.  The 16th
edition of the \emph{Manual} prefers DOIs to URLs; in the latter case
it allows the use of access dates, particularly in contexts that
require it, but prefers that you use revision dates, if these are
available.  To enable you to specify which date is at stake, I have
provided the \textbf{userd} field, documented below.  If an entry
doesn't have a \textsf{userd}, then the \textsf{urldate} will be
treated, as before, as an access date (14.6--8, 14.184, 15.9;
evanston:library, grove:sibelius, hlatky:hrt, osborne:poison,
sirosh:visualcortex, wikiped:bibtex).  In the default setting of
\cmd{DeclareLabeldate}, any entry without a \textsf{date},
\textsf{eventdate}, or \textsf{origdate} will now use the
\textsf{urldate} to find a year for citations and the list of
references (grove:sibelius, wikiped:bibtex).

% %\enlargethispage{\baselineskip}

\mybigspace A \mymarginpar{\textbf{usera}} supplemental
\textsf{biblatex} field which functions in \textsf{biblatex-chicago}
almost as a \enquote{\textsf{journaltitleaddon}} field.  In
\textsf{article}, \textsf{periodical}, and \textsf{review} entries
with \textsf{entrysubtype} \texttt{maga\-zine}, the contents of this
field will be placed, unformatted and between commas, after the
\textsf{journaltitle} and before the date.  The main use is for
identifying the broadcast network when you cite a radio or television
program (14:221; bundy:macneil).

\mybigspace I \mymarginpar{\textbf{userc}} have now implemented this
supplemental \textsf{biblatex} field as part of the Chicago
author-date style's handling of cross-references within the list of
references.  (The \enquote{c} part is meant as a sort of mnemonic for
this latter function.)  In the 16th edition of the \emph{Manual}, you
no longer need to use the \textbf{customc} entry type to include
alphabetized expansions of \textsf{shorthands} in the reference list,
but you may still need to provide cross-references of some sort to
separate entries in that list, perhaps when a single author uses
multiple pseudonyms.  In such a case it is unlikely that you will cite
the \textsf{customc} entry itself in the body of your text.
Therefore, in order for it to appear in the reference list, you have
two choices.  You can either include the entry key of the
\textsf{customc} entry in a \cmd{nocite} command inside your document,
or you can place that entry key in the \textsf{userc} field of the
.bib entry that actually contains one of the full citations.  In the
latter case, \textsf{biblatex-chicago} will call \cmd{nocite} for you
when you cite the main entry.  (See 14.84, 14.86; creasey:ashe:blast,
creasey:morton:hide, creasey:york:death, lecarre:quest.)

\mybigspace The \colmarginpar{\textbf{userd}} \textsf{userd} field,
recently added to the package, acts as a sort of
\enquote{\textsf{datetype}} field, allowing you in most entry types to
identify whether a \textsf{urldate} is an access date or a revision
date.  The general usage is fairly simple.  If this field is absent,
then a \textsf{urldate} will be treated as an access date, as has long
been the default in \textsf{biblatex} and in
\textsf{biblatex-chicago}.  If you need to identify it in any other
way, what you include in \textsf{userd} will be printed \emph{before}
the \textsf{urldate}, so phrases like \enquote{\texttt{last modified}}
or \enquote{\texttt{last revised}} are what the field will typically
contain (14.7--8; wikiped:bibtex).  In \colmarginpar{New!} the absence
of a \textsf{urldate}, you can now, in most entry types, include a
\textsf{userd} field to qualify a \textsf{date} in the same way it
would have modified a \textsf{urldate}.

\mylittlespace Because of the rather specialized needs of some
audio-visual references, this basic schema changes for \textsf{music}
and \textsf{video} entries.  In \textsf{music} entries where an
\textsf{eventdate} is present, \textsf{userd} will modify that date
instead of any \textsf{urldate} that may also be present, and it will
modify an \textsf{origdate} if it is present and there is no
\textsf{eventdate}.  It will modify a \textsf{date} only in the
absence of the other three.  In \textsf{video} entries it will modify
an \textsf{eventdate} if it is present, and in its absence the
\textsf{urldate}.  Given the absence of those two, it can modify a
\textsf{date}.  In all these cases, \textsf{userd} will modify what
remains of any date, i.e., the month and the day, if that date's year
has been printed at the head of the entry.  Please see the
documentation of the \textsf{music} and \textsf{video} entry types,
and especially of the \textsf{eventdate}, \textsf{origdate}, and
\textsf{urldate} fields, above (14.276--279, 15.53; nytrumpet:art).

\mylittlespace In all cases, you can start the \textsf{userd} field
with a lowercase letter, and \textsf{biblatex} will take care of
automatic contextual capitalization for you.

\mybigspace Another \mymarginpar{\textbf{usere}} supplemental
\textsf{biblatex} field, which \textsf{biblatex-chicago} uses
specifically to provide a translated \textsf{title} of a work,
something that may be needed if you deem the original language
unparseable by a significant portion of your likely readership.  The
\emph{Manual} offers two alternatives in such a situation: either you
can translate the title and use that translation in your
\textsf{title} field, providing the original language in
\textsf{language}, or you can give the original title in
\textsf{title} and the translation in \textsf{usere}.  Cf.\
\textbf{language}, above.  (See 14.108--110, 14.194; kern,
pirumova:russian, weresz.)

%%\enlargethispage{-\baselineskip}

\mybigspace This \mymarginpar{\textbf{userf}} is the last of the
supplemental fields which \textsf{biblatex} provides, and is used by
\textsf{biblatex-chicago} for a very specific purpose.  When you cite
both a translation and its original, the \emph{Manual} (14.109)
recommends that, in a reference list at least, you combine references
to both texts in one entry.  Lacking specific instructions about the
author-date style, I have nonetheless chosen to implement this
possibility also for a list of references, though in-text citations
will still only refer to individual works.  In order to follow this
specification, I have provided a third cross-referencing system (the
others being \textsf{crossref} and \textsf{xref}), and have chosen the
name \textsf{userf} because it might act as a mnemonic for its
function.

\mylittlespace In order to use this system, you should start by
entering both the original and its translation into your .bib file,
just as you normally would.  The mechanism works for any entry type,
and the two entries need not be of the same type.  In the entry for
the \emph{translation}, you put the cite key of the original into the
\textsf{userf} field.  In the \emph{original's} entry, you need to
include some means of preventing it appearing separately in the list
of references, either a toggle in the \textsf{keywords} field or
perhaps \texttt{skipbib} in the \textsf{options} field.  In this
standard case, the data for the translation will be printed first,
followed by the string \texttt{orig. pub. as}, followed by the
original, author omitted (furet:passing:eng, furet:passing:fr).  As
explained above (\textbf{origlanguage}), I have also included a way to
modify the string printed before the original.  In the entry for the
\emph{translation}, you put the original's language in
\textsf{origlanguage}, and instead of \texttt{originally published
  as}, you'll get \texttt{French edition:} or \texttt{Latin edition:},
etc.\ (aristotle:metaphy:gr, aristotle:metaphy:trans).

\mybigspace Standard \mymarginpar{\textbf{venue}} \textsf{biblatex}
offers this field for use in \textsf{proceedings} and
\textsf{inproceedings} entries, but I haven't yet implemented it,
mainly because the \emph{Manual} has nothing to say about it.  Perhaps
the \textsf{organization} field could be used, for the moment,
instead.  Anything in a \textsf{venue} field will be ignored.

\mybigspace Standard \mymarginpar{\textbf{version}} \textsf{biblatex}
field, currently only available in \textsf{misc} and \textsf{patent}
entries in \textsf{biblatex-chicago}.

\mybigspace Standard \colmarginpar{\textbf{volume}} \textsf{biblatex}
field.  It holds the volume of a \textsf{journaltitle} in
\textsf{article} entries, and also the volume of a multi-volume work
in many other sorts of entry.  The treatment and placement of
\textsf{volume} information in \textsf{book}-like entries is rather
complicated in the \emph{Manual} (14.121--27, 15.39).  In the
reference list, the \textsf{volume} appears either before the
\textsf{maintitle} or before the publication information, while in
citations you may need to provide it in the \textsf{postnote} field
--- see the \textsf{volumes} field, just below.  In a number of these
contexts, and in both books and periodicals, \textsf{volume}
information can appear \emph{immediately before} the page number(s).
In such a case, the \emph{Manual} (14.121) prescribes the same
treatment for both sorts of sources, that is, that \enquote{a colon
  separates the volume number from the page number with no intervening
  space.}  I have implemented this, but at the request of Clea~F.\
Rees I have made this punctuation customizable, using the new command
\mycolor{\cmd{postvolpunct}} \colmarginpar{\cmd{postvolpunct}}.  By
default it prints \cmd{addcolon}, but you can use
\cmd{renewcommand\{\textbackslash postvolpunct\}\{\ldots\}} in your
preamble to redefine it.  Cf.\ \textsf{part}, and the command
documentation in section~\ref{sec:formatting:authdate};
conway:evolution shows how sometimes this field may hold series
information, as well.

\mybigspace Standard \colmarginpar{\textbf{volumes}} \textsf{biblatex}
field.  It holds the total number of volumes of a multi-volume work,
and in such references you should provide the volume and page numbers
in the \textsf{postnote} field of the relevant \cmd{cite} command,
e.g.:

\begin{quote}
\cmd{autocite}\texttt{[3:25]\{bibfile:key\}}.  
\end{quote}

Cf.\ 15.22; meredith:letters, tillich:system, weber:saugetiere,
wright:evolution.  The entry wright:theory presents one volume of such
a multi-volume work, so you would no longer need to give the volume in
any \textsf{postnote} field when citing it.  If both a \textsf{volume}
and a \textsf{volumes} field are present, as may occur particularly in
cross-referenced entries, then \textsf{biblatex-chicago} will
ordinarily suppress the \textsf{volumes} field, except in some cases
when a \textsf{maintitle} is present.  In this latter case, if the
\textsf{volume} appears before the \textsf{maintitle}, the new option
\mycolor{\texttt{hidevolumes}}, \colmarginpar{\texttt{hidevolumes}}
set to \texttt{true} by default, controls whether to print the
\textsf{volumes} field after that title or not.  Set it to
\texttt{false} either in the preamble or in the \textsf{options} field
of your entry to have it appear after the \textsf{maintitle}.  See the
option's documentation in section~\ref{sec:authpreset}, below.

\mybigspace A \mymarginpar{\textbf{xref}} modified \textsf{crossref}
field provided by \textsf{biblatex}, which prevents inheritance of any
data from the parent entry.  See \textbf{crossref}, above.

\mybigspace Standard \mymarginpar{\textbf{year}} \textsf{biblatex}
field, especially important for the author-date specification.  Please
see all the details under \textbf{date} above.  Unlike the
\textsf{date} field \textsf{year} allows non-numeric input, so you can
put \cmd{bibstring\{nodate\}} here if required, or indeed any other
sort of non-numerical date information.  If you can guess the date
then you can include that guess in square brackets instead of
\cmd{bibstring\{nodate\}}.  Cf.\ bedford:photo, clark:mesopot,
leo:madonna, ross:thesis.

\subsection{Commands}
\label{sec:commands:authdate}

In this section I shall attempt to document all those commands you may
need when using \textsf{biblatex-chicago-authordate} that I have either
altered with respect to the standard provided by \textsf{biblatex} or
that I have provided myself.  Some of these, unfortunately, will make
your .bib file incompatible with other \textsf{biblatex} styles, but
I've been unable to avoid this.  Any ideas for more elegant, and more
compatible, solutions will be warmly welcomed.

\subsubsection{Formatting Commands}
\label{sec:formatting:authdate}

These commands allow you to fine-tune the presentation of your
references in both citations and list of references.  You can find
many examples of their usage in \textsf{dates-test.bib}, and I shall
try to point you toward a few such entries in what follows.
\textbf{NB:} \textsf{biblatex's} \cmd{mkbibquote} command is now
mandatory in some situations.  See its entry below.

\mybigspace Version \mymarginpar{\textbf{\textbackslash autocap}} 0.8
of \textsf{biblatex} introduced the \cmd{autocap} command, which
capitalizes a word inside a citation or list of references entry if
that word follows sentence-ending punctuation, and leaves it lowercase
otherwise.  The whole question of capitalization is considerably more
complicated in the notes \&\ bibliography style, where the former uses
commas and the latter (often) periods to separate blocks of
information, whereas the more streamlined author-date specification
has few such issues.  In \textsf{dates-test.bib} there are only two
places where the \cmd{autocap} macro is necessary, and they both
involve the string \texttt{forthcoming} in the \textsf{year} field
(author:forthcoming, contrib:contrib).

\mylittlespace I have nonetheless retained the system developed,
following Lehman's example, for the notes \&\ bibliography style,
which automatically tracks the capitalization of certain fields in
your .bib file.  I chose these fields after a non-scientific survey of
entries in my own databases, so of course if you have ideas for the
extension of this facility I would be most interested to hear them.
In order to take advantage of this functionality, all you need do is
begin the data in the appropriate field with a lowercase letter,
e.g.,\ \texttt{note = \{with the assistance of X\}}.  If the data
begins with a capital letter --- and this is not infrequent --- that
capital will always be retained.  (cf., e.g., creel:house,
morgenson:market.)  If, on the other hand, you for some reason need
such a field always to start with a lowercase letter, then you can try
using the \cmd{isdot} macro at the start, which turns off the
mechanism without printing anything itself.  Here, then, for reference
purposes, is the complete list of fields where this functionality is
active:

\begin{enumerate}
\item The \textbf{addendum} field in all entry types.
\item The \textbf{booktitleaddon} field in all entry types.
\item The \textbf{edition} field in all entry types.  (Numerals work
  as you expect them to here.)
\item The \textbf{maintitleaddon} field in all entry types.
\item The \textbf{note} field in all entry types.
\item The \mycolor{\textbf{part}} field in entry types that use it.
\item The \mycolor{\textbf{prenote}} field prefixed to citation
  commands.
\item The \textbf{shorttitle} field in the \textsf{review}
  (\textsf{suppperiodical}) entry type and in the \textsf{misc} type,
  in the latter case, however, only when there is an
  \textsf{entrysubtype} defined, indicating that the work cited is
  from an archive.
\item The \textbf{title} field in the \textsf{review}
  (\textsf{suppperiodical}) entry type and in the \textsf{misc} type,
  in the latter case, however, only when there is an
  \textsf{entrysubtype} defined, indicating that the work cited is
  from an archive.
\item The \textbf{titleaddon} field in all entry types.
\item The \textbf{type} field in \textsf{artwork}, \textsf{audio},
  \textsf{image}, \textsf{music}, \textsf{suppbook},
  \textsf{suppcollection}, and \textsf{video} entry types.
\end{enumerate}

If you accidentally use the \cmd{autocap} macro in one of the above
fields, it frankly shouldn't matter at all, and you'll still get what
you want, but taking advantage of the automatic provisions should at
least save some typing.

\mybigspace This \mymarginpar{\textbf{\textbackslash bibstring}} is
Lehman's very powerful mechanism to allow \textsf{biblatex}
automatically to provide a localized version of a string, and to
determine whether that string needs capitalization, depending on where
it falls in an entry.  \textsf{Biblatex} also provides functionality
which allows you sometimes simply to input, for example,
\texttt{newseries} instead of \cmd{bib\-string\{newseries\}}, the
package auto-detecting when a bibstring is involved and doing the
right thing, though in all such cases either form will work.  This
functionality is available in the \textsf{series} field of
\textsf{article}, \textsf{periodical}, and \textsf{review} entries; in
the \textsf{type} field of \textsf{manual}, \textsf{patent},
\textsf{report}, and \textsf{thesis} entries; in the \textsf{location}
field of \textsf{patent} entries; in the \textsf{language} field in
all entry types; and in the \textsf{nameaddon} field in
\textsf{customc} entries.  These are the places, as far as I can make
out, where \textsf{biblatex's} standard styles support this feature,
though I have added the last, style-specific, one.  If Lehman
generalizes it still further in a future release, I shall do the same,
if possible.

%\enlargethispage{-2\baselineskip}

\mybigspace This \mymarginpar{\textbf{\textbackslash mkbibquote}} is
the standard \textsf{biblatex} command, which requires attention here
because it is a crucial part of the mechanism of Lehman's
\enquote{American} punctuation system.  Quotation marks around the
\textsf{title} field in various entry types are automatically provided
by \textsf{biblatex-chicago}, but titles-within-titles frequently also
require them, so it is best to get accustomed to using this command to
make sure any periods or commas appearing in the neighborhood of the
closing quotes will appear inside them automatically.  A few examples
from \textsf{dates-test.bib} should help to clarify this.

\mylittlespace In an \textsf{article} entry, the \textsf{title}
contains a quoted phrase:

\begin{quotation}
  \noindent\texttt{title = \{Diethylstilbestrol and Media Coverage of the \\
    \indent\cmd{mkbibquote}\{Morning After\} Pill\}}
\end{quotation}

Here, because the quoted text doesn't come at the end of title, and no
punctuation will ever need to be drawn within the closing quotation
mark, you could instead use \texttt{\cmd{enquote}\{Morning After\}} or
even \texttt{`Morning After'}. (Note the single quotation marks here
--- the other two methods have the virtue of taking care of nesting
for you.)  All of these will produce the formatted:
\enquote{Diethylstilbestrol and Media Coverage of the \enquote{Morning
    After} Pill.}

\mylittlespace Here, by contrast, is a \textsf{book title}:

\begin{quotation}
  \noindent \texttt{title = \{Annotations to
    \cmd{mkbibquote}\{Finnegans Wake\}\}}
\end{quotation}

Because the quoted title within the title comes at the end of the
field, and because this reference unit will be separated from
what follows by a period in the list of references, then the
\cmd{mkbibquote} command is necessary to bring that period within the
final quotation marks, like so: \emph{Annotations to
  \enquote{Finnegans Wake.}}

\mylittlespace Note in both cases that you only need to be careful
with the capitalization inside the curly brackets if you are using
\textsf{authordate-trad}, as the 16th edition of the \textsf{Manual}
has unified the title formatting for the two remaining styles, which
means that, for them, all lower- and uppercase letters remain as they
are typed in your .bib file.

\mylittlespace Let me also add that this command interacts well with
Lehman's \textsf{csquotes} package, which I highly recommend, though
the latter isn't strictly necessary in texts using an American style,
to which \textsf{biblatex} defaults when \textsf{csquotes} isn't
loaded.

\mybigspace The \colmarginpar{\textbf{\textbackslash postvolpunct}}
\emph{Manual} (14.121) unequivocally prescribes that when a
\textsf{volume} number appears immediately before a page number,
\enquote{the abbreviation \emph{vol.}\ is omitted and a colon
  separates the volume number from the page number with no intervening
  space.}  The treatment is basically the same whether the citation is
of a book or of a periodical, and it appears to be a surprising and
unwelcome feature for many users, conflicting as it may do with
established typographic traditions in a number of contexts.  Clea~F.\
Rees has requested a way to customize this, so I have provided the new
\cmd{postvolpunct} command, which prints the punctuation between a
\textsf{volume} number and a page number.  It is set to \cmd{addcolon}
by default, except when the current language of the entry is French,
in which case it defaults to \cmd{addcolon\textbackslash addspace}.
You can use \cmd{renewcommand\{\textbackslash
  postvolpunct\}\{\ldots\}} in your preamble to redefine it, but
please note that the command only applies in this limited context, not
more generally to the punctuation that appears between, e.g., a
\textsf{volume} and a \textsf{part} field.

\mybigspace This \mymarginpar{\textbf{\textbackslash partcomp}} and
the following 6 macros were all designed to help
\textsf{biblatex-chicago} cope with the fact that many bibstrings in
the notes \&\ bibliography style differ between notes and
bibliography, the former sometimes using abbreviated forms when the
latter prints them in full.  These problems do not arise in the
author-date styles, but using these macros will make your .bib database
more portable across languages and across both Chicago styles, and may
be slightly easier to remember than the strings themselves.  On the
other hand, of course, they will make your .bib file less portable
across multiple \textsf{biblatex} styles.

\mylittlespace These macros allow you to provide an \texttt{editor}, a
\texttt{translator}, and/or a \texttt{compiler} in situations where
the available fields (\textsf{editor}, \textsf{namea},
\textsf{translator}, \textsf{nameb}, and \textsf{namec}) aren't
adequate.  Their names all begin with \cmd{part}, as originally I
intended them for use when a particular name applied only to a
specific \textsf{title}, rather than to a \textsf{maintitle} or
\textsf{booktitle} (cf.\ \textbf{namea} and \textbf{nameb}, above).

% %\enlargethispage{\baselineskip}

\mylittlespace In the present instance, you can use \cmd{partcomp} to
identify a compiler when \textsf{namec} (or \textsf{editortype}) won't
do, e.g., in a \textsf{note} field or the like.  In such a case,
\textsf{biblatex-chicago} will print the appropriate string in your
references.

\mybigspace Use \mymarginpar{\textbf{\textbackslash partedit}} this
macro when identifying an editor whose name doesn't conveniently fit
into the usual fields (\textsf{editor} or \textsf{namea}).  (N.B.: If
you are writing in French and using \textsf{cms-french.lbx}, then
currently you'll need to add either \texttt{de} or \texttt{d'} after
this command in your .bib files to make the references come out right.
I'm working on this.)  See howell:marriage.

\mybigspace As \mymarginpar{\textbf{\textbackslash
    partedit-\\andcomp}} before, but for use when an editor is also a
compiler.

\vspace{1.3\baselineskip} As \mymarginpar{\textbf{\textbackslash
    partedit-\\andtrans}} before, but for when when an editor is also a
translator (ratliff:review).

\vspace{1.3\baselineskip} As \mymarginpar{\textbf{\textbackslash
    partedit-\\transandcomp}} before, but for when an editor is also a
translator and a compiler.

\vspace{1.4\baselineskip} As \mymarginpar{\textbf{\textbackslash
    parttrans-\\andcomp}} before, but for when a translator is also a
compiler.

\vspace{1.3\baselineskip} As \mymarginpar{\textbf{\textbackslash
    parttrans}} before, but for use when identifying a translator
whose name doesn't conveniently fit into the usual fields
(\textsf{translator} and \textsf{nameb}).

\subsubsection{Citation Commands}
\label{sec:cite:authordate}

The \textsf{biblatex} package is particularly rich in citation
commands, most of which, in \textsf{biblatex-chicago-authordate} and
\textsf{authordate-trad}, function as they do in the standard
author-date styles.  If you are getting unexpected behavior when using
them please have a look in your .log file.  A command like
\cmd{supercite}, listed in �~3.6.2 of the \textsf{biblatex} manual but
not defined by \textsf{biblatex-chicago-authordate} or by core
\textsf{biblatex}, defaults to \cmd{cite}, and leaves a warning in the
.log.  The following commands may require some minimal explanation,
but if there are standard commands that don't work for you, or new
commands that would be useful, please let me know, and it should be
possible to fix or add them.

\mybigspace I \mymarginpar{\textbf{\textbackslash autocite}} haven't
adapted this in the slightest, but I thought it worth pointing out
that \textsf{biblatex-chicago-authordate} sets this command to use
\cmd{parencite} as the default option.  It is, in my experience, much
the most common citation command you will use, and also works fine in
its multicite form, \textbf{\textbackslash autocites}.

\mybigspace In \mymarginpar{\textbf{\textbackslash textcite}} standard
\textsf{biblatex} this command searches first for a
\textsf{labelname}, usually taken from the \textsf{author} or
\textsf{shortauthor} field, then uses the \textsf{shorthand} field if
the former doesn't exist.  Because of the way the Chicago author-date
specification recommends handling abbreviations, I have switched this
around, and the command now searches for a \textsf{shorthand} first.
This holds also for the multicite form \textbf{\textbackslash
  textcites}, though both commands revert to their standard
\textsf{biblatex} behavior when you give the \texttt{cmslos=false}
option in the preamble.

\subsection{Package Options}
\label{sec:opts:authdate}

\subsubsection{Pre-set \textsf{biblatex} Options}
\label{sec:preset:authdate}

Although a quick glance through \textsf{biblatex-chicago.sty} will
tell you which \textsf{biblatex} options the package sets for you, I
thought I might gather them here also for your perusal.  These
settings are, I believe, consistent with the specification, but you
can alter them in the options to \textsf{biblatex-chicago} in your
preamble or by loading the package using
\cmd{usepackage[style=chicago-authordate]\{biblatex\}}, which gives
you the \textsf{biblatex} defaults unless you redefine them yourself
inside the square brackets.

\mylittlespace \textsf{Biblatex-chicago-authordate}
\mymarginpar{\texttt{autocite=\\inline}} and \textsf{authordate-trad}
place references in parentheses by default.

\mybigspace The \mymarginpar{\texttt{citetracker=\\true}} citetracker
for the \cmd{ifciteseen} test is enabled globally.

\mybigspace The \mymarginpar{\texttt{alldates=comp}} specification
calls for the long format when presenting dates, slightly shortened
when presenting date ranges.  Please note that because of the
author-date style's complicated requirements with respect to dates,
there will be cases when printed ranges don't look exactly right ---
cf., e.g., nass:address.  I'm working on this.

\mylittlespace This \mymarginpar{\texttt{ibidtracker=\\constrict}}
enables an \emph{ibidem} mechanism in citations, but only in the most
strictly-defined circumstances.  The Chicago author-date style doesn't
print \enquote{Ibid} in citations, but in general a repeated citation
on the same page will print only the page reference.  Technically,
this should only occur when a source is cited \enquote{more than once
  in one paragraph} (15.26), so you can use the \cmd{citereset}
command from \textsf{biblatex} to achieve the greatest compliance, as
the package only offers automatic resetting on part, chapter, section,
and subsection boundaries, while \textsf{biblatex-chicago}
automatically resets the tracker at page breaks.  (Cf.\
\textsf{biblatex.pdf} �3.1.2.1.)  Whenever there might be any
ambiguity, \textsf{biblatex} should default to printing a more
informative reference.

\mylittlespace If you are going to repeat a source, make sure that the
cite command provides a postnote --- from this release of
\textsf{biblatex-chicago} you'll no longer get any annoying empty
parentheses, but you will get another standard citation, which may add
too much clutter.

%%\enlargethispage{\baselineskip}

\mylittlespace This \mymarginpar{\texttt{labelyear=\\true}} option
tells \textsf{biblatex} to provide the special \textsf{labelyear} and
\textsf{extrayear} fields for author-date styles.

\mylittlespace These \mymarginpar{\textsf{\texttt{maxbibnames\\=10\\
      minbibnames\\=7}}} two options are new, and control the number
of names printed in the list of references when that number exceeds
10.  These numbers follow the recommendations of the \emph{Manual}
(17.29--30), and they are different from those for use in citations.
With \textsf{biblatex} 1.6 you can no longer redefine
\texttt{maxnames} and \texttt{minnames} in the \cmd{printbibliography}
command at the bottom of your document, so \textsf{biblatex-chicago}
now does this automatically for you, though of course you can change
them in your document preamble.  Please see
section~\ref{sec:otherhints:auth} below (and the file
\textsf{cms-dates-sample.pdf}) for hints on dealing with entries with
more than three authors.

\mylittlespace This \mymarginpar{\texttt{pagetracker=\\true}} enables
page tracking for the \cmd{iffirstonpage} and \cmd{ifsamepage}
commands for controlling, among other things, the \emph{ibidem}
mechanism.  It tracks individual pages if \LaTeX\ is in oneside mode,
or whole spreads in twoside mode.

\mylittlespace This \mymarginpar{\texttt{punctfont=\\true}} fixes a
minor problem with punctuation in titles, ensuring that the colon
between a title and a subtitle appears in the correct, matching font.

\mylittlespace This \mymarginpar{\texttt{sortcase=\\false}} turns off
the sorting of uppercase and lowercase letters separately, a practice
which the \emph{Manual} doesn't appear to recommend.

\mylittlespace This \mymarginpar{\texttt{sorting=cms}} setting takes
advantage of the \cmd{DeclareSortingScheme} command provided by
\textsf{biblatex} and \textsf{Biber}, in effect implementing a default
sorting order in the list of references tailored to comply with the
Chicago author-date specification.  Please see the documentation of
\cmd{DeclareSortingScheme} in section~\ref{sec:authformopts}, below.

\mylittlespace This \mymarginpar{\texttt{uniquelist=\\minyear}} option
enables \textsf{biblatex-chicago-authordate} to disambiguate entries
which have more than three \textsf{authors}, but which differ
\emph{after} the first name in the list.  This will only occur when
two such entries have the same \textsf{year} (15.28).  The option is
\textsf{Biber}-only, like the following, which means that this
next-generation \textsc{Bib}\TeX\ replacement is required for the
author-date styles.  Please see \textsf{cms-dates-sample.pdf} (or
\textsf{cms-trad-sample.pdf}) and section~\ref{sec:otherhints:auth},
below, for further details.

\mylittlespace This \mymarginpar{\texttt{uniquename=\\minfull}}
enables the package to distinguish different authors who share a
surname, using initials in the first instance, and whole names if
initials aren't enough (15.21).  The option is \textsf{Biber}-only,
like the previous one.

\enlargethispage{\baselineskip}

\mylittlespace This \mymarginpar{\texttt{usetranslator\\=true}}
enables automatic use of the \textsf{translator} at the head of
entries in the absence of an \textsf{author} or an \textsf{editor}.
In the list of references, the entry will be alphabetized by the
translator's surname.  You can disable this functionality on a
per-entry basis by setting \texttt{usetranslator=false} in the
\textsf{options} field.  Cf.\ silver:gawain.

\subsubsection*{Other \textsf{biblatex} Formatting Options}
\label{sec:authformopts}

I've chosen defaults for many of the general formatting commands
provided by \textsf{biblatex}, including the vertical space between
items in the list of references and between items in the list of
shorthands (\cmd{bibitemsep} and \cmd{lositemsep}).  I define many of
these in \textsf{biblatex-chicago.sty}, and of course you may want to
redefine them to your own needs and tastes.  It may be as well you
know that the \emph{Manual} does state a preference for two of the
formatting options I've implemented by default: the 3-em dash as a
replacement for repeated names in the list of references (15.17--19,
and just below); and the formatting of note numbers, both in the main
text and at the bottom of the page / end of the essay (superscript in
the text, in-line in the notes; 14.19).  The code for this last
formatting is also in \textsf{biblatex-chicago.sty}, and I've wrapped
it in a test that disables it if you are using the \textsf{memoir}
class, which I believe has its own commands for defining these
parameters.  You can also disable it by using the \texttt{footmarkoff}
package option, on which see below.

\mylittlespace Gildas Hamel pointed out that my default definition, in
\textsf{biblatex-chicago.sty}, of \textsf{biblatex's}
\cmd{bibnamedash} didn't work well with many fonts, leaving a line of
three dashes separated by gaps.  He suggested an alternative, which
I've adopted, with a minor tweak to make the dash thicker, though you
can toy with all the parameters to find what looks right with your
chosen font.  The default definition is:
\cmd{renewcommand*\{\textbackslash bibnamedash\}\{\textbackslash
  rule[.4ex]\{3em\}\{.6pt\}\}}.

\mylittlespace At \mymarginpar{\texttt{losnotes}
  \&\\\texttt{losendnotes}} the request of Kenneth Pearce, I have
added two new \texttt{bibenvironments} to
\textsf{chicago-authordate.bbx}, for use with the \texttt{env} option
to the \cmd{printshorthands} command.  The first, \texttt{losnotes},
is designed to allow a list of shorthands to appear inside footnotes,
while \texttt{losendnotes} does the same for endnotes.  Their main
effect is to change the font size, and in the latter case to clear up
some spurious punctuation and white space that I see on my system when
using endnotes.  (You'll probably also want to use the option
\texttt{heading=none} in order to get rid of the [oversized] default,
providing your own within the \cmd{footnote} command.)  Please see the
documentation of \textsf{shorthand} in
section~\ref{sec:fields:authdate} above for further options available
to you for presenting and formatting the list of shorthands.

\mylittlespace The next-generation backend \textsf{Biber} offers
enhanced functionality in many areas, three of which I've implemented
in this release.  If the default definitions don't work well for you,
you can redefine all of them in your document preamble --- see
\textsf{biblatex.pdf} ��4.5.8 and 4.5.5.

\enlargethispage{\baselineskip}

\mylittlespace This \mymarginpar{\cmd{Declare-}\\\texttt{Labelname}}
option allows you to add name fields for consideration when
\textsf{biblatex} is attempting to find a shortened name for in-text
citations.  This, for example, allows a compiler (=\textsf{namec}) to
appear in citations without any other intervention from the user,
rather than requiring a \textsf{shortauthor} field as previous
releases of \textsf{biblatex-chicago} did.  The default definition
currently is
\texttt{\{shortauthor,author,\\shorteditor,namea,editor,nameb,%
  translator,namec\}}.

\mylittlespace This \colmarginpar{\cmd{Declare-}\\\texttt{Labeldate}}
option allows you to alter the order in which \textsf{Biber} and
\textsf{biblatex} search for the year to use both in citations and at
the head of entries in the list of references.  This will also be the
year to which an alphabetical suffix will be appended when an author
has published more than one work in the same year, and the year by
which works will be sorted in the list of references.  In the default
configuration, a year will be searched for in the order \textsf{date,
  eventdate, origdate, urldate}.  This generally suits the Chicago
author-date styles well, except for two situations.  First, when a
reference apparatus contains many entries with multiple dates, it may
be simplest to promote the \textsf{origdate} to the head of the list,
which you can do using the new \mycolor{\texttt{cmsdate} preamble}
option.  This changes the order to \mycolor{\textsf{origdate, date,
    eventdate, urldate}}.  Second, in \textsf{music} and
\textsf{video} entries, and, exceptionally, some \textsf{review}
entries, the general rule is to emphasize the earliest date.  For
these three entry types, then, \cmd{DeclareLabeldate} uses the order
\textsf{eventdate, origdate, date, urldate}.  See \texttt{avdate} in
section~\ref{sec:authpreset}, \texttt{cmsdate} in
section~\ref{sec:authuseropts}, and the \textbf{date} docs in
section~\ref{sec:fields:authdate}.

\mylittlespace The
\colmarginpar{\cmd{Declare-}\\\texttt{SortingScheme}} third
\textsf{Biber} enhancement I have implemented allows you to include
almost any field whatsoever in \textsf{biblatex's} sorting algorithms
for the list of references, so that a great many more entries will be
sorted correctly automatically rather than requiring manual
intervention in the form of a \textsf{sortkey} field or the like.
Code in \textsf{biblatex-chicago.sty} sets the \textsf{biblatex}
option \texttt{sorting=cms}, which is a custom scheme, basically a
Chicago-specific variant of the default \texttt{nyt}.  You can find
its definition in \textsf{chicago-authordate.cbx}.  (Please note that
it now uses the \textsf{labelyear} as its main year component, which
should help improve the automatic sorting of entries by the same
\textsf{author}.)

\mylittlespace The advantages of this scheme are, specifically, that
any entry headed by one of the supplemental name fields
(\textsf{name[a-c]}), a \textsf{manual} entry headed by an
\textsf{organization}, or an \textsf{article} or \textsf{review} entry
with an \textsf{entrysubtype} and headed by a \textsf{journaltitle}
will no longer need a \textsf{sortkey} set.  The two main
disadvantages should only occur very rarely.  First, in
\textsf{author}-less \textsf{article} and \textsf{review} entries
without an \textsf{entrysubtype}, the \textsf{title} will appear
instead of the \textsf{journaltitle}, and since the latter appears
before the former in the sorting scheme, you'll need a
\textsf{sortkey} for proper alphabetization.  The second occurs
because the supplemental name fields are treated differently from the
standard name fields by \textsf{biblatex}.  Ordinarily, you can set,
for example, \texttt{useauthor=false} in the \textsf{options} field to
remove the \textsf{author's} name from consideration for sorting
purposes.  The Chicago-specific option \textsf{usecompiler=false},
however, doesn't remove \textsf{namec} from such consideration, so in
some rare corner cases you may need a \textsf{sortkey}.

\subsubsection{{Pre-set \textsf{chicago} Options}}
\label{sec:authpreset}

At \mymarginpar{\texttt{bookpages=\\true}} the request of Scot Becker,
I have included this rather specialized option, which controls the
printing of the \textsf{pages} field in \textsf{book} entries.  Some
bibliographic managers, apparently, place the total page count in that
field by default, and this option allows you to stop the printing of
this information in the reference list.  It defaults to true, which
means the field is printed, but it can be set to false either in the
preamble, for the whole document, or on a per-entry basis in the
\textsf{options} field (though rather than use this latter method it
would make sense to eliminate the \textsf{pages} field from the
affected entries).

\mylittlespace This \mymarginpar{\texttt{doi=true}} option controls
whether any \textsf{doi} fields present in the .bib file will be
printed in the reference list.  At the request of Daniel Possenriede,
and keeping in mind the \emph{Manual's} preference for this field
instead of a \textsf{url} (15.9), I have added a third switch,
\texttt{only}, which prints the \textsf{doi} if it is present and the
\textsf{url} only if there is no \textsf{doi}.  The package default
remains the same, however --- it defaults to true, which will print
both \textsf{doi} and \textsf{url} if both are present.  The option
can be set to \texttt{only} or to \texttt{false} either in the
preamble, for the whole document, or on a per-entry basis in the
\textsf{options} field.  In \textsf{online} entries, the \textsf{doi}
field will always be printed, but the \texttt{only} switch will still
eliminate any \textsf{url}.

\mylittlespace This \mymarginpar{\texttt{eprint=true}} option controls
whether any \textsf{eprint} fields present in the .bib file will be
printed in the list of references.  It defaults to true, and can be
set to false either in the preamble, for the whole document, or on a
per-entry basis, in the \textsf{options} field.  In \textsf{online}
entries, the \textsf{eprint} field will always be printed.

\mylittlespace This \mymarginpar{\texttt{isbn=true}} option controls
whether any \textsf{isan}, \textsf{isbn}, \textsf{ismn},
\textsf{isrn}, \textsf{issn}, and \textsf{iswc} fields present in the
.bib file will be printed in the list of references.  It defaults to
true, and can be set to false either in the preamble, for the whole
document, or on a per-entry basis, in the \textsf{options} field.

%%\enlargethispage{-\baselineskip}

\mylittlespace Once \mymarginpar{\texttt{numbermonth=\\true}} again
at the request of Scot Becker, I have included this option, which
controls the printing of the \textsf{month} field in all the
periodical-type entries when a \textsf{number} field is also present.
Some bibliographic software, apparently, always includes the month of
publication even when a \textsf{number} is present.  When all this
information is available the \emph{Manual} (17.181) prints everything,
so this option defaults to true, which means the field is printed, but
it can be set to false either in the preamble, for the whole document,
or on a per-entry basis in the \textsf{options} field.

\mylittlespace This \mymarginpar{\texttt{url=true}} option controls
whether any \textsf{url} fields present in the .bib file will be
printed in the reference list.  It defaults to true, and can be set to
false either in the preamble, for the whole document, or on a
per-entry basis, in the \textsf{options} field.  Please note that, as
in standard \textsf{biblatex}, the \textsf{url} field is always
printed in \textsf{online} entries, regardless of the state of this
option.

\mylittlespace This \mymarginpar{\texttt{includeall=\\true}} is the
one option that rules the six preceding, either printing all the
fields under consideration --- the default --- or excluding all of
them.  It is set to \texttt{true} in \textsf{chicago-authordate.cbx},
but you can change it either in the preamble for the whole document
or, for specific fields, in the \textsf{options} field of individual
entries.  The rationale for all of these options is the availability
of bibliographic managers that helpfully present as much data as
possible, in every entry, some of which may not be felt to be entirely
necessary.  Setting \texttt{includeall} to \texttt{true} probably
works just fine for those compiling their .bib databases by hand, but
others may find that some automatic pruning helps clear things up, at
least to a first approximation.  Some per-entry work afterward may
then polish up the details.

\mylittlespace For \mymarginpar{\texttt{avdate=true}} \textsf{music}
and \textsf{video} entries, the 16th edition of the \emph{Manual}
(15.53) strongly recommends both that you provide a recording,
release, or broadcast date for your references and also that this
earlier date should appear in citations and at the head of reference
list entries.  In the default setting of \cmd{DeclareLabeldate},
\textsf{biblatex} searches for dates in the following order:
\textsf{year, eventyear, origyear, urlyear}.  This option changes the
default ordering in \textsf{music} and \textsf{video} entries to the
following: \textsf{eventyear, origyear, year, urlyear}.
\textsf{Review} entries presenting on-line comments have similar
needs, so the same reordering applies to that entry type, too.  If you
simply want to apply the defaults to these three entry types, you can
use \texttt{avdate=false} in the options when loading
\textsf{biblatex-chicago}.  If, however, you want to tailor the
algorithm to your own needs, then you can use \cmd{DeclareLabeldate}
commands in your preamble.  Please be aware, however, that some parts
of the style hard-code the search syntax, and although they take
account of the \texttt{avdate} setting, if you use your own
definitions of \cmd{DeclareLabeldate} the results may, in some corner
cases, surprise.  Please see \textsf{music}, \textsf{review}, and
\textsf{video} in section~\ref{sec:types:authdate}; \textsf{date},
\textsf{eventdate}, \textsf{origdate}, and \textsf{urldate} in
section~\ref{sec:fields:authdate}; and \cmd{DeclareLabeldate} in
section~\ref{sec:authformopts}.

\mylittlespace At \colmarginpar{\texttt{booklongxref=\\true}} the
request of Bertold Schweitzer, I have included two new options for
controlling whether and where \textsf{biblatex-chicago} will print
abbreviated references when you cite more than one part of a given
collection or series.  This option controls whether multiple
\textsf{book}, \textsf{bookinbook}, \textsf{collection}, and
\textsf{proceedings} entries which are part of the same collection
will appear in this space-saving format.  The parent collection itself
will usually be presented in, e.g., a \textsf{book},
\textsf{bookinbook}, \textsf{mvbook}, \textsf{mvcollection}, or
\textsf{mvproceedings} entry, and using \textsf{crossref} or
\textsf{xref} in the child entries will allow such presentation
depending on the value of the option:

\begin{description}
\item[\qquad true:] This is the default.  If you use \textsf{crossref}
  or \textsf{xref} fields in these entry types, by default you will
  \emph{not} get any abbreviated citations in the reference list.
\item[\qquad false:] You'll get abbreviated citations in these entry
  types in the reference list.
\item[\qquad notes,bib:] These two options are carried over from the
  notes \&\ bibliography style; here they are synonymous with
  \texttt{false} and \textsf{true}, respectively.
\end{description}

This option can be set either in the preamble or in the
\textsf{options} field of individual entries.  For controlling the
behavior of \textsf{inbook}, \textsf{incollection},
\textsf{inproceedings}, and \textsf{letter} entries, please see
\mycolor{\texttt{longcrossref}}, below, and also the documentation of
\textsf{crossref} in section~\ref{sec:fields:authdate}.

\mylittlespace This \mymarginpar{\texttt{cmslos=true}} option alters
\textsf{biblatex's} standard behavior when processing the
\textsf{shorthand} field.  Chicago's author-date style only seems to
recommend the use of shorthands as abbreviations for long authors'
names, particularly institutional names, which means the shorthand
will replace only the name part in citations rather than the whole
citation (15.36; bsi:abbreviation, iso:electrodoc).  The 16th edition
now suggests placing the abbreviation at the head of the entry,
followed by its expansion inside parentheses, an arrangement
automatically provided by \textsf{biblatex-chicago-authordate} when
you use the \textsf{shorthand} field, assuming you retain the default
setting of this option.  Please note that you can still print a list
of shorthands if you wish, and you can also get back something
approaching the \enquote{standard} behavior of shorthands if you give
the \texttt{cmslos=false} option to \textsf{biblatex-chicago} in your
document preamble.  Cf.\ section~\ref{sec:fields:authdate},
s.v. \enquote{\textbf{shorthand}} above, and also
\textsf{cms-dates-sample.pdf}.

\mylittlespace If \colmarginpar{\texttt{hidevolumes=\\true}} both a
\textsf{volume} and a \textsf{volumes} field are present, as may occur
particularly in cross-referenced entries, then
\textsf{biblatex-chicago} will ordinarily suppress the
\textsf{volumes} field.  In some instances, when a \textsf{maintitle}
is present, this may not be the desired result.  In this latter case,
if the \textsf{volume} appears before the \textsf{maintitle}, this new
option, set to \texttt{true} by default, controls whether to print the
\textsf{volumes} field after that title or not.  Set it to
\texttt{false} either in the preamble or in the \textsf{options} field
of your entry to have it appear after the \textsf{maintitle}.

\mylittlespace This \colmarginpar{\texttt{longcrossref=\\false}} is
the second option, requested by Bertold Schweitzer, for controlling
whether and where \textsf{biblatex-chicago} will print abbreviated
references when you cite more than one part of a given collection or
series.  It controls the settings for the entry types more-or-less
authorized by the \emph{Manual}, i.e., \textsf{inbook},
\textsf{incollection}, \textsf{inproceedings}, and \textsf{letter}.
The mechanism itself is enabled by multiple \textsf{crossref} or
\textsf{xref} references to the same parent, whether that be, e.g., a
\textsf{collection}, an \textsf{mvcollection}, a \textsf{proceedings},
or an \textsf{mvproceedings} entry.  Given these multiple cross
references, the presentation in the reference apparatus will be
governed by the following options:

\begin{description}
\item[\qquad false:] This is the default.  If you use
  \textsf{crossref} or \textsf{xref} fields in the four mentioned
  entry types, you'll get the abbreviated entries in the reference
  list.
\item[\qquad true:] You'll get no abbreviated citations of these entry
  types in the reference list.
\item[\qquad none:] This switch is special, allowing you with one
  setting to provide abbreviated citations not just of the four entry
  types mentioned but also of \textsf{book}, \textsf{bookinbook},
  \textsf{collection}, and \textsf{proceedings} entries.
\item[\qquad notes,bib:] These two options are carried over from the
  notes \&\ bibliography style; here they are synonymous with
  \texttt{false} and \textsf{true}, respectively.
\end{description}

This option can be set either in the preamble or in the
\textsf{options} field of individual entries.  For controlling the
behavior of \textsf{book}, \textsf{bookinbook}, \textsf{collection},
and \textsf{proceedings} entries, please see
\mycolor{\texttt{booklongxref}}, above, and also the documentation of
\textsf{crossref} in section~\ref{sec:fields:authdate}.

%\enlargethispage{\baselineskip}

\mylittlespace This \mymarginpar{\texttt{nodates=true}} option means
that for all entry types except \textsf{inreference}, \textsf{misc},
and \textsf{reference}, \textsf{biblatex-chicago} will automatically
provide \cmd{bibstring\{nodates\}} for any entry that doesn't
otherwise provide a date for citations and for the heads of entries in
the list of references.  If you set \texttt{nodates=false} in your
preamble, then the package won't perform this substitution in any
entry type whatsoever.  (The bibstring expands to
\enquote{\texttt{n.d.}} in English.)

\mylittlespace This \mymarginpar{\texttt{usecompiler=\\true}} option
enables automatic use of the name of the compiler (in the
\textsf{namec} field) at the head of an entry, usually in the absence
of an \textsf{author}, \textsf{editor}, or \textsf{translator}, in
accordance with the specification (\emph{Manual} 15.35).  It may also,
like \texttt{useauthor}, \texttt{useeditor}, and
\texttt{usetranslator}, be disabled on a per-entry basis by setting
\texttt{usecompiler=false} in the \textsf{options} field.  The only,
subtle, difference between this switch and those standard
\textsf{biblatex} switches is that this one won't remove
\textsf{namec} from the sorting list, whereas \texttt{useauthor=false}
and \texttt{useeditor=false} do remove the \textsf{author} and
\textsf{editor}.  You may, therefore, in corner cases, require a
\textsf{sortkey} in the entry.

\subsubsection{Style Options -- Preamble}
\label{sec:authuseropts}

These are parts of the specification that not everyone will wish to
enable.  All except the fourth can be used even if you load the
package in the old way via a call to \textsf{biblatex}, but most users
can just place the appropriate string(s) in the options to the
\cmd{usepackage\{biblatex-chicago\}} call in your preamble.

\mylittlespace At \mymarginpar{\texttt{annotation}} the request of
Emil Salim, I have added to this version of \textsf{biblatex-chicago}
the ability to produce annotated reference lists.  If you turn this
option on then the contents of your \textsf{annotation} (or
\textsf{annote}) field will be printed after the reference.  (You can
also use external files to store annotations -- please see
\textsf{biblatex.pdf} �~3.11.8 for details on how to do this.)  This
functionality is currently in a beta state, so before you use it
please have a look at the documentation for the \textsf{annotation}
field, in section~\ref{sec:fields:authdate} above.

\mylittlespace With \colmarginpar{\texttt{cmsdate}} this release, I am
providing a new method for simplifying the creation of databases with
a great many multi-date entries: the \texttt{cmsdate} option \emph{in
  the preamble}.  Despite warnings in previous releases, users have
plainly already been setting this option in their preambles, so I
thought I might at least attempt to make it work as
\enquote{correctly} as I can.  The switches for this option are
basically the same as for the entry-only option, that is, assuming an
entry presents a reprinted edition of a work by Smith, first published
in 1926 (the \textsf{origdate}) and reprinted in 1985 (the
\textsf{date}):

\begin{enumerate}
\item \texttt{cmsdate=off} is the default: (Smith 1985).
\item \texttt{cmsdate=both} prints both the \textsf{origdate} and the
  \textsf{date}, using the \emph{Manual's}\ standard format: (Author
  [1926] 1985) in parenthetical citations, Author (1926) 1985 outside
  parentheses, e.g., in the reference list.
\item \texttt{cmsdate=on} prints the \textsf{origdate} at the head of
  the entry in the list of references and in citations: (Author 1926).
  NB: The \emph{Manual} no longer includes this among the approved
  options.  If you want to present the \textsf{origdate} at the head
  of an entry, then generally speaking you should probably use
  \texttt{cmsdate=both}.  I have nevertheless retained this option for
  certain cases where it has proved useful.  The 15th-edition options
  \texttt{new} and \texttt{old} now work like \texttt{both}.
\end{enumerate}

The important change for the user is that, when you set this option in
your preamble to \texttt{on} or \texttt{both} (or to the 15th-edition
synonyms for the latter, \texttt{new} or \texttt{old}), then
\textsf{biblatex-chicago-authordate} (and \textsf{authordate-trad})
will now change the default \cmd{DeclareLabel\-date} definition so
that the \textsf{labelyear} search order will be
\mycolor{\textsf{origdate, date, eventdate, urldate}}.  This means
that for entry types not covered by the \texttt{avdate} option, and
for those types as well if you turn off that option, the
\textsf{labelyear} will, in any entry containing an \textsf{origdate},
be that very date.  If you want \emph{every} such entry to present its
\textsf{origdate} in citations and at the head of reference list
entries, then setting the option this way makes sense, as you should
automatically get the proper \textsf{extrayear} letter
(1926\textbf{a}) and the correct sorting, without having to use the
counter-intuitive .bib file date switching that sometimes accompanied
the entry-only \texttt{cmsdate} option.  A few clarifications may yet
be in order.

\mylittlespace Obviously, any entry with only a \textsf{date} should
behave as usual.  Also, since \textsf{patent} entries have fairly
specialized needs, I have exempted them from this change to
\cmd{DeclareLabeldate}.  Third, the per-entry \texttt{cmsdate} options
will still affect which dates are printed in citations and at the head
of reference list entries, but they cannot change the search order for
the \textsf{labeldate}.  This will be fixed by the preamble option.
Fourth, if you have been used to switching the \textsf{date} and the
\textsf{origdate} to get the correct results, then you should be aware
that this mechanism may actually still be useful when using the
\texttt{on} switch to \texttt{cmsdate} in the preamble, but it
produces incorrect results when the \texttt{cmsdate} option is
\texttt{both} in the preamble and the individual entry.  The preamble
option is designed to make the need for this switching as rare as
possible, so some editing of existing databases may be necessary.
Fifth, the entry-only option \texttt{full} has no effect at all when
used in the preamble; you must set it in individual entries.  Finally,
please see the documentation of the \textbf{date} field in
section~\ref{sec:fields:authdate} for the fullest discussion of date
presentation in the \textsf{authordate} styles.

\mylittlespace Although \colmarginpar{\texttt{cmsorigdate}} I can't
currently think of any reason why anyone would want to use it on its
own, I should nonetheless mention that the \texttt{cmsorigdate} option
in your preamble will change the default \cmd{DeclareLabeldate}
settings to \mycolor{\textsf{origdate, date, eventdate, urldate}}.
Setting \texttt{cmsdate} to \texttt{on} or \texttt{both} in the
preamble --- see the previous option --- sets this to true, but if for
some reason you want to set it to true without any of the other
effects of the \texttt{cmsdate} option, then you can.  The effects may
surprise.

\mylittlespace When \colmarginpar{\texttt{compresspages}} set to
\texttt{true}, any page ranges in your .bib file or in the
\textsf{postnote} field of your citation commands will be compressed
in accordance with the \emph{Manual's} specifications (9.60).
Something like 321-{-}328 in your .bib file would become 321--28 in
your document.  See the \textsf{pages} field in
section~\ref{sec:fields:authdate}, above.

\mylittlespace Although \mymarginpar{\texttt{footmarkoff}} the
\emph{Manual} (14.19) recommends specific formatting for footnote (and
endnote) marks, i.e., superscript in the text and in-line in foot- or
endnotes, Charles Schaum has brought it to my attention that not all
publishers follow this practice, even when requiring Chicago style.  I
have retained this formatting as the default setup, but if you include
the \texttt{footmarkoff} option, \textsf{biblatex-chicago} will not
alter \LaTeX 's (or the \textsf{endnote} package's) defaults in any
way, leaving you free to follow the specifications of your publisher.
I have placed all of this code in \textsf{biblatex-chicago.sty}, so if
you load the package with a call to \textsf{biblatex} instead, then
once again footnote marks will revert to the \LaTeX\ default, but of
course you also lose a fair amount of other formatting, as well.  See
section~\ref{sec:loading:auth}, below.

\mylittlespace Several \mymarginpar{\texttt{headline}\\\texttt{(trad
    only)}} users requested an option that turned off the automatic
transformations that produce sentence-style capitalization in the
title fields of the 15th-edition author-date style.  I have,
therefore, also included it in \textsf{authordate-trad}.  If you set
this option, the word case in your title fields will not be changed in
any way, that is, this doesn't automatically transform your titles
into headline-style, but rather allows the .bib file to determine
capitalization.  It works by redefining the command
\cmd{MakeSentenceCase}, so in the unlikely event you are using the
latter anywhere in your document please be aware that it will also be
turned off there.

\enlargethispage{\baselineskip}

\mylittlespace The \mymarginpar{\texttt{juniorcomma}} \emph{Manual}
(6.47) states that \enquote{commas are not required around \emph{Jr.}\
  and \emph{Sr.},} so by default \textsf{biblatex-chicago} has
followed standard \textsf{biblatex} in using a simple space in names
like \enquote{John Doe Jr.}  Charles Schaum has pointed out that
traditional \textsc{Bib}\TeX\ practice was to include the comma, and
since the \emph{Manual} has no objections to this, I have provided an
option which allows you to turn this behavior back on, either for the
whole document or on a per-entry basis.  Please note, first, that
numerical suffixes (John Doe III) never take the comma.  The code
tests for this situation, and detects cardinal numbers well, but if
you are using ordinals you may need to set this to \texttt{false} in
the \textsf{options} field of some entries.  Second, I have fixed a
bug in older releases which always printed the \enquote{Jr.}\ part of
the name immediately after the surname, even when the surname came
before the given names (as in a reference list).  The package now
correctly puts the \enquote{Jr.}\ part at the end, after the given
names, and in this position it always takes a comma, the presence of
which is unaffected by this option.

\mylittlespace This \mymarginpar{\texttt{natbib}} may look like the
standard \textsf{biblatex} option, but to keep the coding of
\textsf{biblatex-chicago.sty} simpler for the moment I have
reimplemented it there, from whence it is merely passed on to
\textsf{biblatex}.  If you load the Chicago style with
\cmd{usepackage\{biblatex-chicago\}}, then the option should simply
read \texttt{natbib}, rather than \texttt{natbib=true}.  The shorter
form also works if you use \cmd{usepackage}\\
\texttt{[style=chicago-authordate]\{biblatex\}}, so I hope this
requirement isn't too onerous.

\mylittlespace At \mymarginpar{\texttt{noibid}} the request of an
early tester, I have included this option to allow you globally to
turn off the \texttt{ibidem} mechanism that
\textsf{biblatex-chicago-authordate} uses by default.  This mechanism
doesn't actually print \enquote{Ibid,} but rather includes only the
\textsf{postnote} information in a citation, i.e., it will print (224)
instead of (Author 2000, 224).  Setting this option will mean that
none of these shortened citations will appear automatically.  For more
fine-grained control of individual citations you'll probably want to
use the \cmd{citereset} command, allied possibly with the
\textsf{biblatex}\ \texttt{citereset} option, on which see
\textsf{biblatex.pdf} �3.1.2.1.

\mylittlespace Originally
\colmarginpar{\texttt{postnotepunct}\\(experimental)} designed for the
notes \&\ bibliography style, this option may in fact be more useful in
the \textsf{authordate} styles.  If set to \texttt{true}, it allows
you to alter the punctuation that appears just before the
\textsf{postnote} argument of citation commands, simplifying in
particular the provision of comments within parenthetical citations.
In previous releases, you either needed to include the comment after a
page number, e.g., \cmd{autocite[16; some comment]\{citekey\}}, or
provide a separate .bib entry using the \textsf{customc} entry type,
e.g., \cmd{autocites\{chicago:man\-ual\}\{chicago:comment\}}.  Now, with
this option enabled, \cmd{autocite[; some comment]\{citekey\}} will
do.  More generally, the \mycolor{\texttt{postnotepunct}} option
allows you to start the \textsf{postnote} field with a punctuation
mark (.\,,\,;\,:) and have it appear as the \cmd{postnotedelim} in
place of whatever the package might otherwise automatically have
chosen.  Please note that this functionality relies on a very nifty
macro by Philipp Lehman which I haven't extensively tested, so I'm
labeling this option \mycolor{experimental}.  Note also that the
option only affects the \textsf{postnote} field of citation commands,
not the \textsf{pages} field in your .bib file.

\mylittlespace Kenneth Pearce \mymarginpar{\texttt{shorthandfull}} has
suggested that, in some fields of study, a list of shorthands
providing full bibliographical information may replace the list of
references itself.  This option, which must be used in tandem with
\texttt{cmslos=false}, prints this full information in the list of
shorthands, though of course you should remember that any .bib entry
not containing a \textsf{shorthand} field won't appear in such a list.
Please see the documentation of the \textsf{shorthand} field in
section~\ref{sec:fields:authdate} above for information on further
options available to you for presenting and formatting the list of
shorthands.

\mylittlespace This \mymarginpar{\texttt{strict}} still-experimental
option attempts to follow the \emph{Manual}'s recommendations (14.36)
for formatting footnotes on the page, using no rule between them and
the main text unless there is a run-on note, in which case a short
rule intervenes to emphasize this continuation.  I haven't tested this
code very thoroughly, and it's possible that frequent use of floats
might interfere with it.  Let me know if it causes problems.

\subsubsection{Style Options -- Entry}
\label{sec:authentryopts}

These options are settable on a per-entry basis in the
\textsf{options} field; both relate to the presentation of dates in
citations and the list of references.

\mylittlespace The \colmarginpar{\texttt{cmsdate}} 16th edition of the
\emph{Manual} has simplified the options for entries with more than
one date (15.38).  You can choose among them using the
\texttt{cmsdate} entry option.  It has 3 possible states relevant to
this problem, alongside a fourth which I discuss below.  An example
should make this clearer.  Let us assume that an entry presents a
reprinted edition of a work by Smith, first published in 1926 (the
\textsf{origdate}) and reprinted in 1985 (the \textsf{date}):

\begin{description}
\item[\qquad off:] This is the default.  The citation will look like
  (Smith 1985).
\item[\qquad both:] The citation will look like (Smith [1926] 1985).
\item[\qquad on:] The citation will look like (Smith 1926).  NB: The
  \emph{Manual} no longer includes this among the approved options.
  If you want to present the \textsf{origdate} at the head of an
  entry, then generally speaking you should probably use
  \texttt{cmsdate=both}.  I have retained the option because in some
  cases it is still useful.  The 15th-edition options \texttt{new} and
  \texttt{old} now work like \texttt{both}.
\end{description}

As I explained in detail above in section~\ref{sec:fields:authdate},
s.v.\ \enquote{\textbf{date},}\ because \textsf{biblatex's} sorting
algorithms and automatic creation of the \textsf{extrayear} field
refer by default to the \textsf{date} before the \textsf{origdate}
when both are present, there may be situations when you need to have
the \emph{earlier} year in the \textsf{date} field, and the later one
in \textsf{origdate}, e.g., if you have another reprinted work by the
same author originally printed in the same year.
\textsf{Biblatex-chicago-authordate} will automatically detect this
switch, and given the same reprinted work as above, the results will
be as follows:

\begin{description}
\item[\qquad off:] This is the default.  The citation will look like
  (Smith 1926a).  This style is no longer recommended by the 16th
  edition of the \emph{Manual}.
\item[\qquad both:] The citation will look like (Smith [1926a] 1985).
  The 15th-edition options \texttt{old} and \texttt{new} are now
  synonyms for this.
\item[\qquad on:] The citation will look like (Smith 1926a).  As noted
  above, this style is no longer recommended by the 16th edition of
  the \emph{Manual}.
\end{description}

If, \mymarginpar{\texttt{switchdates}} for any reason, simply
switching the \textsf{date} and the \textsf{origdate} isn't possible
in a given entry, then you can put \texttt{switchdates} in the
\textsf{options} field to achieve the same result.  Also, now you can
use the new preamble version \colmarginpar{\texttt{cmsdate}\\\emph{in
    preamble}} of \mycolor{\texttt{cmsdate}} to change the default
order of \cmd{DeclareLabeldate}, generally making this date-switching
in your .bib file unnecessary.  Please take a look at the full
documentation of the \textbf{date} field to which I referred just
above, at the preamble \texttt{cmsdate} documentation in
section~\ref{sec:authuseropts}, and also at
\textsf{cms-dates-sample.pdf} and \textsf{dates-test.bib} for examples
of how all this works.

\mylittlespace Bertold Schweitzer has brought to my attention certain
difficult corner cases involving cross-referenced works with more than
one date.  In order to facilitate the accurate presentation of such
sources, I made a slight change to the way \texttt{cmsdate=on}
and \texttt{cmsdate=both} work.  If, and only if, a work has only one
date, and there is no \texttt{switchdates} in the \textsf{options}
field, then \texttt{cmsdate=on} and \texttt{cmsdate=both} will both
result in the suppression of the \textsf{extrayear} field in that
entry.  Obviously, if the same options are set in the preamble, this
behavior is turned off, so that single-date entries will still work
properly without manual intervention.

\mylittlespace The 16th edition of the \emph{Manual} now specifies
that it is \enquote{usually sufficient to cite newspaper and magazine
  articles entirely within the text} (15.47).  This will apply mainly
to \textsf{article} and \textsf{review} entries with
\textsf{entrysubtype} \texttt{magazine}, and involves a parenthetical
citation giving the \textsf{journaltitle} and then the full
\textsf{date}, not just the year, with any other relevant identifying
information incorporated into running text.  (Cf.\ 14.206.)\ In order
to facilitate this, I have added a further switch to the
\texttt{cmsdate} option \mymarginpar{\texttt{cmsdate=full}} ---
\texttt{full} --- which \emph{only} affects the presentation of
citations, and causes the printing of the full date specification
there.  You can use the standard \textsf{biblatex} \texttt{skipbib}
option to keep such entries from appearing in the list of references,
and you may, if your .bib entry is a complete one, also need
\texttt{useauthor=false} in order to ensure that the
\textsf{journaltitle} appears in the citations rather than the
\textsf{author}.

\subsection{General Usage Hints}
\label{sec:hints:auth}

\subsubsection{Loading the Styles}
\label{sec:loading:auth}

With the addition of the \textsf{authordate-trad} style to the
package, there are now three keys for choosing which style to load,
\texttt{notes}, \texttt{authordate}, and \textsf{authordate-trad}, one
of which you put in the options to the \cmd{usepackage} command.  With
early versions of \textsf{biblatex-chicago}, the standard way of
loading the package was via a call to \textsf{biblatex}, e.g.:
\begin{quote}
  \cmd{usepackage[style=chicago-authordate,strict,backend=biber,\%\\
    babel=other,bibencoding=inputenc]\{biblatex\}}
\end{quote}
Now, the default way to load the style, and one that will in the
vast majority of standard cases produce the same results as the old
invocation, will look like this:
\begin{quote}
  \cmd{usepackage[authordate,strict,backend=biber,autolang=other,\%\\
    bibencoding=inputenc]\{biblatex-chicago\}}
\end{quote}

If you read through \textsf{biblatex-chicago.sty}, you'll see that it
sets a number of \textsf{biblatex} options aimed at following the
Chicago specification, as well as setting a few formatting variables
intended as reasonable defaults (see section~\ref{sec:preset:authdate},
above).  Some parts of this specification, however, are plainly more
\enquote{suggested} than \enquote{required,} and indeed many
publishers, while adopting the main skeleton of the Chicago style in
citations, nonetheless maintain their own house styles to which the
defaults I have provided do not conform.

\mylittlespace If you only need to change one or two parameters, this
can easily be done by putting different options in the call to
\textsf{biblatex-chicago} or redefining other formatting variables in
the preamble, thereby overriding the package defaults.  If, however,
you wish more substantially to alter the output of the package,
perhaps to use it as a base for constructing another style altogether,
then you may want to revert to the old style of invocation above.
You'll lose all the definitions in \textsf{biblatex-chicago.sty},
including those to which I've already alluded and also the code that
sets the note number in-line rather than superscript in endnotes or
footnotes.  Also in this file is the code that calls all of the
package's localization files, which means that you'll lose all the
Chicago-specific bibstrings I've defined unless you provide, in your
preamble, a \cmd{DeclareLanguageMapping} command, or several, adapted
for your setup, on which see section~\ref{sec:international} below and
also ��~4.9.1 and 4.11.8 in Lehman's \textsf{biblatex.pdf}.

\mylittlespace What you \emph{will not} lose is the ability to call
the package options \texttt{annotation, strict, cmslos=false} and
\texttt{noibid} (section~\ref{sec:authuseropts}, above), in case these
continue to be useful to you when constructing your own modifications.
There's very little code, therefore, actually in
\textsf{biblatex-chicago.sty}, but I hope that even this minimal
separation will make the package somewhat more adaptable.  Any
suggestions on this score are, of course, welcome.

\subsubsection{Other Hints}
\label{sec:otherhints:auth}

Starting with \textsf{biblatex} version 1.5, in order to adhere to the
author-date specification you will need to use \textsf{Biber} to
process your .bib files, as \textsc{Bib}\TeX\ (and its more recent
variants) will no longer provide all the required features.  This
document assumes that you are using \textsf{Biber}; if you wish to
continue using \textsc{Bib}\TeX\ then you need \textsf{biblatex}
version 1.4c and, if you have any problems with the current release,
possibly \textsf{biblatex-chicago} 0.9.7a.

\mylittlespace If your .bib file contains a large number of entries
with more than three authors, then you may run into some limitations
of the \textsf{biblatex-chicago} code.  The default settings in the
package are \texttt{maxnames=3,minnames=1} in citations and
\texttt{max\-bibnames=10,minbibnames=7} in the list of references.  In
practice, this means that an entry like hlatky:hrt, with 5 authors,
will present all of them in the list of references but will truncate
to one in citations, like so: (Hlatky et al. 2002).  For the vast
majority of circumstances, these settings are exactly right for the
Chicago author-date specification.  However, if \enquote{a reference
  list includes another work of the same date that would also be
  abbreviated as [\enquote{Hlatky et al.}] but whose coauthors are
  different persons or listed in a different order, the text citations
  must distinguish between them} (15.28).  The new
(\textsf{Biber}-only) \textsf{biblatex} option \texttt{uniquelist},
set for you in \textsf{biblatex-chicago.sty}, will automatically
handle many of these situations for you, but it is as well to
understand that it does so by temporarily suspending the limits,
listed above, on how many names to print in a citation.  Without
\texttt{uniquelist}, \textsf{biblatex} would present such a work as,
e.g., (Hlatky et al. 2002b), while hlatky:hrt would be (Hlatky et
al. 2002a).  This does distinguish between them, but inaccurately, as
it suggests that the two different author lists are exactly the same.
With \texttt{uniquelist}, the two citations might look like (Hlatky,
Boothroyd et al.\ 2002) and (Hlatky, Smith et al.\ 2002), which is
what the specification requires.

\mylittlespace If, however, the distinguishing name occurs further
down the author list --- in fourth or fifth position in our examples
--- then the default settings would produce citations with all 4 or 5
names printed, which can become awkwardly long.  In such a situation,
you can provide \textsf{shortauthor} fields that look like this:
\{\{Hlatky et al., \textbackslash mkbibquote\{Quality of Life,\}\}\}
and \{\{Hlatky et al., \textbackslash mkbibquote\{Depressive
Symptoms,\}\}\}, using a shortened title to distinguish the
references.  This would produce (Hlatky et al., \enquote{Quality of
  Life,} 2002) and (Hlatky et al., \enquote{Depressive Symptoms,}
2002), again as the spec requires.  There is, unfortunately, no
simpler way that I know of to deal with this situation.

\mylittlespace One useful rule, when you are having difficulty
creating a .bib entry, is to ask yourself whether all the information
you are providing is strictly necessary.  The Chicago specification is
a very full one, but the \emph{Manual} is actually, in many
circumstances, fairly relaxed about how much of the data from a work's
title page you need to fit into a reference.  Authors of introductions
and afterwords, multiple publishers in different countries, the real
names of authors more commonly known under pseudonyms, all of these
are candidates for exclusion if you aren't making specific reference
to them, and if you judge that their inclusion won't be of particular
interest to your readers.  Of course, any data that may be of such
interest, and especially any needed to identify and track down a
reference, has to be present, but sometimes it pays to step back and
reevaluate how much information you're providing.  I've tried to make
\textsf{biblatex-chicago} robust enough to handle the most complex,
data-rich citations, but there may be instances where you can save
yourself some typing by keeping it simple.

% %\enlargethispage{\baselineskip}

\mylittlespace Scot Becker has pointed out to me that the inverse
problem not only exists but may well become increasingly common, to
wit, .bib database entries generated by bibliographic managers which
helpfully provide as much information as is available, including
fields that users may well wish not to have printed (ISBN, URL, DOI,
\textsf{pagetotal}, inter alia).  The standard \textsf{biblatex}
styles contain a series of options, detailed in \textsf{biblatex.pdf}
�3.1.2.2, for controlling the printing of some of these fields, and
with this release I have implemented the ones that are relevant to
\textsf{biblatex-chicago}, along with a couple that Scot requested and
that may be of more general usefulness.  There is also a general
option to excise with one command all the fields under consideration
-- please see section~\ref{sec:authpreset} above.

\mylittlespace Finally, allow me to reiterate what Philipp Lehman says
in \textsf{biblatex.pdf}, to wit, if you aren't going to use
\textsf{Biber}, use \textsf{bibtex8}, rather than standard
\textsc{Bib}\TeX, and avoid the cryptic errors that ensue when your
.bib file gets to a certain size.

\section{Internationalization}
\label{sec:international}

Several users have requested that, in line with analogous provisions
in other \enquote{American} \textsf{biblatex} styles (e.g.,
\textsf{biblatex-apa} and \textsf{biblatex-mla}), I include facilities
for producing a Chicago-like style in other languages.  I have
supplied three lbx files, \textsf{cms-german.lbx}, its clone
\textsf{cms-ngerman.lbx}, and \textsf{cms-french.lbx}, in at least
partial fulfillment of this request.  For this release, Antti-Juhani
Kaijahano has very kindly provided \textsf{cms-finnish.lbx} for
speakers of that language, thereby adding to the generous
contributions of Baldur Kristinsson (\textsf{cms-icelandic.lbx}) and
H�kon Malmedal (\textsf{cms-norsk.lbx}, \textsf{cms-norwegian.lbx},
and \textsf{cms-nynorsk.lbx}).  I include \textsf{cms-british.lbx} in
order to simplify and to improve the package's handling of
non-American typographical conventions in English.  This means that
all --- or at least most --- of the Chicago-specific bibstrings are
now available for documents and reference apparatuses written in these
languages, with, as I intend, more languages to follow, limited mainly
by my finite time and even-more-finite competence.  (If you would like
to provide bibstrings for a language in which you want to work, or
indeed correct deficiencies in the lbx files contained in the package,
please contact me.)

\mylittlespace Using \mymarginpar{\textbf{babel}} these facilities is
fairly simple.  By default, and this functionality remains the same as
it was in the previous release of \textsf{biblatex-chicago}, calls to
\cmd{DeclareLanguage\-Mapping} in \textsf{biblatex-chicago.sty} will
automatically load the American strings, and also \textsf{biblatex's}
American-style punctuation tracking, when you:
\begin{enumerate}
\item Load \textsf{babel} with \texttt{american} as the main text
  language.
\item Load \textsf{babel} with \texttt{english} as the main text
  language.
\item[] \qquad \emph{or}
\item Do not load \textsf{babel} at all.
\end{enumerate}
(This last is a change from the \textsf{biblatex} defaults --- cp.\
�~3.10.1 in \textsf{biblatex.pdf} --- but it seems to me reasonable,
in an American citation style, to expect this arrangement to work well
for the majority of users.)

\mylittlespace If, for whatever reason, you wanted to use
\textsf{biblatex-chicago} but retain British typographical conventions
--- punctuation outside of quotation marks, outer quotes single rather
than double, etc.\ --- then you no longer need to follow the
complicated rules outlined in previous releases of
\textsf{biblatex-chicago}.  Instead, simply load \textsf{babel} with
the \texttt{british} option.

\mylittlespace If you want to use Finnish, French, German, Icelandic,
or Norwegian strings in the reference apparatus, then you can load
\textsf{babel} with \texttt{finnish}, \texttt{french},
\texttt{german}, \texttt{icelandic}, \texttt{ngerman}, \texttt{norsk},
or \texttt{nynorsk} as the main document language.  You no longer need
any calls to \cmd{DeclareLanguageMapping} in your document preamble,
since \textsf{bib\-latex-chicago.sty} automatically provides these if
you load the package in the standard way.

\mylittlespace You can also define which bibstrings to use on an
entry-by-entry basis by using the \textsf{hyphenation} field in your
bib file, but you will have to make sure that the Chicago-specific
strings for the given language are loaded using a
\cmd{DeclareLanguageMapping} call in the preamble.  Indeed, if
\texttt{american} isn't the main text language when loading
\textsf{babel}, then in order to have access to those strings you'll
need \cmd{DeclareLanguageMapping\{american\}\{cms-american\}} in your
preamble, as \textsf{biblatex-chicago.sty} won't load it for you.

\mylittlespace Three other hints may be in order here.  Please note,
first, that I haven't altered the standard punctuation procedures used
in any of the other available languages, so commas and full stops will
appear outside of quotation marks, and those quotation marks
themselves will be language-specific.  If, for whatever reason, you
wish to follow the Chicago specification and move punctuation inside
quotation marks, then you'll need a declaration of this sort in your
preamble:

\begin{quote}
  \cmd{DefineBibliographyExtras\{german\}\{\%}\\
  \hspace*{2em}\cmd{DeclareQuotePunctuation\{.,\}\}}
\end{quote}

Second, depending on the nature of your bibliography database, it will
only rarely be possible to process the same bib file in different
languages and obtain completely satisfactory results.  Fields like
\textsf{note} and \textsf{addendum} will often contain
language-specific information that won't be translated when you switch
languages, so manual intervention will be necessary.  If you suspect
you may have a need to use the same bib file in different languages,
you can minimize the amount of manual intervention required by using
the bibstrings defined either by \textsf{biblatex} or by
\textsf{biblatex-chicago}.  Here, a quick read through
\textsf{notes-test.bib} and/or \textsf{dates-test.bib} should give you
an idea of what is available for this purpose --- see esp.\ the
strings \texttt{by}, \texttt{nodate}, \texttt{newseries},
\texttt{number}, \texttt{numbers}, \texttt{oldseries},
\texttt{pseudonym}, \texttt{reviewof}, \texttt{revisededition}, and
\texttt{volume}, and also section \ref{sec:formatcommands} above,
esp.\ s.v.\ \enquote{\cmd{partedit}.}

\mylittlespace Finally, the French and German bibstrings I have
provided may well break with established bibliographical traditions in
those languages, but my main concern when choosing them was to remain
as close as possible to the quirks of the Chicago specification.  I
have entirely relied on the judgment of the creators of the Finnish,
Icelandic and Norwegian localizations in those instances.  If you have
strong objections to any of the strings, or indeed to any of my
formatting decisions, please let me know.

\section{One .bib Database, Two Chicago Styles}
\label{sec:twostyles}

I have, when designing this package, attempted to keep at least half
an eye on the possibility that users might want to re-use a .bib
database in documents using the two different Chicago styles.  The
extensive unification of the two styles in the 16th edition of the
\emph{Manual} has simplified things, and though I have no idea whether
this will even be a common concern, I still thought I might gather in
this section the issues that a hypothetical user might face.  The two
possible conversion vectors are by no means symmetrical, so I provide
two lists, items within the lists appearing in no particular order.
These may well be incomplete, so any additions are welcome.

\subsection{Notes -> Author-Date }
\label{sec:conv:notesauth}

This is, I believe, the simpler conversion, as most well-constructed
.bib entries for the notes \&\ bibliography style will nearly
\enquote{just work} in author-date, but here are a few caveats
nonetheless:

\begin{enumerate}
\item \textbf{NB:} Unless you are using \textsf{authordate-trad}, the
  formatting of titles in the two styles is now the same, which means
  you would no longer need to worry about extra curly brackets and
  their effects on capitalization.  If you are using
  \textsf{authordate-trad}, please see the caveats in the
  documentation of the \textsf{title} field in
  section~\ref{sec:fields:authdate}, above.
\item You may need to reevaluate your use of shorthands, given that by
  default the author-date styles use them in place of authors rather
  than in place of the whole citation.  The preamble option
  \textsf{cmslos=false} may help, but this may leave your document
  out-of-spec.
\item The potential problem with multiple author lists containing more
  than three names doesn't arise in the notes \&\ bibliography style,
  so the \textsf{shortauthor} fields in such entries may need
  alteration according to the instructions in
  section~\ref{sec:otherhints:auth} above.
\item Date presentation is relatively simple in notes \&\
  bibliography, so you'll need to contemplate the \texttt{cmsdate}
  options from sections~\ref{sec:authuseropts} and
  \ref{sec:authentryopts} when doing the conversion to author-date.
\end{enumerate}

\subsection{Author-date -> Notes}
\label{sec:conv:authnotes}

It is my impression that an author-date .bib database is somewhat
easier to construct in the first instance, but subsequently converting
it to notes \&\ bibliography is a little more onerous.  Here are some
of the things you may need to address:

\begin{enumerate}
\item If you've decided against using the \cmd{partedit} macro and
  friends from section~\ref{sec:formatting:authdate} above, commands
  not strictly necessary for author-date, you'll need to insert them
  now.
\item In general, you need to be more careful in notes \&\ bibliography
  about capitalization issues.  Fields which only appear once in
  author-date --- in the list of references --- may appear in both
  long notes and in the bibliography, in different syntactic contexts,
  so a quick perusal of the documentation of the \cmd{autocap} macro
  in section~\ref{sec:formatting:authdate} above may help.
\item You also need to be more careful about the use of abbreviations,
  e.g., in journal names, where the author-date style is more liberal
  in their use than the notes \&\ bibliography style.  (Cf.\ 14.179.)
  The bibstrings mechanism and package options sort much of this out
  automatically, but not all.
\item The \textsf{shorttitle} field is used extensively in notes \&\
  bibliography to keep short notes short, so you may find that you
  need to add a fair number of these to an author-date database.  In
  general this field is ignored by the latter style, so this, too,
  will be a one-time conversion.
\item You may need to add \textsf{letter} entries if you are citing
  just one letter from a published collection.  See
  section~\ref{sec:entrytypes}, s.v. \enquote{letter,} above.
\item The default shorthand presentation differs from one style to the
  other.  You may need to reconsider how you use this field when
  making the conversion.
\item As I explained above in section~\ref{sec:entryfields}, s.v.\
  \enquote{date,} I have included compatibility code in
  \textsf{biblatex-chicago-notes} for the \texttt{cmsdate} (silently
  ignored) and \texttt{switchdates} options, along with the automatic
  mechanism for reversing \textsf{date} and \textsf{origdate}.  This
  means that you can, in theory, leave all of this alone in your .bib
  file when making the conversion, though I'm retaining the right to
  revise this if the code in question demonstrably interferes with the
  functioning of the notes \&\ bibliography style.
\end{enumerate}

\section{Interaction with Other Packages}
\label{sec:otherpacks}

For \mymarginpar{\textbf{endnotes}} users of the \textsf{endnotes}
package --- or of \textsf{pagenote} --- \textsf{biblatex} 0.9 offers
considerably enhanced functionality.  Please read Lehman's RELEASE
file and the documentation of the \texttt{notetype} option in
\textsf{biblatex.pdf} �~3.1.2.1.

\mylittlespace Another \mymarginpar{\textbf{memoir}} problem I have
found occurs because the \textsf{memoir} class provides its own
commands for the formatting of foot- and end-note marks.  By default,
\textsf{biblatex-chicago} uses superscript numbers in the text, and
in-line numbers in foot- or end-notes, but I have turned this off when
the \textsf{memoir} class is loaded, reasoning that users of that
package may well have their own ideas about such formatting.

\mylittlespace The \mymarginpar{\textbf{ragged2e}} footnote mark code
I've just mentioned also causes problems for the \textsf{rag\-ged2e}
package, but in this case a simple workaround is to load
\textsf{biblatex} \emph{after} you've loaded \textsf{ragged2e} in your
document preamble.

\enlargethispage{\baselineskip}

\mylittlespace Nick \mymarginpar{\textbf{Xe\LaTeX}} Andrewes alerted
me to problems that appeared when he used the Xe\LaTeX\ engine to
process his files.  These included spurious punctuation after
quotation marks in some situations, and also failures in the automatic
capitalization routines.  Some of these problems disappeared when I
switched to using Lehman's punctuation-tracking code for
\enquote{American} styles, but some remained.  A bug report from
J. P. E.~Harper-Scott suggested a new way of addressing the issue, and
newer versions of Lehman's \textsf{csquotes} package incorporate a
full fix.  This, thankfully, doesn't require turning off any of
Xe\LaTeX 's features, and indeed merely involves upgrading to the
latest version of \textsf{csquotes}, which I recommend doing in any
case.  Compatibility with the EU1 encoding is now standard in that
package.

\section{TODO \&\ Known Bugs}
\label{sec:bugs}

This release implements the 16th edition of the \emph{Chicago Manual
  of Style}.  It now also contains a version of the author-date style
(\textsf{authordate-trad}) with traditional title formatting,
alongside the \textsf{authordate} code which unifies the treatment of
titles between itself and the notes \&\ bibliography style.  I hope
that users will migrate to one of these styles implementing the most
recent specification, as I am focusing my development and testing time
there.  With the current release, I am calling the 15th-edition styles
\enquote{strongly deprecated,} but if you still have urgent feature
requests for them, I'll do what I can.

\mylittlespace Regardless of which edition you are considering, there
are a number of things I haven't implemented.  The solution in
brown:bre\-mer to multi-part journal articles obviously isn't optimal,
and I should investigate a way of making it simpler.  If the kludge
presented there doesn't appeal, you can always, for the time being,
refer separately to the various parts.  Legal citations are another
thorny issue, and implementing them would involve choosing a
particular documentation scheme (for which there exist at least three
widely-used standards in the US), then providing what would be, it has
seemed to me, an entirely separate \textsf{biblatex} style, bearing
little or no relation to the usual look of Chicago citations.  Indeed,
the \emph{Manual} (14.281) even makes it clear that you should be
using a different reference book if you are presenting work in the
field, so I've thought it prudent to stay clear of those waters so
far.  I have received a request for this feature, however, so when I
have finished the updates for the 16th edition I shall look at it more
closely.  If you have other issues with particular sorts of citation,
I'm of course happy to take them on board.  The \emph{Manual} covers
an enormous range of materials, but if we exclude the legal citations
it seems to me that the available entry types could be pressed into
service to address the vast majority of them.  If this optimism proves
misguided, please let me know.

\mylittlespace I haven't yet explored the possible uses in
\textsf{biblatex-chicago} for the new(-ish) \textsf{biblatex} fields
\textsf{related}, \textsf{relatedtype}, and \textsf{relatedstring}.
It's possible they will solve some issues more simply and elegantly
than my own kludges do, so I hope to address this for the next
release.  The same holds for the \textsf{datelabelsource} field, new
in \textsf{biblatex} 2.8.

\mylittlespace Kenneth L. Pearce has reported a bug that appears when
using multiple citation commands inside the \textsf{annotation} field
of annotated bibliographies.  As late as I am with the file for the
16th edition, I shall attempt to address this in a future release.  If
you run into this problem, he suggests placing all the citations
together in parentheses at the end of the annotation, though on my
machine this doesn't always work too well, either.

\mylittlespace Version 1.5 of \textsf{biblatex} revised the way the
package deals with breaking long URLs and DOIs across lines.  The new
code is designed to deal as elegantly as possible with as wide a
variety of cases as possible, but in a few of my test entries it has
caused some line-breaking issues of its own.  Depending on the nature
of your cited sources, it may be useful for you to revert to the
older, pre-1.5 \textsf{biblatex} behavior, something which is easily
done by copying and pasting the old definition of the
\cmd{biburlsetup} command into your document preamble.  If you look in
the preambles of \textsf{cms-notes-sample.tex} or
\textsf{cms-dates-sample.tex}, you can see the redefinition and copy
it from there, just to see whether it helps your situation.  If it
would be generally useful, I could also easily turn it into a package
option.  Feedback welcome.

\mylittlespace The switch to \textsf{Biber} for the author-date
specification means that \textsf{biblatex} now provides considerably
enhanced handling of the various date fields.  I have attempted to
document the relevant changes in \textsf{cms-dates-sample.pdf} and in
the \textbf{date} discussion in section~\ref{sec:fields:authdate},
above, but it's possible the package may need some changes to cope
with all the permutations.  Please let me know if you find something
that looks like a bug.

\mylittlespace Speaking of \textsf{dates}, now that there's an option
for the automatic compression of page ranges, something similar could
perhaps be provided for year ranges.  I shall be looking into this.

\mylittlespace Roger Hart, Pierric Sans, and a number of other users
have reported a bug in the formatting of title fields.  This, as far
as I can tell, has to do with the interaction between
\cmd{MakeSentenceCase} and certain characters at the start of the
title, particularly Unicode ones.  If you are using
\textsf{authordate-trad}, it may help for the moment to put an empty
set of curly braces \{\}\ at the start of the field, but I shall look
into this further.

\mylittlespace Roger Hart has requested that I incorporate some means
of changing the punctuation before \textsf{titleaddon} fields, perhaps
using a customizable command like \cmd{titleaddonpunct}.  I hope to
provide this in the next release.  He has also requested, despite the
\emph{Manual's} objections, the possibility of using both
\texttt{Idem} and \texttt{Ibid.}\ in notes.  I shall look into this
for the next major release.

\enlargethispage{-4\baselineskip}

\mylittlespace This release fixes the formatting errors of which I am
aware, though users writing in French should be aware of problems with
the \cmd{partedit} command in section~\ref{sec:formatcommands} above.
There also remain the larger issues I've discussed throughout this
documentation, which mainly represent my inability to make all of
\textsf{biblatex-chicago's} formatting functions transparent for the
user, but thankfully Lehman's superb punctuation-tracking code has
preemptively fixed a great many small errors, some of which I hadn't
even noticed before I began testing that functionality.  That there
are other micro-bugs seems certain --- if you report them I'll do my
best to fix them.

\mylittlespace I haven't looked closely at the standard
\textsc{Bib}\TeX\ style by Glenn Paulley, contained in
the \textsf{chicago} package on CTAN, which implements the
author-date specification from the 13th edition of the \emph{Manual}.
If anyone is still using the style, and requires some compatibility
code for it, let me know, and I'll look into it.

\section{Revision History}
\label{sec:history}

\textbf{0.9.9g: Released \today}
\begin{itemize}
\item Alexandre Roberts found a showstopper in the functionality
  related to the new \mycolor{\texttt{inheritshorthand}} option in the
  notes \&\ bibliography style, and I found an unpleasant bug in the
  formatting of abbreviated cross-references in the same style.  This
  release, I hope, fixes both, but is in all other respects identical
  to 0.9.9f.
\end{itemize}

\textbf{0.9.9f: Released August 15, 2014}
\begin{itemize}
\item I've \label{deprec:obsol} made the alterations needed to bring
  the styles into line with the latest version of \textsf{biblatex}
  (2.9a).  This is the version that has been tested most thoroughly
  with \textsf{biblatex-chicago}, so I strongly recommend using it.
\item I fixed several inaccuracies in the presentation of abbreviated
  cross-referen\-ces in all the Chicago styles, and while I was working
  on that portion of the code it seemed an opportune moment to fulfill
  some feature requests bearing on the same area of functionality.
\item First, following a request from Alexandre Roberts, I have added
  the \mycolor{\texttt{inher\-itshorthand}} option to the notes \&\
  bibliography style, which allows child entries to inherit the
  \textsf{shorthand} field from their parents.  This in turn allows
  the \textsf{shorthand} itself to appear in place of the usual
  abbreviated citation of parent entries cross-referenced by several
  different child entries, thereby saving some space.  (This behavior
  was already available in the author-date styles, so the option is
  unnecessary there.)  You'll need to use \texttt{skipbiblist} in the
  \textsf{options} field of child entries to make the list of
  shorthands work correctly.  Please see the documentation of the
  \textsf{shorthand} field for the full explanation.
\item Second, following a request from Kenneth Pearce, I have added to
  all Chicago styles the capacity to combine abbreviated
  cross-references with the presentation of the original text of
  translations (via the \textsf{userf} field) or of the original
  publication details of an essay or chapter you are citing from a
  subsequent reprint (via the \textsf{reprinttitle} field).  See the
  documentation of those fields, and also of \textsf{crossref}, and
  note that you can now, taking certain precautions as outlined in the
  \textsf{shorthand} docs, combine the \textsf{userf},
  \textsf{crossref}, and \textsf{shorthand} fields.  This mechanism
  contains a great many moving parts, so please report any problems
  you might have with it.
\item Third, and finally, following a bug report by Mark van Atten I
  have fixed all Chicago styles so that the \textsf{biblatex}
  \texttt{backref} mechanism works properly in
  \textsf{biblatex-chicago}, including in those entries that use
  abbreviated cross-references, and in those that use the
  \textsf{userf} or \textsf{reprinttitle} fields.  I can't see any
  instructions concerning this in the \emph{Manual}, so I've left the
  formatting of \texttt{backref} lists in the hands of
  \textsf{biblatex} itself.  If the default behavior doesn't match
  your needs, let me know, as it's possible I could add some further
  options for modifying it.
\item I have added a new \mycolor{\texttt{compresspages}} option to
  all the Chicago styles.  If set to \texttt{true} it automatically
  compresses page ranges in the \textsf{pages} and \textsf{postnote}
  fields, allowing you to type ranges naturally, e.g., 101-{-}109, and
  letting the package follow the \emph{Manual's} rules for you.  (In
  this case, it would yield 101--9 in the document.)  Thanks are due
  to David Gohlke who brought to my attention a discussion that took
  place a couple of years ago on
  \href{http://tex.stackexchange.com/questions/44492/biblatex-chicago-style-page-ranges}{Stackexchange}
  regarding the automatic compression of page ranges.
  \textsf{Biblatex} has long had the facilities for providing this,
  and though the \emph{Manual's} rules (9.60) are fairly complicated,
  Audrey Boruvka fortunately provided in that discussion code that
  implements the specifications.  As some users may well be accustomed
  to compressing page ranges themselves in their .bib files, and in
  their \textsf{postnote} fields, I have made the activation of this
  code a package option.
\item Several users, most recently David Gohlke, have requested a way
  to alter the punctuation that appears just before the
  \textsf{postnote} argument of citation commands.  This allows, in
  the notes \&\ bibliography style, citations to fit better into the
  flow of text, while in the \textsf{authordate} styles it allows you
  very easily to insert comments, which follow a semi-colon, inside
  parenthetical text citations.  This punctuation is a complex issue
  in the \emph{Manual}, but as a first stab at enabling this greater
  flexibility, I have introduced the \mycolor{\texttt{postnotepunct}}
  package option.  Set to \texttt{true}, it allows you to start the
  \textsf{postnote} field with a punctuation mark (.\,,\,;\,:) and
  have it appear as the \cmd{postnotedelim} in place of whatever the
  package might otherwise automatically have chosen.  Please note that
  this functionality relies on a very nifty macro by Philipp Lehman
  which I haven't extensively tested, so I'm labeling this option
  \mycolor{experimental}.  Note also that the option only affects the
  \textsf{postnote} field of citation commands, not the \textsf{pages}
  field in your .bib file.  Note, finally, that if you are using the
  new \mycolor{\texttt{compresspages}} option then any
  \textsf{postnote} field starting with a punctuation mark will
  require you to do the compression of page ranges yourself.
\item I've added a new inheritance declaration so that
  \textbf{incollection} entries can inherit from \textbf{book} entries
  the same way they inherit from \textbf{mvbook}.
\item I've fixed a fair number of other bugs, including two in the
  \emph{Ibidem} mechanism pointed out by Bernd Rellermeyer, one in the
  printing of dates, and one in the \cmd{textcite} command in the
  notes \&\ bibliography style, these last two pointed out by Kenneth
  Beesley.  The presentation of all the periodical entry types
  (without an \textsf{entrysubtype}) has also been made more accurate.
\end{itemize}

\textbf{0.9.9e: Released January 29, 2014}
\begin{itemize}
\item This minor release fixes a regression in the \emph{Ibidem}
  mechanism in the notes \&\ bibliography style, spotted by Harold
  Bellemare, and present in the package since version 0.9.9c.  In all
  other respects this release is identical to 0.9.9d.
\end{itemize}

\textbf{0.9.9d: Released October 30, 2013}
\begin{itemize}
\item Following requests by Kenneth~L.\ Pearce and Bertold Schweitzer,
  I have modified and extended the mechanism for creating abbreviated
  citations when several parts of the same collection are included in
  a reference apparatus.  To the \textbf{incollection},
  \textbf{inproceedings}, and \textbf{letter} entries of previous
  releases, I have added \textbf{inbook}, \textbf{book},
  \textbf{bookinbook}, \textbf{collection}, and \textbf{proceedings}
  entries.  Only \textbf{inbook} entries join the former three in
  having this functionality turned on by default --- if you don't want
  this, it will require intervention either in the preamble or in the
  \texttt{options} field of individual entries.  This intervention
  will be via the new \mycolor{\texttt{longcrossref}} option, which
  controls the behavior of the four essay-like entry types and
  defaults to \texttt{false}, while the new
  \mycolor{\texttt{booklongxref}} option controls the four book-like
  types and defaults to \texttt{true}.  The useful settings for the
  options differ slightly between the author-date and the notes \&\
  bibliography specifications, so please see all the details in the
  docs of the \textsf{crossref} field in
  sections~\ref{sec:entryfields} and \ref{sec:fields:authdate}, above.
\item On the same subject, in the notes \&\ bibliography style, I
  should mention that in the first, full citation of one part of a
  collection in a note, the code no longer uses a separate citation of
  the parent entry to supply parts of what you see printed.  (This led
  to numerous inaccuracies.)  If your setup uses a side-effect of the
  old code to print data that hasn't even been inherited by the child,
  you may find that you need to change some \textsf{xref} fields to
  \textsf{crossref} fields to make it work correctly now.  This
  situation will, I imagine, be very rare, but you can look at
  white:ross:memo in \textsf{notes-test.bib} to see an example.
\item In the author-date styles, several users have been frustrated by
  the lack of an approved way of setting the \texttt{cmsdate} option
  in the preambles of their documents, and Kenneth~L.\ Pearce
  requested that I attempt to ease the burden on users by looking at
  this again.  With this release, you can now set \texttt{cmsdate}
  either to \texttt{both} or \texttt{on} in the preamble, and it will
  affect all entries (except \textbf{music}, \textbf{review}, and
  \textbf{video}) with multiple dates.  You can still change this
  setting in the \texttt{options} field of individual entries, but
  what you won't be able to change there is the new call to
  \cmd{DeclareLabeldate} which puts the \textsf{origdate} first in the
  list of dates when \textsf{Biber} searches for a \textsf{labelyear}
  to use in citations and in the list of references.  If you have been
  using the \texttt{switchdates} mechanism to get the
  \textsf{origdate} as the \textsf{labeldate}, your .bib files may
  need some editing in order to use the new preamble options.  Please
  see the documentation of the \textsf{date} field in
  section~\ref{sec:fields:authdate} above for all the (voluminous)
  details.
\item Following a request by Rasmus Pank Rouland, I adapted new
  \textsf{biblatex} code in the \cmd{textcite(s)} commands in all
  styles to make them fit more elegantly in the flow of text.  Upon
  reconsideration of the commands in the notes \&\ bibliography style,
  I slightly modified them, but \emph{only} when used inside a foot-
  or endnote.  In this context, by default, for both \cmd{textcite}
  and \cmd{textcites}, you'll now get the \textsf{author's} name(s)
  followed by a headless \emph{short} citation (or citations) placed
  within parentheses.  You can use \cmd{renewcommand} in the preamble
  of your document to redefine the new \mycolor{\cmd{foottextcite}}
  and \mycolor{\cmd{foottextcites}} commands to change this
  formatting.  See section~\ref{sec:formatcommands}, above.
%\enlargethispage{\baselineskip}
\item This release includes support, in all styles, for
  \textsf{biblatex's} multi-volume entry types: \textbf{mvbook},
  \textbf{mvcollection}, \textbf{mvproceedings}, and
  \textbf{mvreference}.  See sections~\ref{sec:entrytypes} and
  \ref{sec:types:authdate}.
\item If you use \textsf{Biber}, I have added several new inheritance
  schemes to all styles to make cross-referenced entries work more
  smoothly: \textbf{incollection} entries can now inherit from
  \textbf{mvbook} just as they do from \textbf{mvcollection} entries;
  \textbf{letter} entries now inherit from \textbf{book},
  \textbf{collection}, \textbf{mvbook}, and \textbf{mvcollection}
  entries the same way an \textbf{inbook} or an \textbf{incollection}
  entry would; the \textsf{namea}, \textsf{nameb}, \textsf{sortname},
  \textsf{sorttitle}, and \textsf{sortyear} fields, all highly
  single-entry specific, are no longer inheritable; and the
  \textsf{date} and \textsf{origdate} fields of any \textbf{mv*} entry
  will \emph{not} be inherited by any other entry type.
\item Following a bug report by Henry~D.\ Hollithron, I've added to
  \textbf{unpublished} entries in all styles the possibility of
  including an \textsf{editor}, \textsf{translator}, etc.
\item Thanks to bug reports from Denis Maier and Bertold Schweitzer, I
  corrected inaccuracies and outright bugs in many entry types in all
  Chicago styles that appeared when there was a \textsf{booktitle} and
  not a \textsf{maintitle} or vice versa.  This also involved another
  rewrite of the code handling the \textsf{volume} field and other
  related fields in all non-periodical entry types that use them.
\item On the subject of the \textsf{volume} field, I added a new
  preamble and entry option, \mycolor{\texttt{delayvolume}}, to the
  notes \&\ bibliography style.  In long notes where this data isn't
  printed before a \textsf{maintitle}, this option allows you to print
  it \emph{after} the publication information rather than
  \emph{before} it, as may sometimes help clarify things, according to
  the \emph{Manual}.  This applies to the non-periodical entry types
  only.  See section~\ref{sec:useropts}.
\item On the same subject, in all styles, I have added a new preamble
  and entry option, \mycolor{\texttt{hidevolumes}}.  This controls
  whether, in entries where a \textsf{volume} has been printed before
  a \textsf{maintitle}, any \textsf{volumes} field present will also
  be printed, in this case \emph{after} the \textsf{maintitle}.  By
  default, this is set to \texttt{true}, so that the \textsf{volumes}
  field won't appear in such circumstances.  See
  sections~\ref{sec:chicpreset} and \ref{sec:authpreset}.
\item On the same subject, I have modified, in all styles, the field
  format for the \textbf{part} field, so that if the field contains
  something other than a number, \textsf{biblatex-chicago} will print
  it as is, capitalizing it if necessary, rather than supplying the
  usual bibstring, thus providing a mechanism for altering the string
  to your liking.  I have also decoupled the \textsf{part} field from
  the \textsf{volume} field, allowing it to be printed even in the
  absence of the latter, thereby providing a means to refer to
  segments of a larger work that don't easily fit the established
  schemes.  The iso:electrodoc entry in \textsf{dates-test.bib} shows
  an example of how this might work.
\item There is a new \mycolor{\texttt{omitxrefdate}} preamble and
  entry option in the notes \&\ bibliography style.  It turns off the
  printing of the child's \textsf{date} next to its \textsf{title} in
  abbreviated book-like entries \emph{only}, in both notes and
  bibliography.  See section~\ref{sec:useropts}.
\item Clea~F.\ Rees requested a way to customize the punctuation when
  a \textsf{volume} and a \textsf{page} number appear together like
  so: \enquote{2:204.}  You can use \cmd{renewcommand} in your
  preamble to redefine the new \mycolor{\cmd{postvolpunct}} command to
  achieve this, in all styles.  If your document language is French,
  \textsf{cms-french.lbx} redefines this already and prints something
  like \enquote{2 : 204.}  See sections~\ref{sec:formatcommands} and
  \ref{sec:formatting:authdate}.
\item I extended, in all styles, the functions of the \textsf{userd}
  field, allowing it to modify a \textsf{date} field if it hasn't
  already been captured by another date specification in the entry.
  See the documentation of the field in sections~\ref{sec:entryfields}
  and \ref{sec:fields:authdate}.
\item A bug report from Mathias Legrand helped clear up inaccuracies
  in the presentation of ordinal numbers in all styles.
\item For the author-date styles, another bug report by Kenneth Pearce
  resulted in the addition of the \textsf{labelyear} to the default
  \texttt{cms} sorting scheme so that more entries in the reference
  list are sorted properly without further user intervention.
\item George Pigman found an odd punctuation-tracking bug in the
  author-date styles.  This has been fixed.
\item Marc Sommer found a bug in the presentation of the
  \textsf{prenote} field in the author-date styles.  This has been
  fixed.
\item In the notes \&\ bibliography style, I improved the behavior of
  abbreviated foot- and endnotes when using the \textsf{hyperref}
  package.
\item I modified the date-presentation code in all the language files
  (\textsf{cms-*.lbx}) provided by the package.  Now, if an entry
  contains a \textsf{(*)year} and an \textsf{(*)endyear} that are
  exactly the same, and there aren't any further month or day
  specifications, then the \textsf{year} alone will be printed.  This
  allows for the clearing of spurious \textsf{endyears} inherited from
  parent entries.
\item I discovered some unpleasant side effects of my arrangement of
  the \textsf{.lbx} files devoted to Norwegian, and reverted to the
  arrangement as originally provided by H�kon Malmedal.
\end{itemize}

\textbf{0.9.9c: Released March 15, 2013}

\begin{itemize}
\item Antti-Juhani Kaijahano has very kindly provided a new Finnish
  localization for \textsf{biblatex-chicago}, called
  \mycolor{\textsf{cms-finnish.lbx}}.  As you will see if you look
  through it, it is still something of a work in progress.  If you
  would like to fill some of its lacunae, please do let me know.
\item Following a report by Bertold Schweitzer, I have added the
  \textbf{namea} and \textbf{nameb} fields to \textbf{article} and
  \textbf{review} entries in all three Chicago styles.  As in all the
  book-like entry types, they allow you to associate an editor or a
  translator specifically with a \textsf{title}, rather than, in these
  cases, with an \textsf{issuetitle}.  See the docs on these entry
  types in sections~\ref{sec:entrytypes} and \ref{sec:types:authdate},
  above.
\item Thanks to another report by Bertold I have, in all three Chicago
  styles, corrected inaccuracies in the presentation of the
  \textbf{report} entry type.  The \textsf{number} now appears
  immediately after the \textsf{type}, and the \textsf{type} itself is
  now capitalized properly depending on its context in an entry.
\item A third report by Bertold, detailing inaccuracies in the
  treatment of the \textbf{volume} and \textbf{volumes} fields in
  certain contexts, has resulted in a complete rewrite of the
  presentation of these (and several related) fields in all
  non-periodical entry types in all three Chicago styles.  This won't
  require any changes to your .bib files, but the output you see may,
  in some reasonably unusual situations, change.  Please let me know
  if something doesn't look right to you.
\item A fourth report by Bertold revealed some inadequacies with
  multiple \textsf{date} presentation in the two Chicago author-date
  styles, issues that particularly involved cross-referenced entries.
  In addition to some general fixes in the code, I have also slightly
  changed the functioning of the \texttt{cmsdate=both} and
  \texttt{cmsdate=on} switches.  If, and only if, a work has only one
  date, and there is no \texttt{switchdates} in the \textsf{options}
  field, then \texttt{cmsdate=on} and \texttt{cmsdate=both} will both
  result in the suppression of the \textsf{extrayear} field in that
  entry.  See the \textbf{date} field docs in
  section~\ref{sec:fields:authdate}, above.
\item Following a report by Antti-Juhani Kaijahano, I have modified
  the presentation of author-less \textbf{article} and \textbf{review}
  entries in the reference list of both Chicago author-date styles.
  If such a source had a \texttt{magazine} \textsf{entrysubtype}, the
  styles would already use the \textsf{journaltitle} at the head of
  the entry in the list of references, but if there was no
  \textsf{entrysubtype} the entry would appear in the list
  \textsf{date} first.  Now, in keeping with the \emph{Manual}
  (14.175), the \textsf{title} will appear first, in both reference
  lists and in-text citations.  See especially under \textbf{article}
  in section~\ref{sec:types:authdate}, above.
\item Several users have pointed out annoying formatting errors in the
  styles.  Evan Cortens spotted two bugs in the notes \&\ bibliography
  style, one of which, under various circumstances, introduced extra
  spaces into long notes and the other of which affected the
  formatting of the \textsf{type} field in \textsf{thesis} entries.  I
  have fixed both, also applying the latter fix to several other entry
  types that use the \textsf{type} field.  Bertold Schweitzer pointed
  out a formatting bug with the \textsf{issuesubtitle} field in the
  author-date style, now fixed.  Mark Sprevak reported some spurious
  spaces appearing in headers and footers when using the
  \textsf{titleps} package; the culprits were errors in the
  \textsf{cms-*.lbx} files, now cleaned up.
\item I have rectified a number of other errors, in particular making
  the automatic provision of abbreviated cross-references more robust
  in \textsf{incollection}, \textsf{inproceedings}, and
  \textsf{letter} entries, improving the behavior of the
  \textsf{postnote} field in certain corner cases, fixing bugs in the
  handling of \textsf{pagination} and \textsf{bookpagination} fields,
  and slightly altering the placement of the \textsf{addendum} field
  in book-like entries to bring it closer to the \emph{Manual's}
  specification.  A number of other, smaller improvements should also
  bring the styles into closer conformity with the specification.
\end{itemize}

\textbf{0.9.9b: Released December 6, 2012}
\begin{itemize}
\item This release contains a new variant of the author-date style,
  available as the \mycolor{\texttt{authordate-trad}} option when
  loading \textsf{biblatex-chicago}.  This provides the traditional,
  plain, pre-16th-edition Chicago title handling --- sentence-style
  capitalization, absence of quotation marks in \textsf{article}
  titles and the like --- but in all other respects follows the
  16th-edition specification, as suggested by the \emph{Manual}
  (15.45).  Remember that the \texttt{headline} package option can be
  used to turn off the automatic sentence-style capitalization,
  meaning that titles will appear as presented in the .bib file, at
  least as far as capitalization is concerned.  Please see especially
  the documentation of \textsf{\textbf{title}} in
  section~\ref{sec:fields:authdate}, above, for the details.
\item I have updated calls to \cmd{DeclareLabelname} and
  \cmd{DeclareLabelyear} in several .cbx files so that the package
  works correctly with the most recent version (2.4) of
  \textsf{biblatex}.
\item Following a request by Norman Gray, I have included a
  \cmd{textcite} (and a \cmd{textcites}) command in the notes \&\
  bibliography style for the first time.  Please see
  section~\ref{sec:citecommands}, above, for the details.
\item Following a request by Daniel Possenriede, I have added in all
  three 16th-edition styles a new switch, \mycolor{\texttt{only}}, to
  the \texttt{doi} option, which prints the \textsf{doi} when present
  and the \textsf{url} only when there is no \textsf{doi}.  The
  package default remains, however, \texttt{true}.
\item I am grateful to Baldur Kristinsson for providing an Icelandic
  localization file for \textsf{biblatex-chicago}, called
  \mycolor{\textsf{cms-icelandic.lbx}}.  You'll see if you look
  through it that it is still something of a work in progress, but it
  should cover most needs in that language very well.  If you would
  like to fill in some of the gaps please let me know.
\item I am also grateful to H�kon Malmedal for providing Norwegian
  localizations for \textsf{biblatex-chicago}, contained in the files
  \mycolor{\textsf{cms-norsk.lbx}},
  \mycolor{\textsf{cms-norwe\-gian.lbx}}, and
  \mycolor{\textsf{cms-nynorsk.lbx}}.
\item I have added a new British localization
  (\mycolor{\textsf{cms-british.lbx}}) that should make it much
  simpler for users to produce documents adhering to that tradition.
  For further details on the usage of all these localizations please
  see section~\ref{sec:international}, above.
\item Several users have reported a bug that resulted in doubled
  bibstrings in certain contexts.  This happened only when using
  localizations for which \textsf{biblatex-chicago} didn't have
  explicit support, and it should now be fixed.
\item I have changed the way the 16th-edition author-date styles
  handle the \emph{Ibidem} mechanism.  In the absence of a
  \textsf{postnote} field you no longer get empty parentheses, but
  rather a standard in-text citation.  If you do have a
  \textsf{postnote} field, then only that will appear.
\end{itemize}

%%\enlargethispage{-2\baselineskip}

\textbf{0.9.9a: Released July 30, 2012}
\begin{itemize}
\item I have made a few changes to \textsf{biblatex-chicago.sty} to
  allow the package to work with the latest version (2.0) of
  \textsf{biblatex}.  In all other respects this release is identical
  to 0.9.9.  If you do use the package with \textsf{biblatex} 2.0,
  please let me know if there are issues I need to address.  Thanks to
  Charles Schaum for alerting me to some of them.
\end{itemize}

\textbf{0.9.9: Released July 5, 2012}

\mylittlespace Converting 15th-Edition .bib Files to Use the 16th
Edition:

\mylittlespace \textbf{Notes and Bibliography Style}

\begin{itemize}
\item The specification for \textbf{music} entries has been
  significantly altered for the new edition.  You no longer need to
  worry about the \texttt{\textcircledP} and \texttt{\textcopyright}
  signs in the \textsf{howpublished} field, which will be silently
  ignored, and the \textsf{pubstate} field now reverts to its usual
  function of identifying reprints or, in this case, reissues.  The
  spec really only requires a record label (\textsf{series}) and
  catalog number (\textsf{number}), though \textsf{publisher} is still
  available if you need it.  There is a new emphasis, finally, on the
  dating of musical recordings, so that the \textsf{eventdate} gives
  the recording date of a particular song or other portion of a
  recording, the \textsf{origdate} the recording date of an entire
  album, and the \textsf{date} the publishing date of that album.
  Please see the full documentation in section~\ref{sec:entrytypes},
  above.
\item The specification for \textbf{video} entries has also been
  clarified.  For television series, the episode and series numbers go
  in \textsf{booktitleaddon} instead of \textsf{titleaddon} and, as
  with \textsf{music} entries, the \textsf{eventdate} will hold the
  original broadcast date of such an episode, or perhaps the
  recording/performance date of, e.g., an opera on DVD.  The
  \textsf{origdate} will still hold the original release date of a
  film, and the \textsf{date} the publishing or copyright date of the
  medium you are referencing.  Please see the full documentation in
  section~\ref{sec:entrytypes}, above.
\item You should add \textbf{customc} entries to provide
  bibliographical cross-references from multiple pseudonyms back to
  the author's name.
\item In \textbf{suppbook} entries, the \emph{Manual} now requires you
  to provide the page range (in the \textsf{pages} field) for the
  specific part you are citing, e.g., an introduction, foreword, or
  afterword.%%\enlargethispage{-2\baselineskip}
\item In \textbf{patent} entries, the \emph{Manual} now prefers
  sentence-style capitalization for titles, which you'll need to
  provide yourself by hand.
\item When a descriptive phrase is used as an \textsf{author}, you can
  now omit an initial definite or indefinite article, which will help
  with alphabetization in the bibliography.
\item A DOI is now preferred to a URL, if both are available.
\item On the same subject, a revision date (or similar) is preferred
  to an access date for online material.  You can use the new
  \textbf{userd} field to change the string introducing the
  \textsf{urldate}, which defaults to being an access date.
\item Special imprints are now separated from their parent press by a
  forward slash rather than a comma, so can just be added to the
  \textsf{publisher} field with the usual keyword \texttt{and}.
\item I have implemented a reasonable, less-flexible facsimile of the
  \textsf{Biber}-only command \mycolor{\cmd{DeclareLabelname}} which
  should work for those using any backend.  It allows
  \textsf{biblatex} to find a name for short notes outside the
  standard name fields, including, notably, in the \textsf{name[a-c]}
  fields.  This should reduce the instances where you need a
  \textsf{shortauthor} field to provide such a name.
\item The Chicago-specific setting of another \textsf{Biber}-only
  command, \mycolor{\cmd{Declare\-SortingScheme=cms}}, allows
  non-standard fields to be considered by \textsf{biblatex's} sorting
  algorithms, which should reduce the instances where you need a
  \textsf{sortkey} or the like in your entries.  If you aren't using
  \textsf{Biber}, the package reverts to the standard \texttt{nty}
  sorting scheme.
\end{itemize}

\textbf{Author-Date Style}

\begin{itemize}
\item All title fields now follow the rules for the notes \&\
  bibliography style as far as punctuation, formatting, and
  capitalization are concerned.  \textsf{Biblatex-chicago-authordate}
  will deal with most of this automatically, but if you have any hand
  formatting of lowercase letters within curly braces in your .bib
  file, you will need to restore the headline-style capitalization
  there.  Also, you'll need to be more careful when you provide
  quotation marks inside titles, remembering to use \cmd{mkbibquote}
  so that punctuation can be brought inside nested quotation marks.
  These revisions will apply particularly to \textbf{title},
  \textbf{booktitle}, and \textbf{maintitle} fields.
\item The one exception to these rules is in \textbf{patent} entries,
  where sentence-style capitalization of the \textsf{title} is now
  specified.  You'll have to provide this by hand yourself, as in the
  notes \&\ bibliography style.
\item Because of these changes to title formatting, you'll need to
  observe the difference between \textbf{article} and \textbf{review}
  entries, where the latter contain generic, \enquote{Review of
    \ldots} titles and the former standard, specific titles.
\item The presentation of \textbf{shorthand} fields has changed.  You
  no longer need to use the \textbf{customc} entry type to include
  cross-references from shorthands to expansions in the list of
  references.  Now, simply using a \textsf{shorthand} field in an
  entry places that \textsf{shorthand} in citations and at the head of
  the entry in the list of references, where it will be followed by
  its expansion within parentheses.  The new system will require help
  with sorting in the reference list --- placing the
  \textsf{shorthand} also in a \textsf{sortkey} should do the trick.
\item On the subject of \textbf{customc} entries, the \emph{Manual}
  now recommends using cross-references in several contexts,
  particularly when a single author uses more than one pseudonym.
  Adding \textsf{customc} entries makes this happen.
\item There have been significant changes when presenting book-like
  entries with more than one date.  If you are using the
  \texttt{cmsdate=on} option, or indeed simply placing the earlier
  date in the \textbf{date} field and the later one in
  \textbf{origdate}, the presentation will be the same as before, but
  you should understand that the \emph{Manual} no longer recommends
  this \textsf{origdate}-only style.  It prefers, instead, to present
  either the \textsf{date} alone or both dates in citations and at the
  head of reference list entries.  When presenting both dates, there
  is now no longer a choice between the \texttt{old} and \texttt{new}
  options for \texttt{cmsdate}, but only the \texttt{both} option.  If
  you have \texttt{old} or \texttt{new} in your .bib files, they will
  be treated as synonyms of \texttt{both}.
\item The specification for \textbf{music} entries has been
  significantly altered for the new edition.  You no longer need to
  worry about the \texttt{\textcircledP} and \texttt{\textcopyright}
  signs in the \textsf{howpublished} field, which will be silently
  ignored, and the \textsf{pubstate} field reverts to its more usual
  function of identifying reprints or, in this case, reissues.  The
  spec really only requires a record label (\textsf{series}) and
  catalog number (\textsf{number}), though \textsf{publisher} is still
  available if you need it.  There is a new emphasis, finally, on the
  dating of musical recordings, which means that such entries will fit
  better with the author-date style.  It also means that I have had to
  redefine the various date fields.  The \textsf{eventdate} gives the
  recording date of a particular song or other portion of a recording,
  the \textsf{origdate} the recording date of an entire album, and the
  \textsf{date} the publishing date of that album.  The earlier date
  is the one that will appear in citations and at the head of
  reference list entries.  Please see the full documentation in
  section~\ref{sec:types:authdate}, above.
\item The specification for \textbf{video} entries has also been
  clarified.  For television series, the episode and series numbers go
  in \textsf{booktitleaddon} instead of \textsf{titleaddon} and, as
  with \textsf{music} entries, the \textsf{eventdate} will hold the
  original broadcast date of such an episode, or perhaps the
  recording/performance date of, e.g., an opera on DVD.  The
  \textsf{origdate} will still hold the original release date of a
  film, and the \textsf{date} the publishing or copyright date of the
  medium you are referencing.  The earlier date, once again, is the
  one that will appear in citations and at the head of reference list
  entries.  Please see the full documentation in
  section~\ref{sec:types:authdate}, above.
\item In \textbf{suppbook} entries, the \emph{Manual} now requires you
  to provide the page range (in the \textsf{pages} field) for the
  specific part you are citing, e.g., an introduction, foreword, or
  afterword.
\item The author-date style now prefers longer bibstrings in the list
  of references, bringing it into line with the notes \&\ bibliography
  style.  Generally, the package will take care of this for you, but
  if you've been using abbreviated strings in \textsf{note} fields,
  for example, you may want to change them so that they conform with
  the strings the package provides.  In some circumstances the
  \cmd{partedit} macro, and its relatives, may help.  See
  section~\ref{sec:formatting:authdate}.
\item When a descriptive phrase is used as an \textsf{author}, you can
  now omit an initial definite or indefinite article, which will help
  with alphabetization in the bibliography.
\item A DOI is now preferred to a URL, if both are available.
\item On the same subject, a revision date (or similar) is preferred
  to an access date for online material.  You can use the new
  \textbf{userd} field to change the string introducing the
  \textsf{urldate}, which defaults to being an access date.
\item Special imprints are now separated from their parent press by a
  forward slash rather than a comma, so can just be added to the
  \textsf{publisher} field with the usual keyword \texttt{and}.
\item The 16th edition of the \emph{Manual} is less than enthusiastic
  about the use of \enquote{Anon.}\ as the \textsf{author}, preferring
  instead that the \textsf{title} or the \textsf{journaltitle} take
  its place.  If you do decide to get rid of \enquote{Anon.,} new
  facilities provided by \textsf{Biber} --- see next entry --- should
  mean that \textsf{biblatex} no longer requires assistance when
  alphabetizing such author-less entries.
\item The Chicago-specific setting of the \textsf{Biber}-only command,
  \mycolor{\cmd{DeclareSort\-ing\-Scheme=cms}}, allows non-standard
  fields to be considered by \textsf{biblatex's} sorting algorithms,
  which should reduce the instances where you need a \textsf{sortkey}
  or the like in your entries.
\item The Chicago-specific setting of the \textsf{Biber}-only command
  \mycolor{\cmd{DeclareLabel\-name}} allows \textsf{biblatex} to find
  a name (\enquote{\textsf{label}}) for citations outside the standard
  name fields, including, notably, in the \textsf{name[a-c]} fields.
  This should reduce the instances where you need a
  \textsf{shortauthor} field to provide such a name.
\end{itemize}

Other New Features:

\begin{itemize}
\item For reprinted books, you can now present more detailed
  publishing information about the original edition using the new
  \mycolor{\textbf{origlocation}} and \mycolor{\textbf{origpublisher}}
  fields.  You can also use the \textsf{origlocation} in
  \textsf{letter} or \textsf{misc} (with \textsf{entrysubtype})
  entries to identify where a published or unpublished letter was
  written.  These uses apply to both Chicago styles.
\item Thanks to a patch sent by Kazuo Teramoto, you can now take
  advantage of \textsf{biblatex's} facilities for citing
  \mycolor{\textbf{eprint}} resources.  There is also a new
  \mycolor{\texttt{eprint}} option, set to \texttt{true} by default,
  which controls the printing of this field in both Chicago styles.
  You can set the option both in the preamble and in the
  \textsf{options} field of individual entries.  The field will always
  print in \textbf{online} entries.
\item I have added a new citation command,
  \mycolor{\cmd{citejournal}}, to the notes \&\ bibliography style to
  allow you to present journal articles using an alternative short
  note form, which may be a clearer form of reference in certain
  circumstances.  Such short notes will present the name of the
  \textsf{author}, the \textsf{journaltitle}, and the \textsf{volume}
  number.
\item I have included a very slightly modified version of the standard
  \textsf{biblatex} \cmd{citeauthor} command, which may be useful for
  references to works from classical antiquity.
\item I have added a new \texttt{cmsdate=\mycolor{full}} switch to the
  author-date style, which only affects citations in the text, and
  means that a full date specification will appear there, rather than
  just the year.  If you follow the \emph{Manual's} recommendations
  concerning newspaper and magazine articles only appearing in running
  text and not in the reference list, this option will help.
\item I have added a new \mycolor{\texttt{avdate}} option to the
  author-date style, set to \texttt{true} by default in
  \textsf{biblatex-chicago.sty}.  This alters the default setting of
  \cmd{Declare\-Labelyear} in \textbf{music}, \textbf{review}, and
  \textbf{video} entries to take account of specialized instructions
  in the \emph{Manual} for finding dates to appear in citations and at
  the head of reference list entries.  Setting \texttt{avdate=false}
  in the options when you load \textsf{biblatex-chicago} restores the
  default settings for all entry types.  See \texttt{avdate} in
  section~\ref{sec:authpreset}.
\item The \emph{Manual} has added recommendations for citing blogs,
  which generally will need an \textbf{article} entry with
  \texttt{magazine} \textsf{entrysubtype}.  You can identify a blog as
  such by placing \enquote{blog} in the \textsf{location} field.  If
  you want to cite a comment to a blog or to other online material,
  the \textbf{review} entry type, \textsf{entrysubtype}
  \texttt{magazine} will serve.  The \textbf{eventdate} dates the
  comment, and any timestamp that is required can go in
  \textsf{nameaddon}.  These instructions work in both specifications.
\item Photographs are no longer presented differently from other sorts
  of artworks so, in effect, in both styles, the \textbf{image} type
  is now a clone of \textbf{artwork}, though retained for backward
  compatibility.
\item Following a request by Kenneth Pearce, I have added new
  facilities for presenting \textbf{shorthands} in both Chicago
  styles.  In both, there are two new \texttt{bibenvironments} which
  you can set using the \texttt{env} option to the
  \cmd{printshorthands} command: \mycolor{\texttt{losnotes}} formats
  the list of shorthands so that it can be presented in a footnote,
  while \mycolor{\texttt{losendnotes}} does the same for endnotes.  In
  both styles, there is a new preamble option,
  \mycolor{\texttt{shorthandfull}}, which prints the full
  bibliographical information of each entry inside the list of
  shorthands, allowing such a list effectively to replace a
  bibliography or list of references.  In the author-date style, you
  need to set the \texttt{cmslos=false} option as well, in order for
  this to work.  In the notes \&\ bibliography style, I have added a
  new citation command, \mycolor{\cmd{shorthandcite}}, which prints
  the \textsf{shorthand} even in the first citation of a given work.
\item Following suggestions by Roger Hart, I have implemented three
  new field-exclusion options in the notes \&\ bibliography style.  In
  all three cases, the field in question will always appear in the
  bibliography, but not in long notes, which may help to save space.
  The fields at stake are \textsf{addendum}, \textsf{note}, and
  \textsf{series}, controlled respectively by the new
  \mycolor{\texttt{addendum}}, \mycolor{\texttt{notefield}}, and
  \mycolor{\texttt{bookseries}} options.  All of these are set to
  \texttt{true} using the new \mycolor{\texttt{completenotes}} option
  in \textsf{chicago-notes.cbx}, but you can change the settings
  either in the preamble or in the \textsf{options} field of
  individual entries.  Please see the documentation of these options
  in section~\ref{sec:chicpreset}, above, for details on which entry
  types are excluded from their scope.
\item Thanks to a coding suggestion from Gildas Hamel, I have
  redefined the \cmd{bibnamedash} in \textsf{biblatex-chicago.sty},
  which should now by default look a little better in a wider variety
  of fonts.
\item At the request of Baldur Kristinsson, I have added
  \cmd{DeclareLanguageMap\-ping} commands to
  \textsf{biblatex-chicago.sty} for all the languages
  \textsf{biblatex-chicago} currently provides.  If you load the style
  in the standard way, you no longer need to provide these mappings
  manually yourself.
\item I have improved the date handling in both styles, particularly
  with regard to date ranges.
\end{itemize}

\textbf{0.9.8d: Released November 15, 2011}
\begin{itemize}
\item Some minor fixes to both styles for compatibility with
  \textsf{biblatex} 1.7.
\item Kenneth Pearce found an error in the formatting of
  \textsf{bookinbook} titles in the author-date style's list of
  shorthands.  This should work properly now.
\item Jonathan Robinson spotted some inconsistencies in the way the
  notes \&\ bibliography style interacts with the \textsf{hyperref}
  package.  Following his suggestion, short notes now point to long
  notes when the latter are available, but to bibliography entries
  instead when you have set the \texttt{short} option.
\end{itemize}

\textbf{0.9.8c: Released October 12, 2011}
\begin{itemize}
\item Emil Salim pointed out some rather basic errors in the
  presentation of \textsf{inproceedings} and \textsf{proceedings}
  entries, errors that have been present from the first release of the
  style(s).  These should now, belatedly, have been put right.
\item Minor improvements to coding and documentation.
\end{itemize}

\textbf{0.9.8b: Released September 29, 2011}
\begin{itemize}
\item Bad Dates: Christian Boesch alerted me to some date-formatting
  errors produced when using the styles with the \texttt{german}
  option to \textsf{babel}.  A little further investigation revealed
  similar problems with \texttt{french}, and before long it became
  clear that date handling in \textsf{biblatex-chicago} was generally,
  and significantly, sub-optimal.  The whole system should now be more
  robust and more accurate.
\item The new date-handling code shouldn't require any changes to your
  .bib files, but users of the author-date style may want to have a
  look at the documentation of the \textsf{letter} and \textsf{misc}
  entry types, and of the four date fields, for some information about
  how the changes could simplify the creation of their databases.
\item Various other minor improvements.
\end{itemize}

\textbf{0.9.8a: Released September 21, 2011}
\begin{itemize}
\item Fixed a series of unsightly errors in the author-date style,
  discovered while working on the pending update to the 16th edition.
\item Fixed bugs uncovered in both the author-date and the notes \&\
  bibliography styles thanks to Charles Schaum's adventurous use of
  the \textsf{origyear} field.
\item Added two new bibstrings to the cms-*.lbx files to fix potential
  bugs in some of the audiovisual entry types.
\end{itemize}

\textbf{0.9.8: Released August 31, 2011}

\begin{itemize}
\item Starting with \textsf{biblatex} version 1.5, in order to adhere
  to the author-date specification you will need to use \textsf{Biber}
  to process your .bib files, as \textsc{Bib}\TeX\ (and its more
  recent variants) will no longer provide all the required features.
  Unfortunately, however, the current release of \textsf{Biber}
  (0.9.5) contains bugs that make it tricky to use with
  \textsf{biblatex-chicago}.  These bugs have been addressed in 0.9.6
  beta, which is available for various operating systems in the
  \texttt{development} subdirectory of your SourceForge mirror, e.g.,
  \href{http://www.mirrorservice.org/sites/download.sourceforge.net/pub/sourceforge/b/project/bi/biblatex-biber/biblatex-biber/development/binaries/}{UK
    mirror}.  (If, by the time you read this, \textsf{Biber} 0.9.6 has
  already been released, then so much the better.)  Please see the
  start of \textsf{cms-dates-sample.pdf} for more details.
\item The switch to \textsf{Biber} for the author-date specification
  means that \textsf{biblatex} now provides considerably enhanced
  handling of the various date fields.  I have attempted to document
  the relevant changes in \textsf{cms-dates-sample.pdf} and in the
  \textbf{date} discussion in section~\ref{sec:fields:authdate},
  above, but in my testing the only alterations I've so far had to
  make to my .bib files involve adhering more closely to the
  instructions for specifying date ranges.  \textsf{Biber} doesn't
  like \{\texttt{1968/75}\}, and will ignore it.  Either use
  \{\texttt{1968/1975}\} or use \{\texttt{1968-{}-75}\} in the
  \textsf{year} field.
\item In the notes \&\ bibliography style, and mainly in
  \textsf{article}, \textsf{letter}, \textsf{misc}, and
  \textsf{review} entries, previous releases of
  \textsf{biblatex-chicago} recommended using the \cmd{isdot} macro
  when you needed both to define a field and not have it appear in the
  printed output.  This mechanism no longer works in \textsf{biblatex}
  1.6, and while addressing the problem I realized that relying on it
  covered over some inconsistencies and bugs in my code, so from this
  release forward you will need to modify your .bib and .tex files to
  use other, more standard mechanisms to achieve the same ends, in
  particular the \cmd{headlesscite} commands and declaring
  \texttt{useauthor=false} in the \textsf{options} field.  Please
  consult the documentation in section~\ref{sec:formatcommands}, s.v.\
  \enquote{\cmd{isdot},} for a list of example entries where you can
  see these changes at work.
\end{itemize}

Other New Features:

\begin{itemize}
\item Fixed the \cmd{smartcite} citation command in, and added a
  \cmd{smartcites} command to, \textsf{chicago-notes.cbx}, so that the
  notes \&\ bibliography style no longer prints parentheses around
  citations produced using \cmd{autocite(s)} commands inside
  \cmd{footnote} commands.  Many thanks to Louis-Dominique Dubeau for
  pointing out this error.
\item Rembrandt Wolpert and Aaron Lambert pointed out an issue with a
  command (\cmd{lbx@fromlang}) that \textsf{biblatex} no longer
  defines, and Charles Schaum very kindly suggested a temporary
  workaround in a newsgroup post, a workaround that should no longer
  be necessary.
\item Version 1.6 of \textsf{biblatex} no longer allows you to
  redefine the \texttt{minnames} and \texttt{maxnames} options in the
  \cmd{printbibliography} command, so I've defined
  \texttt{minbibnames} and \texttt{maxbibnames} in
  \textsf{biblatex-chicago.sty}, instead.  These parameters have been
  available since version 1.1, so this is now the earliest version of
  \textsf{biblatex} that will work with the Chicago styles.  Of
  course, if the (Chicago-recommended) values of these options don't
  suit your needs, you can redefine them in your document preamble.
\end{itemize}

\textbf{0.9.7a: Released March 17, 2011}
\begin{itemize}
\item Added \cmd{smartcite} command to \textsf{chicago-notes.cbx} so
  that the notes \&\ bibliography style will work with
  \textsf{biblatex} 1.3.
\item Added bibstrings \texttt{byconductor} and \texttt{cbyconductor}
  to the .lbx files, mistakenly omitted in version 0.9.7.
\item Minor fixes to the docs.
\end{itemize}

\textbf{0.9.7: Released February 15, 2011}

\mylittlespace Obsolete and Deprecated Features:
\begin{itemize}
\item The \textbf{customa} and \textbf{customb} entry types are now
  obsolete.  Any such entries will be ignored.  Please change any that
  remain to \textbf{letter} and \textbf{bookinbook}, respectively.
\item If you still have any \textbf{customc} entries containing
  introductions, prefaces, or the like, please change them to
  \textbf{suppbook}.  I have recycled \textsf{customc} for another
  purpose, on which see below.
\end{itemize}

Other New Features:

\begin{itemize}
\item At the request of Johan Nordstrom, I have added three new
  audiovisual entry types to both styles, \textbf{audio},
  \textbf{music}, and \textbf{video}.  The documentation of
  \textsf{audio} in sections~ \ref{sec:entrytypes} and
  \ref{sec:types:authdate} above contains an overview of the three,
  and the details for each type are to be found under their individual
  headings.
\item I have transformed the \textbf{customc} entry type to enable
  alphabetized cross-references --- the \enquote{c} is meant to be
  mnemonic --- to other, separate entries in a reference list or
  bibliography.  In particular, this facilitates cross-references to
  other names in a list, rather than to other works.  In author-date,
  in a procedure recommended by the \emph{Manual}, this now allows you
  to expand shorthands inside the reference list rather than in a list
  of shorthands.  In both styles, you can now provide a pointer to the
  main entry if a reader is looking an author up under, e.g., a
  pseudonym or other alternative name.
\item I have introduced the \textbf{userc} field, intended to simplify
  the printing of the cross-references provided by \textsf{customc}
  entries.  The standard \cmd{nocite} command works as well, but the
  additional mechanism may be more convenient in some circumstances.
\item You can now provide an \textbf{eventdate} in \textsf{music}
  entries to identify, e.g., a particular recording session.  It will
  be printed just after the \textsf{title}.
\item In the notes \&\ bibliography style, I have now implemented the
  \textbf{shorthandintro} field, which allows you to change the string
  introducing a shorthand in the first, long note.  It works just as
  it does in the standard \textsf{biblatex} styles.
\item At the request of Scot Becker, I have added six new
  field-exclusion options to both styles, all of which can be set both
  in the document preamble and/or in the \textsf{options} field of
  individual .bib entries.  Three of these --- \texttt{doi},
  \texttt{isbn}, and \texttt{url} --- are standard \textsf{biblatex}
  options, the others --- \texttt{bookpages}, \texttt{includeall}, and
  \texttt{numbermonth} --- are \textsf{chicago}-specific.  See the
  docs in sections~\ref{sec:chicpreset} and \ref{sec:authpreset},
  above.
\item At the request of Charles Schaum, I've added the
  \texttt{juniorcomma} option to both styles, which can be set in the
  document preamble and/or in the \textsf{options} field of individual
  entries.  It allows you to get the traditional comma between a
  surname and \enquote{Jr.} or \enquote{Sr.}
\item Fixed an old inaccuracy in the presentation of \enquote{Jr.} and
  \enquote{Sr.,} so that they now appear at the end of names printed
  surname first in bibliographies and reference lists.
\item Thanks to Andrew Goldstone, I fixed some old inaccuracies in the
  syntax of shortened notes and bibliography entries presenting
  multiple contributions to one multi-author (or single-author)
  volume.
\item I've altered the directory structure of the archive containing
  this release.  Files were multiplying, and look set to multiply
  still further, so I've copied the structure used by Lehman for
  \textsf{biblatex} itself.
\item Fixed an old bug, which I'd guess was triggered quite rarely, in
  the formatting of publication information in long notes.
\item Fixed another bug in author-date where the colon separating
  titles and subtitles was in the wrong font.  The \textsf{biblatex}
  \texttt{punctfont} option solved this.
\item Fixed a punctuation bug in \textsf{InReference} entries in the
  notes \&\ bibliography style.  Also fixed \textsf{title}
  presentation in \textsf{Reference} entries in author-date.
\item Fixed some inaccuracies in the tests establishing priority
  between \textsf{date} and \textsf{origdate} fields.  These arose
  when date ranges were involved, and it's possible I haven't yet
  addressed all possible permutations of the problem.
\item Added several new bibstrings to the \textsf{cms-*.lbx} files for
  the new audiovisual entry types.  This means that the
  \textsf{editortype} fields can now be set to \texttt{director},
  \texttt{producer}, or \texttt{conductor}, depending on your needs.
  You can also set the fields to \texttt{none}, which eliminates all
  identifying strings, and which is useful for identifying performers
  of various sorts.
\item Minor improvements to documentation.
\end{itemize}

\textbf{0.9.5a: Released September 7, 2010}
\begin{itemize}
\item Quick fix for an elementary and show-stopping mistake in
  \textsf{biblatex-chica\-go.sty}, a mistake disguised if you load
  \textsf{csquotes}, which I do in all my test files.  Mea culpa.
  Many thanks indeed to Israel Jacques and Emil Salim for pointing
  this out to me.
\end{itemize}

\textbf{0.9.5: Released September 3, 2010}

\mylittlespace Obsolete and Deprecated Features:
\begin{itemize}
\item All the custom entry types --- \textbf{customa},
  \textbf{customb}, and \textbf{customc} --- are now deprecated.  They
  will still work for the time being, but please be aware that in the
  next major release they will no longer function, at least not as you
  might be expecting.  Please change your .bib files to use
  \textbf{letter} (=\textbf{customa}), \textbf{bookinbook}
  (=\textbf{customb}), and \textbf{suppbook} (=\textbf{customc})
  instead.
\item If by some chance anyone is still using the old \cmd{custpunctc}
  macro, it is now obsolete.  It really shouldn't be needed, but let
  me know if I'm wrong.
\end{itemize}

%\vspace{2\baselineskip}

Other New Features:
\begin{itemize}
\item The Chicago author-date style is now implemented in the
  package, and is fully documented in section~\ref{sec:authdate},
  above.
\item The default way of loading the style(s) has slightly changed.
  You should put either \texttt{notes} or \texttt{authordate} in the
  options to \textsf{biblatex-chicago}, e.g.:
  \begin{quote}
    \cmd{usepackage[authordate,more options%
     \,\ldots]\{biblatex-chicago\}}
  \end{quote}
\item With the addition of the second Chicago style, I have thought it
  appropriate to alter both the name of the package and the names of
  the files it contains.  The package is now \textsf{biblatex-chicago}
  instead of \textsf{biblatex-chicago-notes-df}, and the following
  files have been renamed:
  \begin{itemize}
  \item \textsf{chicago-notes-df.cbx} is now \textsf{chicago-notes.cbx}
  \item \textsf{chicago-notes-df.bbx} is now \textsf{chicago-notes.bbx}
  \item \textsf{sample.tex} is now \textsf{cms-notes-sample.tex}
  \item \textsf{sample.pdf} is now \textsf{cms-notes-sample.pdf}
  \item \textsf{chicago-test.bib} is now \textsf{notes-test.bib}
  \item \textsf{biblatex-chicago-notes-df.pdf} (this file) is now
    \textsf{biblatex-chicago.pdf}
  \end{itemize}
  The following files have been added:
  \begin{itemize}
  \item \textsf{chicago-authordate.cbx}
  \item \textsf{chicago-authordate.bbx}
  \item \textsf{cms-dates-sample.tex}
  \item \textsf{cms-dates-sample.pdf}
  \item \textsf{dates-test.bib}
  \end{itemize}
  The following files have retained their old names:
  \begin{itemize}
  \item \textsf{cms-american.lbx}
  \item \textsf{cms-french.lbx}
  \item \textsf{cms-german.lbx}
  \item \textsf{cms-ngerman.lbx}
  \item \textsf{biblatex-chicago.sty}
  \end{itemize}
\item I have implemented the \textsf{pubstate} field, slightly
  differently yet compatibly in the two styles, to provide a simpler
  mechanism for identifying a reprinted book.  In the author-date
  style, it is highly recommended you use it, as it sorts out some
  complicated formatting questions automatically.  In the notes \&\
  bibliography style it isn't strictly necessary, but may be useful
  anyway and easier to remember than the old system.  See the
  documentation under \textsf{pubstate} in
  sections~\ref{sec:entryfields} and \ref{sec:fields:authdate}, above.
\item Users of \textsf{biblatex-chicago-notes} no longer need a
  \textsf{shortauthor} field in author-less \textsf{manual} entries,
  or in author-less \textsf{article} or \textsf{review} entries with a
  \texttt{maga\-zine} \textsf{entrysubtype}.  The package will now
  automatically take an author for short notes from the
  \textsf{organization} field for \textsf{manual} entries and from the
  \textsf{journaltitle} field for the others.  You can still use a
  \textsf{shortauthor} field if you want, but it's no longer
  necessary.  (This also holds for \textsf{chicago-authordate}.)
\item Date presentation in the \textsf{misc} entry type (with
  \textsf{entrysubtype}) has changed to fix an inaccuracy.  You can
  now use the \textsf{date} and \textsf{origdate} fields to
  distinguish between two sorts of archival source: letters and
  \enquote{letter-like} sources use \textsf{origdate}, interviews and
  other non-letters use \textsf{date}.  The only difference is in how
  the date is printed, so current .bib entries will continue to work
  fine, albeit with minor inaccuracies in the case of non-letter-like
  sources.  See the docs on \textbf{misc} in
  sections~\ref{sec:entrytypes} and \ref{sec:types:authdate}, above.
\item When only one date is presented in a \textsf{patent} entry ---
  either in the \textsf{date} or \textsf{origdate} field --- this will
  now always be used as the filing date.  In
  \textsf{biblatex-chicago-notes}, this makes a change from the
  previous (incorrect) behavior.
\item I have included the option \texttt{dateabbrev=false} in the
  default settings for \textsf{biblatex-chicago-notes}.  This ensures
  that the long month names are printed, as otherwise recent releases
  of \textsf{biblatex} print the abbreviated ones by default.
\item The provision of punctuation in \textsf{entrysubtype}
  \texttt{classical} entries has been improved, allowing the comma to
  appear before certain kinds of location specifiers even when citing
  works by their traditional divisions.  See \emph{Manual} 17.253.
  (This applies to both Chicago styles.)
\item The \textsf{number} field in \textsf{article},
  \textsf{periodical}, and \textsf{review} entries now allows you to
  include a series or range of numbers in the field, with the style
  automatically providing the correct bibstring (singular or plural).
\item I have removed and altered bibstrings in the .lbx files to take
  advantage of the new \cmd{bibsstring} and \cmd{biblstring} commands
  in \textsf{biblatex}, and added one new string
  (\texttt{origpubyear}) needed by
  \textsf{biblatex-chicago-authordate}.
\end{itemize}

\textbf{0.9a: Released March 20, 2010}
\begin{itemize}
\item Quick fixes for compatibility with \textsf{biblatex} 0.9a.
\end{itemize}

\textbf{0.9: Released March 18, 2010}

\mylittlespace Obsolete and Deprecated Features:
\begin{itemize}
\item The \textbf{userd} field is now obsolete.  All information it
  used to hold should be placed in the \textsf{edition} field.
\item The \textbf{origyear} field is now obsolete in
  \textsf{biblatex}.  It has been replaced by \textbf{origdate}, and
  because the latter allows a full date specification, I have been
  able to make the operation of \textsf{customa} (=\,\textsf{letter}),
  \textsf{misc} (with an \textsf{entrysubtype}), and \textsf{patent}
  entries more intuitive.  The RELEASE file contained in this package
  gives the short instructions on how to update your .bib files, and
  you can also consult the documentation of those entry types above.
\item The modified \textsf{csquotes.cfg} file I provided in earlier
  releases is now obsolete, and has been removed from the package.
  Please upgrade to the latest version of \textsf{csquotes} and, if
  you are still using my modified .cfg file, remove it from your \TeX\
  search path, or at the very least excise the code I provided.
\end{itemize}

Other New Features:
\begin{itemize}
\item Added the files \textsf{cms-german.lbx} (with its clone
  \textsf{cms-ngerman.lbx}) and \textsf{cms-french.lbx}, which allow
  the creation of Chicago-like references in those languages.  See
  section \ref{sec:international} above for details on usage.
\item Added the \texttt{annotation} package option to allow the
  creation of annotated bibliographies.  This code is still not
  entirely polished yet, but it is usable.  Please see page
  \pageref{sec:annote} above for instructions and hints.
\item Added \textsf{biblatex's} new \textbf{bookinbook} entry type,
  which currently functions as an alias of the \textsf{customb} type.
  As \textsf{biblatex} now provides standard equivalents for all of
  the custom types I initially found it necessary to provide ---
  \textsf{letter}~= \textsf{customa}, \textsf{bookinbook}~=
  \textsf{customb}, and \textsf{suppbook} \& \textsf{suppcollection}~=
  \textsf{customc} --- it may soon be time to prune out the custom
  types to enhance compatibility with other \textsf{biblatex} styles.
  I shall give plenty of warning before I do so.
\item In line with the new system adopted in \textsf{biblatex} 0.9,
  using the \textsf{editortype} field turns off the usual string
  concatenation mechanisms of the Chicago style.  See Lehman's RELEASE
  file for a discussion of this.
\item I have added support for the new \textsf{editor[a--c]} and
  \textsf{editor[a--c]type} fields, and they work just as in standard
  \textsf{biblatex}, though I'm uncertain how much use they'll get
  from users of the Chicago style.
\item I have added many bibstrings to the .lbx files to help with
  internationalization.  The new ones that you might want to use in
  your .bib files include: \texttt{pseudonym}, \texttt{nodate},
  \texttt{revisededition}, \texttt{numbers}, and \texttt{reviewof}.
  Please see section~\ref{sec:international} for a fuller list.
\end{itemize}

\textbf{0.8.9d: Released February 17, 2010}
\begin{itemize}
\item Chris Sparks and Aaron Lambert both found formatting bugs in the
  0.8.9c code.  I've fixed these bugs, and am releasing this version
  now, the last in the 0.8.9 series.  The next release of
  \textsf{biblatex-chicago-notes-df}, due as soon as possible, will
  contain many more significant changes, including those necessary for
  it to function properly with the recently-released \textsf{biblatex}
  version 0.9.  In the meantime, at least version 0.8.9d should produce
  more accurate output.
\end{itemize}

\textbf{0.8.9c: Released November 4, 2009}
\begin{itemize}
\item Emil Salim noticed that the \emph{ibidem} mechanism wasn't
  working properly, printing the page number after \enquote{Ibid} even
  when the page reference of the preceding citation was identical.
  The fix for this involved setting \texttt{loccittracker=constrict}
  in \textsf{biblatex-chicago.sty}, something you'll have to do
  manually yourself if you're loading the package via a call to
  \textsf{biblatex} rather than to \textsf{biblatex-chicago}.
\item Several users have reported unwanted behavior when repeated
  names in bibliographies are replaced with the \texttt{bibnamedash}.
  This release should fix both when the \texttt{bibnamedash} appears
  and what punctuation follows it.
\end{itemize}

\textbf{0.8.9b: Released September 9, 2009}
\begin{itemize}
\item Fixed a long-standing bug in formatting names in the
  bibliography.  The package now correctly places a comma after the
  reversed name that begins the entry, using \textsf{biblatex's}
  \cmd{revsdnamedelim} command.  Many thanks to Johanna Pink for
  catching my rather egregious error.
\item While fixing some formatting errors that cropped up when using
  the newest version of \textsf{biblatex} (0.8h at time of writing), I
  also spotted some more venerable bugs in the code for using
  shortened cross-references for citing multiple entries in a
  collection of essays or letters.  I believe this now works
  correctly, but please let me know if you discover differently.
\item Joseph Reagle noticed that endnote marks (produced using the
  \textsf{endnotes} package) did not receive the
  same treatment as footnote marks.  I have rectified this, placing
  the code in \textsf{biblatex-chicago.sty} so that you can turn it
  off either by using the old package-loading system or by setting the
  \texttt{footmarkoff} package option when loading
  \textsf{biblatex-chicago}.
\item Updates to Lehman's \textsf{csquotes} package have rendered my
  modifications in \textsf{csquotes.cfg} obsolete.  Please use the
  latest version of \textsf{csquotes} (4.4a at time of writing) and
  ignore my file, which will disappear in a later release.
\item At the request of Will Small, I have included some code, still
  in an alpha state, to allow you to specify, in the bibliography, the
  original publication details of essays which you are citing from
  later reprints (a \emph{Collected Essays} volume, for example).  See
  the documentation above under the \textsf{\mycolor{reprinttitle}}
  field if you would like to test this functionality.
\end{itemize}

%%\enlargethispage{-3\baselineskip}

\textbf{0.8.9a: Released July 5, 2009}
\begin{itemize}
\item Slight changes for compatibility with \textsf{biblatex} 0.8e.
  The package still works with 0.8c and 0.8d, as well.
\end{itemize}

\textbf{0.8.9: Released July 2, 2009}

\mylittlespace Obsolete and Deprecated Features:
\begin{itemize}
\item The \textbf{single-letter bibstrings} (\cmd{bibstring\{a\}},
  \cmd{bibstring\{b\}}, etc.) are now obsolete.  You should replace
  any still present in your .bib file with \cmd{autocap} commands ---
  see �~3.8.4 of \textsf{biblatex.pdf}.
\end{itemize}

Other New Features:
\begin{itemize}
\item The default way of loading the package is now with

  \cmd{usepackage[further-options]\{biblatex-chicago\}}

  rather than

  \cmd{usepackage[style=chicago-notes-df,further-options]\{biblatex\}}.

  Please see section~\ref{sec:loading} above for details and hints.
\item Package-specific bibstrings have been removed from the .cbx and
  .bbx files and are now gathered in a new file,
  \textbf{cms-american.lbx}, which changes the way the package
  interacts with \textbf{babel}.  It is now somewhat simpler if you
  want the defaults, but somewhat more complex if you require
  non-standard features.  Please see section~\ref{sec:otherpacks}
  above for more details.
\item Two new entry types have been added: \textbf{artwork} for works
  of visual art excluding photographs, and \textbf{image} for
  photographs.  See the documentation of \textsf{artwork} for how to
  create .bib entries for both types.
\item Added the new bibliography and entry option
  \textbf{usecompiler}, set to \texttt{true} by default.  This
  streamlines the code that finds a name to head an entry
  (\textbf{author -> editor [or namea] -> translator [or nameb] ->
    compiler [namec] -> title}).  The whole system should work more
  consistently now, but do see the \textsf{author} and \textsf{namec}
  documentation for improved notes on how to use it.
\item Added the new bibliography option \textbf{footmarkoff}, to turn
  off the optional in-line (as opposed to superscript) formatting of
  the marks in foot- or endnotes.  You only need this if you load the
  package with the new default \cmd{usepackage\{biblatex-chicago\}};
  users loading it the old way get default \LaTeX\ formatting.
\item At Matthew Lundin's request, I have added the citation command
  \textbf{\textbackslash head\-lesscite}, which works like
  \cmd{headlessfullcite} but allows \textsf{biblatex} to decide
  whether to print the full or the short note.
\item Fully adopted \textsf{biblatex's} system for providing
  end-of-entry punctuation, which should solve some of the bugs users
  have been finding.  See section~\ref{sec:otherhints}, above, and do
  please let me know if inconsistencies remain.
\item Added a modified \textbf{csquotes.cfg} file to address issues
  users were having when using the \textbf{Xe\LaTeX} engine in
  combination with \textsf{biblatex-chicago}.  See
  section~\ref{sec:otherpacks}, above.
\item Added \texttt{natbib} option to allow users of the default setup
  to continue to benefit from \textsf{biblatex's} \textsf{natbib}
  compatibility code.  Thanks to Bennett Helm for pointing out this
  issue.
\item Added a \textbf{shorthandibid} option to allow the printing of
  \emph{ibid.}\ in consecutive references to an entry that contains a
  \textsf{shorthand} field.  Thanks to Chris Sparks for calling my
  attention to this problem.
\item While investigating the preceding, I noticed failures when
  combining the \texttt{short} option with a \textsf{shorthand} field.
  The package now actually does what it has always claimed to do under
  \textbf{shorthand}.
\item Many small bug fixes and improvements to the documentation.
\end{itemize}

To Do:

\begin{itemize}
\item The shorthand vs \emph{ibid.}\ question may need more careful
  addressing in some cross references, and also in relation to the
  \texttt{noibid} package option.
\item Charles Schaum has quite rightly pointed out the inconsistency
  in my naming conventions --- \textsf{biblatex-chicago.sty} as
  opposed to \textsf{chicago-notes-df.cbx}, for example.  I'm going to
  delay a decision on which way to go with this until a later release.
\end{itemize}

\textbf{0.8.5a: Released June 14, 2009}

\begin{itemize}
\item Quick and dirty fixes to bibliography strings to allow
  compatibility with \textsf{biblatex} version 0.8d.  If you are still
  using 0.8c, then I would wait for the next version of
  \textsf{biblatex-chicago-notes-df}, which is due soon.  See README.
\end{itemize}

\textbf{0.8.5: Released January 10, 2009}

\mylittlespace Obsolete and Deprecated Features:

  \begin{itemize}
  \item The \textbf{\textbackslash custpunct} commands are now
    deprecated --- Lehman's \enquote{American} punctuation tracking
    facilities should handle quoted text automatically, assuming you
    remember always to use \textbf{\textbackslash mkbibquote} in your
    database.  If you still need \cmd{custpunct}, please let me know,
    as it may be an error in the style.
  \item With \cmd{custpunct} no longer needed, the toggles activated
    by placing \enquote{\texttt{plain}} in the \textbf{type} or
    \textbf{userb} fields are also deprecated.
  \end{itemize}

Other New Features:

\begin{itemize}
\item At least \textbf{biblatex 0.8b} is now required --- 0.8c works
  fine, as well.
\item I now \emph{strongly recommend} that you use \textbf{babel} with
  \enquote{\texttt{american}} as the main text language.  See
  section~\ref{sec:otherpacks} above for further details.
\item The \textbf{customc} entry type has been revised, allowing you
  to cite any sort of supplementary material using the \textbf{type}
  field instead of relying on toggles in the \textsf{introduction},
  \textsf{afterword}, and \textsf{foreword} fields, though these
  latter still work.  The two new entry types \textbf{suppbook} and
  \textbf{suppcollection} are both aliased to \textsf{customc}, and
  therefore work in exactly the same way.
\item The new entry type \textbf{suppperiodical} is aliased to
  \textbf{review}.
\item The new entry type \textbf{letter} is aliased to
  \textbf{customa}.
\item In \textbf{inreference} entries the \textsf{postnote} field of
  all \cmd{cite} commands is now treated like data in \textsf{lista},
  that is, it will be placed within quotation marks and prefaced with
  the appropriate string.  The only difference is that you can only
  put one such article name in \textsf{postnote}, as it isn't a list
  field.
\item I've set the new \textsf{biblatex} option \texttt{usetranslator}
  to \texttt{true} by default, which means entries will automatically
  be alphabetized by their \textsf{translator} in the absence of an
  \textsf{author} or an \textsf{editor}.
\item A host of small formatting errors were eliminated, nearly all of
  them through adopting Lehman's punctuation tracker.
\item In the main body of this documentation, I've added some
  \mycolor{\textbf{color coding}} to help you more quickly to identify
  entry types and fields that are either new or that have undergone
  significant revision.
\end{itemize}

To Do:

\begin{itemize}
\item Separate out \enquote{options} from the basic citation
  \enquote{style,} using a \LaTeX\ style file.  This is an
  architectural change recommended by Lehman.
\end{itemize}

\textbf{0.8.2.2: Released November 24, 2008}

\begin{itemize}
\item Fixed spurious commas appearing in some bibliography entries,
  spotted by Nick Andrewes.  While investigating this I noticed a more
  general problem with punctuation after italicized titles ending with
  question marks or exclamation points.  This will be addressed in
  forthcoming revisions both of \textsf{biblatex} and of this package.
\item Nick also reported some problems with spurious punctuation in
  the bibliography when using XeLaTeX.  I haven't yet been able to pin
  down the exact cause of these, but if you are using XeLaTeX and are
  having (or have solved) similar problems I'd be interested to hear
  from you.
\end{itemize}

\textbf{0.8.2: Released November 3, 2008}

\begin{itemize}
\item Fixed several formatting glitches between citations in multicite
  commands (spotted by Joseph Reagle) and also after some prenotes. 
\end{itemize}

\textbf{0.8.1: Released October 22, 2008}

\mylittlespace Obsolete and Deprecated Features:

\begin{itemize}
\item The \textbf{origlocation} field is now obsolete, and has been
  replaced by \textbf{lista}.  Please update your .bib files
  accordingly.
\item The single-letter \textbf{\textbackslash bibstring} commands I
  provided in version 0.7 are now deprecated.  In most cases, you'll
  be able to take advantage of the automatic contextual capitalization
  facilities introduced in this release, but if you still need the
  single-letter \cmd{bibstring} functionality then you should switch
  to \cmd{autocap}, as I shall be removing the single-letter
  \texttt{bibstrings} in a future release.  See above under
  \textbf{\textbackslash autocap} for all the details.
\item The \textbf{userd} field is now deprecated, as \textsf{biblatex}
  0.8 allows all forms of data to be included in the \textsf{edition}
  field.  I shall be removing \textsf{userd} in a future release, so
  please update your .bib files as soon as is convenient.
\end{itemize}

Other New Features:

\begin{itemize}
\item Updated the .bbx and .cbx files to work with \textsf{biblatex}
  0.8.  This most recent version of \textsf{biblatex} is now required
  for \textsf{biblatex-chicago-notes-df} to work.
\item Added the \textbf{usera} field, which holds supplemental
  information about a \textsf{journaltitle} in \textsf{article} and
  \textsf{review} entries.  See the documentation of the field for
  details.
\item Added the \textbf{\textbackslash citetitles} multicite command
  to fix a problem with spurious punctuation when multiple titles were
  listed.
\item Added the \textbf{\textbackslash Citetitle} command to help with
  automatic capitalization of titles when they occur at the beginning
  of a note.
\item Minor punctuation fixes in \textsf{biblatex-chicago-notes-df.bbx}.
\end{itemize}

To Do:

\begin{itemize}
\item Integrate \textsf{biblatex's} American punctuation facilities.
\item Separate out \enquote{options} from the basic citation
  \enquote{style,} using a \LaTeX\ style file.  This is an
  architectural change recommended by Lehman.
\item Investigate and possibly integrate the new entry types provided
  in \textsf{biblatex} 0.8.
\end{itemize}

\textbf{0.7: First public release, September 18, 2008}

\end{document}
